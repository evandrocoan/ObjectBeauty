

% Is it possible to keep my translation together with original text?
% https://tex.stackexchange.com/questions/5076/is-it-possible-to-keep-my-translation-together-with-original-text
\chapter{Manual do Formatador de Código}
\label{manualDoFormatadorDeCodigo}


Esse apêndice apresenta um manual da ferramenta.
Não recomendada~=se que usuários que não possuam conhecimentos sobre a semântica das linguagens de programação,
utilizem esta ferramenta para realizar a formatação de código~=fonte.
Neste caso,
o mais indicado é que seja utilizado as demais ferramentas de formatação,
que são mais especificas para cada linguagem individualmente e
possuem inerentemente os conhecimentos específicos da sintaxe e
semântica da linguagem a ser formatada.

Não foi criada nenhuma interface gráfica ou
de linha de comando que faça a entrada do programa a ser formatado e
das configurações do formatador.
Existem duas implementações que utilizam a metalinguagem (\typeref{grammarsGrammarPy}).
Uma utiliza a metalinguagem para adicionar cores (\typeref{codeHighlighterPy}),
como feito em editores de texto e
a outra realiza a formatação de código~=fonte (\typeref{codeFormatterPy}).

Nos \typeref{mainHighlighterPy,mainFormatterPy},
encontra~=se um exemplo simples de programa que pode ser construído para executar o Formatador de Código e
a Adição de Cores.
Sua construção é a mesma utilizada para a criação dos testes de unidade (\typeref{unitTestsPy}).
Tanto o \typeref{mainFormatterPy} quanto do o \typeref{mainHighlighterPy} geram como resultado um arquivo HTML,
contendo como conteúdo o resultado de seu trabalho.

Para se realizar a execução de qualquer arquivo deste projeto,
é necessário ter um interpretador ``Python 3.6'' instalado,
junto com as bibliotecas ``pip3'',
``debug\_tools'', ``dominate'' e
``pushdown''.
Em uma instalação tradicional ``Ubuntu'',
estes pacotes podem ser instalados com os seguintes comandos:
\begin{enumerate}[1)]
\item \mintinline{shell}{sudo apt-get install python3 python3-pip}
\item \mintinline{shell}{pip3 install -r requirements.txt} (\typeref{requirementsTxt})
\item \mintinline{shell}{python3 main_formatter.py} (\typeref{mainHighlighterPy})
\item \mintinline{shell}{python3 main_highlighter.py} (\typeref{mainFormatterPy})
\end{enumerate}
\begin{code}
\caption{Arquivo ``source/requirements.txt''}
\label{requirementsTxt}
\inputminted{python3}{../source/requirements.txt}
\end{code}
\begin{code}
\caption{Arquivo ``source/main\_formatter.py''}
\label{mainHighlighterPy}
\inputminted{python3}{../source/main_formatter.py}
\end{code}
\begin{code}
\caption{Arquivo ``source/main\_highlighter.py''}
\label{mainFormatterPy}
\inputminted{python3}{../source/main_formatter.py}
\end{code}
\begin{code}
\caption{Arquivo ``source/utilities.py''}
\label{utilitiesPy}
\inputminted{python3}{../source/utilities.py}
\end{code}


\chapter{Código dos Testes de Unidade}

Para se executar os testes de unidade,
basta executar o comando ``\mintinline{shell}{python3 unit_tests.py}''.
No \typeref{unitTestsResults},
pode~=se encontrar um exemplo de execução dos testes de unidade apresentados no \typeref{unitTestsPy}.
Primeiro foi desenvolvida a ferramenta de Adição de Cores para facilitar os testes da metalinguagem e
do Analisador Semântico.
Uma vez que se comprovou o funcionamento da metalinguagem,
foi realizada um implementação mínima de um formatador de código~=fonte.

Ambas as implementações do módulo de Adição de Cores quanto a do Formatador de Código são somente uma prova de conceito do que pode ser feito com a metalinguagem desenvolvida.
No \typeref{unitTestsPy},
a classe ``TestingGrammarUtilities'' é uma classe abstrata \cite{understandingDataAbstraction} que contém características necessárias a todos os tipos de testes de unidades implementados,
tanto os testes de unidade de Adição de Cores quanto os testes de unidade do Formatador de Código.
No total existem 22 testes:
\begin{enumerate}[i)]
\item 16 testes do Analisador Semântico \typeref{semanticAnalyzerPy} na classe testes ``TestSemanticRules''.
\item 5 testes de Adição de Cores \typeref{codeHighlighterPy} na classe testes ``TestCodeHighlighterBackEnd''.
\item 1 testes de Formatação de Código \typeref{codeFormatterPy} na classe testes ``TestCodeFormatterBackEnd''.
\end{enumerate}%

Como a ferramenta de Adição de Cores foi a primeira desenvolvida,
ela obteve a criação de mais testes de unidade para assegurar que a implementação do Analisador Semântico \typeref{semanticAnalyzerPy} e
da Metagramática estão funcionais.
Uma vez feito isso,
duplicou~=se o código~=fonte da implementação de adição de cores \typeref{codeHighlighterPy} e
realizou~=se as alterações necessárias para que se faça a formatação de código~=fonte \typeref{codeFormatterPy}.
\begin{code}
\caption{Resultado da execução dos Testes de Unidade''}
\label{unitTestsResults}
\begin{minted}{text}
$ python3 unit_tests.py
00:02:41:738.736391 1.52e-04 - source.<module>:52 - Importing __main__
......................
----------------------------------------------------------------------
Ran 22 tests in 2.086s

OK
\end{minted}
\end{code}
\begin{code}
\caption{Arquivo ``source/unit\_tests.py''}
\label{unitTestsPy}
\inputminted{python3}{../source/unit_tests.py}
\end{code}


\chapter{Código do Analisador Semântico}

De todos os códigos~=fonte criados neste trabalho,
o Analisador Semântico (\typeref{semanticAnalyzerPy}) é o maior deles.
Sua implementação é utilizada diretamente pelo módulo de Adição de Cores e
Formatação de Código.
A regras do Analisador Semântico estão divididas entre dois tipos,
erros e
alertas.
Um erro é algo que impede completamente o programa final de funcionar.
Um alerta é algo que não impede que o programa final funcione corretamente,
mas é algo que precisa ser revisto.

Os testes das regras do Analisador Semântico podem ser encontradas na classe ``TestSemanticRules'' do \typeref{unitTestsPy}.
A seguir pode~=se encontrar algumas das regras semânticas implementadas pelo Analisador Semântico do \typeref{semanticAnalyzerPy}.
O nome de cada uma das regras a seguir começa com o prefixo ``test\_'' que corresponde ao nome do teste de unidade,
criado para verificar tal regra semântica na classe de testes ``TestSemanticRules'' (\typeref{unitTestsPy}).
% https://tex.stackexchange.com/questions/513020/how-can-i-stop-this-useless-empty-line-popping-up-in-the-middle-of-my-lists
\begin{enumerateoptional}[1.]
    \item[\bfseries\mintinline{text}{test_duplicatedContext}] Detecção de contextos duplicados e
    emissão de um erro semântico.

    \item[\bfseries\mintinline{text}{test_duplicatedIncludes}] Detecção de inclusões duplicadas  e
    emissão de um erro semântico.

    \item[\bfseries\mintinline{text}{test_invalidRegexInput}] Detecção de expressões regulares inválidas e
    emissão de um erro semântico.

    \item[\bfseries\mintinline{text}{test_missingIncludeDetection}] Detecção da inclusão um bloco inexistente e
    emissão de um erro semântico.

    \item[\bfseries\mintinline{text}{test_duplicatedGlobalNames}] Detecção de múltiplas definições do nome da gramática e
    emissão de um erro semântico.

    \item[\bfseries\mintinline{text}{test_missingScopeGlobalName}] Detecção da falta da definição do nome do escopo global da gramática e
    emissão de um erro semântico.

    \item[\bfseries\mintinline{text}{test_missingNameGlobal}] Detecção de esquecer de definir o nome da gramática e
    emissão de um erro semântico.

    \item[\bfseries\mintinline{text}{test_unsusedInclude}] Detecção de criação de um contexto e
    em esquecer de utilizar ele e
    emissão de um alerta.

    \item[\bfseries\mintinline{text}{test_redifinedConst}] Detecção de redefinir um valor constante e
    emissão de um erro semântico.

    \item[\bfseries\mintinline{text}{test_unsusedConstantDeclaration}] Detecção de definir um valor constante e esquecer de utilizar ele e
    emissão de um alerta.

    \item[\bfseries\mintinline{text}{test_usingConstOutOfScope}] Detecção de tentativa de usar um valor constante fora do escopo dele e
    emissão de um erro semântico.
\end{enumerateoptional}%
\begin{code}
\caption{Arquivo ``source/semantic\_analyzer.py''}
\label{semanticAnalyzerPy}
\inputminted{python3}{../source/semantic_analyzer.py}
\end{code}


\chapter{Código de Adição de Cores}

\begin{code}
\caption{Arquivo ``source/code\_highlighter.py''}
\label{codeHighlighterPy}
\inputminted{python3}{../source/code_highlighter.py}
\end{code}


\chapter{Código do Formatador}

\begin{code}
\caption{Arquivo ``source/code\_formatter.py''}
\label{codeFormatterPy}
\inputminted{python3}{../source/code_formatter.py}
\end{code}


\chapter{Código da Metagramática}

\begin{code}
\caption{Arquivo ``source/grammars\_grammar.pushdown''}
\label{grammarsGrammarPy}
\inputminted{antlr}{../source/grammars_grammar.pushdown}
\end{code}
