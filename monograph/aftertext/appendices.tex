

% Is it possible to keep my translation together with original text?
% https://tex.stackexchange.com/questions/5076/is-it-possible-to-keep-my-translation-together-with-original-text
\chapter{Manual do Formatador de Código}
\label{manualDoFormatadorDeCodigo}

Não foi criado nenhuma interface gráfica ou
de linha de comando que faça a entrada do programa a ser formatado e
das configurações do formatador.
Para realizar os testes de funcionamento do formatador,
foram realizados testes de unidade automatizados (\typeref{unitTestsPy}).
Existem duas implementações que utilizam a metalinguagem (\typeref{grammarsGrammarPy}).
Uma utiliza a metalinguagem para adicionar cores (\typeref{codeHighlighterPy}),
como feito em editores de texto e
a outra realiza a formatação de código~=fonte (\typeref{codeFormatterPy}).

Nos \typeref{mainHighlighterPy,mainFormatterPy},
encontra~=se um exemplo simples de programa que pode ser construído para executar o Formatador de Código e
a Adição de Cores.
Sua construção é a mesma utilizada para a criação dos testes de unidade (\typeref{unitTestsPy}).
Tanto o \typeref{mainFormatterPy} quanto do o \typeref{mainHighlighterPy} geram como resultado um arquivo HTML,
contendo como conteúdo o resultado de seu trabalho.
\begin{code}
\caption{Arquivo ``source/main\_formatter.py''}
\label{mainHighlighterPy}
\inputminted[fontsize=\small,linenos=true,numberblanklines=true,breaklines=true]{python3}{../source/main_formatter.py}
\end{code}
\begin{code}
\caption{Arquivo ``source/main\_highlighter.py''}
\label{mainFormatterPy}
\inputminted[fontsize=\small,linenos=true,numberblanklines=true,breaklines=true]{python3}{../source/main_formatter.py}
\end{code}
\begin{code}
\caption{Arquivo ``source/utilities.py''}
\label{utilitiesPy}
\inputminted[fontsize=\small,linenos=true,numberblanklines=true,breaklines=true]{python3}{../source/utilities.py}
\end{code}


\chapter{Código dos Testes de Unidade}

\begin{code}
\caption{Arquivo ``source/unit\_tests.py''}
\label{unitTestsPy}
\inputminted[fontsize=\small,linenos=true,numberblanklines=true,breaklines=true]{python3}{../source/unit_tests.py}
\end{code}


\chapter{Código do Analisador Semântico}

\begin{code}
\caption{Arquivo ``source/semantic\_analyzer.py''}
\label{semanticAnalyzerPy}
\inputminted[fontsize=\small,linenos=true,numberblanklines=true,breaklines=true]{python3}{../source/semantic_analyzer.py}
\end{code}


\chapter{Código de Adição de Cores}

\begin{code}
\caption{Arquivo ``source/code\_highlighter.py''}
\label{codeHighlighterPy}
\inputminted[fontsize=\small,linenos=true,numberblanklines=true,breaklines=true]{python3}{../source/code_highlighter.py}
\end{code}


\chapter{Código do Formatador}

\begin{code}
\caption{Arquivo ``source/code\_formatter.py''}
\label{codeFormatterPy}
\inputminted[fontsize=\small,linenos=true,numberblanklines=true,breaklines=true]{python3}{../source/code_formatter.py}
\end{code}


\chapter{Código da Metagramática}

\begin{code}
\caption{Arquivo ``source/grammars\_grammar.pushdown''}
\label{grammarsGrammarPy}
\inputminted[fontsize=\small,linenos=true,numberblanklines=true,breaklines=true]{antlr}{../source/grammars_grammar.pushdown}
\end{code}
