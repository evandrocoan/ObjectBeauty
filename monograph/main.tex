%https://github.com/SublimeText/LaTeXTools/issues/1439
%!TEX output_directory=latexcache

% You can build this using the command:
% latexmk -pdf -jobname=output -output-directory=cache -aux-directory=cache -pdflatex="pdflatex -interaction=nonstopmode" -use-make main.tex

% When the bibliography includes a cyclic reference to another bibliography,
% you need to run `pdflatex` 5 times on the following order:
% 1. `pdflatex`,
% 2. `biber`,
% 3. `pdflatex`
% 4. `pdflatex`
% 5. `pdflatex`
% 6. `biber`
% 7. `pdflatex`

% Monograph LaTeX Template for UFSC based on:
% 1. https://github.com/royertiago/tcc
% 2. http://portal.bu.ufsc.br/normalizacao/
% 3. https://github.com/evandrocoan/ufscthesisx
% 4. http://www.latextemplates.com/template/simple-sectioned-essay

% Initially translated from Portuguese with help of https://github.com/omegat-org/omegat <Computer Assisted Translation of LaTeX document>
% https://tex.stackexchange.com/questions/313732/computer-assisted-translation-of-latex-document

% Allows you to write your thesis both in English and Portuguese
% https://tex.stackexchange.com/questions/5076/is-it-possible-to-keep-my-translation-together-with-original-text
\newif\ifenglish\englishfalse
\newif\ifadvisor\advisorfalse

% Uncomment the line `\englishtrue` to set the document default language to English
% \englishtrue
% \advisortrue

% https://tex.stackexchange.com/questions/131002/how-to-expand-ifthenelse-so-that-it-can-be-used-in-parshape
\newcommand{\lang}[2]{\ifenglish#1\else#2\fi}
\newcommand{\advisor}[2]{\ifadvisor#1\else#2\fi}

% https://tex.stackexchange.com/questions/385895/how-to-make-passoptionstopackage-add-the-option-as-the-last
% https://tex.stackexchange.com/questions/484400/changing-the-cleveref-package-language-conjunction-definition
% https://tex.stackexchange.com/questions/516058/why-isnt-my-biblatex-language-changing-when-passing-the-language-on-my-document
\ifenglish
    \PassOptionsToPackage{brazil,main=english,spanish,french}{babel}
\else
    \PassOptionsToPackage{main=brazil,english,spanish,french}{babel}
\fi

% Simple alias for English and Portuguese words
% https://tex.stackexchange.com/questions/513019/argument-of-bbltempd-has-an-extra
\newcommand{\brazilword}[1]{\protect\foreignlanguage{brazil}{#1}}
\newcommand{\englishword}[1]{\protect\foreignlanguage{english}{#1}}

% Allow you to write `Evandro's house` in latex as `Evandro\s house` instead of `Evandro\textquotesingle{}s house`
% https://tex.stackexchange.com/questions/31091/space-after-latex-commands
\newcommand{\s}[0]{\textquotesingle{}s\xspace}
\newcommand{\q}[0]{\textquotesingle{}\xspace}

% Uncomment the following line if you want to use other biblatex settings
% \PassOptionsToPackage{style=numeric,repeatfields=true,backend=biber,backref=true,citecounter=true}{biblatex}
\documentclass[
\lang{english}{brazilian,brazil}, % https://tex.stackexchange.com/questions/484400/changing-the-cleveref-package-language-conjunction-definition
12pt, % Padrão UFSC para versão final
a4paper, % Padrão UFSC para versão final
oneside, % Impressão nos dois lados da folha
chapter=TITLE, % Título de capítulos em caixa alta
section=TITLE, % Título de seções em caixa alta
]{setup/ufscthesisx}

% Utilize o arquivo aftertext/references.bib para incluir sua bibliografia.
% http://tug.ctan.org/tex-archive/macros/latex/contrib/cleveref/cleveref.pdf
\addbibresource{aftertext/references.bib}

% https://www.overleaf.com/learn/latex/Inserting_Images
\graphicspath{{pictures/}}

% FIXME: Preencha com seus dados
\autor{\brazilword{Evandro Coan}}
\titulo{\lang{A Grammar Programmable\protect\\Source Code Formatting Tool}{Uma Ferramenta de Formatação\protect\\Programável Por Gramáticas}}

% FIXME: Se houver subtítulo, descomente a linha abaixo
% \subtitulo{\lang{Subtitle}{Subtítulo}}

% FIXME: Siglas para grau de formação Dr./Dra., Me./Ma, Bel. Bela. (inglês: PhD., MSc., Bs.)
\orientador[\lang{Supervisor}{Orientador}]{\brazilword{Rafael de Santiago}, \lang{Phd.}{Dr.}}

% FIXME: Se houver coorientador, descomente a linha abaixo
% \coorientador[\lang{Co-supervisor}{Coorientador(a)}]{\brazilword{Nome do coorientador(a)}, \lang{Phd.}{Dr.}}

% FIXME: Preencher com o nome do Coordenador de TCCs/Teses do seu curso
\coordenador[Coordenador]{\brazilword{José Francisco Danilo De Guadalupe Correa Fletes}, \lang{Phd.}{Dr.}}

% FIXME: Local da sua defesa
\local{\brazilword{Florianópolis, Santa Catarina} -- \lang{Brazil}{Brasil}}

% FIXME: Ano da sua defesa
\ano{2019}
\biblioteca{\lang{University Library}{Biblioteca Universitária}}

% FIXME: Sigla da sua instituição
\instituicaosigla{UFSC}
\instituicao{\brazilword{Universidade Federal de Santa Catarina}}

% FIXME: Preencha com Tese, Dissertação, Monografia ou Trabalho de Conclusão de Curso, Bachelor's Thesis, etc
\tipotrabalho{\lang{Bachelor\s Thesis}{Trabalho de Conclusão de Curso}}

% FIXME: Se houver Área de Concentração, descomente a linha abaixo
\area{\lang{Formal Languages}{Linguagens Formais}}

% FIXME: Preencha com Doutor, Bacharel ou Mestrando
\formacao{\lang
    {Bachelor of Science degree in Computer Science}
    {Bacharel em Ciências da Computação}%
}
\programa{\lang
    {Undergraduate Program in Computer Science}
    {Programa de Graduação em Ciências da Computação}%
}

% FIXME: Preencha com Departamento de XXXXXX, Centro de XXXXXX
\centro{\lang
    {INE -- Department of Informatics and Statistics, CTC -- Technological Center}
    {INE -- Departamento de Informática e Estatística, CTC -- Centro Tecnológico}%
}

% FIXME: Preencha com Campus XXXXXX     ou     Centro de XXXXXX
\campus{\brazilword{Campus Reitor João David Ferreira Lima}}

% FIXME: Data da sua defesa
\data{\lang{25 of November of}{25 de Novembro de} 2019}

% O preambulo deve conter tipo do trabalho, objetivo, nome da instituição e a área de concentração.
\preambulo{\lang%
    {%
        \imprimirtipotrabalho~submitted to the \imprimirprograma~of
        \imprimirinstituicao~for degree acquirement in \imprimirformacao.%
    }{%
        \imprimirtipotrabalho~submetido ao \imprimirprograma~da
        \imprimirinstituicao~para a obtenção do Grau de \imprimirformacao.%
    }%
}

% Allows you to use ~= instead of `\hyp{}`
% https://tex.stackexchange.com/questions/488008/how-to-create-an-alternative-to-shortcut-or-hyp
% https://tex.stackexchange.com/questions/405718/depending-on-babel-language-setting-i-get-biblatex-error-argument-of-language
% https://tex.stackexchange.com/questions/340661/argument-of-languageactivearg-has-an-extra-i-use-includegraphics-and-r
\useshorthands{~}\defineshorthand{~=}{\hyp{}}

\palavraschaveufsc{palavraschaveingles}   {Text Formatter}
\palavraschaveufsc{palavraschaveportugues}{Formatador do texto}

\palavraschaveufsc{palavraschaveingles}   {Source Code Beautifier}
\palavraschaveufsc{palavraschaveportugues}{Embelezador de código~=fonte}

\palavraschaveufsc{palavraschaveingles}   {Pretty~=printing}
\palavraschaveufsc{palavraschaveportugues}{Impressão~=bonita}

\palavraschaveufsc{palavraschaveingles}   {Context~=free Grammars}
\palavraschaveufsc{palavraschaveportugues}{Gramáticas Livre de Contexto}

\palavraschaveufsc{palavraschaveingles}   {Programming Languages Syntax}
\palavraschaveufsc{palavraschaveportugues}{Sintaxe de Linguagens de Programação}


\hypersetup
{
    pdfsubject={Thesis' Abstract},
    pdfcreator={LaTeX with abnTeX2 for UFSC},
    pdftitle={\imprimirtitulo},
    pdfauthor={\imprimirautor},
    pdfkeywords={\lang{\palavraschaveinglessemitem}{\palavraschaveportuguessemitem}},
}

% Altere o arquivo 'settings.tex' para incluir customizações de aparência da sua tese


% Thesis settings
\newcommand{\brazil}[1]{\foreignlanguage{brazil}{#1}}
\newcommand{\english}[1]{\foreignlanguage{english}{#1}}

\newcommand{\s}[0]{\textquotesingle{}s{ }}

% What is the difference between \def and \newcommand?
% https://tex.stackexchange.com/questions/655/what-is-the-difference-between-def-and-newcommand
\def\mytextpreliminarylistname{\chooselang{Brief Table of Contents}{Breve Sumário}}

% How to manually set where a word is split?
% https://tex.stackexchange.com/questions/182569/how-to-manually-set-where-a-word-is-split
\hyphenation{Ge-la-im}

% Informações de dados para CAPA e FOLHA DE ROSTO
\titulo
{%
    \chooselang
    {Good Programming Practices \& Style}
    {Boas Práticas de Programação \& Estilo}
}
\subtitulo
{%
    \chooselang
    {Universal Programming Tools}
    {Ferramentas Universais de Programação}
}

\data{\today}
\autor{\brazil{Evandro Coan}}
\local{\chooselang{\brazil{Florianópolis, Santa Catarina} -- Brazil}{Florianópolis, Santa Catarina -- Brasil}}

\biblioteca{\chooselang{University Library}{Biblioteca Universitária}}
\orientador{\chooselang{Prof. PhD. Ricardo Azambuja Silveira}{Prof. Dr. Ricardo Azambuja Silveira}}
\coorientador{\chooselang{M.S. Thiago Ângelo Gelaim}{M.S. Thiago Ângelo Gelaim}}

\instituicaosigla{UFSC}
\instituicao{\chooselang{Federal University of \brazil{Santa Catarina}}{Universidade Federal de Santa Catarina}}
\tipotrabalho{\chooselang{Bachelor's Thesis}{Tese de Graduação}}

\area{\chooselang{Formal Languages}{Linguagens Formais}}
\formacao{\chooselang{Bachelor of Science Degree in Computer Science}{Grau de Bacharel em Ciência da Computação}}
\programa{\chooselang{Undergraduate Program in Computer Science}{Trabalho de Conclusão de Curso}}
\centro{\chooselang{Department of Informatics and Statistics}{Departamento de Informática e Estatística}}

% O preambulo deve conter tipo do trabalho, objetivo, nome da instituição e a área de concentração.
\preambulo
{%
    \chooselang
    {Thesis submitted to the~\imprimirprograma~of the~\imprimirinstituicao~to obtain the~\imprimirformacao.}
    {Tese submetida ao \imprimirprograma da \imprimirinstituicao para a obtenção do Título de \imprimirformacao.}
}

% Keywords
\newcommand{\palavraschaveingles}
{%
    \item source. \item code. \item formatter. \item beautifier. \item prettyprint. \item universal.
    \item reuse. \item blocks. \item object. \item oriented. \item programming. \item structured.
    \item parsing. \item parse. \item regular. \item expression. \item regex. \item C. \item C++.
    \item grammar. \item Turing. \item machine. \item automata. \item lexer. \item syntax. \item
    Sublime. \item Java. \item Rust. \item Shell. \item script. \item obfuscators. \item learning.
    \item syntec. \item teamicide. \item concensus. \item indentação. \item settings.
}
\newcommand{\palavraschaveportugues}
{%
    \item fonte. \item código. \item formatador. \item embelezante. \item prettyprint. \item
    universal. \item reuso. \item blocos. \item objeto. \item orientado. \item programação. \item
    estruturada. \item análise. \item analisador. \item regular. \item expressão. \item regex. \item
    C. \item C++. \item gramática. \item Turing. \item máquina. \item autômatos. \item lexer. \item
    sintaxe. \item Sublime. \item Java. \item Rust. \item Shell. \item roteiro. \item ofuscadores.
    \item aprendizado. \item Syntec. \item teamicide. \item consenso. \item indentation. \item
    configurações.
}

% Remove the colon appended to theses variables, allowing us to use other separators
\addto\captionsbrazil
{
    \renewcommand{\orientadorname}{Orientador}
    \renewcommand{\coorientadorname}{Coorientador}
}

% Create caption English translations as the sections headers
% https://tex.stackexchange.com/questions/8564/what-is-the-right-way-to-redefine-macros-defined-by-babel
\addto\captionsenglish
{
    %% adjusts names from abnTeX2
    \renewcommand{\folhaderostoname}{Title page}
    \renewcommand{\epigraphname}{Epigraph}
    \renewcommand{\dedicatorianame}{Dedication}
    \renewcommand{\errataname}{Errata sheet}
    \renewcommand{\agradecimentosname}{Acknowledgements}
    \renewcommand{\anexoname}{ANNEX}
    \renewcommand{\anexosname}{Annex}
    \renewcommand{\apendicename}{APPENDIX}
    \renewcommand{\apendicesname}{Appendix}
    \renewcommand{\orientadorname}{Supervisor}
    \renewcommand{\coorientadorname}{Co\hyp{}supervisor}
    \renewcommand{\folhadeaprovacaoname}{Approval}
    \renewcommand{\resumoname}{Abstract}
    \renewcommand{\listadesiglasname}{List of abbreviations and acronyms}
    \renewcommand{\listadesimbolosname}{List of symbols}
    \renewcommand{\fontename}{Source}
    \renewcommand{\notaname}{Note}
    %% adjusts names used by \autoref
    \renewcommand{\pageautorefname}{page}
    \renewcommand{\sectionautorefname}{section}
    \renewcommand{\subsectionautorefname}{subsection}
    \renewcommand{\subsubsectionautorefname}{subsubsection}
    \renewcommand{\paragraphautorefname}{subsubsubsection}
}

% Source Code Settings in Document
\makeatletter
\@ifpackageloaded{listings}
{
\ifenglish
    % These default values are already in English
\else
    % Listing -> Codigo fonte
    \renewcommand{\lstlistingname}{Código--fonte}

    % List of Listings -> Lista de códigos-fonte
    \renewcommand{\lstlistlistingname}{Lista de códigos--fonte}

    % Calculate the size of the header
    \calculatelisteningsheader
\fi
}{}
\makeatother


% Backref package settings, pages with citations in bibliography
\makeatletter
\@ifpackageloaded{backref}
{
\ifenglish
    % Used without the backref hyperpageref option
    \renewcommand{\backrefpagesname}{Cited on page(s):~}

    % Default text before page number
    \renewcommand{\backref}{}

    % Sets the text of the citation
    \renewcommand*{\backrefalt}[4]
    {
        \ifcase #1
            No citation in the text.
        \or
            Cited on page #2.
        \else
            Cited #1 times on pages #2.
        \fi
    }
\else
    % Usado sem a opção hyperpageref de backref
    \renewcommand{\backrefpagesname}{Citado na(s) página(s):~}

    % Texto padrão antes do número das páginas
    \renewcommand{\backref}{}

    % Define os textos da citação
    \renewcommand*{\backrefalt}[4]
    {
        \ifcase #1
            Nenhuma citação no texto.
        \or
            Citado na página #2.
        \else
            Citado #1 vezes nas páginas #2.
        \fi
    }
\fi
}{}
\makeatother


% Espaçamentos entre linhas e parágrafos
%
% ifpackageloaded question
% https://tex.stackexchange.com/questions/70212/ifpackageloaded-question
\makeatletter
\@ifclassloaded{memoir}
{
    % Estilo de capítulos, ver classe para maiores detalhes.Veja outros estilos em:
    % http://mirrors.ibiblio.org/CTAN/macros/latex/contrib/memoir/memman.pdf
    \chapterstyle{VZ14}
    \setlength\beforechapskip{0pt}
    \setlength\midchapskip{15pt}
    \setlength\afterchapskip{15pt}

    % O tamanho do parágrafo é dado por:
    \setlength{\parindent}{1.3cm}

    % Controle do espaçamento entre um parágrafo e outro. Tente também
    % \onelineskip
    \setlength{\parskip}{0.2cm}

    % Memoir: Warnings “The material used in the headers is too large” w/ accented titles
    % https://tex.stackexchange.com/questions/387293/how-to-change-the-page-layout-with-memoir
    \setheadfoot{30.0pt}{\footskip}
    \checkandfixthelayout
}{}
\makeatother


% Color settings across the document
\makeatletter
\@ifpackageloaded{xcolor}
{
    % RGB colors in absolute values from 0 to 255 by using `RGB` tag
    \definecolor{darkblue}{RGB}{26,13,178}

    % Definição de cores, RGB colors in percentage notation by using `rgb` tag
    \definecolor{mygreen}{rgb}{0,0.6,0}
    \definecolor{mygray}{rgb}{0.5,0.5,0.5}
    \definecolor{mymauve}{rgb}{0.58,0,0.82}

    % Configurações de aparência do PDF final
    \definecolor{figcolor}{rgb}{1,0.4,0}  % orange
    \definecolor{tabcolor}{rgb}{1,0.4,0}  % orange
    \definecolor{eqncolor}{rgb}{1,0.4,0}  % orange
    \definecolor{linkcolor}{rgb}{1,0.4,0} % orange
    \definecolor{citecolor}{rgb}{1,0.4,0} % orange
    \definecolor{seccolor}{rgb}{0,0,1}    % blue
    \definecolor{abscolor}{rgb}{0,0,1}    % blue
    \definecolor{titlecolor}{rgb}{0,0,1}  % blue
    \definecolor{biocolor}{rgb}{0,0,1}    % blue

    % Alterando o aspecto da cor azul
    \definecolor{blue}{RGB}{41,5,195}

    % Informações do PDF
    \@ifpackageloaded{hyperref}
    {
        \hypersetup
        {
            pdftitle={\@title},
            colorlinks=true, % false: boxed links; true: colored links
            linkcolor=darkblue, % color of internal links
            citecolor=darkgreen, % color of links to bibliography
            filecolor=black, % color of file links
            urlcolor=linkcolor,
            bookmarksdepth=4
        }
        \ifenglish
            \hypersetup
            {
                pdfauthor={Author},
                pdfsubject={Thesis' Abstract},
                pdfcreator={LaTeX with abnTeX2 for UFSC},
                pdfkeywords={abnt}{latex}{UFSC}{abntex2}{thesis},
            }
        \else
            \hypersetup
            {
                pdfauthor={Autores},
                pdfsubject={Resumo da tese},
                pdfcreator={LaTeX com abnTeX2 para UFSC},
                pdfkeywords={abnt}{latex}{UFSC}{abntex2}{tese},
            }
        \fi
    }
}{}
\makeatother


% Fontes das entradas do sumario
\makeatletter
\renewcommand*{\l@chapter}[2]
{%
    \l@chapapp{\uppercase{#1}}{#2}{\cftchaptername}
}
\renewcommand*{\l@section}[2]
{%
    \l@chapapp{\ABNTEXsectionfont\uppercase{#1}}{#2}{\cftsectionname}
}
\makeatother

% Changing the font of the numbers in the ToC in the memoir class
% https://tex.stackexchange.com/questions/14314/changing-the-font-of-the-numbers-in-the-toc-in-the-memoir-class
\renewcommand{\cftpartfont}{\ABNTEXpartfont\color{darkblue}}
\renewcommand{\cftpartpagefont}{\ABNTEXpartfont\color{black}}

\renewcommand{\cftchapterfont}{\ABNTEXchapterfont\color{darkblue}}
\renewcommand{\cftchapterpagefont}{\ABNTEXchapterfont\color{black}}

\renewcommand{\cftsectionfont}{\ABNTEXsectionfont\color{darkblue}}
\renewcommand{\cftsectionpagefont}{\ABNTEXsectionfont\color{black}}

\renewcommand{\cftsubsectionfont}{\ABNTEXsubsectionfont\color{darkblue}}
\renewcommand{\cftsubsectionpagefont}{\ABNTEXsubsectionfont\color{black}}

\renewcommand{\cftsubsubsectionfont}{\ABNTEXsubsubsectionfont\color{darkblue}}
\renewcommand{\cftsubsubsectionpagefont}{\ABNTEXsubsubsectionfont\color{black}}

\renewcommand{\cftparagraphfont}{\ABNTEXsubsubsubsectionfont\color{darkblue}}
\renewcommand{\cftparagraphpagefont}{\ABNTEXsubsubsubsectionfont\color{black}}




% When writing a large document, it is sometimes useful to work on selected sections of the document
% to speed up compilation time: https://en.wikibooks.org/wiki/TeX/includeonly
\newif\ifforcedinclude\forcedincludefalse

% \addtoincludeonly{beforetext/agradecimentos}
% \addtoincludeonly{beforetext/epigrafe}
% \addtoincludeonly{beforetext/fichacatalografica}
% \addtoincludeonly{beforetext/folhadeaprovacao}
% \addtoincludeonly{beforetext/resumos}
% \addtoincludeonly{beforetext/siglas}
% \addtoincludeonly{beforetext/simbolos}

% Part 1
% \addtoincludeonly{chapters/introduction}
% \addtoincludeonly{chapters/motivation}
% \addtoincludeonly{chapters/beautifiers}

% Part 2
\addtoincludeonly{chapters/object_beautifier}
% \addtoincludeonly{chapters/conclusion}
% \addtoincludeonly{aftertext/aftertext}

% Control whether the full document will be generated
% Note: It will also generate severals errors like the following, which can be ignored
%       Latexmk: Missing input file: 'chapters/test.aux'
%
% You can make latex stop generate these errors, if you generate a full version
% of the document, before uncommenting these lines.
%
% Uncomment these two lines, to only partially generate the document
% \doincludeonly
% \forcedincludetrue


% https://tex.stackexchange.com/questions/85113/xrightarrow-text
\makeatletter
\newcommand{\xRightarrow}[2][]{\ext@arrow 0359\Rightarrowfill@{#1}{#2}}
\newcommand{\xLeftarrow}[2][]{\ext@arrow 0359\Leftarrowfill@{#1}{#2}}
\makeatother

% https://tex.stackexchange.com/questions/32208/footnote-runs-onto-second-page
\interfootnotelinepenalty=10000

% Disable the empty pages automatically put by memoir class, except the ones by \cleardoublepage
\ifforcedinclude\openany\else\fi

% https://tex.stackexchange.com/questions/171999/overfull-hbox-in-biblatex
% https://tex.stackexchange.com/questions/499457/why-my-document-is-not-hyphenation-on-words-starting-with-upper-case-letter-i
\emergencystretch=5em

% https://tex.stackexchange.com/questions/23313/how-can-i-reduce-padding-after-figure
% https://tex.stackexchange.com/questions/499580/how-to-keep-my-default-floating-environment-spacing-before-them-while-reducing
% \xpretocmd{\figure}{\setlength{\belowcaptionskip}{-10pt}}{}{}


\begin{document}
    % FIXME: Comment this after finishing the thesis, so you can start fixing the \flushbottom vs \raggedbottom
    % https://tex.stackexchange.com/questions/65355/flushbottom-vs-raggedbottom
    \raggedbottom

    % https://tex.stackexchange.com/questions/4705/double-space-between-sentences
    \frenchspacing

    % Uncomment this to put a ←← | ← (Go To Top/Go Back) on each section header
    \advisor{}{\addGoToSummary}

    % ELEMENTOS PRÉ-TEXTUAIS
    

% ELEMENTOS PRÉ-TEXTUAIS
\ifforcedinclude\else
    % Fix the \textpreliminarycontents not showing up when @twoside is disabled
    \newboolean{ufscThesisXisMemoirTwoSidesEnabled}
    \if@twoside
        \setboolean{ufscThesisXisMemoirTwoSidesEnabled}{true}
    \else
        \setboolean{ufscThesisXisMemoirTwoSidesEnabled}{false}
    \fi
    \setboolean{@twoside}{true}

    % pretextual settings
    % https://tex.stackexchange.com/questions/386446/how-to-fix-destination-with-the-same-identifier-namepage-a-has-been-already
    % https://tex.stackexchange.com/questions/67989/pdftex-warning-has-been-referenced-but-does-not-exist-replaced-by-a-fixed-one
    \hypersetup{pageanchor=false}
    \PRIVATEbookmarkthis{Capa}
    \addtotextpreliminarycontent{Capa}
    \pretextual

    % Capa
    % \includepdf{pictures/FrenteCapaUFSC.pdf}
    % https://tex.stackexchange.com/questions/227711/blank-page-after-titlingpage
    \advisor{}{\AtBeginShipoutNext{\AtBeginShipoutNext{\AtBeginShipoutDiscard}}}
    \imprimircapa

    % https://tex.stackexchange.com/questions/386446/how-to-fix-destination-with-the-same-identifier-namepage-a-has-been-already
    % https://tex.stackexchange.com/questions/67989/pdftex-warning-has-been-referenced-but-does-not-exist-replaced-by-a-fixed-one
    \hypersetup{pageanchor=true}

    % Custom list throw LaTeX Error: Command \mycustomfiction already defined?
    % https://tex.stackexchange.com/questions/388489/custom-list-throw-latex-error-command-mycustomfiction-already-defined/
    \advisor{}{%
        % Manually add the `\textpreliminarycontents` to the Table of Contents here
        % to keep the hyper link pointing to the beginning of the page, instead of
        % the beginning of `\textpreliminarycontents`
        % https://tex.stackexchange.com/questions/44088/when-do-i-need-to-invoke-phantomsection
        \phantomsection\addcontentsline{toc}{chapter}{\mytextpreliminarylistname}

        % https://tex.stackexchange.com/questions/234398/list-of-figures-and-tables-when-there-are-no-figures-or-tables
        \whenlistisnotempty{\mytextpreliminarylistname}{%
            \begin{KeepFromToc}
                \textpreliminarycontents
            \end{KeepFromToc}
        }

        \clearpage
    }

    % Fix the \textpreliminarycontents not showing up when @twoside is disabled
    \ifthenelse{\boolean{ufscThesisXisMemoirTwoSidesEnabled}}
    {\setboolean{@twoside}{true}}
    {\setboolean{@twoside}{false}}

    % Folha de rosto (o * indica que haverá a ficha bibliográfica)
    % https://tex.stackexchange.com/questions/74439/table-of-contents-incorrect-page-numbering
    \addtotextpreliminarycontent{\folhaderostoname}
    \imprimirfolhaderosto*{}

    % Inserir a ficha bibliografica
    %
    % Isto é um exemplo de Ficha Catalográfica, ou ``Dados internacionais de
    % catalogação-na-publicação''. Você pode utilizar este modelo como referência.
    % Porém, provavelmente a biblioteca da sua universidade lhe fornecerá um PDF
    % com a ficha catalográfica definitiva após a defesa do trabalho. Quando estiver
    % com o documento, salve-o como PDF no diretório do seu projeto e substitua todo
    % o conteúdo de implementação deste arquivo pelo comando abaixo:
    \PRIVATEbookmarkthis{Ficha Catalográfica}
    \addtotextpreliminarycontent{Ficha Catalográfica}

    

\ifenglish

Legal Notes

There is no warranty for any part of the documented software. The authors have taken care in the
preparation of this thesis, but make no expressed or implied warranty of any kind and assume no
responsibility for errors or omissions. No liability is assumed for incidental or consequential
damages in connection with or arising out of the use of the information or programs contained here.
\cite{koma-scrguien}

\else

Notas legais

Não há garantia para qualquer parte do software documentado. Os autores tomaram cuidado na
preparação desta tese, mas não fazem nenhuma garantia expressa ou implícita de qualquer tipo e não
assumem qualquer responsabilidade por erros ou omissões. Não se assume qualquer responsabilidade por
danos incidentais ou consequentes em conexão ou decorrentes do uso das informações ou programas aqui
contidos. \cite{koma-scrguien}

\fi


% http://portalbu.ufsc.br/ficha
% http://portal.bu.ufsc.br/servicos/ficha-de-identificacao-da-obra/
\begin{fichacatalografica}
    \vspace*{\fill}

    \begin{center}

        \chooselang
        {Cataloging at source by the University Library of the Federal University of Santa Catarina.}
        {Catalogação na fonte pela Biblioteca Universitária da Universidade Federal de Santa Catarina.}

        \chooselang
        {File compiled at \currenttime h of the day \today.}
        {Arquivo compilado às \currenttime h do dia \today.}

        \framebox[\textwidth]
        {
            \begin{minipage}{0.98\textwidth}

                \ttfamily
                \imprimirautor

                \hspace{0.5cm} \imprimirtitulo~:~\imprimirsubtitulo~/~\imprimirautor;
                \imprimirorientadorRotulo,~\imprimirorientador;~\imprimircoorientadorRotulo,~\imprimircoorientador
                ~--~\imprimirlocal,~\currenttime,~\imprimirdata.

                % Prints how much pages there are on the document and links to the last page
                \hspace{0.5cm} \pageref{LastPage} p.
                \bigskip

                \hspace{0.5cm} \imprimirtipotrabalho~--~\imprimirinstituicao,
                \imprimircentro,~\imprimirprograma.
                \bigskip

                \hspace{0.5cm} \chooselang{Includes references}{Inclui referências}
                \bigskip

                \hspace{0.5cm}
                \begin{inparaenum}
                    \chooselang{\palavraschaveingles}{\palavraschaveportugues}
                \end{inparaenum}
                I. \imprimirorientador~
                II. \imprimircoorientador~
                III. \imprimirprograma~
                IV. \imprimirtitulo~
                \bigskip

                \hspace{7.75cm} CDU 02:141:005.7

            \end{minipage}
        }

    \end{center}

\end{fichacatalografica}


    % https://tex.stackexchange.com/questions/91440/how-to-include-multiple-pages-in-latex
    % \includepdf{pictures/Ficha_Catalografica.pdf}
    \ifforcedinclude\else\cleardoublepage\fi
\fi


% Inserir errata

% Inserir folha de aprovação. Isto é um exemplo de Folha de aprovação, elemento obrigatório da
% NBR 14724/2011 (seção 4.2.1.3). Você pode utilizar este modelo até a aprovação do trabalho.
% Após isso, substitua todo o conteúdo deste arquivo por uma imagem da página assinada pela
% banca com o comando abaixo:
\ifforcedinclude\else\cleardoublepage\fi


\addtotextpreliminarycontent{\lang{Approval Sheet}{Folha de Aprovação}}

\begin{folhadeaprovacao}

    \begin{center}
        {\imprimirautor}

        \begin{center}
            \ABNTEXchapterfont\bfseries\MakeUppercase{\imprimirtitulo}\ifnotempty{\imprimirsubtitulo}{: \imprimirsubtitulo}
        \end{center}

        \begin{minipage}{\textwidth}
            \lang
            {
                This \imprimirtipotrabalho~ was considered appropriate to get the \imprimirformacao,
                \ifnotempty{\imprimirarea}{in the area of \imprimirarea,}
                and it was approved by the \imprimirprograma~ of \imprimircentro~ of \imprimirinstituicao.
            }
            {
                Este(a) \imprimirtipotrabalho~ foi julgado adequado(a) para obtenção do Título de \imprimirformacao,
                \ifnotempty{\imprimirarea}{na área de concentração \imprimirarea,}
                e foi aprovado em sua forma final pelo \imprimirprograma~
                do \imprimircentro~ da \imprimirinstituicao.
            }
         \end{minipage}%
    \end{center}

    \begin{center}
        \imprimirlocal, \imprimirdata.
    \end{center}

    \assinatura{%
        \textbf{\imprimircoordenador} \\
        \imprimircoordenadorRotulo~\lang{of}{do} \imprimirprograma
    }

    % \newpage
    \begin{flushleft}
        \textbf{\lang{Examination Board}{Banca Examinadora}:}
    \end{flushleft}

    \assinatura{%
        \textbf{\imprimirorientador} \\ \imprimirorientadorRotulo\\
        \imprimirinstituicao~--~\imprimirinstituicaosigla
    }

    \ifnotempty{\imprimircoorientador}{%
        \assinatura{%
            \textbf{\imprimircoorientador} \\ \imprimircoorientadorRotulo \\
            \imprimirinstituicao~--~\imprimirinstituicaosigla
        }
    }

    \assinatura{%
        \textbf{Prof. Convidado 1, \lang{PhD.}{Dr.}} \\
        Instituição 1 -- Sigla 1
    }

    \assinatura{%
        \textbf{Prof. Convidado 2, \lang{PhD.}{Dr.}} \\
        Instituição 2 -- Sigla 2
    }

\end{folhadeaprovacao}


% \includepdf{pictures/folhadeaprovacao_final.pdf}


% % Dedicatória
% \ifforcedinclude\else\cleardoublepage\fi
% \ifforcedinclude\else

\addtotextpreliminarycontent{\chooselang{Dedication}{Dedicatória}}

\begin{dedicatoria}

    \vspace*{\fill}
    \centering
    \noindent
    \textit{\chooselang
    {
        This work is dedicated to adult children who, \\
        When small, dreamed of becoming scientists.
    }
    {
        Este trabalho é dedicado às crianças adultas que,\\
        quando pequenas, sonharam em se tornar cientistas.
    }}
    \vspace*{\fill}

\end{dedicatoria}


\fi

% % Agradecimentos
% \ifforcedinclude\else\cleardoublepage\fi
% \include{beforetext/agradecimentos}

% % Epígrafe
% \ifforcedinclude\else\cleardoublepage\fi
% 

\addtotextpreliminarycontent{\lang{Epigraph}{Epigrafe}}

\begin{epigrafe}

\vspace*{\fill}\lang
{
    \begin{flushright}
        \textit{``Learn from yesterday, live for today, hope for tomorrow. The important thing is not to stop questioning.''} \\ Albert Einstein
    \end{flushright}
    \begin{flushright}
        \textit{``The true sign of intelligence is not knowledge but imagination.''} \\  Albert Einstein
    \end{flushright}
    \begin{flushright}
        \textit{``Peace cannot be kept by force; it can only be achieved by understanding.''} \\ Albert Einstein
    \end{flushright}
    \begin{flushright}
        \textit{``Whoever is careless with the truth in small matters cannot be trusted with important matters.''} \\ Albert Einstein
    \end{flushright}
    \begin{flushright}
        \textit{``Extraordinary claims require extraordinary evidence''} \\ Carl Sagan
    \end{flushright}
    \begin{flushright}
        \textit{``Catholic, which I was until I reached the age of reason.''} \\ George Carlin
    \end{flushright}
    \begin{flushright}
        \textit{``We made too many wrong mistakes.''} \\ Yogi Berra
    \end{flushright}
}
{
    \begin{flushright}
        \textit{``Assim como aquele pecado da juventude, este documento te perseguirá pelo resto da vida. \showfont''} \\ Enio Valmor Kassick
    \end{flushright}
    \begin{flushright}
        \textit{``Estupidez trará mais autoconfiança do que o conhecimento e a bravura juntas.''} \\ Adriano Ruseler
    \end{flushright}
}

\end{epigrafe}





% Ajusta o espaçamento dos parágrafos do resumo
\setlength{\absparsep}{18pt}

% RESUMOS
\ifforcedinclude\else\cleardoublepage\fi


\newcommand{\imprimirbrazilabstract}{%
    \cleardoublepage\phantomsection
    \addtotextpreliminarycontent{Resumo em Português}
    \begin{otherlanguage*}{brazil}
    \begin{resumo}[Resumo]

        Faz~=se um estudo sobre o que é e
        para que servem os Formatadores de Código,
        assim como as abordagens utilizadas nas mais variadas Ferramentas de Formatação de Código.
        Os softwares Formatadores de Código~=Fonte atuais,
        também conhecidos como \textit{Source Code Beautifiers},
        são limitados a um conjunto similar,
        ou mesmo à uma única linguagem de programação,
        além de muitos serem limitados no que eles podem fazer ao formatar o código~=fonte.
        Nesse contexto,
        propõe~=se uma ferramenta que permita por meio de gramáticas,
        a especificação de quais linguagens de programação deseja~=se realizar a formatação.
        Assim,
        centralizando em um único programa a formatação de código das mais diversas linguagens de programação pela especificação de suas gramáticas.

        \imprimirpalavraschave{Palavras~=chaves}{\begin{inparaitem}[]\palavraschaveportugues\end{inparaitem}}

    \end{resumo}
    \end{otherlanguage*}
}


\newcommand{\imprimirenglishabstract}{%
    % https://tex.stackexchange.com/questions/20987/changing-babel-package-inside-a-single-chapter
    % https://tex.stackexchange.com/questions/36526/multiple-language-document-babel-selectlanguage-vs-begin-endotherlanguage
    \cleardoublepage\phantomsection
    \addtotextpreliminarycontent{English's Abstract}
    \begin{otherlanguage*}{english}
    \begin{resumo}[Abstract]

        A study about nowadays used Source Code Formatters,
        and as well,
        the programming algorithms used in most Source Code Formatting Tools.
        Cutting edge Source Code Formatting Softwares,
        also known as Source Code Beautifiers,
        are limited to a common set,
        or even to a single programming language,
        and many formatters are limited in what they can do.
        In this context,
        it is also proposed a new Source Code Formatting Tool allowing users to input their favourite languages grammars.
        Therefore,
        formatting all programming languages they would like a single formatting software,
        by the input of their languages grammars specification.

        \imprimirpalavraschave{Keywords}{\begin{inparaitem}[]\palavraschaveingles\end{inparaitem}}

    \end{resumo}
    \end{otherlanguage*}
}


% \newcommand{\imprimirfrenchabstract}{%
%     \addtotextpreliminarycontent{Français Résumé}
%     \begin{resumo}[Résumé]
%       \begin{otherlanguage*}{french}
%           Il s'agit d'un résumé en français.

%           \imprimirpalavraschave{Mots-clés}{latex. abntex. publication de textes.}
%       \end{otherlanguage*}
%     \end{resumo}
% }


% \newcommand{\imprimirspanishabstract}{%
%     \addtotextpreliminarycontent{Español Resumen}
%     \begin{resumo}[Resumen]
%       \begin{otherlanguage*}{spanish}
%           Este es el resumen en español.

%           \imprimirpalavraschave{Palabras clave}{latex. abntex. publicación de textos.}
%       \end{otherlanguage*}
%     \end{resumo}
% }


\makeatletter
\ifenglish
    \@ifundefined{imprimirbrazilabstract}{}{\imprimirbrazilabstract}

    % https://tex.stackexchange.com/questions/331108/times-new-roman-in-latex-just-some-text
    % https://tex.stackexchange.com/questions/11707/how-to-force-output-to-a-left-or-right-page
    % https://tex.stackexchange.com/questions/132966/do-not-display-chapter-title-in-memoir-class
    \cleardoublepage\phantomsection
    \pretextualchapter{Resumo Expandido}
    \addtotextpreliminarycontent{Resumo Expandido}

    \begin{otherlanguage*}{brazil}
        \setlength{\parskip}{0.2cm}
        \setlength{\parindent}{0.0cm}
        \fontfamily{ptm}\selectfont

        \section*{Introdução}
        O resumo expandido é previsto na Resolução Normativa nº 95/CUn/2017, Art. 55, § 2, de 4 de
        abril de 2017, e exigido para teses e dissertações escritas em idiomas estrangeiros (com
        exceção dos cursos pertinentes ao estudo de idiomas estrangeiros – Programa de Pós-Graduação
        em Estudos da Tradução e Programa de Pós-Graduação em Inglês: Estudos Linguísticos e
        Literários).

        O resumo expandido é considerado um elemento pré-textual e deverá ser incluído no trabalho
        após o resumo e antes do abstract. Deverá iniciar em página impar (no anverso de uma folha)
        continuando no verso da folha. O texto deverá seguir o formato A5, com margens espelhadas:
        superior 2,0 cm, inferior 1,5 cm, interna 2,5 cm e externa 1,5. Deve ser empregada a fonte
        Time New Roman.  Todo o texto deve ser digitado em tamanho 10,5. O espaçamento entre as
        linhas deverá ser simples. A expressão “resumo expandido” deve seguir a mesma tipografia das
        demais sessões primárias do trabalho.

        O texto do resumo expandido deve ser redigido em português e conter as seguintes seções (ver
        modelo): Introdução, Objetivos, Metodologia, Resultados e Discussão e Considerações Finais.
        Deve apresentar no mínimo duas (02) e, no máximo, cinco (05) páginas contendo a mesma
        formatação em A5 do resumo e do abstract, bem como palavras-chave. \englishword{\showfont}

        \section*{Objetivos}
        Lorem ipsum dolor sit amet, consectetur adipiscing elit. Phasellus vitae dolor lacus. Ut
        accumsan vitae felis nec porttitor. Integer interdum fringilla feugiat. Nullam pulvinar sit
        amet tellus eget maximus. Donec sit amet magna eget justo semper fermentum vel eget velit.
        In iaculis imperdiet mauris, ac ornare libero placerat non. Nulla libero lectus, ullamcorper
        ac ornare eget, pulvinar ac nulla. Curabitur vestibulum non nisl eget sagittis. Proin
        gravida lacus id eros bibendum interdum. Mauris ullamcorper elementum tortor sed consequat.
        Integer tempus, est a lobortis vehicula, nisi mi fringilla augue, non semper leo metus in
        quam. Etiam in leo maximus, pulvinar mi eget, vehicula risus. Donec sed dui semper, dictum
        eros at, suscipit felis.

        Nam sagittis vel orci at tempus. Nulla non pellentesque eros.
        Quisque cursus leo massa, eu ultricies nisl lacinia a. Nulla sit amet elementum ligula.
        Proin sodales venenatis dictum. Ut et est cursus, vulputate velit et, viverra odio. Interdum
        et malesuada fames ac ante ipsum primis in faucibus. Maecenas purus diam, tempor a semper
        et, finibus a ex. Cras sagittis felis urna, et consequat arcu lacinia ut. Praesent blandit
        venenatis ante nec porta. Duis rutrum, tellus vitae ullamcorper auctor, lectus ex laoreet
        est, ac tristique ipsum arcu vitae nibh. Nam efficitur felis ut mi consectetur, nec auctor
        odio ornare. In tempor vulputate urna, vitae cursus enim egestas eu. Proin diam augue,
        dignissim vitae ligula eget, lobortis ornare odio. Duis quis elit augue. Fusce quis rhoncus
        tortor. Donec hendrerit at massa a mattis. Sed ipsum neque, aliquam ut sem sed, ultrices
        varius ligula. Suspendisse blandit, dolor ac rhoncus lacinia, dolor purus cursus purus, et
        accumsan orci neque a leo.

        \section*{Metodologia}
        Quisque efficitur dolor in lectus dapibus elementum. Nam ultrices blandit consectetur.
        Nullam ultricies sit amet odio quis placerat. Aenean eget est elit. Maecenas et nulla dolor.
        Orci varius natoque penatibus et magnis dis parturient montes, nascetur ridiculus mus. In
        pulvinar velit sed mi sagittis ornare. Aenean rutrum suscipit egestas. Phasellus pharetra
        eget ex in volutpat. Quisque eu arcu nunc. Vivamus arcu ligula, pharetra at rhoncus sit
        amet, pulvinar sed eros. Sed porta ipsum ipsum, et fermentum magna volutpat sed. Vivamus
        pharetra facilisis orci, sit amet luctus nisl pretium id. Sed consequat, arcu et congue
        pulvinar, risus enim aliquet purus, eget venenatis libero leo sit amet metus. Maecenas vitae
        elit sapien. Fusce mollis libero et gravida placerat. Proin ut quam quis justo aliquam
        dictum. Donec volutpat convallis suscipit. Vivamus metus nisl, placerat ac enim vitae,
        tempus ultricies odio.

        Aliquam ac vehicula arcu, non bibendum nulla. Morbi libero sem,
        imperdiet vel quam et, posuere tempus nunc. Maecenas dictum magna sit amet ligula facilisis
        commodo. Aliquam tellus diam, ornare vel elementum in, dignissim id purus. Ut at tortor non
        sem molestie euismod non at turpis. Phasellus vitae bibendum tellus. Suspendisse odio enim,
        faucibus eget congue quis, semper sit amet tortor. Sed ac lectus est. Pellentesque nec
        mattis mi, et varius dolor. Aliquam quis massa ac tellus malesuada sollicitudin. Maecenas
        ultrices risus massa, nec auctor risus sagittis id. Praesent a sapien nulla. Donec
        tincidunt, metus quis hendrerit facilisis, enim augue convallis elit, sed consequat lacus
        odio vitae magna.

        \section*{Resultados e Discussão}
        Nullam sed cursus leo. Donec commodo volutpat hendrerit. Fusce et tempus lectus, feugiat
        consequat est. Class aptent taciti sociosqu ad litora torquent per conubia nostra, per
        inceptos himenaeos. Nam quis cursus mauris, non tempus orci. Phasellus lobortis et mauris at
        vulputate. Sed nec nisl elementum lorem commodo gravida non a enim. Phasellus neque erat,
        aliquet ac ligula ac, maximus vestibulum sem. Vestibulum vel tincidunt turpis. Donec lacinia
        rutrum dolor dapibus bibendum. Mauris pharetra nibh nec tincidunt iaculis. Vivamus pharetra
        bibendum nisl eget blandit. In lobortis diam non justo eleifend, id lobortis ante fringilla.
        Donec libero tortor, suscipit vestibulum vestibulum id, rutrum accumsan turpis. Phasellus
        sollicitudin luctus tincidunt. Suspendisse potenti. Nam semper metus et mi pharetra, in
        pretium ligula fermentum. Integer consectetur, orci non placerat feugiat, dui ex gravida
        augue, vel placerat ligula augue vel velit. Aliquam sollicitudin pellentesque congue. Donec
        vitae turpis in ante posuere posuere. Pellentesque eu justo leo. Donec quis elit vitae leo
        varius luctus quis eget justo.

        Vestibulum elementum ex neque, quis commodo tortor porttitor
        mattis. Mauris vel sagittis turpis. Aenean ligula turpis, eleifend at felis sed, cursus
        condimentum orci. Fusce accumsan est odio, eu venenatis massa sodales in. Curabitur a tempor
        nisl. Quisque consequat sed arcu a congue. In viverra, ex ut hendrerit condimentum, urna sem
        euismod eros, nec suscipit turpis dolor eget augue. Aenean posuere tellus et consectetur
        condimentum. Mauris et massa et nulla fringilla interdum. Duis quis posuere elit. Donec at
        ex non arcu faucibus rutrum et vel lectus. Vivamus pellentesque vestibulum rutrum. Sed
        pretium, purus sed efficitur feugiat, nisi justo eleifend nibh, id suscipit nunc massa nec
        lectus. In euismod enim eu sapien dictum sodales. Fusce sit amet vulputate orci. Nulla
        rutrum mauris at purus aliquet, ac sollicitudin leo laoreet. Etiam elementum posuere
        feugiat. Maecenas sed libero non augue fermentum ultricies eget at mi. Aenean auctor
        bibendum lacus, dignissim aliquet est tempus eget. Maecenas tempus, nulla id rhoncus
        suscipit, augue leo auctor mi, eget tincidunt magna mi quis dui. Maecenas ut elit in turpis
        tincidunt ultrices. Nulla id nulla aliquet, porttitor eros quis, egestas justo. Nunc nisi
        quam, egestas a accumsan fermentum, ultricies ac elit.

        Nulla porta auctor vestibulum. Sed
        consectetur lacus molestie iaculis ullamcorper. Proin porta posuere massa a lacinia. Nunc a
        lacinia orci, non vehicula ante. Vestibulum ipsum velit, congue et neque aliquam, imperdiet
        ornare augue. Donec et congue sapien. Pellentesque consequat consectetur neque ut varius. In
        aliquam ex quis ante venenatis dapibus. Vivamus et imperdiet urna. Vestibulum quis nibh
        magna. In a congue lectus, eu sodales nunc. Suspendisse id.

        \section*{Considerações Finais}
        Lorem ipsum dolor sit amet, consectetur adipiscing elit. Phasellus vitae dolor lacus. Ut
        accumsan vitae felis nec porttitor. Integer interdum fringilla feugiat. Nullam pulvinar sit
        amet tellus eget maximus. Donec sit amet magna eget justo semper fermentum vel eget velit.
        In iaculis imperdiet mauris, ac ornare libero placerat non. Nulla libero lectus, ullamcorper
        ac ornare eget, pulvinar ac nulla. Curabitur vestibulum non nisl eget sagittis. Proin
        gravida lacus id eros bibendum interdum. Mauris ullamcorper elementum tortor sed consequat.
        Integer tempus, est a lobortis vehicula, nisi mi fringilla augue, non semper leo metus in
        quam. Etiam in leo maximus, pulvinar mi eget, vehicula risus. Donec sed dui semper, dictum
        eros at, suscipit felis.

        Nam sagittis vel orci at tempus. Nulla non pellentesque eros.
        Quisque cursus leo massa, eu ultricies nisl lacinia a. Nulla sit amet elementum ligula.
        Proin sodales venenatis dictum. Ut et est cursus, vulputate velit et, viverra odio. Interdum
        et malesuada fames ac ante ipsum primis in faucibus. Maecenas purus diam, tempor a semper
        et, finibus a ex. Cras sagittis felis urna, et consequat arcu lacinia ut. Praesent blandit
        venenatis ante nec porta. Duis rutrum, tellus vitae ullamcorper auctor, lectus ex laoreet
        est, ac tristique ipsum arcu vitae nibh. Nam efficitur felis ut mi consectetur, nec auctor
        odio ornare. In tempor vulputate urna, vitae cursus enim egestas eu. Proin diam augue,
        dignissim vitae ligula eget, lobortis ornare odio. Duis quis elit augue. Fusce quis rhoncus
        tortor. Donec hendrerit at massa a mattis. Sed ipsum neque, aliquam ut sem sed, ultrices
        varius ligula. Suspendisse blandit, dolor ac rhoncus lacinia, dolor purus cursus purus, et
        accumsan orci neque a leo.


        \imprimirpalavraschave{Palavras-chaves}{\begin{inparaitem}[]\palavraschaveportugues\end{inparaitem}}

    \end{otherlanguage*}

    \@ifundefined{imprimirenglishabstract}{}{\imprimirenglishabstract}

\else
    \@ifundefined{imprimirbrazilabstract}{}{\imprimirbrazilabstract}
    \@ifundefined{imprimirenglishabstract}{}{\imprimirenglishabstract}
\fi

\@ifundefined{imprimirfrenchabstract}{}{\imprimirfrenchabstract}
\@ifundefined{imprimirspanishabstract}{}{\imprimirspanishabstract}
\makeatother



% Some tables of contents
\ifforcedinclude\else
{
    % https://tex.stackexchange.com/questions/179506/disable-colorlinks-locally-or-just-for-the-toc
    \hypersetup{hidelinks}

    % inserir lista de figuras
    \ifforcedinclude\else\cleardoublepage\fi
    % https://tex.stackexchange.com/questions/234398/list-of-figures-and-tables-when-there-are-no-figures-or-tables
    \whenlistisnotempty{\listfigurename}{%
        \addtotextpreliminarycontent{\listfigurename}
        % https://tex.stackexchange.com/questions/121879/remove-spacing-between-per-chapter-figures-in-lof
        {\renewcommand{\addvspace}[1]{}
        \listoffigures*}
    }{\pdfbookmark[0]{\listfigurename}{lof}}

    % inserir lista de quadros
    \ifforcedinclude\else\cleardoublepage\fi
    % https://tex.stackexchange.com/questions/234398/list-of-figures-and-tables-when-there-are-no-figures-or-tables
    \whenlistisnotempty{\listofquadrosname}{%
        \addtotextpreliminarycontent{\listofquadrosname}
        % https://tex.stackexchange.com/questions/121879/remove-spacing-between-per-chapter-figures-in-lof
        {\renewcommand{\addvspace}[1]{}
        \listofquadros*}
    }{\pdfbookmark[0]{\listofquadrosname}{loq}}

    % inserir lista de tabelas
    \ifforcedinclude\else\cleardoublepage\fi
    % https://tex.stackexchange.com/questions/234398/list-of-figures-and-tables-when-there-are-no-figures-or-tables
    \whenlistisnotempty{\listtablename}{%
        \addtotextpreliminarycontent{\listtablename}
        % https://tex.stackexchange.com/questions/121879/remove-spacing-between-per-chapter-figures-in-lof
        {\renewcommand{\addvspace}[1]{}
        \listoftables*}
    }{\pdfbookmark[0]{\listtablename}{lot}}

    % inserir códigos fonte (List of Listings `lol`)
    % https://tex.stackexchange.com/questions/511519/latex-keeps-showing-minted-environment-as-figures-instead-of-listening/511579#511579
    \ifforcedinclude\else\cleardoublepage\fi
    % https://tex.stackexchange.com/questions/234398/list-of-figures-and-tables-when-there-are-no-figures-or-tables
    \whenlistisnotempty{\lstlistlistingname}{%
        \addtotextpreliminarycontent{\lstlistlistingname}
        % https://tex.stackexchange.com/questions/121879/remove-spacing-between-per-chapter-figures-in-lof
        {\renewcommand{\addvspace}[1]{}
        \lstlistoflistings*}
    }{\pdfbookmark[0]{\lstlistlistingname}{lol}}
}
\fi


% % inserir lista de abreviaturas e siglas
% \ifforcedinclude\else\cleardoublepage\fi
% 

\addtotextpreliminarycontent{\chooselang{List of Acronyms}{Lista de Siglas}}

\begin{siglas}
    \item[ABNT] Associação Brasileira de Normas Técnicas \chooselang{}{, Brazilian Association of Technical Standards}
    \item[abnTeX] ABsurdas Normas para TeX \chooselang{}{, Absurd Standards for TeX}
\end{siglas}



% % Inserir lista de símbolos
% \ifforcedinclude\else\cleardoublepage\fi
% 

\addtotextpreliminarycontent{\chooselang{List of Symbols}{Lista de Símbolos}}

% Devam aparecer na mesma ordem de ocorrência no texto.
\begin{simbolos}
    \item[$ \Gamma $] \chooselang{Greek letter Gama}{Letra grega Gama}
    \item[$ \Lambda $] \chooselang{Lambda}{Lambda}
    \item[$ \zeta $] \chooselang{Minimal Greek letter zeta}{Letra grega minúscula zeta}
    \item[$ \in $] \chooselang{Belongs}{Pertence}
\end{simbolos}


% How to remove the self-reference of the ToC from the ToC?
% https://tex.stackexchange.com/questions/10943/how-to-remove-the-self-reference-of-the-toc-from-the-toc
\ifforcedinclude\else\cleardoublepage\fi

\begin{KeepFromToc}
    % https://tex.stackexchange.com/questions/35/what-does-overfull-hbox-mean
    % https://tex.stackexchange.com/questions/59122/how-to-avoid-using-sloppy-document-wide-to-fix-overfull-hbox-problems
    % https://tex.stackexchange.com/questions/257007/adding-color-to-table-of-contents-and-section-headings
    {
        % https://tex.stackexchange.com/questions/179506/disable-colorlinks-locally-or-just-for-the-toc
        \hypersetup{hidelinks}

        % https://tex.stackexchange.com/questions/65711/underfull-vbox-badness-10000-with-memoir
        \raggedbottom

        % https://tex.stackexchange.com/questions/49887/overfull-hbox-warning-for-toc-entries-when-using-memoir-documentclass
        % \makeatletter
            % \renewcommand{\@pnumwidth}{2em}
            % \renewcommand{\@tocrmarg}{3em}
        % \makeatother

        % https://tex.stackexchange.com/questions/57544/memoir-mysterious-overfull-hbox-in-toc-when-mathptmx-is-used
        % \setlength{\cftchapternumwidth}{2.25em}

        % Add the table of contents to the brief table of contents
        % https://tex.stackexchange.com/questions/234398/list-of-figures-and-tables-when-there-are-no-figures-or-tables
        \whenlistisnotempty{\contentsname}{%
            \addtotextpreliminarycontent{\contentsname}
            \tableofcontents
        }{\pdfbookmark[0]{\contentsname}{toc}}
    }

\end{KeepFromToc}


    % \hypersetup{hidelinks}

    % ELEMENTOS TEXTUAIS
    \textual
    \setlength\beforechapskip{50pt}
    \setlength\midchapskip{20pt}
    \setlength\afterchapskip{20pt}

    % PARTE 1
    \advisor{}{\ifforcedinclude\else\part{\lang{Research}{Pesquisa}}\fi\label{primeira_parte}}

    % Introdução (exemplo de capítulo sem numeração, mas presente no Sumário)
    

% The \phantomsection command is needed to create a link to a place in the document that is not a
% figure, equation, table, section, subsection, chapter, etc.
%
% When do I need to invoke \phantomsection?
% https://tex.stackexchange.com/questions/44088/when-do-i-need-to-invoke-phantomsection
\phantomsection


% Is it possible to keep my translation together with original text?
% https://tex.stackexchange.com/questions/5076/is-it-possible-to-keep-my-translation-together-with-original-text
\chapter{\lang{Introduction}{Introdução}}

The work first part will be based on research in
articles, books, theses, dissertations, trusted authors websites,
and through new demonstrated evidences based on arguments
in the monograph evolution.
Also, present results after building a new tool
which proposes a solution for the problems presented and detailed.

In this proposal last chapter which lies on the part
called `\nameref{sec:software_implementation}', which holds the implementation of a
tool for code `Beautifying'.



\section{Context}

Questions like ``What are good programming practices?'' Or ``Why are these
practices are good?''Are not easy to answer. But each programmer learns to
write their codes in a certain way, with certain features like using 4 or 8
spaces to indent lines, always leave a blank line before each control
structure as if or for statements, and alike rules.
\cite{naturalCodingConventions}

But entering the universe of good practices, there are many things for
discoursing. Nonetheless, in this work is presented the implementation of
tool called `Object Beautifier', which specifically dedicates on how to
perform the best layout/display of programming code on the computer screen,
so that maximize and facilitate the understanding of same
\cite{automaticSynthesis}.
Therefore, allowing the programmer to disperse
more time and efforts thinking about its coding algorithms problem,
other than trying to decipher the information that is presented
to it on the screen through infinit different code layouts
\cite{usingVersionControlData}.

Within this work\s area, we need to also think long and hard about how to
share the programming code of the programmers among you. Now, the problem of
human diversity, like all big scientific questions -- how do you explain
something like that -- It can be broken down into sub-questions. It happens
many times, which is a good practice for a `Programmer A', is not the same
to another `Programmer B'. For example, imagine some code where a programmer
decided to put before each `if' statement, a blank line. It is therefore
expected that whenever we see a blank line we can potentially find a
matching `if', which can be considered a quite useful pattern matching as
empty line may call better your attention. \cite{aPrettyGoodFormatting}

But again this is something heavily dependent of what each one learning
through their life time. Imagine another programmer do not liked this rule,
and when he was writing your code involving an `if', he did not put such
blank line another programmer is expecting. So when the first programmer
start reading its the code and look for `if', he will be expecting for blank
lines before its if\s. But will lose some time searching until realize
another programmer does not put them, or perhaps he forgot to insert them.
\cite{quantifyingProgramComprehension}

These differences are due to the diversity of ways we learn programming,
i.e., to the ways we are used to doing coding, as much as the abilities and
objectives of every programmer developed. Hence, nowadays it becomes a big
problem because we increasingly need more and more programmers working
together developing several and diverse computing systems. Where the latter
is due to the fact of the complexity of computer systems growing
increasingly, therefore over requiring programmers working and sharing their
codes and ideas. \cite{howProgrammersRead}

Moreover besides only worrying about how the code is displayed on their
computer screen, we need to worry about on how it will be saved in the file
system on its `plain-text' mode. Since for code sharing, it is vital for you
to use a versioning control system\footnote{\url{http://www.codeservedcold.com/version-control-importance/}}
which enable project manager\s and
programmers themselves, take control of their code changes
\cite{redesignOfGit}. It does allow to easily perform the tracking of code
changes \cite{gettingProductive} and
allow you to better understand what each programmer is doing
every time he formalizes a change in the code through a `commit', as in
`git' systems for example. \cite{usingSourceControl}

\begin{citacao}
I'd say there are two main reasons to enforce a single code format in a project. First has
to do with version control: with everybody formatting the code identically, all changes in
the files are guaranteed to be meaningful. No more just adding or removing a space here or
there, let alone reformatting an entire file as a `side effect' of actually changing just a
line or two. \cite{Geukens}
\end{citacao}

That is because while working with a versioning system like `git', we need
to keep the code among a single style or which we may call a `good practice'
set as standard for everybody, due the fact of letting each programmer to
write as he pleases, there will be plenty of noise on the code review and we
are figuring out what actually each programmer did \cite{quitDiffCalculating}.
Hence, if every programmer re-writes the history making changes
like inserting new lines
before each if, we end up with too much noise and focus of a versioning
system is to look at only those changes that are significant to the code,
such as the creation of new functions and not the addition of new blank
lines. \cite{findingRegressionsInProjects}

Talking about the last ideia pointed out, we could also think about an
approach to creating a new version control system which focuses only on
significant changes to the code, while reviewing code changes. However, this
approach could not be ideal, as for example, it would allow programmers to
start tedious wars of unproductive code adjustments. For example, imagine
how it would be for your every day and have to go through your code
re-adding new lines before each one of your beloved if\s, just because some
night shift programmer\footnote{\url{https://blog.codinghorror.com/who-wrote-this-crap/}}
had just removed them?



\section{Research Goals}

Beforehand due the scope limitation for a Graduation Thesis,
we should only think about a basic, simple,
and yet reusable core of features.

\begin{enumerate}
    \item A Software Product with a great Object Orientation and possibilities of extension of features,
    decent research on the state of the art.

    \item Ranking all code formatting classes (beautifying) applicable.
    Including a study on what does is source beautifying,
    how to do such and why.

    \item Establish relationships between good programming practices and efficiency in programming,
    in addition to a new tool to support programmers in order to automate the long and diverse
    programming process in teams of developers with different programming `best practices'.

    \item Define, determine and classify which one are good programming practices and
    perform an in-depth study on the good practices on visual layout area,
    also known as code `Beautifying'.

    \item The definition of a flow pattern of development allowing teams of
    developers with different programming best practices,
    to work without intervene with each other up to start wars of `best good practices'.

    \item Discourse on the variety of existing tools for the support of good programming practices,
    with a comparative analysis between them,
    determining their weaknesses and strengths.
\end{enumerate}



\section{Implementation Goals}

Propose a unique tool that allowing several and distinct
programming `best practices' being implemented in several programming
languages, which can be configured and set accordingly to their wishes,
from a single software working well behaved across all programming languages.

Moreover, explain the differences for other softwares and the benefits
of a unique tool, instead of several heavily different ones.

From this point, a sketch is presented on the problem, solutions,
information as for why to want make such software, or even why do we want to
beautifying things:

\begin{enumerate}[leftmargin=*]
    \item There are many different tools, sometimes paid, and difficult to
          complete. \cite{universalCodeFormatter}

    \item Many programming languages exist, so always having Beautifier
          software for each of them is very laborious
          \cite{universalCodeFormatter}. But the approach to a Universal
          Beautifier proposed in this work, would allow easily new languages to be
          added, being completely different from previous ones, or alike. And in
          case of similarities between them, it is enough to reuse their
          configuration structures already implemented.

    \item Looking for a Beautifier for each one of them because programmers
          currently work daily with several of these languages, and they are not
          similar. So you need to configure several beautifiers to do the
          formatting. This is a problem because only a few beautifiers are more
          complete, and every time you need to make a change in the formatting
          style, you must manually propagate the same change over several
          different program configuration files, which is bad because it takes the
          user a lot of time to learn how to handle many different types of
          settings. \cite{Schweitzer}

    \item In the case of ideal Beautifier, a change in your styling is
          propagated to all languages. And if you want to leave some language out
          of it, you just need to remove it from the list on which the
          configuration block applies to, and `a)' leave it out so no change is
          applied to. Or `b)' create a new block including only the block within
          the desired settings.

\end{enumerate}

The difference from this proposal to remaining formatting tools,
is the tradeoff between end\hyp{}users and developers responsibilities.
Most tools rarely expose to end\hyp{}users their language syntax specification,
in contrast,
this proposal completely exposes the language to the end\hyp{}user as simple plain\hyp{}text,
not requiring the tool to know any language syntax neither semantics.
Moreover,
with no syntax knowledge required,
the tool be can used with any languages their user wishes to.



\section{Related Works}


\url{http://editorconfig.org/}


    

% Is it possible to keep my translation together with original text?
% https://tex.stackexchange.com/questions/5076/is-it-possible-to-keep-my-translation-together-with-original-text
\chapter{\lang{Theory base}{Fundamentação Teórica}}

Fazer depois que a fundamentação teórica estiver concluída.


\section{Compiladores e Tradutores}
\label{compiladoresEtradutores}

\lang{%
    This work aims to propose a translator \cite{generatingInterpretiveTranslators},
    where the input and output languages are the same language.
    Such translation objective is to change the language representational structure,
    but without affecting the language lex,
    syntactic or semantics, i.e.,
    the language meaning.

    This program class is commonly know as text formatters.
    The differential from this work from others is the goal of a single expandable tool,
    capable of manipulating all existent and future programming languages,
    based on the use of deterministic \cite{introductionToContextFreeGrammars}
    and controlled nondeterministic
    % \cite{TODO:section explaning what does controlled means}
    context free grammars.
}{%
    Em linguagens formais,
    tradutores são ferramentas que operam realizando a transformação de um programa de entrada,
    em um programa de saída \cite{generatingInterpretiveTranslators}.
    Diferente de um compilador,
    a linguagem de destino da ``tradução'' é do \textbf{mesmo nível} que a linguagem de origem.
    Por exemplo,
    dado um programa de entrada em C++ e
    um programa de saída em Java,
    tem~=se um processo de tradução (Figura \ref{fig:pictures/ProcessoTraducao.png}).
    \advisor{A tradução é diferente de um processo de compilação,
    que é dotado de mais etapas \cite{translatorGenerationCompilier}.
    }{Pelo outro lado,
    dado um programa de entrada em C++ e
    um programa de saída em \textit{Assembly},
    tem~=se um processo de compilação \cite{translatorGenerationCompilier}.
    }
    \begin{figure}[h]
    \centering
    \includegraphics[width=1.0\textwidth]{pictures/ProcessoTraducao.png}
    \caption[Processo de Tradução]{Processo de Tradução -- Fonte Própria,
    \citeonline{ahoCompilerDragonBook}}
    \label{fig:pictures/ProcessoTraducao.png}
    \end{figure}

    No processo de compilação ou
    tradução,
    um Analisador Léxico cria múltiplos \textit{tokens}.
    Um token é composto por diversos atributos como a posição e
    o \textit{lexema}, i.e.,
    a sequencia de caracteres que este token representa no programa de entrada.
    Uma vez que o programa é ``\textit{tokenizado}'' pelo Analisador Léxico,
    o Analisador Sintático constrói a Árvore Sintática do programa.

    Utilizando a Árvore Sintática do programa de entrada,
    o tradutor constrói uma nova Árvore Sintática correspondente a Árvore Sintática da linguagem do programa destino,
    utilizada para construir o código~=fonte do programa destino.
    Em um processo de compilação,
    não seria criado uma nova Árvore Sintática como no processo de tradução,
    mas sim a geração de código objeto ou
    binário \cite{ahoCompilerDragonBook}.

    Analisadores Sintáticos podem ser Ascendentes\footnote{
    Do inglês, \textit{Bottom~=Up}
    }
    ou Descendentes\footnote{
    Do inglês, \textit{Top~=Down}
    }.
    Devido a essa característica ambos possuem as suas vantagens e
    desvantagens.
    Um Analisador Ascendente realiza a construção da Árvore Sintática das folhas até a raíz,
    o contrário de um Analisador Descendente que realiza a construção da Árvore Sintática a partir da raíz até as folhas\footnote{
    Como pode ser observados em seus nomes,
    ambos os analisadores tanto da familia LL (Descendentes,
    \textit{Left-to-right, Leftmost derivation}) ou LR (Ascendentes, \textit{Left~=to~=right,
    Rightmost derivation}) fazem a leitura do programa de entrada da esquerda para a direita.
    }.

    Uma vantagem de um Analisador Ascendente é o suporte de uma maior classe de Gramáticas Determinísticas.
    Uma vantagem de um Analisador Descendente é a facilidade da recuperação de erros em relação aos Analisadores Ascendentes\footnote{
    Conceito abordado na
    \fullref{analisadoresSintaticos}
    }
    \cite{sippu1982,lr1ErrorRecovery,larkJosefGrosch}.


\section{Gramáticas}

    Gramáticas são conjuntos de regras que definem uma linguagem.
    Em linguagens formais,
    sendo $\alpha$ um não~=terminal,
    $\beta$ um terminal,
    $V_n$ um conjunto de não~=terminais,
    $V_t$ um conjunto de terminais e
    $V = V_n \cup V_t$,
    uma gramática é definida por quatro componentes:
    \begin{enumerate}%[nosep,nolistsep]
        \item \advisor{O}{Um} conjunto $V_t$ de símbolos terminais\advisor{ (também chamados
        de tokens ou símbolos do alfabeto da linguagem).
        Cada terminal corresponde a um símbolo presente na linguagem.
        }{,
        chamados algumas vezes de ``\textit{tokens}'' devido a sua forte conexão.
        Cada terminal corresponde a um símbolo presente no alfabeto da linguagem.
        }%
        Durante a Análise Léxica,
        os símbolos terminais serão utilizados definir os lexemas que são a base principal dos tokens.
        Na composição da Árvore Sintática,
        os ``\textit{tokens}'' ou terminais,
        serão sempre as folhas da Árvore Sintática.

        \item \advisor{O}{Um} conjunto $V_n$ de símbolos
        não~=terminais\advisor{ (algumas vezes chamados de ``variáveis sintáticas'').}{,
        algumas vezes chamados de ``variáveis sintáticas''.
        }
        não~=terminais servem para agrupar vários não~=terminais e\slash{}ou terminais.
        Na composição da Árvore Sintática,
        os símbolos não~=terminais sempre serão os nós da Árvore Sintática\footnote{
        Desde que a gramática da linguagem não contenha símbolos inúteis,
        i.e.,
        todos os símbolos da gramática são férteis e
        permitem a geração de palavras além do conjunto vazio $\varnothing$ \cite{hopcroftBook}
        }.
        Por convenção,
        e para evitar confusões entre quais são os símbolos terminais e não~=terminais,
        a intersecção entre o conjunto de símbolos terminais e
        não~=terminais é sempre vazia, i.e.,
        $V_n \cap V_t = \varnothing$.

        \item \label{definicaoDeGramatica}Um conjunto de produções $P$.
        Uma produção consiste em uma dupla elementos.
        O primeiro elemento é a cabeça ou
        lado esquerdo e
        representa a substituição ou
        consumo que será feito no programa de entrada.
        Ele é obrigatoriamente constituído de no mínimo um não~=terminal e
        um ou mais não~=terminais ou
        terminais.
        O segundo elemento é a cauda ou
        lado direito da produção,
        composto de terminais e\slash{}ou não~=terminais.
        Formalmente defini~=se uma produção pela seguinte regra,
        onde ``*'' representa o operador de fechamento do conjunto \cite{hopcroftBook}:
        $$P = \{\; \alpha ::= \beta \;|\; \alpha \in V^* V_n V^* \land \beta \in V^* \;\}$$

        \item Um símbolo inicial selecionado a partir do conjunto de símbolos não~=terminais.
        O símbolo inicial é utilizado para definir qual será a raíz da Árvore Sintática,
        e.g.,
        o última regra de produção utilizada para terminar o reconhecimento do programa de entrada em um Analisador Ascendente (ver seção \ref{reducoesEderivacoes}),
        e a primeira regra utilizada em um Analisador Descendente ou
        gerar~=se palavras desta linguagem\footnote{
        Processo natural quanto um Analisador Sintático realiza o reconhecimento de um programa de entrada.
        Para mais informações,
        veja a \fullref{reducoesEderivacoes}.
        }.
    \end{enumerate}


\subsection{Hierarquia de Chomsky}
\label{hierarquiaDeChomsky}

    Todas as gramáticas que existem são no mínimo\footnote{
    Caso contrário não serão gramáticas,
    mas qualquer outra definição no qual a Teoria de Linguagens Formais e
    Compiladores pode não se aplicar.
    }
    Gramáticas Tipo 0,
    também conhecidas como Gramáticas Irrestritas por que não possuem nenhuma restrição de complexidade de tempo,
    como os outros tipos de gramáticas a serem definidos nas próximas seções{}.
    A partir da adição restrições sobre a definição formal de gramática recém apresentada,
    também pode~=se \advisor{compreender}{realizar diversas classificações como} a hierarquia de \citeonline{chomskyGrammars1956},
    onde uma linguagem pode ser classificada como Regular,
    Livre de Contexto,
    Sensível ao Contexto e
    Irrestrita (Figura \ref{fig:pictures/HierarquiaDeChomsky.png}).
    \begin{figure}[h]
    \centering
    \includegraphics[width=1.0\textwidth]{pictures/HierarquiaDeChomsky.png}
    \begin{minipage}{\textwidth} \footnotesize
    *Para Gramáticas Regulares Determinísticas,
    complexidade linear ao tamanho da palavra de entrada para determinar se uma dada palavra pertence ou
    não a linguagem.
    Para Gramáticas Regulares Não~=Determinísticas,
    complexidade polinomial para construir as Árvores de Derivações e
    determinar se dada palavra pertence ou
    não a linguagem com algoritmos como CYK \cite{hopcroftBook,cykParsingAlgorithm}.
    Por fim,
    para Automatos Finitos Não~=Determinísticos ou
    Analisadores com Backtracking,
    tempo exponencial.

    **Para Gramáticas Livres de Contexto Determinísticas,
    também conhecidas como LR(K),
    complexidade linear ao tamanho da palavra de entrada (veja a seção \ref{gramaticasVersusLinguagens}).
    Para Gramáticas Livre de Contexto Não~=Determinísticas,
    vale o mesmo que para Linguagens Regulares Não~=Determinísticas logo acima,
    mas no lugar de Automatos Finitos,
    utilizam~=se Máquinas de Pilha.

    ***Para verificar se uma dada sentença pertence ou não a linguagem.
    \end{minipage}
    \caption[Hierarquia de Chomsky]{Hierarquia de Chomsky -- Fonte Própria\protect\footnotemark,
    \citeonline{sipserBook,ahoTheoryOfParsing,efficientNonDeterministicParsers,johnCocke}}
    \label{fig:pictures/HierarquiaDeChomsky.png}
    \end{figure}
    \footnotetext{
    Veja \citeonline{computationalComplexityAuroraBarak,complexityClasses} para aprender mais sobre Classes de Complexidade.
    }

    Toda Gramática Regular ou
    Livre de Contexto,
    é também uma Gramática Irrestrita ou
    Sensível ao Contexto,
    uma vez que Gramáticas Livres de Contexto ou
    Regulares são um subconjunto das Gramáticas Irrestritas ou
    Sensíveis ao Contexto como apresentado na Figura \ref{fig:pictures/HierarquiaDeChomsky.png}.
    Por isso,
    também pode~=se chamar uma dada Gramática Regular de Irrestrita ou
    Livre de Contexto.

    Quando diz~=se que existe Gramática Livre de Contexto para uma dada linguagem,
    pode~=se ter o equívoco de pensar que este é o melhor tipo,
    i.e.,
    o tipo mais eficiente em tempo computational de gramática no qual dada linguagem pode ser representada.
    Entretanto,
    precisa~=se tomar cuidado quando fala~=se sobre gramáticas e
    linguagens.

    Não pode~=se dizer que uma dada Linguagem é Livre de Contexto simplesmente por que existe uma Gramática Livre de Contexto para dada linguagem.
    Pois também é preciso que esta gramática seja o tipo mínimo no qual esta linguagem pode ser escrita.
    Sempre pode~=se escrever uma gramática menos eficiente do que o tipo mínimo de gramática que uma linguagem pode ser escrita.
    Para saber se este tipo de gramática é o mínimo,
    utiliza~=se o Lema do Bombeamento\footnote{
    Do inglês, \textit{Pumping Lemma} \cite{hopcroftBook,sipserBook}
    }
    para determinar e
    provar formalmente que dada gramática é o tipo mínimo de gramática para dada linguagem.


\subsection{Gramáticas Regulares}

    Gramáticas Regulares (também conhecidas como Tipo 3) são todas aquelas reconhecidas por Automatos Finitos Determinísticos e\slash{}ou Não~=Determinísticos.
    Gramáticas de Linguagens Regulares pela definição formal,
    são todas aquelas nos quais todas as Produções $P$ da gramática possuem a seguinte forma:
    $$ P = \{\; \alpha ::= a \beta \;|\; \alpha \in V_n \land a \in V_t
                \land \beta \in \{\; V_n \cup \varepsilon\; \} \;\} $$

\subsection{Gramáticas Livres de Contexto}

    Gramáticas Livres de Contexto (também conhecidas como Tipo 2) \cite{hopcroftBook} são todas aquelas reconhecidas por Automatos de Pilha Não~=Determinísticos.
    Gramáticas de Linguagens Livre de Contexto pela definição formal,
    são todas aquelas nos quais todas as Produções $P$ da gramática possuem a seguinte forma:
    $$ P = \{\; \alpha ::= \beta \;|\; \alpha \in V_n \land \beta \in V^* \;\} $$


\subsection{Gramáticas Sensíveis ao Contexto}

    Gramáticas Sensíveis ao Contexto (também conhecidas como Tipo 1) são todas aquelas reconhecidas por Automatos Linearmente Limitados,
    que tratam~=se somente de Máquinas de Turing \cite{sipserBook} com Fita (ou memória) Finita.
    Gramáticas de Linguagens Sensíveis ao Contexto pela definição formal,
    são todas aquelas nos quais todas as Produções $P$ da gramática possuem a seguinte forma:
    $$ P = \{\; \alpha ::= \beta \;|\; \alpha \in V^* V_n V^* \land \beta \in V^*
                \land \vert\alpha\vert \leq \vert\beta\vert \;\} $$


\subsection{Gramáticas Irrestritas}

    Por fim,
    as Gramáticas Irrestritas ou (também conhecidas como Tipo 0) possuem a mesma definição do
    que a definição válida de uma gramática como apresentado anteriormente (no
    \fullref{definicaoDeGramatica}).
    Gramáticas Irrestritas são reconhecidas somente por Máquinas de Turing\footnote{
    Máquinas de Turing possuem por definição fita (ou memória) ilimitada,
    mas não infinita,
    pois em um dado momento,
    somente uma quantidade finita de símbolos podem estar na fita,
    que continuamente pode crescer ilimitadamente.
    },
    e diferente das Gramáticas Sensíveis ao Contexto,
    a Máquina de Turing não possui parada garantida.

    Linguagens do Tipo 0 (ou Irrestritas) representam problemas incomputáveis e
    que podem ser representados de procedimentos \cite{sipserBook}.
    Já Linguagens do Tipo 1 (ou Sensíveis ao Contexto),
    representam todos os problemas computáveis e
    sua implementação pode ser representada por algoritmos,
    pois possuem parada garantida,
    apesar de terem em pior caso,
    tempo exponential ao contrário de tempo infinito como nas Linguagens Irrestritas.


\section{Analisadores Sintáticos}
\label{analisadoresSintaticos}

    Analisadores são equivalentes a Mecanismos Reconhecedores como Automatos Finitos,
    Automatos de Pilha ou
    Máquinas de Turing.
    No caso de outros Mecanismos como Automatos Finitos,
    o reconhecimento é feito a partir da especificação ou
    construção do automato que reconhece palavras de dada linguagem.
    Ambos gramáticas e
    automatos são equivalentes e
    existem algoritmos de conversão entre um e
    outro \cite{hopcroftBook}.

    Analisador Sintático\footnote{
    Além de Analisadores Sintáticos (Gramáticas Livre de Contexto),
    existem muitos outros como Analisadores Semânticos (Gramáticas Sensíveis ao Contexto) \cite{contextSensitiveParsing}.
    }
    é um nome dado para analisadores que recebem como entrada uma gramática que representa os aspectos estruturais de uma linguagem,
    i.e.,
    sua sintaxe \cite{ahoCompilerDragonBook}.
    Analisadores Sintáticos possuem muito mais utilidade do que somente checar se a sintaxe do programa de entrada está correta.
    Uma vez que eles também geram a Árvore Sintática do programa\footnote{
    Como visto no começo desde capítulo na \fullref{compiladoresEtradutores}.
    }
    que é utilizada para realizar a análise semântica e
    geração de código.


\subsection{Gramáticas $versus$ Linguagens}
\label{gramaticasVersusLinguagens}

    É importante fazer a distinção entre Gramáticas Livre de Contexto e
    as Linguagens Livre de Contexto.
    \citeonline{parikh1966} provou que existem linguagens nas quais não existem Gramáticas Não~=Ambíguas que representem estas linguagens.
    Tais linguagens são conhecidas como Linguagens Inerentemente Ambíguas\footnote{
    Do inglês,
    \textit{Inherently Ambiguous Languages}.
    }
    onde não existem Gramáticas Livre de Contexto Determinísticas capazes de representa~=las e
    tais Linguagens somente podem ser reconhecidas por Analisadores com Backtracking \cite{ahoCompilerDragonBook} ou
    Automatos de Pilha Não~=Determinísticos.

    A maior classe de Gramáticas Determinísticas suportadas por Analisadores Sintáticos são as Gramáticas LR(K)\footnote{
    Do inglês, \textit{Left~=to~=right,
    Rightmost derivation} em reverso com K símbolos de \textit{lookahead}.
    \textit{Rightmost} significa que ao realizar as derivações,
    escolhe~=se sempre o não~=terminal mais a direita.
    }.
    Analisadores LR(K) \cite{ahoCompilerDragonBook} são Ascendentes e
    reconhecem um subconjunto das Linguagens Livre de Contexto (Figura \ref{fig:pictures/LinguagensDeterministicas.png}).
    Já os Analisadores LL(K)\footnote{
    Do inglês, \textit{Left-to-right,
    Leftmost derivation} com K símbolos de \textit{lookahead}.
    \textit{Leftmost} significa que ao realizar as derivações,
    escolhe~=se sempre o não~=terminal mais a esquerda.
    }
    são Descendentes \cite{antlrBookTerrentParr,llStarAntlr} e
    reconhecem somente um subconjunto das Linguagens LR(K)\footnote{
    Diz~=se que uma linguagem é LR(K) ou
    LL(K) quando ela é reconhecida por este analisador
    }.

    A Figura \ref{fig:pictures/LinguagensDeterministicas.png} não é inteiramente um Diagrama de Venn \cite{generalizedVennDiagrams},
    inicialmente,
    nas camadas mais externas,
    ele é uma relação abstrata entre Linguagens Ambíguas e
    Gramáticas Determinísticas.
    O Conjunto das Gramáticas Livre de Contexto Determinísticas está contido dentro das Linguagens Livre de Contexto\footnote{
    Também existem Gramáticas Sensíveis ao Contexto Determinísticas \cite{contextSensitiveParsing},
    entretanto,
    algoritmos de análise possuem em pior caso,
    complexidade exponencial.
    }.
    O primeiro nível significa que todas as Linguagens Inerentemente Ambíguas\footnote{
    É comum confundir~=se e
    chamar Gramáticas de Inerentemente Ambíguas,
    mas esse termo não existe para gramáticas.
    Ou elas são Ambíguas ou
    Não.
    Somente uma linguagem pode ser Inerentemente Ambígua.
    }
    são representáveis somente por Gramáticas Ambíguas.

    O segundo nível significa que Linguagens Não~=Inerentemente Ambíguas\footnote{
    Somente utilizado para enfatizar o conjunto de Linguagens no qual existem Gramáticas Ambíguas e
    Determinísticas (ou Não~=Ambíguas).
    }
    podem ser representadas por Gramáticas Ambíguas e\slash{}ou Determinísticas.
    No terceiro nível encontra~=se as gramáticas que são mais importantes,
    as Gramáticas Determinísticas\footnote{
    As Gramáticas Determinísticas representam o conjunto de Linguagens que podem ser Analisadas Deterministicamente e
    tais Linguagens também podem ser conhecidas como LR(K),
    LR(K).
    Reveja os parágrafos após a Figura \ref{fig:pictures/HierarquiaDeChomsky.png}
    },
    que podem ser classificadas como LR(K),
    LL(K)\advisor{etc, }{ entre outros, }i.e.,
    de acordo com o tipo de analisador que pode ser construído.
    \begin{figure}[h]
    \centering
    \includegraphics[width=1.0\textwidth]{pictures/LinguagensDeterministicas.png}
    \caption[Gramáticas Determinísticas \textit{versus} suas Linguagens]{Gramáticas Determinísticas \textit{versus} suas Linguagens -- Fonte Própria,
    \citeonline{llVersusLrContainment,llContainmentInLalr,beatty1982,ahoCompilerDragonBook}}
    \label{fig:pictures/LinguagensDeterministicas.png}
    \end{figure}

    Diz~= que um Analisador Determinístico para dada gramática pode ser construído quando a criação de sua Tabela de Análise\footnote{
    Do inglês, \textit{Parsing Table}
    }
    \cite{ahoCompilerDragonBook} acontece sem conflitos.
    É importante notar que usualmente o processo de análise por um analisador,
    seja ele LR(K) ou
    LL(K),
    acontece em duas etapas.
    Com a exceção dos Analisadores LL(K) que também podem ser facilmente construídos programaticamente,
    i.e.,
    com o programador construindo manualmente como deve acontecer cada transição de estado do analisador \cite{ahoCompilerDragonBook}.

    Na primeira etapa do analisador utilizam~=se algoritmos de construção da Tabela de Análise.
    Quando a tabela está construída sem conflitos (este analisador portanto,
    é determinístico),
    entra em cena o algoritmo de análise na segunda etapa,
    que utilizando a Tabela de Análise,
    realiza o reconhecimento do programa de entrada.

    Como única diferença entre os Analisadores LR(K),
    LALR(K) e
    SLR(K) é exatamente construção da Tabela de Análise,
    ambos possuem a mesma complexidade de análise linear \cite{knuthLrParser1965} ao tamanho do programa\footnote{
    Quando refere~=se a programa,
    fala~=se da \textit{string} ou
    texto que será analisado e
    decidir se tal programa é um programa da linguagem que se está analisando.
    Um ponto curioso,
    caso o programa não seja aceito pelo analisador,
    ele não é um programa com erros,
    mas um programa inválido,
    i.e.,
    de uma outra linguagem,
    que não é a linguagem que está sendo analisada.
    Comumente ou
    informalmente,
    chamamos estes programas como programas com erros (de sintaxe).
    }
    de entrada que será analisado \cite{linearLL1AndLR1Grammars,generalContextFreeParsingAlgorithm}.

    No caso de conflitos na Tabela de Análise,
    a gramática não pode ser analisada deterministicamente e
    algoritmos de análise com backtracking (ou CYK,
    reveja a seção \ref{hierarquiaDeChomsky}) precisam ser utilizados para construção da Árvore Sintática.
    Como mostrado para Máquinas de Turing Não~=Determinísticas na seção \ref{mecanismosReconhecedores},
    Analisadores com Backtracking também funcionam em pior caso,
    com tempo exponencial e
    podem escolher uma estratégia como Busca em Profundidade\footnote{
    Veja a \fullref{buscaEmLarguraEProfundidade} para saber mais.
    } para executar os Ramos de Computação Não~=Determinístico.


\subsection{Reduções e Derivações}
\label{reducoesEderivacoes}

    Diferente Máquinas Reconhecedoras específicas como Automatos Finitos,
    os analisadores recebem diretamente como entrada uma gramática de uma dada linguagem.
    Mas diferente de gramáticas e
    Analisadores LL(K),
    Analisadores LR(K) especificamente funcionam de modo contrário.
    Eles operam por meio de Reduções ao invés de Derivações como no caso das Gramáticas e
    Analisadores LL(K) \cite{sipserBook}.

    Uma Derivação acontece quando uma regra de produção como ``$S \Rightarrow a a $'' de uma gramática expande e
    tem~=se como resultado ``$a a$'' a partir do símbolo de origem ``$S$''.
    Já uma Redução acontece quanto a dada regra de produção como ``$S \Rightarrow a a $'' de uma gramática reduz e
    tem~=se como resultado ``$S$'' a partir do símbolo de origem ``$a a$''.

    Tanto Derivações quanto Reduções podem ser descritas em termos que quantos passos são necessários para que se possa sair de um ponto até outro \cite{ahoCompilerDragonBook}:
    \begin{enumerate}%[nosep,nolistsep]
        \item Quanto um derivação é denotada como ``$S \Rightarrow a a $'',
        isso significa que somente um passo é necessário para sair do símbolo inicial ``$S$'' e
        chegar no símbolo final ``$a a$''.
        \item Quanto um derivação é denotada como ``$S \xRightarrow{*} a a $'',
        isso significa que são necessários,
        desde zero (nenhum) até infinitos passos para sair do símbolo inicial ``$S$'' e
        chegar no símbolo final ``$a a$''.
        \item Quanto um derivação é denotada como ``$S \xRightarrow{+} a a $'',
        isso significa que são necessários,
        desde um passo até infinitos passos para sair do símbolo inicial ``$S$'' e
        chegar no símbolo final ``$a a$''.
    \end{enumerate}

    Para reduções,
    estas mesmas condições se aplicam,
    mas em ordem reversa,
    i.e., $\Leftarrow$, $\xLeftarrow{*}$ e $\xLeftarrow{+}$,
    ao invés de $\Rightarrow$,
    $\xRightarrow{*}$ e
    $\xRightarrow{+}$.
    Enquanto gramáticas são geradores de palavras que partem do símbolo inicial da gramática até gerarem uma palavra da linguagem,
    Analisadores Ascendentes como LR(K) são reconhecedores de palavras.

    Diferente de gramáticas,
    Analisadores Ascendentes partem de uma palavra da linguagem até chegarem no símbolo inicial da gramática,
    consumindo toda a palavra de entrada e
    chegando em um Estado de Aceitação.
    Já Analisadores Descendentes como LL(K),
    partem do símbolo inicial da gramática até consumirem toda palavra de entrada,
    também chegando um em Estado de Aceitação.

    Ambos os Analisadores Ascendentes ou
    Descendentes,
    terminam no final do processo,
    gerando toda a Árvore de Derivação.
    Entretanto,
    caso no final do processo,
    não chegue~=se em um Estado de Aceitação,
    i.e.,
    no símbolo inicial da gramática.
    Tem~=se somente a construção de uma Árvore de Derivação partial \cite{allStarAntlr}.

    No caso dos Analisadores Ascendentes,
    será uma floresta de árvores,
    por que somente no final da análise,
    com a chegada ao símbolo inicial da gramática,
    completa~=se custura de todas a árvores que foram parcialmente construídas durante o processo de análise (\textit{Bottom~=Up}).

    Já no caso dos Analisadores Descendentes,
    não existe uma floresta de árvores.
    Como parte~=se diretamente do símbolo inicial da gramática,
    a Árvore de Derivação desde o começo é construído como sendo uma única árvore (\textit{Top~=Down}).


\subsection{Analisadores LR(K)}

    Como pode ser observado na Figura \ref{fig:pictures/LinguagensDeterministicas.png},
    existem Gramáticas SLR(K) que não são Gramáticas LL(K) por que para uma gramática ser LL(K),
    ela precisa respeitar 3 propriedades,
    \begin{inparaenum}
        \item Não possuir Recursão a Esquerda,
        \item Estar fatorada e
        \item $\forall\; A\, \in\, V_n\; |\; A\,
                \xRightarrow{*}\, \varepsilon\,
                \land\, First(A)\, \cap\, Follow(A) = \varnothing$
    \end{inparaenum}
    \cite{ahoCompilerDragonBook}.

    Entretanto,
    Gramáticas LR(K), LALR(K) e
    SLR(K) não precisam de nenhuma dessas restrições.
    No caso da Recursão a Esquerda,
    o algoritmo de criação da Tabela de Análise Sintática da Gramática LR(K),
    LALR(K) ou SLR(K),
    não possui o problema de entrar em um loop infinito assim como acontecem com as Gramáticas LL(K),
    portanto aceitando~=se Gramáticas com Recursão a Esquerda.

    Uma vez que Analisadores LR(K) requerem uma quantidade de memória exponencial \cite{complexityOfLRKTesting} ao tamanho da gramáticas de entrada para operar,
    \citeonline{lalrDeRemer1982} criaram os Analisadores LALR(K)\footnote{
    Do inglês, \textit{Look~=Ahead} LA(K) LR(0),
    onde LR(0) é um Analisador LR(K) com $K=0$
    }
    e SLR(K)\footnote{
    Do inglês, \textit{Simple LR(K) parser}
    }
    com o objeto de viabilizar a implementação de Analisadores Ascendentes Determinísticos.

    Gramáticas de Linguagens Determinísticas são chamadas de LR,
    por que todas as Linguagens Determinísticas são reconhecidas por Analisadores LR(K),
    uma vez que \citeonline{knuthLrParser1965} provou que todas as Gramáticas Determinísticas são aceitas por um Analisador LR(K).

    Assim,
    além da hierarquia de Chomsky,
    também classifica~=se as gramáticas de acordo com o tipo de analisador que reconhece as linguagens representadas por elas.
    Como mostrado na Figura \ref{fig:pictures/LinguagensDeterministicas.png},
    nem todas as Gramáticas Livre de Contexto são de Determinísticas e
    uma gramática é Determinística somente se ela pode ser reconhecida por um Analisador LR(K).

    Portanto uma maneira fácil de decidir se uma dada gramática é determinística ou
    não,
    é tentar construir a sua tabela de um Analisador LR(K).
    Caso consiga~=se construir com sucesso (sem conflitos) a Tabela de Análise Sintática \cite{ahoCompilerDragonBook},
    a gramática é LR(K),
    caso contrário a gramática não é determinística.

    A mesma técnica pode ser aplicada no caso de analisadores menos poderosos como LALR(K),
    entretanto,
    uma vez que não se consiga construir a Tabela de Análise Sintática,
    não se pode ter certeza se dada gramática é ou não determinística.


\subsection{Análise Semântica}

    Usualmente,
    somente depois que a Árvore Sintática é construída,
    realiza~=se o processo de Análise Semântica \cite{ahoCompilerDragonBook},
    i.e.,
    a verificação da corretude do programa escrito em relação os aspectos Não~=estruturais,
    por exemplo,
    é sintaticamente correto escrever a declaração de uma mesma variável duas vezes ou
    mais.

    Entretanto,
    para algumas linguagens é semanticamente errado redeclarar uma variável duas vezes ou
    mais.
    O Analisador Sintático representado por uma Gramática Livre de Contexto não tem poder suficiente para realizar tais verificações devido as limitações desse tipo de gramática,
    que restringem~=se a estrutura do programa e
    não a seu significado (semântica).

    Nem todas as linguagens podem ser analisadas completamente em diferentes etapas,
    como Análise Léxica, Sintática e Semântica. Muitas vezes,
    estas três etapas acontecem em paralelo como realizado na implementação do compilador da Linguagem C \cite{jourdan2017,whyCcannotBeParsedWithALR1Parser}.

    A Gramática da Linguagem C não é Livre de Contexto devido as ambiguidades (conhecido também como Não~=Determinismo) existentes como a expressão ``\textit{x * y ;}''.
    Tal sentença pode ser ou
    a declaração de um ponteiro chamado \textit{y} do tipo \textit{x},
    ou a multiplicação de dois números armazenados nas variáveis \textit{x} e
    \textit{y},
    portanto ela não pode ser aceita por um Analisador LR(K) tradicional.

    Uma otimização que o compilador C faz para poder fazer o Analisador ``Determinístico'' da linguagem C,
    e assim saber se a expressão ``\textit{x * y ;}'' trata~=se de de uma mera multiplicação ou
    a declaração de uma variável,
    é exatamente a realização simultânea da Análise Léxica,
    Sintática e
    Semântica.
    Uma vez que um novo \textit{token} é reconhecido,
    ele é alimentado para o Analisador Sintático,
    que também o alimenta para o Analisador Semântico.

    Assim,
    o Analisador Sintático é capaz de consultar a Tabela de Símbolos \cite{ahoCompilerDragonBook} e
    descobrir se dado token ou
    tratar~=se de um tipo ou
    uma variável numérica.
    Entretanto,
    requer~=se cuidado sobre como estas alterações são feitas,
    pois pode~=se pensar que as gramáticas de todas as linguagens de programação são ``Livres de Contexto'' e
    Determinísticas.
    E uma vez que a gramática não é mais Livre de Contexto ou
    Determinística,
    pode~=se mover Aspectos Sensíveis ao Contexto para o Analisador Semântico,
    assim,
    deixando a gramática somente com aspectos determinísticos.


\subsection{Alterações nos Analisadores Sintáticos}

    Dependendo de como o Analisador Sintático de Gramáticas Livre de Contexto é alterado,
    o conjunto de gramáticas aceitos por tal analisador pode deixar de serem Livres de Contexto.
    As gramáticas somente continuarão Livre de Contexto caso estas alterações sejam somente mover checagens da etapa de Análise Sintática para a etapa de Análise Semântica sem realizar alterações no Analisador Sintático.

    Quanto se adiciona suporte a Aspectos Sensíveis ao Contexto \cite{contextSensitiveParsing} a Gramáticas Livre de Contexto por meio de alterações do Analisador Sintático,
    como feito no Analisador da Linguagem C,
    o analisador da gramática deixa de ser Livre de Contexto,
    suportando assim,
    algumas Gramáticas Sensíveis ao Contexto e\slash{}ou também algumas Gramáticas Não~=Determinísticas.

    Note que,
    a pesar disso não impede~=se que a gramática da Linguagem C,
    como mostrado na seção anterior,
    seja analisada com eficiência.
    Mas isso deixa a brechas para que ela possa não ser analisada com eficiência.
    A diferença para um analisador onde a gramática é inteiramente Livre de Contexto,
    é que elas tem performance \textit{garantida} pela sua Classe de Complexidade (Veja seção \ref{classesDeComplexidade}).

    Sintaxe e
    Semântica de Linguagens são completamente ortogonais.
    Gramáticas de Linguagens Irrestritas\footnote{
    Não a linguagem no qual elas representam,
    mas a própria gramática em si \cite{finiteAutomataTuringComplete}
    }
    podem ser Turing Completas\footnote{
    A Turing Completude acontece quando uma dada linguagem pode simular o funcionamento completo de uma Máquina de Turing
    }
    devido a sua equivalência com Máquinas de Turing e
    são capazes de realizar qualquer operação computacional.
    Mas,
    isso não pode ser confundido com as \textit{Strings} ou
    Programas gerados por essas gramáticas \cite{areThereDomainSpecificLanguages}.

    Tais programas podem ou
    não ser Turing Completos.
    Do lado oposto,
    até Linguagens Regulares podem gerar programas que são Turing Completos,
    mesmo que seu dispositivo reconhecedor equivalente,
    os Automatos Finitos,
    não tenham Turing Completude\footnote{
    Caso isso esteja confuso,
    reveja a Figura \ref{fig:pictures/HierarquiaDeChomsky.png} e
    note que de todas as Linguagens,
    quem tem Turing Completude são as Linguagens Irrestritas,
    enquanto Automatos Finitos são um subconjunto das Máquinas de Turing \cite{finiteAutomataTuringComplete}
    }
    \cite{turingCompleteRegularLanguages,finiteAutomataTuringComplete}.

    Na seção \ref{sec:software_implementation},
    será mostrado a implementação de uma Gramática ``Livre de Contexto'' em um Analisador LALR(1),
    onde Aspectos Sensíveis ao Contexto serão analisados pelo Analisador Semântico,
    tal como feito na implementação do Compilador da Linguagem C apresentado.
    Mas com a diferença de que utiliza~= um Analisador LALR(1) genérico,
    ao contrário de um analisador feito exclusivamente para a linguagem alvo.

    Este Analisador LALR(1),
    possui suporte a pequenos ``\textit{\englishword{hacks}}'' ou
    otimizações que permitem adicionar alguns aspectos Sensíveis ao Contexto ao Analisador LALR(1).
    Assim,
    as gramáticas aceitas por esse analisador incluem somente algumas Gramáticas Não~=Determinísticas,
    não incluindo todas as Gramáticas Livre de Contexto Ambíguas ou
    Sensíveis ao Contexto devido a limitações das alterações do algoritmo de Análise LALR(1) \cite{larkContextualLexer}.


\section{\advisor{Compiladores e }{}Classes de Complexidade}
\label{classesDeComplexidade}

    Como um todo,
    o conjunto de Linguagens Regulares pode ser considerado com complexidade linear\footnote{
    Complexidade linear é um caso particular de complexidade polinomial onde o grau do Polinômio é 1,
    i.e.,
    $\Theta(n)$.
    Aprenda mais sobre complexidade linear com \citeonline{cormenIntroductionToAlgorithms,computationalComplexityAuroraBarak}.
    }
    em tempo computacional para determinar de dada palavra pertence ou
    não a linguagem,
    por que toda Gramática Regular Não~=Determinística pode ser convertida em uma Gramática Regular Determinística \cite{sipserBook}.

    Infelizmente isso não é verdade para Gramáticas Livres de Contexto,
    por que Gramáticas Livre de Contexto Determinísticas $\Theta(n)$ e
    Não~=Determinísticas não são equivalentes.
    Gramáticas Não~=Determinísticas possuem complexidade exponential,
    quando analisadas por um Analisador com Backtracking.
    Em contra~=partida,
    Gramáticas Livre de Contexto Não~=Determinísticas também podem ser analisadas em tempo polinomial $\Theta(n^3)$ utilizando algoritmos de parsing tal como CYK (seção \ref{hierarquiaDeChomsky} e
    \citeonline{larkContextualLexer}).


\advisor{}{%
\subsection{Complexidade Computacional com Computadores Quânticos}

    Com a exceção de alguns problemas específicos \cite{theGoodAndBadQuantumComputing},
    a execução probabilística de Computadores Quânticos \cite{nonlinearQuantumComputers},
    baseados nas leis da Física Quântica \cite{dicke1963QuantumPhysicsIntroduction},
    podem cortar\footnote{
    Devido as probabilidades envolvidas,
    somente um ou
    alguns dos ramos de computação serão seguidos durante a execução do Algoritmo Quântico,
    pelo Computador Quântico.
    }
    caminho ``pulando'' ramos de Computação Não~=Determinísticos (da computação Clássica) com a superposição quântica.
    Assim,
    conseguindo resolver alguns problemas que são exponenciais,
    em tempo polinomial ao tamanho da entrada,
    utilizando algoritmos específicos para computadores quânticos \cite{quantumComputerSurvey,quantumSimulatorChagas}.

    Esta é a gama de problemas nos quais Computadores Quânticos são úteis \cite{quantumComputingForNonPhysicists},
    não sendo assim,
    substitutos completos da Computação Tradicional (ou Clássica) \cite{efficientQuantumComputation},
    somente otimizadores na resolução de alguns problemas que podem ser otimizados devido as propriedades específicas\slash{}probabilísticas das leis Física Quântica \cite{churchTuringQuantumComputer}.

    Pode~=se confundir Computadores Quânticos como equivalentes a Analisadores Não~=Determinísticos devido as nomenclaturas utilizadas.
    Enquanto analisadores são Não~=Determinísticos devido à ambiguidades nas gramáticas de entrada,
    Computadores Quânticos são Não Determinísticos devido à serem baseado em modelos Probabilísticos,
    i.e.,
    Computadores Quânticos não são equivalentes a Analisadores Não~=Determinísticos devido a sua execução ser probabilística \cite{polynomialQuantumComputers,probabilisticQuantumComputation,quantumSimulatorChagas}.

    Diferente dos Computadores Tradicionais,
    Computadores Quânticos são construídos com base nas leis da Física Quântica,
    que são radicalmente diferentes das Leis da Física Tradicional ou
    Clássica, i.e.,
    as Leis de Newton.
    As Leis da Física Clássica regem os elementos muitos grandes na escala galáxias,
    planetas, células e
    virus \cite{halliday2013fundamentals}.
    Já as Leis da Física Quântica regem as elementos muito pequenos na escala de átomos,
    elétrons, prótons, fótons e
    \textit{quarks} \cite{dicke1963QuantumPhysicsIntroduction}.
}

\subsection{Complexidade Teórica $versus$ Real}

    Na Figura \ref{fig:pictures/ParserNonDeterministic.png},
    encontra~=se uma Árvore de Computação de um Analisador Não~=Determinístico.
    Diz~=se que que o tempo de execução de um Analisador Não~=Determinístico é Não~=Determinístico Polinomial $NP$\footnote{
    Do inglês, \textit{Non~=Deterministic Polynomial Time},
    comumente conhecida pela pergunta,
    $P \stackrel{?}{=} NP$, i.e.,
    a classe de problemas com complexidade de tempo polinomial,
    está estritamente contida na classe de problemas $NP$ (Não~=Determinísticos Polinomiais) \cite{computationalComplexityAuroraBarak}?
    }
    ao tamanho da entrada,
    por que um Analisador Não~=Determinístico executa simultaneamente todos os ramos de Computação Não~=Determinísticos \cite{hopcroftBook}.

    Como mostrado na Figura \ref{fig:pictures/ParserNonDeterministic.png},
    após a cada um dos passo de computação 1,
    2, 3 e 4,
    todos os 15 ramos de computação foram concluídos.
    Cada um desses passos é corresponde a um item a ser analisado na entrada do programa.
    E esta computação,
    acontece em tempo Não~=Determinístico Polinomial,
    com expoente de $n$ igual a $1$,
    i.e.,
    $n^1$.
    \begin{figure}[h]
    \centering
    \includegraphics[width=1.0\textwidth]{pictures/ParserNonDeterministic.png}
    \caption[Árvore de Computação com 4 Passos de um Problema da Classe $NP$]{Árvore de Computação com 4 Passos de um Problema da Classe $NP$ -- Fonte Própria}
    \label{fig:pictures/ParserNonDeterministic.png}
    \end{figure}

    O que torna a computação Não~=Determinística\footnote{
    E pertencente a classe dos problemas Não~=Determinístico Polinomial.
    } é o fato de cada um dos itens 1,
    2, 3 e 4 da entrada,
    permitirem simultaneamente a escolha de mais de um caminho na escolha do próximo estado do analisador,
    i.e.,
    mais de um Ramo de Computação,
    devido a ambiguidades da gramática de entrada \cite{antlrBookTerrentParr}.

    Nesse contexto,
    $P$ representa o conjunto de problemas resolvidos tempo Determinístico Polinomial (por Máquinas de Turing Determinísticas),
    enquanto $NP$ o conjunto de Problemas resolvido em tempo Não~=Determinístico Polinomial (por Máquinas de Turing Não~=Determinísticas).
    Assim,
    um problema Não~=Determinístico Polinomial somente pode ser resolvido por uma Máquinas de Turing Determinística em tempo exponencial.

    O tempo de execução será linear ao tamanho da entrada caso o Analisador Não~=Determinístico seja de uma Linguagem Regular e
    implementado através de um Automato Finito.
    O tempo de execução será polinomial ao tamanho da entrada caso o Analisador Não~=Determinístico seja de uma Linguagem Livre de Contexto Ambígua (Gramática Não~=Determinística) e
    implementado através de algum algoritmo com tempo polinomial como CYK \cite{allStarAntlr}.

    Como já explicado nas observações da Figura \ref{fig:pictures/HierarquiaDeChomsky.png},
    existem duas classes distintas de complexidade para Gramáticas Livre de Contexto Não~=Determinísticas.
    Quando faz~=se uma Análise de uma Gramáticas Livre de Contexto Não~=Determinística,
    tem~=se como resultado várias possíveis Árvores de Derivação\footnote{
    Como a gramática é Não~=Determinística,
    existem muitas possíveis Árvores de Derivação,
    (devido à ambiguidade da gramática).
    }.

    Algoritmos de construção das Árvores de Derivação com Backtracking para Gramáticas Livre de Contexto Não~=Determinística,
    possuem tempo exponencial de execução.
    Já algoritmos como CYK,
    que simplesmente dizem se dada palavra pertence ou
    não a linguagem,
    possuem complexidade polinomial ao tamanho da palavra de entrada \cite{hopcroftBook}.


\subsection{Mecanismos Reconhecedores}
\label{mecanismosReconhecedores}

    Uma vez que o conjunto de Linguagens Determinísticas LR(K) (com tempo linear) está contida no conjunto das Linguagens Livres de Contexto,
    não considera~=se tempos Análise Lineares ou
    Polinomiais de execução para Linguagens Sensíveis ao Contexto ou
    Irrestritas,
    por que tudo o que é eficiente é Livre de Contexto e
    Determinístico.

    Algoritmos de Análise para essas outras classes de Linguagens serão exponenciais em pior caso \cite{contextSensitiveParsing} e
    quando analisados por dispositivos equivalentes a Máquinas de Turing Não~=Determinísticas terão no mínimo tempo polinomial\footnote{
    Quando fala~=se de complexidade de tempo polinomial para Máquinas Não~=Determinísticas,
    resulta~=se em uma complexidade de tempo exponencial ao simular o funcionamento dessa Máquina Não~=Determinística em um computador,
    ou seja,
    a Real Complexidade do problema termina sendo exponencial,
    enquanto teoricamente a complexidade é polinomial.
    Veja as figuras \ref{fig:pictures/ParserNonDeterministic.png} e
    \ref{fig:pictures/ParserDeterministic.png} e
    as compare.
    }
    de execução.

    Máquinas de Turing Não~=Determinísticas que resolvem os problemas da Classe $NP$ em tempo polinomial não existem fisicamente,
    portanto sua complexidade de tempo reduzida não pode ser alcançada e
    seu tempo de execução é exponential,
    pois para simular o funcionamento de uma Máquina de Turing Não~=Determinística,
    utiliza~=se uma Máquina de Turing Determinística\footnote{
    Máquinas de Turing Determinísticas são equivalentes aos computadores de propósito geral
    }
    \cite{sipserBook,turingMachinesRoyer}.

    Assim,
    Máquinas de Turing Determinísticas e
    Não~=Determinísticas são equivalentes,
    pois sempre é possível simular o funcionamento de uma Máquina de Turing Não~=Determinística,
    utilizando uma Máquina de Turing Determinística \cite{hopcroftBook}.

    Na Figura \ref{fig:pictures/ParserDeterministic.png},
    encontra~=se a mesma Árvore de Computação apresentada na Figura \ref{fig:pictures/ParserNonDeterministic.png},
    mas com a diferença de que desta vez utiliza~=se uma Máquina de Turing Determinística ao contrário de uma Máquina de Turing Não~=Determinística.
    Com isso,
    ao invés de um tempo polinomial ao tamanho da entrada,
    tem~=se um tempo exponential ao tamanho da entrada.
    \begin{figure}[h]
    \centering
    \includegraphics[width=1.0\textwidth]{pictures/ParserDeterministic.png}
    \caption[Árvore de Computação com 15 Passos]{Árvore de Computação com 15 Passos -- Fonte Própria}
    \label{fig:pictures/ParserDeterministic.png}
    \end{figure}

    Para que uma Máquina de Turing Determinística possa processar uma Gramática Não~=Determinística,
    é necessário execute cada um dos ramos de computação.
    Já Máquinas de Turing Não~=Determinísticas\footnote{
    Máquinas de Turing da classe $NP$,
    somente existem teoricamente
    }
    executam simultaneamente todos os ramos de computação Não~=determinísticos,
    conseguindo assim, desempenho linear ou
    polinomial ao tamanho da entrada compondo os problemas da classe $NP$ (com tempo Não~=Determinístico Polinomial) \cite{hopcroftBook}.


\subsection{Busca em Largura e Profundidade}
\label{buscaEmLarguraEProfundidade}

    Quando uma Máquina de Turing Determinística é utilizado para simular o funcionamento de uma Máquina de Turing Não~=Determinística,
    ela precisa decidir como escolher executar os Ramos de Computação Não~=Determinísticos \cite{sipserBook}.
    Duas principais abordagens distintas e
    conhecidas\footnote{
    Além dessas duas abordagens,
    existem muitas outras técnicas que podem ser cridas como misturas dessas duas estratégias extremas,
    como heurísticas e inteligências artificias
    }
    são a Busca em Largura\footnote{
    Do inglês, \textit{Breadth-First Search (BFS)}
    }
    e Busca em Profundidade\footnote{
    Do inglês, \textit{Depth-First Search (DFS)}
    }.
    Os algoritmos funcionamento desses tipos de busca são detalhadas em \citeonline{cormenIntroductionToAlgorithms}.

    Cada uma delas apresenta suas vantagens e
    desvantagens.
    Uma vantagem da Busca em Profundidade é possibilidade de ``sorte'',
    caso o primeiro ramo não~=determinístico que escolhe~=se seja uma solução para o problema,
    i.e.,
    leve o analisador a um estado de aceitação,
    mas ao mesmo tempo de pode~=se ter ``sorte'',
    pode~=se ter o ``azar'' de que o primeiro ramo não~=determinístico seja um ramo infinito de computações que nunca levarão o analisador à um estado de aceitação.

    A Figura \ref{fig:pictures/ParserDeterministic.png} mostrou um exemplo de uso do algoritmo de Busca em Profundidade.
    Já na Figura \ref{fig:pictures/ParserDeterministicBreadth.png},
    encontra~=se a variação de execução de um Analisador Determinístico que utilizou o algoritmo de Busca em Largura.
    \begin{figure}[h]
    \centering
    \includegraphics[width=1.0\textwidth]{pictures/ParserDeterministicBreadth.png}
    \caption[Árvore de Computação com 15 Passos utilizado Busca em Largura]{Árvore de Computação com 15 Passos utilizado Busca em Largura -- Fonte Própria}
    \label{fig:pictures/ParserDeterministicBreadth.png}
    \end{figure}

    Tanto o algoritmo de Busca em Largura quanto Busca em Profundidade,
    não precisam exatamente seguir resolvendo o problema pela esquerda ou
    direita.
    O que importa é a sua característica de avançar até o fim de algum dado ramo de computação,
    ou seguir executando todos os ramos que fazem parte de um mesmo nível de computação \cite{cormenIntroductionToAlgorithms,efficientBreadthFirstSearch}.
}


    

% The \phantomsection command is needed to create a link to a place in the document that is not a
% figure, equation, table, section, subsection, chapter, etc.
%
% When do I need to invoke \phantomsection?
% https://tex.stackexchange.com/questions/44088/when-do-i-need-to-invoke-phantomsection
\phantomsection


% Is it possible to keep my translation together with original text?
% https://tex.stackexchange.com/questions/5076/is-it-possible-to-keep-my-translation-together-with-original-text
\chapter{Source Code Beautifiers}
\label{source_code_beautifiers}


    Below there are some basic formatting rules for
    illustration:\footnote{\url{https://www.python.org/dev/peps/pep-0008/#should-a-line-break-before-or-after-a-binary-operator}}

    \medskip
    % \begin{bluebox}
    \begin{enumerate}[nosep,nolistsep]
        \item Add new lines after `\{' and before `\}
        \item Add new lines before `\{'
        \item Remove empty lines
        \item Add comment lines before function
        \item Add new lines after `;'
        \item Add new lines after `\}'
        \item Remove new lines
        \item Reduce whitespace
        \item Fix bad indentation
    \end{enumerate}
    % \end{bluebox}
    \vspace{-4mm}\begin{flushright}\textcite{prettyPrinter}\end{flushright}

    Mostly,
    code formatting is recuded into this set of changes.



    \section{For what is their use?}

    For now could not find any strong evidence or correlation about code
    comprehension and source code beautifying \cite{improvingCodeReadability},
    except perhaps for team annoyance:

    \begin{citacao}
    % \setlength{\itemindent}{5pt}
    One of absolute worst, worst methods of teamicide for software developers is to engage
    in these kinds of passive-aggressive formatting wars. I know because I've been there.
    They destroy peer relationships, and depending on the type of formatting, can also damage
    your ability to effectively compare revisions in source control, which is really scary.
    I can't even imagine how bad it would get if the lead was guilty of this behavior. That's
    leading by example, all right. Bad example. \cite{Atwood}
    \end{citacao}

    On \citeonline{improvingCodeReadability} was analyzed the level of program
    comprehension which can be gained by the indentation and was constated that
    the indentation levels of 2 and 4 spaces proved to have the best
    comprehension levels again other levels.

    \begin{citacao}
    So yes, absurd as it may sound, fighting over whitespace characters and other seemingly
    trivial issues of code layout is actually justified. Within reason of course -- when done
    openly, in a fair and concensus building way, and without stabbing your teammates in the
    face along the way. \cite{Atwood}
    \end{citacao}



    \section{Application Fields}

    TODO.

    \url{https://github.com/r-lib/styler}
    \url{https://github.com/github/linguist}
    \url{https://forum.sublimetext.com/t/auto-align-symbolic/34375}
    \url{https://github.com/Thom1729/SmartIndent}
    \url{https://github.com/wbond/package_control_channel/issues/4310} Write a formatter utility

    With this ObjectBeautifier, we can edit XML files like YAML files, because
    the when we load the XML, we create a copy (the YAML version of the XML) on
    the `.object-beauty` folder which is edited by Sublime Text, and let the
    XML on its original place. Then when we save our YAML copy, we translate it
    to the XML original copy.
    \url{https://github.com/Microsoft/XmlNotepad}


    \citeonline{annotationAssistant},
    \citeonline{codePlagiarismDetection},
    \citeonline{softwarePortfolio},
    \citeonline{legacyAssets},
    \citeonline{massMaintenance},
    \citeonline{prettyPrinting},
    \citeonline{architectureFormatting},
    \citeonline{independentFramework},
    \citeonline{programIndentation},
    \citeonline{industrialApplication},
    \citeonline{toolsForProjectManagement},
    \citeonline{codeClassification},
    \citeonline{codeScanningPatterns},
    \citeonline{debuggingIntoExamples},
    \citeonline{programUnderstanding},
    \citeonline{documentingAndSharingKnowledge},
    \citeonline{autofoldingForSourceCode},
    \citeonline{learningSupportSystem},
    \citeonline{syntaxHighlightingInfluencing},
    \citeonline{improvingCodeReadability},
    \citeonline{howNovicesRead},
    \citeonline{theRoleOfMethodChains},
    \citeonline{codeComprehensionComparedToOO},
    \citeonline{enhancingLegacySoftwareSystemAnalysis},
    \citeonline{moldableCodeEditor},
    \citeonline{blindAndSightedProgrammers},
    \citeonline{pushdownAutomata},



    \section{Continuous Integration}

    TODO

    On a distributed version control system,
    continuous integration would be continuously running the system tests
    as long new code is integrated into main system
    from the distributed clients \cite{continuousIntegration}.

    \begin{enumerate}[leftmargin=*]
        \item \citeonline{trackingChanges},
        \item \footnote{\url{https://blog.codinghorror.com/check-in-early-check-in-often/}}
        \item \cite{aspectOriented}
    \end{enumerate}



    \section{Language Server Protocol}

    TODO.

    \url{https://www.youtube.com/watch?v=2GqpdfIAhz8}
    \url{http://langserver.org/}
    \url{https://github.com/Microsoft/language-server-protocol}
    \url{https://github.com/SublimeCodeIntel/SublimeCodeIntel}
    \url{https://code.visualstudio.com/blogs/2016/06/27/common-language-protocol}
    \url{https://www.eclipse.org/community/eclipse_newsletter/2017/may/article1.php}
    \url{https://github.com/Microsoft/language-server-protocol/wiki/Protocol-Implementations}


    % PARTE 2
    \advisor{}{\ifforcedinclude\else\part{\lang{Implementation}{Implementação}}\fi\label{segunda_parte}}

    % Capítulo com exemplos de comandos inseridos de arquivo externo
    

% Is it possible to keep my translation together with original text?
% https://tex.stackexchange.com/questions/5076/is-it-possible-to-keep-my-translation-together-with-original-text
\chapter{Uma Ferramenta de Formatação}
\label{software_implementation}

Neste capítulo,
será explicado o funcionamento e
implementação de uma nova ferramenta de formatação.
A proposta desta nova ferramenta é permitir que usuários possam entrar com a gramática de qualquer linguagem,
por meio de uma metagramática\footnote{
Em Ciências da Computação,
quando algo é prefixado com ``meta'',
isso significa que ele refere~=se sobre o seu tipo ou
categoria \cite{theUseOfMetaRules}.
Por exemplo,
``metadata'' são dados sobre os dados.
} para então formatar o código~=fonte da linguagem descrita pela gramática.


\section{Uma Gramática de Gramáticas}
\label{GrammarsGrammar}

Na \fullref{introducaoGramaticas},
foi explicado o que são gramáticas.
Mas,
como gramáticas podem ser expressadas?
Isso depende de como seu analisador foi implementado\advisor{.}{,
sendo assim,
um detalhe de implementação.%
} \advisor{Analisadores}{Usualmente,
analisadores} seguem uma notação comum como EBNF\footnote{
Do inglês,
\textit{Extended Backus–Naur Form} uma extensão do padrão BNF (\textit{Backus–Naur Form}).
}\cite{teachingEbnf,antlrBookTerrentParr},
\advisor{que diverge de acordo com detalhes de implementação.
}{%
que não difere muito de um analisador para outro,
exceto por detalhes de implementação específicos de cada analisador.
}

Para realizar a implementação da nova ferramenta de formatação de código~=fonte,
foi realizado a construção de uma nova gramática de gramáticas de uma nova linguagem chamada de ``ObjectBeauty'',
uma metalinguagem \cite{compilersCompilerMetaLanguage}.
Na \typeref{MyWorflowForLarkTraduzido},
é apresentado o fluxo de uso comum para um analisador.
Neste processo,
o desenvolvedor da linguagem escreve a gramática de especificação\advisor{}{
desta linguagem} que é entregue a algum analisador e
gera~=se um compilador para tal linguagem.
\begin{figure}[h]
\centering
\includegraphics[width=1.0\textwidth]{MyWorflowForLarkTraduzido.png}
\caption[Fluxo de uso comum de um analisador]{Fluxo de uso comum de um analisador -- Fonte Própria \cite{larkErrorRecovery}}
\label{MyWorflowForLarkTraduzido}
\end{figure}

\advisor{Este trabalho faz um uso diferente}{Já este trabalho faz um uso fora do comum}.
Como mostrado na \typeref{MyWorflowForLarkTraduzido2},
primeiro especifica~=se uma metalinguagem que será utilizada pelos usuários da nova ferramenta de Formatação de Código.
Para escrever esta nova metalinguagem,
utilizou~=se o Analisador Lark \cite{larkContextualLexer}.
Usualmente,
o Analisador Lark é utilizado somente como um gerador de compiladores (\typeref{MyWorflowForLarkTraduzido}),
entretanto,
neste contexto Lark é utilizado como um compilador de compiladores (\typeref{MyWorflowForLarkTraduzido2}).
Para este trabalho,
foi realizado um \textit{fork} \cite{overviewOfGitHubForks,mayTheForkBeWithYou,collaborationAmongGitHubUsers} do Analisador Lark,
renomeado o Analisador Lark para ``pushdown''\footnote{%
O código~=fonte do \textit{fork} pode ser encontrado em \url{https://github.com/evandrocoan/pushdownparser}.
}.
\begin{figure}[h]
\centering
\includegraphics[width=1.0\textwidth]{MyWorflowForLarkTraduzido2.png}
\caption[Uso feito pela nova ferramenta de Formatação de Código]{Uso feito pela nova ferramenta de Formatação de Código -- Fonte Própria \cite{larkErrorRecovery}}
\label{MyWorflowForLarkTraduzido2}
\end{figure}

Foi realizado um \textit{fork} do Analisador Lark para poder~=se realizar pequenas alterações que facilitam o entendimento do funcionamento interno da ferramenta como adição de logs e
alterações nos algoritmos de iteração nas árvores geradas pela ferramenta.
Por isso,
em alguns lugares do código~=fonte é encontrado o nome ``pushdown'' ao invés de ``lark''.
Já em outros,
continua~=se sendo chamado Lark de Lark para simplificar a retrocompatibilidade com a biblioteca original e
facilitar a realização da integração de novos updates vindos do repositório original do Analisador Lark para o \textit{fork} realizado.

Em vez de permitir com que o usuário final da aplicação opere diretamente com o analisador da \typeref{MyWorflowForLarkTraduzido},
foi criado uma nova metagramática (uma gramática de gramáticas) como mostrado nas \typeref{MyWorflowForLarkTraduzido2}.
Esta nova metagramática simplifica o processo de escrita de gramáticas ao criar uma nova especificação de gramáticas,
somente com os recursos necessários para se possa trabalhar com formatação de código~=fonte.
\advisor{Não}{A final, não}
é objetivo deste trabalho fazer a análise completa de programas,
pela sua sintaxe, semântica,
e gerar código~=binário executável.

Na \typeref{ParsersPublicAudienceTraduzido},
pode~=se encontrar uma relação entre o funcionamento das diversas partes da ferramenta de Formatação de Código e
a audiência alvo.
Basicamente existem três grupos distintos de usuários ou
audiência:
\begin{inparaenum}[1)]
\item quem escreve ou
desenvolve a ferramenta de Formatação de Código proposta por este trabalho e
define as regras da metalinguagem (especificada pela sua metagramática,
i.e.,
a gramática de gramáticas);
\item quem escreve ou
desenvolve gramáticas de linguagens para serem formatadas de acordo com as regras da metalinguagem e;
\item quem escreve ou
desenvolve programas de computador e
deseja realizar a formatação de seus códigos~=fonte.
\end{inparaenum}%
\begin{figure}[h]
\centering
\includegraphics[width=1.0\textwidth]{ParsersPublicAudienceTraduzido.png}
\caption[Relacionamentos entre os Diferentes Públicos deste Projeto]{Relacionamentos entre os Diferentes Públicos deste Projeto -- Fonte Própria \cite{larkErrorRecovery}}
\label{ParsersPublicAudienceTraduzido}
\end{figure}

Até este ponto,
já falou~=se de metagramática e
metalinguagem com a exceção dos metaprogramas \cite{tradeoffsInMetaprogramming}.
Nas \typeref{MyWorflowForLarkTraduzido2,ParsersPublicAudienceTraduzido},
por simplificação foram omitidos o relacionamento dos metaprogramas com a metagramática e
metalinguagem.
Metaprogramas fazem parte da entrada do metacompilador (\typeref{MetacompilerMetagrammarMetaprogram}) junto com a metagramática para gerar um novo compilador (ou Formatador de Código).
Neste trabalho,
os metaprogramas serão as gramáticas que serão utilizadas pelos formatadores de código~=fonte.

Os metaprogramas (ou gramáticas) são entradas diretas do metacompilador,
o Analisador Lark na \typeref{MyWorflowForLarkTraduzido2},
um Analisador LALR(1).
Na \typeref{ParsersPublicAudienceTraduzido},
não pode~=se ver diretamente que as gramáticas das linguagens serão os metaprogramas,
mas o quadro em azul mais a esquerda ligado por linhas pontilhadas explica que os erros léxicos e
sintáticos nas gramáticas de entrada serão mostrados pelo Analisador Lark.
Isso acontece por que as gramáticas (ou metaprogramas) são entradas diretamente no Analisador Lark.

Na \typeref{MetacompilerMetagrammarMetaprogram},
encontra~=se uma extensão da \typeref{MyWorflowForLarkTraduzido2},
e pode~=se ver claramente as relações entre Metagramáticas,
Metacompiladores e Metaprogramas. Por simplificação,
mostra~=se o nó ``Árvore de Sintaxe'' sem explicitamente falar sobre sua Análise Semântica e
propriamente a construção do Compilador (ou do Formatador de Código).
Vale lembrar que trata~=se de um Compilador de Compiladores,
e não um Compilador de Analisadores.
Por isso vemos que os Metaprogramas (ou gramáticas) são entradas diretas dos Metacompilador,
e não do Formatador de Código.
\begin{figure}[h]
\centering
\includegraphics[width=1.0\textwidth]{MetacompilerMetagrammarMetaprogram.png}
\caption[Relação entre Metagramáticas, Metacompiladores e Metaprogramas]{Relação entre Metagramáticas, Metacompiladores e Metaprogramas -- Fonte Própria \cite{larkErrorRecovery}}
\label{MetacompilerMetagrammarMetaprogram}
\end{figure}

Esta não é a primeira vez que uma metagramática com simplificações foi escrita.
Em trabalhos como \citeonline{rustSublimeTextSyntaxSyntec,sublimeTextSyntax,vsCodeSyntaxHighlighthing} foram realizados as mesmas simplificações aqui apresentadas.
Existem algumas diferenças técnicas da metagramática deste trabalho com as dos recém~=apresentados.
Como por exemplo,
a implementação da metagramática realizada ainda não suporta a classificação do mesmo trecho de código~=fonte por múltiplos tipos de escopo \cite{vsCodeSyntaxHighlighthing}.

Foi escolhida a criação de uma nova metagramática por que as implementações de metagramáticas já existentes como \citeonline{rustSublimeTextSyntaxSyntec,vsCodeSyntaxHighlighthing}:
\begin{enumerate}[1)]
\item Não utilizam explicitamente nenhum analisador,
realizando a programação das produções da gramática diretamente no código~=fonte (\fullref{gramaticasVersusLinguagens});
\item Não são capazes de reconhecer todas as características de todas as linguagens de programação (devido a optimizações para maior performance);
\item Não possuem sintaxe própria,
i.e.,
utilizam~=se de outras linguagens como YAML,
XML e
JSON para fazer a especificação da metagramática.
\end{enumerate}
Fazendo a especificação de uma nova metagramática,
é possível adaptar~=se a especificação da sintaxe das gramáticas de acordo as necessidades específicas sem ter que depender de características de outras linguagens como YAML,
XML ou
JSON.


\subsection{Escopos}

Na \typeref{TexMateScopes},
é mostrado na primeira linha o trecho de código~=fonte ``function f1 () \{'' e
nas demais linhas são apresentados as diversas classificações de escopos aplicados a cada um dos trechos do código~=fonte de amostra.
Por exemplo,
a palavra ``function'' possui simultaneamente os escopos
\begin{inparaenum}[1)]
\item ``source.js'';
\item ``meta.function.js'' e;
\item ``storage.type.function.js''.
\end{inparaenum}
\begin{figure}[h]
\centering
\includegraphics[width=1.0\textwidth]{TexMateScopes.png}
\caption[Exemplo de Classificação de Código~=Fonte com Múltiplos Escopos]{Exemplo de Classificação de Código~=Fonte com Múltiplos Escopos -- Fonte \citeonline{vsCodeSyntaxHighlighthing}}
\label{TexMateScopes}
\end{figure}

Os nomes utilizados na \typeref{TexMateScopes} podem ser qualquer texto que usuário especificador daquela gramática atribuiu.
Entretanto,
pode~=se perceber que o nome dos escopos recém apresentados parecem seguir um padrão.
Por conversão,
desenvolvedores de gramáticas para os editores de texto como \citeonline{sublimeTextSyntax,vsCodeSyntaxHighlighthing},
seguem uma conversão de nomes para que as utilizações dos escopos gerados pelas gramáticas sejam compatíveis entre si.

Fazendo o uso de uma conversão para nomes de escopos,
as gramáticas ficam compatíveis com um maior número de arquivos de temas (ou configurações de cores),
onde são especificados os nomes dos escopos serão utilizados para especificar as cores a serem utilizadas pelo editor de texto.
Para mais informações sobre a utilização de arquivos de temas em editores de texto veja \citeonline{sublimeTextScopeNaming,vsCodeSyntaxHighlighthing}.


\section{Metalinguagem}
\label{metalinguagemGrammar}

Como já explicado na seção anterior,
uma metagramática é gramática de gramáticas e
foi utilizado o Analisador Lark \cite{larkContextualLexer} como um metacompilador ou
compilador de compiladores.
Nesta seção será discutido como a metalinguagem (especificada pela metagramática) utilizada foi construída,
começando com o seu símbolo inicial.
No \typeref{simboloInicialDaMetagramatica},
defini~=se que o programa é constituído de três grandes áreas,
que devem acontecer uma em sequencia da outra:
\begin{enumerate}
\item A produção ``preamble\_statements'' define características globais da gramática como um nome,
e um escopo que será atribuído a toda gramática;
\item A produção ``language\_construct\_rules'' define qual será o símbolo inicial da gramática.
Em comparação com linguagens de programação como ``C'',
ele pode ser considerado similar ao método ``main'';
\item A produção ``miscellaneous\_language\_rules'' permite a definição de diversos contextos\footnote{
Contexto refere~=se a um bloco de operadores ou
conjunto de instruções como ``include'' e
``match''.
} com grupos de produções da gramática (\fullref{definicaoDeGramatica}),
que podem ser incluídos a partir do símbolo inicial da gramática definido no item ``language\_construct\_rules''.
\end{enumerate}%
\begin{code}
\caption{Simbolo Inicial da Metagramática ``ObjectBeauty''}
\label{simboloInicialDaMetagramatica}
\begin{minted}{antlr}
language_syntax: _NEWLINE? preamble_statements _NEWLINE?
                    language_construct_rules _NEWLINE?
                    ( miscellaneous_language_rules _NEWLINE? )*
                    _NEWLINE?

preamble_statements: ( (
                        target_language_name_statement
                        | master_scope_name_statement
                        | constant_definition
                    ) _NEWLINE )+

language_construct_rules: "contexts" ": " indentation_block
miscellaneous_language_rules: /[^:\n]+/ ": " indentation_block

target_language_name_statement: "name" ": " free_input_string
master_scope_name_statement: "scope" ": " free_input_string
\end{minted}
\end{code}

A sintaxe da metagramática utilizada no \typeref{simboloInicialDaMetagramatica} é uma adaptação especial do padrão EBNF que o Analisador Lark faz.
Símbolos que possuem todas as letras em ``MAIÚSCULA'' definem os símbolos terminais da gramática.
Enquanto símbolos escrito todos em letras ``minúsculas'' definem os símbolos não~=terminais da gramática.
Para mais informações sobre a sintaxe de entrada de gramáticas do Analisador Lark,
consulte sua documentação \cite{larkGrammarReference,larkStyleCheat}.

Entre os \typeref{exemploDeGramaticaPawn1,exemploDeGramaticaPawn2,exemploDeGramaticaPawn3,exemploDeGramaticaPawn4},
encontra~=se pequenos exemplos de gramáticas escritas na metalinguagem ``ObjectBeauty'' brevemente apresentada.
No \typeref{exemploDeGramaticaPawn1},
encontra~=se a definição do símbolo inicial da gramática da linguagem sendo descrita (pela metagramática) e
pode~=se ver a metalinguagem sendo utilizada para definir uma linguagem chamada de ``Abstract Machine Language''.
Por padrão,
toda gramática ``ObjectBeauty'' precisa ter um contexto inicial ou
símbolo inicial chamado de ``contexts''.

O \typeref{exemploDeGramaticaPawn1} faz uso dos operadores ``include'' e
``match''.
O operador ``include'' serve incluir partes de outras gramáticas ou
mesmo gramáticas inteiras no contexto da gramática atual.
Entretanto,
a implementação de ``include'' realizada neste trabalho somente consegue realizar includes de contextos definidos no mesmo arquivo.

No exemplo do \typeref{exemploDeGramaticaPawn1},
o operador ``include'' está incluindo contextos da gramática atual que serão definidas mais tarde neste mesmo arquivo.
Já o operador ``match'' utilizado no final serve para realizar propriamente o reconhecimento do programa de entrada e
atribuir a ele o escopo ``constant.boolean.language.pawn''.

Mais tarde,
as informações de escopo atribuídas por operadores como ``match'' e
``captures'' serão utilizadas pelo formatador de código~=fonte.
Com estas informações,
o Formatador de Código será capaz de realizar as operações de formatação somente sobre os trechos de código~=fonte que o usuário definir.
Realizando assim,
a formatação seletiva de código~=fonte,
contrário da formatação total de código~=fonte como acontece nos demais trabalhos (\fullref{performanceDoFormator}).
\begin{code}
\caption{Exemplo de Gramática -- Símbolo Inicial}
\label{exemploDeGramaticaPawn1}
\begin{minted}{yaml}
name: Abstract Machine Language
scope: source.sma

contexts: {
    include: parens
    include: numbers
    include: check_brackets

    match: (true|false) {
        scope: constant.boolean.language.pawn
    }
}
\end{minted}
\end{code}

No \typeref{exemploDeGramaticaPawn2},
é introduzido o uso dos operadores ``push'',
``meta\_scope'' e
``pop''.
O operadores ``push'' e
``pop'' são responsáveis por manter uma pilha de contextos que permite aplicar um mesmo escopo por várias linhas utilizado o operador ``meta\_scope''.
A diferença entre o operador ``scope'' e
``meta\_scope'' é que o operador ``scope'' atribuí o escopo diretamente ao texto reconhecido pelo um operador ``match''.
Já o operador ``meta\_scope'' permite aplicar o escopo a todo o texto desde o primeiro até o último ``match'',
que desempilha com o operador ``pop'',
o contexto empilhado inicialmente com um ``push''.
\begin{code}
\caption{Exemplo de Gramática -- Contextos}
\label{exemploDeGramaticaPawn2}
\begin{minted}{yaml}
parens: {
    match: \( {
        scope: parens.begin.pawn
        push: {
            meta_scope: meta.group.pawn
            match: \) {
                scope: parens.end.pawn
                pop: true
            }
            include: numbers
        }
    }
}
\end{minted}
\end{code}

No \typeref{exemploDeGramaticaPawn3},
é introduzido o uso do operador ``captures''.
O operador ``captures'' atribuí simultaneamente diversos escopos com uma única expressão regular.
Cada um dos números listados equivalem a um dos grupos de captura da expressão regular utilizada no operador ``captures''.
O operador ``scope'' pode ser considerado um caso especial do operador ``captures'' quando utiliza~=se o Grupo de Captura 0.

Motores de expressões regulares geralmente suportam um recurso chamado de Grupos de Captura \cite{expressionGrammarsWithRegexLikeCaptures}.
Por exemplo,
a expressão regular ``foo(bar)zoo(car)'' possuí 3 grupos de captura quando analisado o texto de entrada ``foobarzoocar'':
\begin{inparaenum}[1)]\setcounter{enumi}{-1}
\item foobarzoocar;
\item bar;
\item car;
\end{inparaenum}%
onde o grupo de captura 0 refere~=se a toda a expressão regular encontrada.
Portanto,
ao invés de utilizar~=se o operador ``scope:
constant.numeric.pawn'',
poderia~=se utilizar equivalentemente o operador ``captures:
0.
constant.numeric.pawn''.
\begin{code}
\caption{Exemplo de Gramática -- Grupos de Captura}
\label{exemploDeGramaticaPawn3}
\begin{minted}{yaml}
numbers: {
    match: '(\d+)(\.\{2\})(\d+)' {
        captures: {
            0: constant.numeric.pawn
            1: constant.numeric.int.pawn
            2: keyword.operator.switch-range.pawn
            3: constant.numeric.int.pawn
        }
    include: numeric
}
\end{minted}
\end{code}

No \typeref{exemploDeGramaticaPawn4},
é mostrado mais alguns exemplos de uso do operador ``match'' classificando diversos tipos de numéricos (da linguagem sendo descrita pela gramática).
É importante notar que a ordem no qual os operadores como ``match'' aparecem é importante.
Ao realizar o reconhecido o programa de entrada utilizando esta gramática,
a Árvore de Sintaxe Abstrata\footnote{%
Do inglês (AST), Abstract Syntax Tree.
} \cite{ahoCompilerDragonBook} será interpretada diversas vezes,
partindo no símbolo inicial até chegar ao último símbolo da gramática.

O processo de interpretação irá reiniciar indefinidamente até que nenhum texto seja mais consumido por nenhum dos operadores da gramática.
Assim,
uma vez que um trecho de código~=fonte já foi classificado,
ele será ignorado quando os próximos operadores forem aplicados,
evitando assim que o programa execute infinitamente.
\begin{code}
\caption{Exemplo de Gramática -- Tipos numéricos}
\label{exemploDeGramaticaPawn4}
\begin{minted}{yaml}
numeric: {
    match: ([-]?0x[\da-f]+) {
        scope: constant.numeric.hex.pawn
    }
    match: \b(\d+\.\d+)\b {
        scope: constant.numeric.float.pawn
    }
    match: \b(\d+)\b {
        scope: constant.numeric.int.pawn
    }
}
\end{minted}
\end{code}

Por fim,
no \typeref{exemploDeGramaticaPawn5} é apresentado um exemplo não relacionado com formatação de código~=fonte.
A construção utilizada é comum para gramáticas que serão utilizadas para realizar a aplicação de cores em editores de texto \cite{vsCodeSyntaxHighlighthing}.
Com ela é possível colorir o código~=fonte,
destacando~=o como inválido no editor de texto,
uma vez que uma inconsistência sintática foi encontrada na linguagem sendo analisada.

Construções como a do \typeref{exemploDeGramaticaPawn5} funcionam usualmente quando elas são a última regra da gramática.
Uma vez que todas as regras que consomem o programa de entrada e
o classifica em escopos terminam seu trabalho,
não deveria existir mais nenhum texto ser reconhecido.
Caso exista,
ou a gramática não estava preparada para reconhecer todo o programa de entrada,
ou estes trechos de código~=fonte são frutos de algum erro no programa de entrada.
\begin{code}
\caption{Exemplo de Gramática -- Reconhecimento de Erros}
\label{exemploDeGramaticaPawn5}
\begin{minted}{yaml}
check_brackets: {
    match: \) {
        scope: invalid.illegal.stray-bracket-end
    }
}
\end{minted}
\end{code}


\section{Analisador Semântico}

Depois que um metaprograma da metalinguagem apresentada na seção anterior é reconhecido pelo Analisador Lark,
o Analisador Lark entrega a Árvore de Sintaxe da gramática da linguagem sendo descrita pelo metaprograma (\typeref{MetacompilerMetagrammarMetaprogram}).
Todas as verificações de corretude da sintaxe da gramática são verificadas pelo Analisador Lark,
com base na metagramática da metalinguagem.
Portanto,
somente resta ser implementado o Analisador Semântico para verificar se a linguagem descrita respeita as regras semânticas da metalinguagem explicadas na seção anterior,
\nameref{metalinguagemGrammar}.

O Diagrama de Classes é apresentado na \typeref{CodeFormatterClassDiagram}.
Nele,
o Analisador Semântico que deriva de ``Transformer'' recebe como entrada a Árvore de Sintaxe do programa de entrada,
e uma vez que o Analisador Semântico termina seu trabalho,
ele devolve Árvore de Sintaxe Abstrata completa.
Então,
utilizando a Árvore de Sintaxe Abstrata,
``Backend'' que deriva de ``Interpreter'',
realiza a formatação de código~=fonte recebendo um programa de entrada e
as configurações do formatador (\typeref{ParsersPublicAudienceTraduzido,MetacompilerMetagrammarMetaprogram}).
\begin{figure}[h]
\centering
\includegraphics[width=1.0\textwidth]{CodeFormatterClassDiagram.png}
\caption[Diagrama das Principais Classes]{Diagrama das Principais Classes -- Fonte Própria}
\label{CodeFormatterClassDiagram}
\end{figure}

No \typeref{semanticAnalizerConstructor},
pode~=se ver o construtor do Analisador Semântico.
Pode~=se notar que seu construtor não recebe como parãmetro a Árvore de Sintaxe.
Entretanto,
a ela não é passada pelo construtor mas sim por um método chamado ``transform(tree)''.
Esta é uma característica do Analisador Lark utilizado.
A função ``transform(tree)'' do Analisador Lark simplesmente inicia a análise do programa visitando todos os nós da Árvore de Sintaxe,
partindo das folhas até chegar na raíz (\fullref{compiladoresEtradutores}).
\begin{code}
\caption{Construtor do Analisador Semântico}
\label{semanticAnalizerConstructor}
\begin{minted}{python}
class TreeTransformer(pushdown.Transformer):
    """
        Transforms the Derivation Tree nodes into meaningful string representations,
        allowing simple recursive parsing and conversion to Abstract Syntax Tree.
    """

    def __init__(self):
        ## Saves all the semantic errors detected so far
        self.errors = []

        ## Saves all warnings noted so far
        self.warnings = []

        ## Whether the mandatory/obligatory global scope name statement was declared
        self.is_master_scope_name_set = False

        ## Whether the mandatory/obligatory global language name statement was declared
        self.is_target_language_name_set = False

        ## Can only be one scope called `contexts`
        self.has_called_language_construct_rules = False

        ## Pending constants declarations
        self.constant_usages = {}

        ## Pending constants usages
        self.constant_definitions = {}

        ## A list of miscellaneous_language_rules include contexts defined for duplication checking
        self.defined_includes = {}

        ## A list of required includes to check for missing includes
        self.required_includes = {}

        ## A list of regular expressions used on match statements,
        ## for validation when the constants definitions are completely know
        self.pending_match_statements = []

        ## Responsible for calculating all open and close commands scoping
        self.open_blocks = {}
        self.indentation_level = 0
        self.indentation_blocks = []
\end{minted}
\end{code}

A visita dos nós da Árvore de Sintaxe pela função ``transform(tree)'' acontece simplesmente chamando os métodos que a classe ``TreeTransformer'' (Definido no \typeref{semanticAnalizerConstructor}) define e
que possuem os mesmos nomes que os símbolos não~=terminais definidos pela metagramática.
Então,
cada nó ou
função deve retornar qual será o novo nó que irá o substituir na Árvore de Sintaxe.
Assim,
no final do processo,
um a um,
cada nó da Árvore de Sintaxe será convertido para um nó da Árvore de Sintaxe Abstrata.

Um jeito fácil de excluir um nó da Árvore de Sintaxe é simplesmente definir uma função com o nome de seu não~=terminal que returna ``null''.
Por exemplo,
o trecho da metagramática apresentado no \typeref{simboloInicialDaMetagramatica} possui alguns símbolos não~=terminais como ``preamble\_statements'' e
``language\_construct\_rules''. Então,
para estes dados símbolos,
serão chamados os métodos da classe ``TreeTransformer'' que possuem os nomes ``preamble\_statements'' e
``language\_construct\_rules''.

Caso não existam os métodos ``preamble\_statements'' e
``language\_construct\_rules'' na classe ``TreeTransformer'',
os nós ``preamble\_statements'' e
``language\_construct\_rules'' da Árvore de Sintaxe não serão visitados e
``poderão'' ser excluídos da Análise Semântica (mas não da Árvore de Sintaxe Abstrata).
Nós não analisados pela classe ``TreeTransformer'' não serão excluídos da Árvore de Sintaxe Abstrata,
eles serão mantidos intactos,
a não ser que algum outro nó os altere diretamente na Árvore de Sintaxe.

Mesmo que não existam os métodos ``preamble\_statements'' e
``language\_construct\_rules'' definidos na classe ``TreeTransformer'',
eles também podem ser visitados diretamente a partir de algum nó pai ou
até algum de seus filhos.
Inclusive,
esta foi uma das alterações realizadas no \textit{fork} ``pushdown'' do Analisador Lark.
Por padrão,
ao iterar pela árvore,
o Analisador Lark somente passa como parâmetro da função o nó correspondente ao método atualmente sendo chamado e
uma lista de nós filhos.
Entretanto,
``pushdown'' também passa um terceiro parâmetro que é uma referência para o nó raíz da árvore e
mantém a variável ``parent'',
acessível como um atributo da classe ``TreeTransformer''.


\section{Formatador de Código}

O Formatador de Código não é composto somente por uma parte única e
altamente acoplada.
\advisor{Um}{Mas pelo contrário,
um} conjunto de partes altamente coesas e
completamente independentes onde cada uma dessas partes recebe a Árvore de Sintaxe Abstrata do Analisador Semântico,
para então realizar a formatação do programa de entrada junto com as configurações que esta parte aceita.


\subsection{Performance}
\label{performanceDoFormator}

Continuamente chamar diversos algoritmos independentes possui uma perda de performance em comparação com os formatadores de código~=fonte apresentados no \fullref{source_code_beautifiers}.
Estes formatadores tem como principal características realizar a formatação de código~=fonte em uma única passada,
i.e.,
a Árvore de Sintaxe de programa de entrada é completamente reconstruída,
para então ser serializada novamente em texto de acordo com as configurações de formatador.

Ao leitor mais desatento,
pode parecer que então não existe muita diferença entre a ferramenta de formatação proposta neste trabalho para as já existentes.
Entretanto,
este trabalho permite que o usuário entre com a gramática da linguagem a ser formatada,
diferente de outros trabalhos onde a gramática já é incluída ao código~=fonte do formatador.
Formadores de código~=fonte como apresentados na \fullref{source_code_beautifiers} são construídos programando~=se as produções da gramática diretamente no código~=fonte do formatador (Descendentes Recursivos,
reveja a \fullref{gramaticasVersusLinguagens}).

Ao não permitir~=se que o usuário possa entrar com a gramática do programa,
restringe~=se o formatador a somente funcionar com as gramáticas que foram programadas dentro do seu código~=fonte.
Uma vez que o usuário da ferramenta precisa programar o formatador de código para ter suporte a sua linguagem,
isso dificulta a adição do suporte de novas linguagens ao formatador,
pois precisa~=se programar as suas gramáticas diretamente dentro do código~=fonte do formatador.


\subsection{Formatação}

No \typeref{construtorDoFormatador},
pode ser encontrado o construtor do formatador de código~=fonte.
Comparando~=o com o construtor do Analisador Semântico \typeref{semanticAnalizerConstructor},
pode~=se notar algumas diferenças.
Em ambos os casos,
o processo todo se completará ao percorrer toda a árvore.
No caso do Analisador Semântico,
a Árvore de Sintaxe,
e no caso do Formatador de Código,
a Árvore de Sintaxe Abstrata.

Diferentemente do Analisador Semântico,
o Formatador de Código faz herança do tipo ``Interpreter'' em vez de ``Transformer'' (\typeref{CodeFormatterClassDiagram}).
A diferença é simples,
``Interpreter'' visita a árvore partindo das folhas até chegar na raíz visitando todos os nós filhos (\fullref{compiladoresEtradutores}),
já ``Transformer'' visita a árvore partindo da raíz até chegar nos nós pais,
i.e.,
ele não visita os nós filhos automaticamente como ``Transformer'' faz.

``Interpreter'' também recebe diretamente no construtor qual será a árvore que ele irá iterar sobre.
Nos demais aspectos,
``Interpreter'' funciona igual ao ``Transformer'',
exceto no ponto onde ``Transformer'' cria uma nova árvore no final do processo,
enquanto ``Interpreter'' não cria árvore alguma.
O parâmetro chamado ``program'',
que o construtor de ``Transformer'' recebe,
é o programa a ser formatado pelo formatador.
No final processo,
``Interpreter'' terá em sua variável ``self.program'' o novo programa completamente formatado.
\begin{code}
\caption{Construtor do Formatador}
\label{construtorDoFormatador}
\begin{minted}{python}
class Backend(pushdown.Interpreter):

    def __init__(self, formatter, tree, program, settings):
        super().__init__()
        self.tree = tree
        self.program = formatter( program, settings )

        ## A list of lists, where each list saves all the matches performed by
        ## the last match_statement on scope_name_statement
        self.last_match_stack = []

        ## This is set to False every push statement, and set to True, after
        ## every match statement. This way we can know whether there is a match
        ## statement after a push statement.
        self.is_there_push_after_match = False
        self.is_there_scope_after_match = False

        self.cached_includes = {}
        self.cache_includes( tree )

        self.visit( tree )
        log( 4, "Tree: \n%s", tree.pretty( debug=0 ) )
\end{minted}
\end{code}

Enquanto ``Interpreter'' é responsável por somente ``passear'' pela Árvore de Sintaxe Abstrata,
a classe ``AbstractFormatter'' \typeref{construtorDeParsedProgram} é responsável por realmente fazer a formatação de código~=fonte de acordo com a instruções vindas da Árvore de Sintaxe Abstrata.
No final do processo,
``AbstractFormatter'' terá na variável ``new\_program'' todos os pedaços do programa formatado.
Uma vez que ``Interpreter'' termina de construir todos os pedaços de código~=fonte formatado,
a função ``get\_new\_program'' irá unir dos os pedaços e
salvá~=los na variável ``cached\_new\_program'',
para evitar ter que recalcular o novo programa toda vez que pedir~=se a sua nova versão.
\begin{code}
\caption{Construtor de ``AbstractFormatter''}
\label{construtorDeParsedProgram}
\begin{minted}{python}
class AbstractFormatter(object):
    """
        Represents a program as chunks of data as (text_chunk_start_position,
        text_chunk).
    """

    def __init__(self, program, settings):
        super().__init__()
        self.initial_size = len( program )
        self.program = program
        self.settings = OrderedDict( sorted( settings.items(), key=lambda item: len( str( item ) ) ) )

        self.new_program = []
        self.cached_new_program = []
        log( 4, "program %s: `%s`", len( str( self.program ) ), self.program )
\end{minted}
\end{code}

A linha ``sorted'' realiza a ordenação do configurações sem nenhuma necessidade prática neste caso.
Ela garante que a ordem no qual as configurações irão ser processadas seja sempre a mesma.
No caso da Adição de Cores,
``sorted'' é uma função obrigatória para garantir que os nomes das cores da configurações sejam atribuídas de acordo com a ordem de funcionamento do tema \cite{vsCodeSyntaxHighlighthing,sublimeTextScopeNaming}.

A linha ``len( program )'' não possui nenhuma influência nos resultados do programa,
seja para formatação ou
adição de cores.
Entretanto,
ela é constantemente verificado durantes as operações que realizam o consumo do programa de entrada pelas regras da metagramática para garantir que o tamanho programa original não seja perdido.
É necessário uma sincronia entre os índices do programa original com o programa formatado para que ambos possam ser sincronizados no final do processo.
Assim,
uma que os algoritmos de adição de cores ou
formatação estivem propriamente testados,
não seria mais necessário manter o uso da variável ``len( program )''.

Durante o processo de consumo,
o programa de entrada é modificado e
os caracteres que foram consumidos são substituídos pelo caractere ``§''.
A existência do caractere de marcação de consumo ``§'' é uma fraqueza do algoritmo implementado,
pois pode levar programas que contenham este caractere ao erro.
Ele também trás uma ineficiência no consumo de novos caracteres.
Os caracteres que já foram substituídos por ``§'',
serão reanalisados continuamente até o final do análise do programa de entrada.
A utilização do caractere ``§'' foi realizada para simplificar a implementação do algoritmo de consumo.

Com o caractere ``§'' colocado no lugar dos caracteres que já foram consumidos,
é possível reconstruir o programa de entrada no final da análise simplesmente ordenando todos pedaços que de programa formatado que estão armazenada na variável ``new\_program''.
Como o tamanho do programa original não fui modificado,
também é possível facilmente integrar no programa formatado,
todas as partes do programa original que não foram formatados,
no caso do formatador,
ou coloridas do caso da adição de cores.


\subsection{Exemplo}

A formatação de código~=fonte recebe como entrada o programa em texto simples,
e como resultado,
retorna uma página HTML \cite{parallelParserForHTML}.
Com isso,
sendo possível observar com mais facilidade o código~=fonte original e
formatado,
e as metainformações atribuídas pela gramática da linguagem como atributos das tags HTML.
No \typeref{exemploDeFormatacaoDeCodigo},
encontra~=se um HTML gerado com as metainformações agregadas ao programa não~=formatado ``if(something) bar'',
onde ``if( \ something \ ) bar'' é o resultado da formatação de código~=fonte.
\begin{quadro}[h]
\caption{Exemplo de Formatação de Código}
\label{exemploDeFormatacaoDeCodigo}
\begin{bluebox}
\begin{code}
\label{exemploDeHTMLGerado}
\caption{Exemplo de HTML Gerado pelo Formatador de Código}
\begin{minted}[xleftmargin=2em]{html}
<body style="white-space: pre; font-family: monospace;">
    <span setting="unformatted" grammar_scope="if.statement.definition" setting_scope="" original_program="if(">if(</span>
    <span setting="1" grammar_scope="if.statement.body" setting_scope="if.statement.body" original_program="something">  something  </span>
    <span setting="unformatted" grammar_scope="if.statement.definition" setting_scope="" original_program=")">)</span>
    <span grammar_scope="none" setting_scope="none"> bar</span>
</body>
\end{minted}
\end{code}

\begin{code}
\label{exemploDeGramaticaUtilizada}
\caption{Exemplo de Gramática Utilizada pelo Formatador de Código}
\begin{minted}[xleftmargin=2em]{yaml}
scope: source.sma
name: Abstract Machine Language
contexts: {
    match: if\( {
        scope: if.statement.definition
        push: {
            meta_scope: if.statement.body
            match: \) {
                scope: if.statement.definition
                pop: true
            }
        }
    }
}
\end{minted}
\end{code}

\begin{code}
\label{exemploDeConfiguracaoUtilizada}
\caption{Exemplo de Configuração Utilizada pelo Formatador de Código}
\begin{minted}[xleftmargin=2em]{python}
{
    "if.statement.body" : 2,
}
\end{minted}
\end{code}

\begin{code}
\label{exemploDeFormatadorDeCodigo}
\caption{Exemplo de Formatador de Código}
\begin{minted}[xleftmargin=2em]{python}
class SingleSpaceFormatter(AbstractFormatter):

    def format_text(self, matched_text, matched_setting):
        matched_text = matched_text.strip( " " )

        if matched_setting:
            return " " * matched_setting + matched_text + " " * matched_setting

        else:
            return matched_text
\end{minted}
\end{code}
\end{bluebox}
\end{quadro}

Como pode ser percebido,
a gramática de entrada no \typeref{exemploDeFormatacaoDeCodigo} não consome a palavra ``bar'' do programa de entrada ``if(something) bar''.
Esta é uma característica importante dos formatadores de código deste trabalho.
Todo texto que não é consumido,
ou pela gramática de entrada,
ou pelo formatador de código ou
adição de cores,
será mantido intacto no final do processo de formatação ou
adição de cores.
Assim,
pode~=se ter o formatador de código já em funcionamento com gramática mais simples possível,
ou que já atenda as características mínimas que deseja~=se formatar ou
adicionar de cores.


    % Finaliza a parte no bookmark do PDF para que se inicie o bookmark na raiz
    % e adiciona espaço de parte no Sumário
    \phantompart

    % Conclusão (outro exemplo de capítulo sem numeração e presente no sumário)
    

\chapter[]{\lang{Conclusion}{Conclusão}}

    The difference from this proposal to remaining formatting tools,
    is the tradeoff between end\hyp{}users and developers responsibilities.
    Most tools rarely expose to end\hyp{}users their language syntax specification,
    in contrast,
    this proposal completely exposes the language to the end\hyp{}user as simple plain\hyp{}text,
    not requiring the tool to know any language syntax neither semantics.
    Moreover,
    with no syntax knowledge required,
    the tool be can used with any languages their user wishes to.

\begin{enumerate}[leftmargin=*]
    \item
        There are many different tools, sometimes paid, and difficult to
        complete. \cite{universalCodeFormatter}
    \item
        Many programming languages exist, so always having Beautifier
        software for each of them is very laborious
        \cite{universalCodeFormatter}. But the approach to a Universal
        Beautifier proposed in this work, would allow easily new languages to be
        added, being completely different from previous ones, or alike. And in
        case of similarities between them, it is enough to reuse their
        configuration structures already implemented.
    \item
        Looking for a Beautifier for each one of them because programmers
        currently work daily with several of these languages, and they are not
        similar. So you need to configure several beautifiers to do the
        formatting. This is a problem because only a few beautifiers are more
        complete, and every time you need to make a change in the formatting
        style, you must manually propagate the same change over several
        different program configuration files, which is bad because it takes the
        user a lot of time to learn how to handle many different types of
        settings. \cite{universalIndentGUI}
    \item
        In the case of ideal Beautifier, a change in your styling is
        propagated to all languages. And if you want to leave some language out
        of it, you just need to remove it from the list on which the
        configuration block applies to, and `a)' leave it out so no change is
        applied to. Or `b)' create a new block including only the block within
        the desired settings.
\end{enumerate}



    % ELEMENTOS PÓS-TEXTUAIS
    \postextual
    \setlength\beforechapskip{0pt}
    \setlength\midchapskip{15pt}
    \setlength\afterchapskip{15pt}

    % Referências bibliográficas
    \begingroup
        % https://tex.stackexchange.com/questions/163559/how-to-modify-line-spacing-per-entry-of-bibliography
        % https://tex.stackexchange.com/questions/19105/how-can-i-put-more-space-between-bibliography-entries-biblatex
        \setlength\bibitemsep{\baselineskip}
        \advisor{}{\linespread{1.18}\selectfont}

        % https://tex.stackexchange.com/questions/17128/using-bibtex-to-make-a-list-of-references-without-having-citations-in-the-body
        % \nocite{*}
        \printbibliography[title=\lang{REFERENCES}{REFERÊNCIAS}]
    \endgroup

    % Glossário, consulte o manual da classe abntex2 para orientações sobre o glossário.
    % \ifforcedinclude\else\glossary\fi

    % Inicia os apêndices
    \begin{apendicesenv}
        % Imprime uma página indicando o início dos apêndices
        \advisor{}{\ifforcedinclude\else\partapendices\fi}
        \setlength\beforechapskip{50pt}
        \setlength\midchapskip{20pt}
        \setlength\afterchapskip{20pt}

        

% Is it possible to keep my translation together with original text?
% https://tex.stackexchange.com/questions/5076/is-it-possible-to-keep-my-translation-together-with-original-text
\chapter{Manual do Formatador de Código}
\label{manualDoFormatadorDeCodigo}

Não foi criado nenhuma interface gráfica ou
de linha de comando que faça a entrada do programa a ser formatado e
das configurações do formatador.
Para realizar os testes de funcionamento do formatador,
foram realizados testes de unidade automatizados (\typeref{unitTestsPy}).
Existem duas implementações que utilizam a metalinguagem (\typeref{grammarsGrammarPy}).
Uma utiliza a metalinguagem para adicionar cores (\typeref{codeHighlighterPy}),
como feito em editores de texto e
a outra realiza a formatação de código~=fonte (\typeref{codeFormatterPy}).

Nos \typeref{mainHighlighterPy,mainFormatterPy},
encontra~=se um exemplo simples de programa que pode ser construído para executar o Formatador de Código e
a Adição de Cores.
Sua construção é a mesma utilizada para a criação dos testes de unidade (\typeref{unitTestsPy}).
Tanto o \typeref{mainFormatterPy} quanto do o \typeref{mainHighlighterPy} geram como resultado um arquivo HTML,
contendo como conteúdo o resultado de seu trabalho.

Para realizar~=se a execução de qualquer arquivo deste projeto,
é necessário ter um interpretador ``Python 3.6'' instalado,
junto com as bibliotecas ``pip3'',
``debug\_tools'', ``dominate'' e
``pushdown''.
Em uma instalação tradicional ``Ubuntu'',
este pacotes podem ser instalados com os comandos:
\begin{enumerate}[1)]
\item \mintinline{shell}{sudo apt-get install python3 python3-pip}
\item \mintinline{shell}{pip3 install -r requirements.txt} (\typeref{requirementsTxt})
\item \mintinline{shell}{python3 main_formatter.py}
\item \mintinline{shell}{python3 main_highlighter.py}
\end{enumerate}
\begin{code}
\caption{Arquivo ``source/requirements.txt''}
\label{requirementsTxt}
\inputminted[fontsize=\small,fontfamily=zi4,linenos=true,numberblanklines=true,breaklines=true]{python3}{../source/requirements.txt}
\end{code}
\begin{code}
\caption{Arquivo ``source/main\_formatter.py''}
\label{mainHighlighterPy}
\inputminted[fontsize=\small,fontfamily=zi4,linenos=true,numberblanklines=true,breaklines=true]{python3}{../source/main_formatter.py}
\end{code}
\begin{code}
\caption{Arquivo ``source/main\_highlighter.py''}
\label{mainFormatterPy}
\inputminted[fontsize=\small,fontfamily=zi4,linenos=true,numberblanklines=true,breaklines=true]{python3}{../source/main_formatter.py}
\end{code}
\begin{code}
\caption{Arquivo ``source/utilities.py''}
\label{utilitiesPy}
\inputminted[fontsize=\small,fontfamily=zi4,linenos=true,numberblanklines=true,breaklines=true]{python3}{../source/utilities.py}
\end{code}


\chapter{Código dos Testes de Unidade}

\begin{code}
\caption{Arquivo ``source/unit\_tests.py''}
\label{unitTestsPy}
\inputminted[fontsize=\small,fontfamily=zi4,linenos=true,numberblanklines=true,breaklines=true]{python3}{../source/unit_tests.py}
\end{code}


\chapter{Código do Analisador Semântico}

\begin{code}
\caption{Arquivo ``source/semantic\_analyzer.py''}
\label{semanticAnalyzerPy}
\inputminted[fontsize=\small,fontfamily=zi4,linenos=true,numberblanklines=true,breaklines=true]{python3}{../source/semantic_analyzer.py}
\end{code}


\chapter{Código de Adição de Cores}

\begin{code}
\caption{Arquivo ``source/code\_highlighter.py''}
\label{codeHighlighterPy}
\inputminted[fontsize=\small,fontfamily=zi4,linenos=true,numberblanklines=true,breaklines=true]{python3}{../source/code_highlighter.py}
\end{code}


\chapter{Código do Formatador}

\begin{code}
\caption{Arquivo ``source/code\_formatter.py''}
\label{codeFormatterPy}
\inputminted[fontsize=\small,fontfamily=zi4,linenos=true,numberblanklines=true,breaklines=true]{python3}{../source/code_formatter.py}
\end{code}


\chapter{Código da Metagramática}

\begin{code}
\caption{Arquivo ``source/grammars\_grammar.pushdown''}
\label{grammarsGrammarPy}
\inputminted[fontsize=\small,fontfamily=zi4,linenos=true,numberblanklines=true,breaklines=true]{antlr}{../source/grammars_grammar.pushdown}
\end{code}

    \end{apendicesenv}

    % % Inicia os anexos
    % \begin{anexosenv}
    %     % Imprime uma página indicando o início dos anexos
    %     \advisor{}{\ifforcedinclude\else\partanexos\fi}
    %     \setlength\beforechapskip{50pt}
    %     \setlength\midchapskip{20pt}
    %     \setlength\afterchapskip{20pt}

    %     


%
% How to fix the Underfull \vbox badness has occurred while \output is active on my memoir chapter style?
% https://tex.stackexchange.com/questions/387881/how-to-fix-the-underfull-vbox-badness-has-occurred-while-output-is-active-on-m
%

% ---

\chooselang
{\chapter[Appendix A]{Since this page is not being completely filled, it is generating the bottom bottom of the page}}
{\chapter[Apêndice A]{Como esta página não está sendo completamente preenchida, ele está gerando a caixa inferior inferior da página}}
% ---


% Multiple-language document - babel - selectlanguage vs begin/end{otherlanguage}
% https://tex.stackexchange.com/questions/36526/multiple-language-document-babel-selectlanguage-vs-begin-endotherlanguage
\begin{otherlanguage*}{english}

\showfont

1. How to display the font size in use in the final output,
2. How to display the font size in use in the final output,
3. How to display the font size in use in the final output,
4. How to display the font size in use in the final output,
5. How to display the font size in use in the final output,
6. How to display the font size in use in the final output,
7. How to display the font size in use in the final output,
8. How to display the font size in use in the final output,
9. How to display the font size in use in the final output,


% As this page is not being completely filled, it is generating the page bottom bad box.
% Fix Underfull \vbox (badness 10000) has occurred while \output is active
%
% \flushbottom vs \raggedbottom
% https://tex.stackexchange.com/questions/65355/flushbottom-vs-raggedbottom
\newpage



\section[Some encoding tests]{\showfont}

1. How to display the font size in use in the final output,
2. How to display the font size in use in the final output,
3. How to display the font size in use in the final output,
4. How to display the font size in use in the final output,
5. How to display the font size in use in the final output,
6. How to display the font size in use in the final output,

7. How to display the font size in use in the final output,
8. How to display the font size in use in the final output,
9. How to display the font size in use in the final output,
10. How to display the font size in use in the final output,
11. How to display the font size in use in the final output,
12. How to display the font size in use in the final output,

\subsection{\showfont}

1. How to display the font size in use in the final output,
2. How to display the font size in use in the final output,
3. How to display the font size in use in the final output,
4. How to display the font size in use in the final output,
5. How to display the font size in use in the final output,
6. How to display the font size in use in the final output,

7. How to display the font size in use in the final output,
8. How to display the font size in use in the final output,
9. How to display the font size in use in the final output,
10. How to display the font size in use in the final output,
11. How to display the font size in use in the final output,
12. How to display the font size in use in the final output,

\subsubsection{\showfont}

1. How to display the font size in use in the final output,
2. How to display the font size in use in the final output,
3. How to display the font size in use in the final output,
4. How to display the font size in use in the final output,
5. How to display the font size in use in the final output,
6. How to display the font size in use in the final output,

7. How to display the font size in use in the final output,
8. How to display the font size in use in the final output,
9. How to display the font size in use in the final output,
10. How to display the font size in use in the final output,
11. How to display the font size in use in the final output,
12. How to display the font size in use in the final output,

\subsubsubsection{\showfont}

1. How to display the font size in use in the final output,
2. How to display the font size in use in the final output,
3. How to display the font size in use in the final output,
4. How to display the font size in use in the final output,
5. How to display the font size in use in the final output,
6. How to display the font size in use in the final output,
7. How to display the font size in use in the final output,

8. How to display the font size in use in the final output,
9. How to display the font size in use in the final output,
10. How to display the font size in use in the final output,
11. How to display the font size in use in the final output,
12. How to display the font size in use in the final output,


Lipsum me [31-35]

\end{otherlanguage*}



    % \end{anexosenv}

    % INDICE REMISSIVO
    \ifforcedinclude\else
        \phantompart
        \printindex
    \fi

\end{document}

