

% Is it possible to keep my translation together with original text?
% https://tex.stackexchange.com/questions/5076/is-it-possible-to-keep-my-translation-together-with-original-text
\chapter{Uma Ferramenta de Formatação}
\label{software_implementation}

Neste capítulo,
será explicado o funcionamento e
implementação de uma nova ferramenta de formatação.
A proposta desta nova ferramenta é permitir que usuários possam entrar com a gramática de qualquer linguagem,
por meio de uma metagramática\footnote{
Em Ciências da Computação,
quando algo é prefixado com ``meta'',
isso significa que ele refere~=se sobre o seu tipo ou
categoria \cite{theUseOfMetaRules}.
Por exemplo,
``metadata'' são dados sobre os dados.
} para então formatar o código~=fonte da linguagem descrita pela gramática.


\section{Uma Gramática de Gramáticas}
\label{GrammarsGrammar}

Na \fullref{introducaoGramaticas},
foi explicado o que são gramáticas.
Mas,
como gramáticas podem ser expressadas?
Isso depende de como seu analisador foi implementado\advisor{.}{,
sendo assim,
um detalhe de implementação.%
} \advisor{Analisadores}{Usualmente,
analisadores} seguem uma notação comum como EBNF\footnote{
Do inglês,
\textit{Extended Backus–Naur Form} uma extensão do padrão BNF (\textit{Backus–Naur Form}).
}\cite{teachingEbnf,antlrBookTerrentParr},
\advisor{que diverge de acordo com detalhes de implementação.
}{%
que não difere muito de um analisador para outro,
exceto por detalhes de implementação específicos de cada analisador.
}

Para realizar a implementação da nova ferramenta de formatação de código~=fonte,
foi realizado a construção de uma nova gramática de gramáticas de uma nova linguagem chamada de ``ObjectBeauty'',
uma metalinguagem \cite{compilersCompilerMetaLanguage}.
Na \typeref{MyWorflowForLarkTraduzido},
é apresentado o fluxo de uso comum para um analisador.
Neste processo,
o desenvolvedor da linguagem escreve a gramática de especificação\advisor{}{
desta linguagem} que é entregue a algum analisador e
gera~=se um compilador para tal linguagem.
\begin{figure}[!htb]
\caption{Fluxo de uso comum de um analisador}
\label{MyWorflowForLarkTraduzido}
\centering
\includegraphics[width=1.0\textwidth]{MyWorflowForLarkTraduzido.png}
\fonte{Própria, traduzido e adaptado de \citeonline{larkErrorRecovery}}
\nota{Própria,
traduzido e
adaptado,
são figuras desenvolvidas pelo autor,
mas inicialmente publicadas em outro local,
idioma e
com o conteúdo um pouco diferente.}
\end{figure}

\advisor{Este trabalho faz um uso diferente}{Já este trabalho faz um uso fora do comum}.
Como mostrado na \typeref{MyWorflowForLarkTraduzido2},
primeiro especifica~=se uma metalinguagem que será utilizada pelos usuários da nova ferramenta de Formatação de Código.
Para escrever esta nova metalinguagem,
utilizou~=se o Analisador Lark \cite{larkContextualLexer}.
Usualmente,
o Analisador Lark é utilizado somente como um gerador de compiladores (\typeref{MyWorflowForLarkTraduzido}),
entretanto,
neste contexto Lark é utilizado como um compilador de compiladores (\typeref{MyWorflowForLarkTraduzido2}).
Para este trabalho,
foi realizado um \textit{fork} \cite{overviewOfGitHubForks,mayTheForkBeWithYou,collaborationAmongGitHubUsers} do Analisador Lark,
renomeado o Analisador Lark para ``pushdown''\footnote{%
O código~=fonte do \textit{fork} pode ser encontrado em \url{https://github.com/evandrocoan/pushdownparser}.
}.
\begin{figure}[!htb]
\caption{Uso feito pela nova ferramenta de Formatação de Código}
\label{MyWorflowForLarkTraduzido2}
\centering
\includegraphics[width=1.0\textwidth]{MyWorflowForLarkTraduzido2.png}
\fonte{Própria, traduzido e adaptado de \citeonline{larkErrorRecovery}}
\end{figure}

Foi realizado um \textit{fork} do Analisador Lark para poder~=se realizar pequenas alterações que facilitam o entendimento do funcionamento interno da ferramenta como adição de logs e
alterações nos algoritmos de iteração nas árvores geradas pela ferramenta.
Por isso,
em alguns lugares do código~=fonte é encontrado o nome ``pushdown'' ao invés de ``lark''.
Já em outros,
continua~=se sendo chamado Lark de Lark para simplificar a retrocompatibilidade com a biblioteca original e
facilitar a realização da integração de novos updates vindos do repositório original do Analisador Lark para o \textit{fork} realizado.

Em vez de permitir com que o usuário final da aplicação opere diretamente com o analisador da \typeref{MyWorflowForLarkTraduzido},
foi criado uma nova metagramática (uma gramática de gramáticas) como mostrado nas \typeref{MyWorflowForLarkTraduzido2}.
Esta nova metagramática simplifica o processo de escrita de gramáticas ao criar uma nova especificação de gramáticas,
somente com os recursos necessários para se possa trabalhar com formatação de código~=fonte.
\advisor{Não}{A final, não}
é objetivo deste trabalho fazer a análise completa de programas,
pela sua sintaxe, semântica,
e gerar código~=binário executável.

Na \typeref{ParsersPublicAudienceTraduzido},
pode~=se encontrar uma relação entre o funcionamento das diversas partes da ferramenta de Formatação de Código e
a audiência alvo.
Basicamente existem três grupos distintos de usuários ou
audiência:
\begin{inparaenum}[1)]
\item quem escreve ou
desenvolve a ferramenta de Formatação de Código proposta por este trabalho e
define as regras da metalinguagem (especificada pela sua metagramática,
i.e.,
a gramática de gramáticas);
\item quem escreve ou
desenvolve gramáticas de linguagens para serem formatadas de acordo com as regras da metalinguagem e;
\item quem escreve ou
desenvolve programas de computador e
deseja realizar a formatação de seus códigos~=fonte.
\end{inparaenum}%
\begin{figure}[!htb]
\caption{Relacionamentos entre os Diferentes Públicos deste Projeto}
\label{ParsersPublicAudienceTraduzido}
\centering
\includegraphics[width=1.0\textwidth]{ParsersPublicAudienceTraduzido.png}
\fonte{Própria, traduzido e adaptado de \citeonline{larkErrorRecovery}}
\end{figure}

Até este ponto,
já falou~=se de metagramática e
metalinguagem com a exceção dos metaprogramas \cite{tradeoffsInMetaprogramming}.
Nas \typeref{MyWorflowForLarkTraduzido2,ParsersPublicAudienceTraduzido},
por simplificação foram omitidos o relacionamento dos metaprogramas com a metagramática e
metalinguagem.
Metaprogramas fazem parte da entrada do metacompilador (\typeref{MetacompilerMetagrammarMetaprogram}) junto com a metagramática para gerar um novo compilador (ou Formatador de Código).
Neste trabalho,
os metaprogramas serão as gramáticas que serão utilizadas pelos formatadores de código~=fonte.

Os metaprogramas (ou gramáticas) são entradas diretas do metacompilador,
o Analisador Lark na \typeref{MyWorflowForLarkTraduzido2},
um Analisador LALR(1).
Na \typeref{ParsersPublicAudienceTraduzido},
não pode~=se ver diretamente que as gramáticas das linguagens serão os metaprogramas,
mas o quadro em azul mais a esquerda ligado por linhas pontilhadas explica que os erros léxicos e
sintáticos nas gramáticas de entrada serão mostrados pelo Analisador Lark.
Isso acontece por que as gramáticas (ou metaprogramas) são entradas diretamente no Analisador Lark.

Na \typeref{MetacompilerMetagrammarMetaprogram},
encontra~=se uma extensão da \typeref{MyWorflowForLarkTraduzido2},
e pode~=se ver claramente as relações entre Metagramáticas,
Metacompiladores e Metaprogramas. Por simplificação,
mostra~=se o nó ``Árvore de Sintaxe'' sem explicitamente falar sobre sua Análise Semântica e
propriamente a construção do Compilador (ou do Formatador de Código).
Vale lembrar que trata~=se de um Compilador de Compiladores,
e não um Compilador de Analisadores.
Por isso vemos que os Metaprogramas (ou gramáticas) são entradas diretas dos Metacompilador,
e não do Formatador de Código.
\begin{figure}[!htb]
\caption{Relação entre Metagramáticas, Metacompiladores e Metaprogramas}
\label{MetacompilerMetagrammarMetaprogram}
\centering
\includegraphics[width=1.0\textwidth]{MetacompilerMetagrammarMetaprogram.png}
\fonte{Própria, traduzido e adaptado de \citeonline{larkErrorRecovery}}
\end{figure}

Esta não é a primeira vez que uma metagramática com simplificações foi escrita.
Em trabalhos como \citeonline{rustSublimeTextSyntaxSyntec,sublimeTextSyntax,vsCodeSyntaxHighlighthing},
foram realizados as mesmas simplificações aqui apresentadas.
Existem algumas diferenças técnicas da metagramática deste trabalho com as dos recém~=apresentados.
Como por exemplo,
a implementação da metagramática realizada ainda não suporta a classificação do mesmo trecho de código~=fonte por múltiplos tipos de escopo \cite{vsCodeSyntaxHighlighthing}.

Foi escolhida a criação de uma nova metagramática por que as implementações de metagramáticas já existentes como \citeonline{rustSublimeTextSyntaxSyntec,vsCodeSyntaxHighlighthing}:
\begin{enumerate}[1)]
\item Não utilizam explicitamente nenhum analisador,
realizando a programação das produções da gramática diretamente no código~=fonte (\fullref{gramaticasVersusLinguagens});
\item Não são capazes de reconhecer todas as características de todas as linguagens de programação (devido a optimizações para maior performance);
\item Não possuem sintaxe própria,
i.e.,
utilizam~=se de outras linguagens como YAML,
XML e
JSON para fazer a especificação da metagramática.
\end{enumerate}
Fazendo a especificação de uma nova metagramática,
é possível adaptar~=se a especificação da sintaxe das gramáticas de acordo as necessidades específicas sem ter que depender de características de outras linguagens como YAML,
XML ou
JSON.


\subsection{Escopos}

Na \typeref{TexMateScopes},
é mostrado na primeira linha o trecho de código~=fonte ``function f1 () \{'' e
nas demais linhas são apresentados as diversas classificações de escopos aplicados a cada um dos trechos do código~=fonte de amostra.
Por exemplo,
a palavra ``function'' possui simultaneamente os escopos
\begin{inparaenum}[1)]
\item ``source.js'';
\item ``meta.function.js'' e;
\item ``storage.type.function.js''.
\end{inparaenum}%
\begin{figure}[!htb]
\caption{Exemplo de Classificação de Código~=Fonte com Múltiplos Escopos}
\label{TexMateScopes}
\centering
\includegraphics[width=1.0\textwidth]{TexMateScopes.png}
\fonte{\citeonline{vsCodeSyntaxHighlighthing}}
\end{figure}

Os nomes utilizados na \typeref{TexMateScopes},
podem ser qualquer texto que usuário especificador daquela gramática atribuiu.
Entretanto,
pode~=se perceber que o nome dos escopos recém apresentados parecem seguir um padrão.
Por conversão,
desenvolvedores de gramáticas para os editores de texto como \citeonline{sublimeTextSyntax,vsCodeSyntaxHighlighthing},
seguem uma conversão de nomes para que as utilizações dos escopos gerados pelas gramáticas sejam compatíveis entre si.

Fazendo o uso de uma conversão para nomes de escopos,
as gramáticas ficam compatíveis com um maior número de arquivos de temas (ou configurações de cores),
onde são especificados os nomes dos escopos serão utilizados para especificar as cores a serem utilizadas pelo editor de texto.
Para mais informações sobre a utilização de arquivos de temas em editores de texto veja \citeonline{sublimeTextScopeNaming,vsCodeSyntaxHighlighthing}.


\section{Metalinguagem}
\label{metalinguagemGrammar}

Como já explicado na seção anterior,
uma metagramática é gramática de gramáticas e
foi utilizado o Analisador Lark \cite{larkContextualLexer} como um metacompilador ou
compilador de compiladores.
Nesta seção será discutido como a metalinguagem (especificada pela metagramática) utilizada foi construída,
começando com o seu símbolo inicial.
No \typeref{simboloInicialDaMetagramatica},
defini~=se que o programa é constituído de três grandes áreas,
que devem acontecer uma em sequencia da outra:
\begin{enumerate}
\item A produção ``preamble\_statements'' define características globais da gramática como um nome,
e um escopo que será atribuído a toda gramática;
\item A produção ``language\_construct\_rules'' define qual será o símbolo inicial da gramática.
Em comparação com linguagens de programação como ``C'',
ele pode ser considerado similar ao método ``main'';
\item A produção ``miscellaneous\_language\_rules'' permite a definição de diversos contextos\footnote{
Contexto refere~=se a um bloco de operadores ou
conjunto de instruções como ``include'' e
``match''.
} com grupos de produções da gramática (\fullref{definicaoDeGramatica}),
que podem ser incluídos a partir do símbolo inicial da gramática definido no item ``language\_construct\_rules''.
\end{enumerate}%
\begin{code}
\caption{Simbolo Inicial da Metagramática ``ObjectBeauty''}
\label{simboloInicialDaMetagramatica}
\begin{minted}{antlr}
language_syntax: _NEWLINE? preamble_statements _NEWLINE?
                    language_construct_rules _NEWLINE?
                    ( miscellaneous_language_rules _NEWLINE? )*
                    _NEWLINE?

preamble_statements: ( (
                        target_language_name_statement
                        | master_scope_name_statement
                        | constant_definition
                    ) _NEWLINE )+

language_construct_rules: "contexts" ": " indentation_block
miscellaneous_language_rules: /[^:\n]+/ ": " indentation_block

target_language_name_statement: "name" ": " free_input_string
master_scope_name_statement: "scope" ": " free_input_string
\end{minted}
\end{code}

Entre os \typeref{exemploDeGramaticaPawn1,exemploDeGramaticaPawn2,exemploDeGramaticaPawn3,exemploDeGramaticaPawn4},
encontra~=se pequenos exemplos de gramáticas escritas na metalinguagem ``ObjectBeauty'' brevemente apresentada.
No \typeref{exemploDeGramaticaPawn1},
encontra~=se a definição do símbolo inicial da gramática da linguagem sendo descrita (pela metagramática) e
pode~=se ver a metalinguagem sendo utilizada para definir uma linguagem chamada de ``Abstract Machine Language''.
Por padrão,
toda gramática ``ObjectBeauty'' precisa ter um contexto inicial ou
símbolo inicial chamado de ``contexts''.

O \typeref{exemploDeGramaticaPawn1},
faz uso dos operadores ``include'' e
``match''.
O operador ``include'' serve incluir partes de outras gramáticas ou
mesmo gramáticas inteiras no contexto da gramática atual.
Entretanto,
a implementação de ``include'' realizada neste trabalho somente consegue realizar includes de contextos definidos no mesmo arquivo.

No exemplo do \typeref{exemploDeGramaticaPawn1},
o operador ``include'' está incluindo contextos da gramática atual que serão definidas mais tarde neste mesmo arquivo.
Já o operador ``match'' utilizado no final serve para realizar propriamente o reconhecimento do programa de entrada e
atribuir a ele o escopo ``constant.boolean.language.pawn''.

Mais tarde,
as informações de escopo atribuídas por operadores como ``match'' e
``captures'' serão utilizadas pelo formatador de código~=fonte.
Com estas informações,
o Formatador de Código será capaz de realizar as operações de formatação somente sobre os trechos de código~=fonte que o usuário definir.
Realizando assim,
a formatação seletiva de código~=fonte,
contrário da formatação total de código~=fonte como acontece nos demais trabalhos (\fullref{performanceDoFormator}).
\begin{lstlisting}[caption={Exemplo de Gramática, Símbolo Inicial},label={exemploDeGramaticaPawn1},style=yaml_style]
name: Abstract Machine Language
scope: source.sma

contexts: {
    include: parens
    include: numbers
    include: check_brackets

    match: (true|false) {
        scope: constant.boolean.language.pawn
    }
}
\end{lstlisting}

No \typeref{exemploDeGramaticaPawn2},
é introduzido o uso dos operadores ``push'',
``meta\_scope'' e
``pop''.
O operadores ``push'' e
``pop'' são responsáveis por manter uma pilha de contextos que permite aplicar um mesmo escopo por várias linhas utilizado o operador ``meta\_scope''.
A diferença entre o operador ``scope'' e
``meta\_scope'' é que o operador ``scope'' atribuí o escopo diretamente ao texto reconhecido pelo um operador ``match''.
Já o operador ``meta\_scope'' permite aplicar o escopo a todo o texto desde o primeiro até o último ``match'',
que desempilha com o operador ``pop'',
o contexto empilhado inicialmente com um ``push''.
\begin{lstlisting}[caption={Exemplo de Gramática, Contextos},label={exemploDeGramaticaPawn2},style=yaml_style]
parens: {
    match: \( {
        scope: parens.begin.pawn
        push: {
            meta_scope: meta.group.pawn
            match: \) {
                scope: parens.end.pawn
                pop: true
            }
            include: numbers
        }
    }
}
\end{lstlisting}

No \typeref{exemploDeGramaticaPawn3},
é introduzido o uso do operador ``captures''.
O operador ``captures'' atribuí simultaneamente diversos escopos com uma única expressão regular.
Cada um dos números listados equivalem a um dos grupos de captura da expressão regular utilizada no operador ``captures''.
O operador ``scope'' pode ser considerado um caso especial do operador ``captures'' quando utiliza~=se o Grupo de Captura 0.

Motores de expressões regulares geralmente suportam um recurso chamado de Grupos de Captura \cite{expressionGrammarsWithRegexLikeCaptures}.
Por exemplo,
a expressão regular ``foo(bar)zoo(car)'' possuí 3 grupos de captura quando analisado o texto de entrada ``foobarzoocar'':
\begin{inparaenum}[1)]\setcounter{enumi}{-1}
\item foobarzoocar;
\item bar;
\item car;
\end{inparaenum}%
onde o grupo de captura 0 refere~=se a toda a expressão regular encontrada.
Portanto,
ao invés de utilizar~=se o operador ``scope:
constant.numeric.pawn'',
poderia~=se utilizar equivalentemente o operador ``captures:
0.
constant.numeric.pawn''.
\begin{lstlisting}[caption={Exemplo de Gramática, Grupos de Captura},label={exemploDeGramaticaPawn3},style=yaml_style]
numbers: {
    match: '(\d+)(\.\{2\})(\d+)' {
        captures: {
            0: constant.numeric.pawn
            1: constant.numeric.int.pawn
            2: keyword.operator.switch-range.pawn
            3: constant.numeric.int.pawn
        }
    include: numeric
}
\end{lstlisting}

No \typeref{exemploDeGramaticaPawn4},
é mostrado mais alguns exemplos de uso do operador ``match'' classificando diversos tipos de numéricos (da linguagem sendo descrita pela gramática).
É importante notar que a ordem no qual os operadores como ``match'' aparecem é importante.
Ao realizar o reconhecido o programa de entrada utilizando esta gramática,
a Árvore de Sintaxe Abstrata\footnote{%
Do inglês (AST), Abstract Syntax Tree.
} \cite{ahoCompilerDragonBook} será interpretada diversas vezes,
partindo no símbolo inicial até chegar ao último símbolo da gramática.

O processo de interpretação irá reiniciar indefinidamente até que nenhum texto seja mais consumido por nenhum dos operadores da gramática.
Assim,
uma vez que um trecho de código~=fonte já foi classificado,
ele será ignorado quando os próximos operadores forem aplicados,
evitando assim que o programa execute infinitamente.
\begin{lstlisting}[caption={Exemplo de Gramática, Tipos numéricos},label={exemploDeGramaticaPawn4},style=yaml_style]
numeric: {
    match: ([-]?0x[\da-f]+) {
        scope: constant.numeric.hex.pawn
    }
    match: \b(\d+\.\d+)\b {
        scope: constant.numeric.float.pawn
    }
    match: \b(\d+)\b {
        scope: constant.numeric.int.pawn
    }
}
\end{lstlisting}

Por fim,
no \typeref{exemploDeGramaticaPawn5},
é apresentado um exemplo não relacionado com formatação de código~=fonte.
A construção utilizada é comum para gramáticas que serão utilizadas para realizar a aplicação de cores em editores de texto \cite{vsCodeSyntaxHighlighthing}.
Com ela é possível colorir o código~=fonte,
destacando~=o como inválido no editor de texto,
uma vez que uma inconsistência sintática foi encontrada na linguagem sendo analisada.

Construções como a do \typeref{exemploDeGramaticaPawn5},
funcionam usualmente quando elas são a última regra da gramática.
Uma vez que todas as regras que consomem o programa de entrada e
o classifica em escopos terminam seu trabalho,
não deveria existir mais nenhum texto ser reconhecido.
Caso exista,
ou a gramática não estava preparada para reconhecer todo o programa de entrada,
ou estes trechos de código~=fonte são frutos de algum erro no programa de entrada.
\begin{lstlisting}[caption={Exemplo de Gramática, Reconhecimento de Erros},label={exemploDeGramaticaPawn5},style=yaml_style]
check_brackets: {
    match: \) {
        scope: invalid.illegal.stray-bracket-end
    }
}
\end{lstlisting}


\section{Analisador Semântico}

Depois que um metaprograma da metalinguagem apresentada na seção anterior é reconhecido pelo Analisador Lark,
o Analisador Lark entrega a Árvore de Sintaxe da gramática da linguagem sendo descrita pelo metaprograma (\typeref{MetacompilerMetagrammarMetaprogram}).
Todas as verificações de corretude da sintaxe da gramática são verificadas pelo Analisador Lark,
com base na metagramática da metalinguagem.
Portanto,
somente resta ser implementado o Analisador Semântico para verificar se a linguagem descrita respeita as regras semânticas da metalinguagem explicadas na seção anterior,
\nameref{metalinguagemGrammar}.

O Diagrama de Classes é apresentado na \typeref{CodeFormatterClassDiagram}.
Nele,
o Analisador Semântico que deriva de ``Transformer'' recebe como entrada a Árvore de Sintaxe do programa de entrada,
e uma vez que o Analisador Semântico termina seu trabalho,
ele devolve Árvore de Sintaxe Abstrata completa.
Então,
utilizando a Árvore de Sintaxe Abstrata,
``Backend'' que deriva de ``Interpreter'',
realiza a formatação de código~=fonte recebendo um programa de entrada e
as configurações do formatador (\typeref{ParsersPublicAudienceTraduzido,MetacompilerMetagrammarMetaprogram}).
\begin{figure}[!htb]
\caption{Diagrama das Principais Classes}
\label{CodeFormatterClassDiagram}
\centering
\includegraphics[width=1.0\textwidth]{CodeFormatterClassDiagram.png}
\fonte{Própria}
\end{figure}

No \typeref{semanticAnalizerConstructor},
pode~=se ver o construtor do Analisador Semântico.
Pode~=se notar que seu construtor não recebe como parãmetro a Árvore de Sintaxe.
Entretanto,
a ela não é passada pelo construtor mas sim por um método chamado ``transform(tree)''.
Esta é uma característica do Analisador Lark utilizado.
A função ``transform(tree)'' do Analisador Lark simplesmente inicia a análise do programa visitando todos os nós da Árvore de Sintaxe,
partindo das folhas até chegar na raíz (\fullref{compiladoresEtradutores}).
\begin{code}
\caption{Construtor do Analisador Semântico}
\label{semanticAnalizerConstructor}
\begin{minted}{python}
class TreeTransformer(pushdown.Transformer):
    """
        Transforms the Derivation Tree nodes into meaningful string representations,
        allowing simple recursive parsing and conversion to Abstract Syntax Tree.
    """

    def __init__(self):
        ## Saves all the semantic errors detected so far
        self.errors = []

        ## Saves all warnings noted so far
        self.warnings = []

        ## Whether the mandatory/obligatory global scope name statement was declared
        self.is_master_scope_name_set = False

        ## Whether the mandatory/obligatory global language name statement was declared
        self.is_target_language_name_set = False

        ## Can only be one scope called `contexts`
        self.has_called_language_construct_rules = False

        ## Pending constants declarations
        self.constant_usages = {}

        ## Pending constants usages
        self.constant_definitions = {}

        ## A list of miscellaneous_language_rules include contexts defined for duplication checking
        self.defined_includes = {}

        ## A list of required includes to check for missing includes
        self.required_includes = {}

        ## A list of regular expressions used on match statements,
        ## for validation when the constants definitions are completely know
        self.pending_match_statements = []

        ## Responsible for calculating all open and close commands scoping
        self.open_blocks = {}
        self.indentation_level = 0
        self.indentation_blocks = []
\end{minted}
\end{code}

A visita dos nós da Árvore de Sintaxe pela função ``transform(tree)'' acontece simplesmente chamando os métodos que a classe ``TreeTransformer'' (Definido no \typeref{semanticAnalizerConstructor}) define e
que possuem os mesmos nomes que os símbolos não~=terminais definidos pela metagramática.
Então,
cada nó ou
função deve retornar qual será o novo nó que irá o substituir na Árvore de Sintaxe.
Assim,
no final do processo,
um a um,
cada nó da Árvore de Sintaxe será convertido para um nó da Árvore de Sintaxe Abstrata.

Um jeito fácil de excluir um nó da Árvore de Sintaxe é simplesmente definir uma função com o nome de seu não~=terminal que returna ``null''.
Por exemplo,
o trecho da metagramática apresentado no \typeref{simboloInicialDaMetagramatica},
possui alguns símbolos não~=terminais como ``preamble\_statements'' e
``language\_construct\_rules''. Então,
para estes dados símbolos,
serão chamados os métodos da classe ``TreeTransformer'' que possuem os nomes ``preamble\_statements'' e
``language\_construct\_rules''.

Caso não existam os métodos ``preamble\_statements'' e
``language\_construct\_rules'' na classe ``TreeTransformer'',
os nós ``preamble\_statements'' e
``language\_construct\_rules'' da Árvore de Sintaxe não serão visitados e
``poderão'' ser excluídos da Análise Semântica (mas não da Árvore de Sintaxe Abstrata).
Nós não analisados pela classe ``TreeTransformer'' não serão excluídos da Árvore de Sintaxe Abstrata,
eles serão mantidos intactos,
a não ser que algum outro nó os altere diretamente na Árvore de Sintaxe.

Mesmo que não existam os métodos ``preamble\_statements'' e
``language\_construct\_rules'' definidos na classe ``TreeTransformer'',
eles também podem ser visitados diretamente a partir de algum nó pai ou
até algum de seus filhos.
Inclusive,
esta foi uma das alterações realizadas no \textit{fork} ``pushdown'' do Analisador Lark.
Por padrão,
ao iterar pela árvore,
o Analisador Lark somente passa como parâmetro da função o nó correspondente ao método atualmente sendo chamado e
uma lista de nós filhos.
Entretanto,
``pushdown'' também passa um terceiro parâmetro que é uma referência para o nó raíz da árvore e
mantém a variável ``parent'',
acessível como um atributo da classe ``TreeTransformer''.


\section{Formatador de Código}

O Formatador de Código não é composto somente por uma parte única e
altamente acoplada.
\advisor{Um}{Mas pelo contrário,
um} conjunto de partes altamente coesas e
completamente independentes onde cada uma dessas partes recebe a Árvore de Sintaxe Abstrata do Analisador Semântico,
para então realizar a formatação do programa de entrada junto com as configurações que esta parte aceita.


\subsection{Performance}
\label{performanceDoFormator}

Continuamente chamar diversos algoritmos independentes possui uma perda de performance em comparação com os formatadores de código~=fonte apresentados no \fullref{source_code_beautifiers}.
Estes formatadores tem como principal características realizar a formatação de código~=fonte em uma única passada,
i.e.,
a Árvore de Sintaxe de programa de entrada é completamente reconstruída,
para então ser serializada novamente em texto de acordo com as configurações de formatador.

Ao leitor mais desatento,
pode parecer que então não existe muita diferença entre a ferramenta de formatação proposta neste trabalho para as já existentes.
Entretanto,
este trabalho permite que o usuário entre com a gramática da linguagem a ser formatada,
diferente de outros trabalhos onde a gramática já é incluída ao código~=fonte do formatador.
Formadores de código~=fonte como apresentados na \fullref{source_code_beautifiers} são construídos programando~=se as produções da gramática diretamente no código~=fonte do formatador (Descendentes Recursivos,
reveja a \fullref{gramaticasVersusLinguagens}).

Ao não permitir~=se que o usuário possa entrar com a gramática do programa,
restringe~=se o formatador a somente funcionar com as gramáticas que foram programadas dentro do seu código~=fonte.
Uma vez que o usuário da ferramenta precisa programar o formatador de código para ter suporte a sua linguagem,
isso dificulta a adição do suporte de novas linguagens ao formatador,
pois precisa~=se programar as suas gramáticas diretamente dentro do código~=fonte do formatador.


\subsection{Formatação}

No \typeref{construtorDoFormatador},
pode ser encontrado o construtor do formatador de código~=fonte.
Comparando~=o com o construtor do Analisador Semântico (\typeref{semanticAnalizerConstructor}),
pode~=se notar algumas diferenças.
Em ambos os casos,
o processo todo se completará ao percorrer toda a árvore.
No caso do Analisador Semântico,
a Árvore de Sintaxe,
e no caso do Formatador de Código,
a Árvore de Sintaxe Abstrata.

Diferentemente do Analisador Semântico,
o Formatador de Código faz herança do tipo ``Interpreter'' em vez de ``Transformer'' (\typeref{CodeFormatterClassDiagram}).
A diferença é simples,
``Interpreter'' visita a árvore partindo das folhas até chegar na raíz visitando todos os nós filhos (\fullref{compiladoresEtradutores}),
já ``Transformer'' visita a árvore partindo da raíz até chegar nos nós pais,
i.e.,
ele não visita os nós filhos automaticamente como ``Transformer'' faz.

``Interpreter'' também recebe diretamente no construtor qual será a árvore que ele irá iterar sobre.
Nos demais aspectos,
``Interpreter'' funciona igual ao ``Transformer'',
exceto no ponto onde ``Transformer'' cria uma nova árvore no final do processo,
enquanto ``Interpreter'' não cria árvore alguma.
O parâmetro chamado ``program'',
que o construtor de ``Transformer'' recebe,
é o programa a ser formatado pelo formatador.
No final processo,
``Interpreter'' terá em sua variável ``self.program'' o novo programa completamente formatado.
\begin{code}
\caption{Construtor do Formatador}
\label{construtorDoFormatador}
\begin{minted}{python}
class Backend(pushdown.Interpreter):

    def __init__(self, formatter, tree, program, settings):
        super().__init__()
        self.tree = tree
        self.program = formatter( program, settings )

        ## A list of lists, where each list saves all the matches performed by
        ## the last match_statement on scope_name_statement
        self.last_match_stack = []

        ## This is set to False every push statement, and set to True, after
        ## every match statement. This way we can know whether there is a match
        ## statement after a push statement.
        self.is_there_push_after_match = False
        self.is_there_scope_after_match = False

        self.cached_includes = {}
        self.cache_includes( tree )

        self.visit( tree )
        log( 4, "Tree: \n%s", tree.pretty( debug=0 ) )
\end{minted}
\end{code}

Enquanto ``Interpreter'' é responsável por somente ``passear'' pela Árvore de Sintaxe Abstrata,
a classe ``AbstractFormatter'' (\typeref{construtorDeParsedProgram}) é responsável por realmente fazer a formatação de código~=fonte de acordo com a instruções vindas da Árvore de Sintaxe Abstrata.
No final do processo,
``AbstractFormatter'' terá na variável ``new\_program'' todos os pedaços do programa formatado.
Uma vez que ``Interpreter'' termina de construir todos os pedaços de código~=fonte formatado,
a função ``get\_new\_program'' irá unir dos os pedaços e
salvá~=los na variável ``cached\_new\_program'',
para evitar ter que recalcular o novo programa toda vez que pedir~=se a sua nova versão.
\begin{code}
\caption{Construtor de ``AbstractFormatter''}
\label{construtorDeParsedProgram}
\begin{minted}{python}
class AbstractFormatter(object):
    """
        Represents a program as chunks of data as (text_chunk_start_position,
        text_chunk).
    """

    def __init__(self, program, settings):
        super().__init__()
        self.initial_size = len( program )
        self.program = program
        self.settings = OrderedDict( sorted( settings.items(), key=lambda item: len( str( item ) ) ) )

        self.new_program = []
        self.cached_new_program = []
        log( 4, "program %s: `%s`", len( str( self.program ) ), self.program )
\end{minted}
\end{code}

A linha ``sorted'' realiza a ordenação do configurações sem nenhuma necessidade prática neste caso.
Ela garante que a ordem no qual as configurações irão ser processadas seja sempre a mesma.
No caso da Adição de Cores,
``sorted'' é uma função obrigatória para garantir que os nomes das cores da configurações sejam atribuídas de acordo com a ordem de funcionamento do tema \cite{vsCodeSyntaxHighlighthing,sublimeTextScopeNaming}.

A linha ``len( program )'' não possui nenhuma influência nos resultados do programa,
seja para formatação ou
adição de cores.
Entretanto,
ela é constantemente verificado durantes as operações que realizam o consumo do programa de entrada pelas regras da metagramática para garantir que o tamanho programa original não seja perdido.
É necessário uma sincronia entre os índices do programa original com o programa formatado para que ambos possam ser sincronizados no final do processo.
Assim,
uma que os algoritmos de adição de cores ou
formatação estivem propriamente testados,
não seria mais necessário manter o uso da variável ``len( program )''.

Durante o processo de consumo,
o programa de entrada é modificado e
os caracteres que foram consumidos são substituídos pelo caractere ``§''.
A existência do caractere de marcação de consumo ``§'' é uma fraqueza do algoritmo implementado,
pois pode levar programas que contenham este caractere ao erro.
Ele também trás uma ineficiência no consumo de novos caracteres.
Os caracteres que já foram substituídos por ``§'',
serão reanalisados continuamente até o final do análise do programa de entrada.
A utilização do caractere ``§'' foi realizada para simplificar a implementação do algoritmo de consumo.

Com o caractere ``§'' colocado no lugar dos caracteres que já foram consumidos,
é possível reconstruir o programa de entrada no final da análise simplesmente ordenando todos pedaços que de programa formatado que estão armazenada na variável ``new\_program''.
Como o tamanho do programa original não fui modificado,
também é possível facilmente integrar no programa formatado,
todas as partes do programa original que não foram formatados,
no caso do formatador,
ou coloridas do caso da adição de cores.


\subsection{Exemplo}

A formatação de código~=fonte recebe como entrada o programa em texto simples,
e como resultado,
retorna uma página HTML \cite{parallelParserForHTML}.
Com isso,
sendo possível observar com mais facilidade o código~=fonte original e
formatado,
e as metainformações atribuídas pela gramática da linguagem como atributos das tags HTML.
No \typeref{exemploDeFormatacaoDeCodigo},
encontra~=se um HTML gerado com as metainformações agregadas ao programa não~=formatado ``if(something) bar'',
onde ``if( \ something \ ) bar'' é o resultado da formatação de código~=fonte.
\begin{quadro}[!htb]
\caption{Exemplo de Formatação de Código}
\label{exemploDeFormatacaoDeCodigo}
\begin{bluebox}
\begin{code}
\label{exemploDeHTMLGerado}
\caption{Exemplo de HTML Gerado pelo Formatador de Código}
\begin{minted}[xleftmargin=2em]{html}
<body style="white-space: pre; font-family: monospace;">
    <span setting="unformatted" grammar_scope="if.statement.definition" setting_scope="" original_program="if(">if(</span>
    <span setting="1" grammar_scope="if.statement.body" setting_scope="if.statement.body" original_program="something">  something  </span>
    <span setting="unformatted" grammar_scope="if.statement.definition" setting_scope="" original_program=")">)</span>
    <span grammar_scope="none" setting_scope="none"> bar</span>
</body>
\end{minted}
\end{code}

\lstset{xleftmargin=2em,aboveskip=0pt}
\begin{lstlisting}[caption={Exemplo de Gramática Utilizada pelo Formatador de Código},label={exemploDeGramaticaUtilizada},style=yaml_style]
scope: source.sma source.c++ source.javascript source.rust
name: Abstract Machine Language
contexts: {
    match: if\( {
        scope: if.statement.definition
        push: {
            meta_scope: if.statement.body
            match: \) {
                scope: if.statement.definition
                pop: true
            }
        }
    }
}
\end{lstlisting}

\begin{code}
\label{exemploDeConfiguracaoUtilizada}
\caption{Exemplo de Configuração Utilizada pelo Formatador de Código}
\begin{minted}[xleftmargin=2em]{python}
{
    "if.statement.body" : 2,
}
\end{minted}
\end{code}

\begin{code}
\label{exemploDeFormatadorDeCodigo}
\caption{Exemplo de Formatador de Código}
\begin{minted}[xleftmargin=2em]{python}
class SingleSpaceFormatter(AbstractFormatter):

    def format_text(self, code_to_format, setting_value):
        code_to_format = code_to_format.strip( " " )

        if setting_value:
            return " " * setting_value + code_to_format + " " * setting_value
        else:
            return code_to_format
\end{minted}
\end{code}
\end{bluebox}
\end{quadro}

Como pode ser percebido,
a gramática de entrada no \typeref{exemploDeFormatacaoDeCodigo},
não consome a palavra ``bar'' do programa de entrada ``if(something) bar''.
Esta é uma característica importante dos formatadores de código deste trabalho.
Todo texto que não é consumido,
ou pela gramática de entrada,
ou pelo formatador de código ou
adição de cores,
será mantido intacto no final do processo de formatação ou
adição de cores.
Assim,
pode~=se ter o formatador de código já em funcionamento com gramática mais simples possível,
ou que já atenda as características mínimas que deseja~=se formatar ou
adicionar de cores.
