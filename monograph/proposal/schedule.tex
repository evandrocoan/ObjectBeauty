


\section{Cronograma}

    \begin{adjustwidth}{-0.5\marginparwidth}{-0.5\marginparwidth}
    \small
    \begin{tabularx}{\linewidth}{|BX|*{11}{^c|}}

        \hline
        \multicolumn{1}{|c|}{\multirow{2}{*}{{\bfseries Etapas}}} & \multicolumn{11}{|c|}{{\bfseries Meses}} \\
        \cline{2-12}

        \rowstyle{\bfseries}
        & ago & set & out & dez & jan & fev & mar & abr & mai & jun & jul   \\ \hline

        Escrita da revisão bibliográfica
        &  x  &  x  &     &     &     &     &     &     &     &     &     \\ \hline

        Classificar todas classes e tipos de formatações
        &     &  x  &  x  &     &     &     &     &     &     &     &     \\ \hline

        Implementação de um núcleo funcional
        &     &     &  x  &  x  &     &     &     &     &     &     &     \\ \hline

        Finalização da escrita do TCC
        &     &     &     &  x  &  x  &     &     &     &     &     &     \\ \hline

        Ajustes finais no texto do TCC
        &     &     &     &     &  x  &  x  &  x  &  x  &     &     &     \\ \hline

        Defesa do TCC
        &     &     &     &     &     &     &     &     &  x  &  x  &     \\ \hline

    \end{tabularx}
    \end{adjustwidth}
    \hfill\cite{Silva}


\section{Custos}

    % How to align a vertical line at the end of the multicolumn in a table?
    % https://tex.stackexchange.com/questions/367075/how-to-align-a-vertical-line-at-the-end-of-the-multicolumn-in-a-table
    \begin{tabular}
    {|
        *1{@{\hspace{3.0pt}}>{ \RaggedRight\arraybackslash\hsize=1.1\hsize }Bp{3.9cm}|} % Item
        *1{@{\hspace{3.0pt}}>{ \RaggedRight\arraybackslash\hsize=1.1\hsize }^p{2.0cm}|} % Quantidade
        *1{@{\hspace{3.0pt}}>{ \RaggedRight\arraybackslash\hsize=1.1\hsize }^p{3.2cm}|} % Valor, Valor
        *1{@{\hspace{3.0pt}}>{ \RaggedRight\arraybackslash\hsize=1.1\hsize }^p{2.8cm}|} % Valor, Valor
    }

        \hline
        \rowstyle{\bfseries}
        Item                    &   Quantidade  &   Valor Unitário (R\$)    &   Valor Total (R\$) \\ \hline
        CD                      &   1           &   5,00                    &   5,00              \\ \hline
        Impressão               &   800         &   0,15                    &   120,00            \\ \hline
        Reserva Gerencial       &   1           &   20,00                   &   20,00             \\ \hline
        Reserva de Contingência &   1           &   20,00                   &   20,00             \\ \hline
        Total                   & \multicolumn{2}{c|@{\hspace{3.0pt}}}{}    &   165,00            \\ \hline

    \end{tabular}

    \medskip
    \hfill\cite{Silva}


\section{Recursos Humanos}

    \begin{tabular}{|Bl|^l|}

        \hline
        \rowstyle{\bfseries}
        Nome                            & Função                  \\ \hline
        \Author                         & Autor                   \\ \hline
        \Advisor                        & Orientador              \\ \hline
        Renato Cislaghi                 & Coordenador de Projetos \\ \hline
        \Supervisor                     & Professor Responsável   \\ \hline

    \end{tabular}


\section{Comunicação}

    \begin{tabular}{|Bl|^l|^l|^l|}

        \hline
        \rowstyle{\bfseries}
        O quê  & De quem & Para Quem & Como                                        \\ \hline
        Proposta de TCC         & Autor     & Renato Cislaghi   & Site de projetos \\ \hline
        Relatório de TCC I      & Autor     & Renato Cislaghi   & Site de projetos \\ \hline
        Prévia do TCC, em TCC I & Autor     & Banca             & E-mail           \\ \hline
        Defesa do TCC           & Autor     & Banca             & Pessoalmente     \\ \hline
        Reunião de Orientação   & Orientadores  & Autor         & Pessoalmente     \\ \hline

    \end{tabular}

    \hfill\cite{Silva}


\section{Riscos}

    % https://tex.stackexchange.com/questions/366156/how-to-change-the-left-padding-for-one-latex-tables-cell
    % https://tex.stackexchange.com/questions/366155/how-to-write-a-table-a-little-larger-than-the-paragraphs-with-centered-columns
    %
    \begin{adjustwidth}{-0.5\marginparwidth}{-0.5\marginparwidth}
    \small
    \begin{tabularx}{\linewidth}
    {|
        *1{                 >{\RaggedRight\arraybackslash\hsize=1.1\hsize }BX       |} % Riscos
        *3{@{\hspace{3.0pt}}>{\Centering\arraybackslash                   }^p{0.9cm}|} % Probabilidade, Impacto, Prioridade
        *2{                 >{\RaggedRight\arraybackslash\hsize=0.95\hsize}^X       |} % Resposta, Prevenção
    }

    \hline

    \rowstyle{\bfseries}
    Riscos  & a & b & c & Estratégia de resposta & Ações de prevenção \\ \hline

    % Row 1
    % Riscos
    Problemas com perda de dados &
    % Probabilidade
    Baixa &
    % Impacto
    Alto &
    % Prioridade
    Alta &
    % Estratégia de resposta
    Uso do backup &
    % Ações de prevenção
    Backup periódicos \\ \hline

    % Row 2
    % Riscos
    Alteração do cronograma ou descontinuidade do projeto onde recebo uma bolsa &
    % Probabilidade
    \rlap{Média} &
    % Impacto
    Alto &
    % Prioridade
    Alta &
    % Estratégia de resposta
    Redefinição da data de entrega do trabalho &
    % Ações de prevenção
    Monitoramento contínuo das informações obtidas com superiores imediatos \\ \hline

    \end{tabularx}

    \hfill {\small {\bfseries a}: Probabilidade, {\bfseries b}: Impacto, {\bfseries c}: Prioridade}

    \end{adjustwidth}

    \hfill\cite{Silva}



