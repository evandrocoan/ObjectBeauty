


\section{Objetivos}

    O objeto neste trabalho de TCC proposto aqui não é inicialmente suportar todas as regras de
    formatação de todas as linguagens de programação, mas a criação de uma estrutura básica inicial
    e robusta que sejam capaz de ser desenvolvida a ponto de ser facilmente expandida, tanto na
    adição de novos módulos de processamento no programa escrito, tanto pelo usuário final na
    escrita dos arquivos de programação.



\subsection{Objetivos Gerais}

    \begin{enumerate}

        \item

        Escrever o programa em C++ ou afins, para permitir também que a formação/beautifying seja
        (em trabalhos futuros/talvez nesse) dinâmico, isto é, na medida que você digita o texto, ele
        é formatado para você. Assim você pode focar mais em escrever o código, ao invés que se
        preocupar com o espaçamento, alinhamento, parenteses, linhas novas, e o que mais que seja.

        \item

        Utilizar o Framework `doctest` para escrita dos Testes de Unidade. Pois após procurar e
        testar alguns frameworks para testes de unidade em C++, entrou-se este como servindo muito
        bem as requisitos do projecto. Ele causa baixíssimo incremento no tempo de compilação e
        permite que os testes possam ser escritos no mesmo arquivo onde encontram-se o código do
        programa, sem que eles sejam compilados.

        \item

        Utilizar uma versão/algoritmo multi-core, assim cada uma das regras pode ser processada em
        paralelo e sobre o mesmo source code original. Essa parte é bastante complexa de ser escrita
        por que as regras entre si podem gerar conflitos sobre o que elas estão fazendo. Assim para
        resolver esse problema, fazer com que cada regra processada gere um objeto de mudanças que
        essa regra está propondo. Assim no final do processamento de todas as regras, será realizado
        um fusão das mudanças que cada uma decidiu realizer, e caso duas regras queriam mudar o
        mesmo pedaço/trecho de código, será lançada um exceção e uma nova classe de mudanças/regra
        deve estar disponível para resolver esse conflito. Caso não exista, ambas as mudanças são
        descartadas e somente as mudanças sem conflitos são refletidas no código.

    \end{enumerate}





