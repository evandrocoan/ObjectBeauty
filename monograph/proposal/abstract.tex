


\begin{abstract}

    Os softwares formatadores de código fonte atuais, também conhecidos como Beautifiers, são
    limitados a um conjunto similar, ou mesmo à uma única linguagem, e além de muitos, serem
    limitados ao que eles podem fazer por você ao processar/formatar o código \cite{Terence}.
    Portanto este trabalho tem como objetivo criar um formatador (software único) de fácil
    configuração e expansão capaz de abranger todas as linguagens de programação que existem,
    baseado em um uso específico de expressões regulares.

    \medskip
    A metodologia abordada será de não ter a necessidade de ter-se conhecimento da sintaxe das
    linguagens de programação que se irão fazer o parsing. Isso porque trataremos elas como texto
    comum, e será o usuário final que fará a configuração das transformações que serão aplicados no
    texto, dando liberdade de facilmente se configurar várias linguagens de programação (senão
    todas), aproveitando o fato de que muitas deles compartilham estruturas semelhantes senão
    idênticas.

    \medskip
    Como resultado espera-se ter um Beautifier Universal capaz de abranger todas as linguagens que
    existem, senão que seja facilmente extensível para abrange-las. Os pontos positivos dessa
    abordagem são a reusabilidade de componentes entre as linguagens. Por exemplo, `if/for/while's
    em C++ e Java são da mesma estrutura. Assim temos que escrever somente uma vez a especificação
    para um componente da linguagem.

    \bigskip
    \bigskip
    \textbf{Palavras-chave:}
    source, code, formatter, beautifier, prettyprint, universal, reuse, blocks, object, oriented,
    programming, structured, parsing, parse, regular, expression, regex, C, C++,  grammar, Turing,
    machine, automata, lexer, syntax, sublime, Java, Rust, shell, script, obfuscators, learning,
    syntec, teamicide, concensus, indent, settings.

\end{abstract}





