



\section{Implementação}
\label{sec:implementation}

    Este trabalho tem como objetivo criar um formatador (software único) de fácil configuração e
    expansão capaz de abranger as linguagens de programação que existem, baseado em um uso
    específico de expressões regulares.

    A metodologia abordada será de não ter a necessidade de ter-se conhecimento da sintaxe das
    linguagens de programação que se irão fazer o parsing. Isso porque trataremos elas como texto
    comum, e será o usuário final que fará a configuração das transformações que serão aplicados no
    texto, dando liberdade de facilmente se configurar várias linguagens de programação (senão
    todas), aproveitando o fato de que muitas deles compartilham estruturas semelhantes senão
    idênticas.

    Como resultado espera-se ter um Beautifier Universal capaz de abranger as linguagens que
    existem, senão que seja facilmente extensível para abrange-las. Os pontos positivos dessa
    abordagem são a reusabilidade de componentes entre as linguagens. Por exemplo, `if/for/while's
    em C++ e Java são da mesma estrutura. Assim temos que escrever somente uma vez a especificação
    para um componente da linguagem.

    A ideia de um software, que em certa extensão pode continuar um ramo do Trabalho de Conclusão de
    Curso do aluno `Lucas Boppre Niehues', orientado do Professor `Olinto José Varela Furtado'
    defendido em 2013/1, com o título: `Estudo e Criação de um Editor de Código Estruturado'. Donde
    durante a leitura de seu TCC, encontra-se o seguinte trecho que faz ligação com uma das
    propostas deste trabalho, no capítulo: `8.1.2 Separação de formato de exibição e de saída':

    \medskip
    \begin{bluebox}
    ``As formas que o código é exibido ao usuário e que ele é salvo em disco são controladas
    por arquivos de configuração distintos. O arquivo ``theme.ini'' contém, entre outras
    configurações, informações de como serializar a árvore sintática.''
    \end{bluebox}

    \vspace{-5mm}
    ...
    \begin{bluebox}
    ``A configuração de formato de saída é dada da mesma forma, mas em um arquivo
    separado, chamado ``output\_format.ini''. A decisão desta separação foi em vista de equipes
    de programadores que queiram utilizar uma convenção única para os arquivos salvos,
    mas manter a exibição a escolha de cada um. Assim os integrantes desta equipe podem
    compartilhar os seus arquivos ``output\_format.ini'' enquanto personalizam o arquivo
    ``theme.ini'' a seu gosto.''
    \end{bluebox}

    Com base nisso, pode-se pensar na escrita de plugins para editores de texto/IDEs comuns como
    Sublime Text. Assim ao carregar o arquivo do disco, este plugin chama o formatter e faz a
    formatação de acordo com as configurações de exibição para o usuário. Após isso, quando o
    usuário for salvar o arquivo, o arquivo com a formatação original é devolvido.

    Para auxilar nesse processo, um módulo de autoconfiguração é de grande ajuda. Ele detecta como o
    source code está formatado e cria arquivos de configuração para ele. Assim ao salvar o arquivo,
    automaticamente ele é salvo no formato que ele foi lido. Então temos o mesmo beneficio de
    editores estruturados, como proposto trabalho de `Lucas Boppre Niehues'. De inicio podemos
    pensar com os seguinte objetivo/ideia para um TCC:

    \medskip
    \begin{bluebox}
    \begin{enumerate}[nolistsep]
        \item Criar um formatador de fácil configuração e expansão para as linguagens de
              programação que existem e que irão existir.
    \end{enumerate}
    \end{bluebox}



\subsection{Problema}

    O problema proposto a se resolver é criar um Beautifier Universal. Os softwares atuais são
    limitados a um conjunto similar, ou mesmo à uma única linguagem, e além de muitos, serem
    limitados ao que eles podem fazer por você ao processar/formatar o código \cite{Terence}.

    To every developer in this world, the closest thing to their heart is the text editor of their
    choice. Over the last few years many new text editors has come into the market in both free and
    paid model, but unfortunately not all of them were able to make a real dent on the developer
    community. I remember in my college days we uses to use Notepad++ as our beloved text editor, as
    at that point of time it was one of the popular and free text editor with a lot of features for
    coding. But as time goes on, the entire development community started to lean towards sublime
    text since it’s launch.
    \url{https://www.isaumya.com/sublime-text-vs-atom-which-one-i-prefer-most-and-why/}

    As a developer, your code editor is one of the most important parts of your setup. It can save
    your wrists and fingers from repetitive strain injuries. It can save your eyes from going blind
    after a coding marathon.
    \url{https://hackernoon.com/virtualstudio-code-the-editor-i-didnt-think-i-needed-16970c8356d5}

    VS Code is an Editor while VS is an IDE.
    \url{https://stackoverflow.com/questions/30527522/what-are-the-differences-between-visual-studio-code-and-visual-studio}

    What is the difference between VS Code and VS Community?
    Visual Studio Code is a streamlined code editor with support for development operations like
    debugging, task running and version control. It aims to provide just the tools a developer needs
    for a quick code-build-debug cycle and leaves more complex workflows to fuller featured IDEs.
    For more details about the goals of VS Code, see Why VS Code.
    \url{https://code.visualstudio.com/docs/supporting/faq#_licensing}

    Reg Replace is a plugin for Sublime Text 2 that allows the creating of commands consisting of
    sequences of find and replace instructions.
    \url{https://forum.sublimetext.com/t/regreplace-plugin/3810}

    The main reason I moved was that I find that it’s much slower, the simple things like opening a
    new window for a project should be instantaneous and sadly it’s far from it. As I've said before
    it's all about personal preference, I've gone back to Sublime but Adam for example is sticking
    with it...
    \url{http://engageinteractive.co.uk/blog/atom-review}

    Logo abaixo há algumas regras de formatação básica encontrados no serviço online
    \url{http://prettyprinter.de/} acessado em março/2017:

    \medskip
    \begin{bluebox}
    \begin{enumerate}[nolistsep]
        \item Add new lines after ``\{'' and before ``\}''
        \item Add new lines before ``\{''
        \item Remove empty lines
        \item Add comment lines before function
        \item Add new lines after ``;''
        \item Add new lines after ``\}''
        \item Remove new lines
        \item Reduce whitespace
        \item Put the code again in the input box above after submit
    \end{enumerate}
    \end{bluebox}

    A partir deste ponto, apresenta-se um esboço sobre o problema, soluções, informações como
    porquês de se querer fazer um software assim, ou ainda de querer-se o beautifying:

    \begin{enumerate}[leftmargin=*]

        \item

        Motivação: Existem muitas ferramentas distintas, por vezes pagas, e dificilmente completas
        \cite{Terence}.

        \item

        Muitas linguagens de programação existem, assim sempre ter fazer um software Beautifier para
        cada uma delas é muito trabalhoso \cite{Terence}. Mas a abordagem para um Beautifier
        Universal proposta nesse trabalho, permite que facilmente novas linguagens sejam
        adicionadas, sendo elas completamente diferentes das anteriores, ou similares. No caso de
        similaridades, basta reutilizar as estruturas de configuração das linguagens já existentes.

        \item

        Preocupa-se de fazer um Beautifier para cada uma delas por que programadores atualmente
        trabalham diariamente com varias dessas linguagens, e elas não são similares. Assim precisa-
        se configurar vários beautifiers para fazer a formatação. Isso é um problema por que,
        somente alguns beautifiers são mais completos, e toda vez que precisa-se fazer uma alteração
        no estilo de formatação, precisa-se propagar manualmente a mesma mudança ao longo de vários
        arquivos de configuração de programas distintos, o que é ruim pois toma ao usuário muito
        tempo de aprender a lidar com várias e muito diferentes tipos de configurações
        \cite{Schweitzer}.

        \item

        No caso do Beautifier que propõem-se, uma mudança no estilo é propagada para todas as
        linguagens. E caso queira-se deixar alguma linguagem fora da regra, basta remover ela da
        lista ao qual esse bloco da configuração se aplica, e `a)' deixar ela de fora assim nenhuma
        mudança é aplicada a ela. Ou `b)' criar um novo bloco que inclua somente ela com a
        configuração desejada.

        \item

        A seguir, temos algumas frases sobre o assunto:

        \begin{bluebox}
        % \setlength{\itemindent}{5pt}
        ``One of absolute worst, worst methods of teamicide for software developers is to engage
        in these kinds of passive-aggressive formatting wars. I know because I've been there.
        They destroy peer relationships, and depending on the type of formatting, can also damage
        your ability to effectively compare revisions in source control, which is really scary.
        I can't even imagine how bad it would get if the lead was guilty of this behavior. That's
        leading by example, all right. Bad example.'', \cite{Atwood}.
        \end{bluebox}
        \vspace{-5mm}
        ...
        \begin{bluebox}
        ``So yes, absurd as it may sound, fighting over whitespace characters and other seemingly
        trivial issues of code layout is actually justified. Within reason of course -- when done
        openly, in a fair and concensus building way, and without stabbing your teammates in the
        face along the way.'', \cite{Atwood}.
        \end{bluebox}

        \begin{bluebox}``
        I'd say there are two main reasons to enforce a single code format in a project. First has
        to do with version control: with everybody formatting the code identically, all changes in
        the files are guaranteed to be meaningful. No more just adding or removing a space here or
        there, let alone reformatting an entire file as a `side effect' of actually changing just a
        line or two.'', \cite{Geukens}.
        \end{bluebox}

    \end{enumerate}



\subsection{Objetivos}

    O objeto neste trabalho de TCC proposto aqui não é inicialmente suportar todas as regras de
    formatação de todas as linguagens de programação, mas a criação de uma estrutura básica inicial
    e robusta que sejam capaz de ser desenvolvida a ponto de ser facilmente expandida, tanto na
    adição de novos módulos de processamento no programa escrito, tanto pelo usuário final na
    escrita dos arquivos de programação.

    A teoria da técnica empregada é muito simples, mas diferente das atuais por que é atribuído ao
    usuário final a responsabilidade de dizer onde será realizado o beautifying do modulo que está
    se configurando. Esse é o preço a pagar para permitir a criação de um Beautifier Universal.
    Quando diz-se fácil configuração, refire-se a não necessidade de recorrer a programação ´C++',
    i.e., alterar o código fonte do programa para permitir/especificar onde devem ser realizadas as
    alterações de beautifying.


\subsubsection{Objetivos Gerais}

    \begin{enumerate}[leftmargin=*]

        \item

        Escrever o programa em C++ ou afins, para permitir também que a formação/beautifying seja
        (em trabalhos futuros/talvez nesse) dinâmico, isto é, na medida que você digita o texto, ele
        é formatado para você. Assim você pode focar mais em escrever o código, ao invés que se
        preocupar com o espaçamento, alinhamento, parenteses, linhas novas, e o que mais que seja.

        \item

        Utilizar o Framework `doctest` para escrita dos Testes de Unidade. Pois após procurar e
        testar alguns frameworks para testes de unidade em C++, entrou-se este como servindo muito
        bem as requisitos do projecto. Ele causa baixíssimo incremento no tempo de compilação e
        permite que os testes possam ser escritos no mesmo arquivo onde encontram-se o código do
        programa, sem que eles sejam compilados.

        \item

        Utilizar uma versão/algoritmo multi-core, então cada uma das regras pode ser processada em
        paralelo e sobre o mesmo source code original. Essa parte é bastante complexa de ser escrita
        por que as regras entre si podem gerar conflitos sobre o que elas estão fazendo. Para
        resolver esse problema, fazer com que cada regra processada gere um objeto de mudanças que
        essa regra está propondo. No final do processamento de todas as regras, será realizado um
        fusão das mudanças que cada uma decidiu realizer, e caso duas regras queriam mudar o mesmo
        pedaço/trecho de código, será lançada um exceção e uma nova classe de mudanças/regra deve
        estar disponível para resolver esse conflito. Caso não exista, ambas as mudanças são
        descartadas e somente as mudanças sem conflitos são refletidas no código.

    \end{enumerate}


\subsubsection{Objetivos Específicos}


    \begin{enumerate}[leftmargin=*]

        \item

        Um Produto de Software com uma ótima orientação a objetos e possibilidades de extensão das
        funcionalidades.

        \item

        Classificar todas classes e tipos de formatações (beautifying) de código aplicáveis com
        facilidade. Uma das partes a serem escritas e entregues na monografia. Um estudo sobre o que
        é beautifying, como fazer e por que fazer.

        \item

        Implementação de um núcleo funcional e de uma pesquisa decente sobre o estado da arte. Um
        dos pontos difíceis seria a marcação dos escopos, mas isso já é implementado pelo núcleo do
        editor Sublime Text, assim provado como possível de ser feito.

        \item

        Inicialmente devido a limitação de tempo em 1 ano e meio para um TCC, podemos pensar somente
        um núcleo básico, simples, reutilizável e que talvez possa ajudar no contexto da linguagem
        que vocês desenvolvem.

    \end{enumerate}


\subsubsection{Trabalhos Futuros}

    O número de recursos/funcionalidades e estratégias de otimizações para serem implementadas, e
    etc, são imensas. Mas esses trabalhos podem ser muito mais para frente depois da entrega do TCC.
    Hoje o controle de espaços em chamadas de funções, declarações de classes, comentários e etc,
    são mais tranquilos de se entender e pensar. Entretanto no requisito e ajuste de indentação,
    inserção/remoção de parenteses redundantes, etc ainda falta estudo sobre como deve ser
    implementado isso.

    Contudo essa especificação por parte do usuário é limitado a linguagens Livres de Contexto
    (máquinas de pilha). Assim caso as especificações de escope precisarem ser feitas em termos de
    linguagens Sensíveis ao Contexto ou ainda Recursivamente Enumeráveis, vai ser preciso tratar
    esses elementos diretamente em C++ (máquina de turing).

    Entretanto não consegue-se pensar facilmente em casos em que precise mais do que tratadores
    Livres de Contexto para realizar a especificação de quais partes do código deve ser necessário
    formatar. Sublime Text faz uso dessa técnica para o Highlight dos códigos das mais diversas
    linguagens e acredita-se que tenha um bom resultado.





\subsection{Método de pesquisa}

    A vantagem nesta abordagem é não ter a necessidade de ter-se conhecimento da sintaxe das
    linguagens de programação que se irão fazer o parsing. Isso porque trataremos elas como texto
    comum, e será o usuário final que fará a configuração das transformações que serão aplicados no
    texto, dando liberdade de facilmente se configurar várias linguagens de programação,
    aproveitando o fato de que muitas deles compartilham estruturas semelhantes senão idênticas.

    A literatura/programas atuais são dependentes de linguagem de programação. Minha proposta é
    fazer este processo independente de linguagem, mas de dialetos como este exemplo tirado do PDF
    em anexo a este e-mail `Initial check list tasks to do.pdf':

    \begin{lstlisting}
    // This is the name used to reference this scope around the settings files.
    Scope Name:
    %c++_like_block_comment

    // This set on which languages this block should be included. Setting it
    // to empty will allow it to be parsed for any languages.
    Language Inclusion:
    Java, C++, Pawn

    // Defines a expression which will map the beginning of a exclusion block.
    Scope Start:
    /\*\*

    // Defines a expression which will map the ending of a exclusion block.
    Scope End:
    \\\*
    \end{lstlisting}
    \vspace*{-4mm}

    A abordagem acima é uma abordagem ingênua, portanto somente brevemente ilustrativa. O real motor
    para o software é baseado em expressões regulares e um pilha de contextos. Esta ideia foi
    inicialmente desenvolvida pelo editor de texto `Sublime Text' \cite{Skinner}. Este editor
    utiliza essa estrutura de blocos para fazer a sintaxe highlighting do códigos das linguagens
    através de expressões regulares alocação de contextos/escopos. Essa mesma abordagem pode ser
    utilizada pelo usuário para definir em quais regiões uma Máquina de Turing (linguagens C++/Rust)
    devem fazer/propor as alterações no código.


\subsubsection{Pontos}

    Os pontos positivos dessa abordagem para um formatador de código são a reusabilidade de
    componentes entre as linguagens pelo usuário final da aplicação ao invés do programador, o que
    torna este software muito mais genérico e abre a possibilidades de maior sucesso para a criação
    definitiva de um formatador Universal de códigos das linguagens de programação, quaisquer sejam
    elas. Por exemplo, `if/for/while'\textquotesingle s em linguagens de programação como C++ e Java
    são da mesma estrutura. Assim temos que escrever somente uma vez a especificação para um
    componente da linguagem sem recorrer a programação de do código do programa. Isso tem a vantagem
    de por der ser configurado pelo usuário final ao invés do programador, assim fica mais simples
    de configurar e expandir o conjunto de linguagens disponíveis ao processamento/beautifying.

    Softwares existentes e similares:

    \medskip
    \begin{bluebox}
    \begin{enumerate}[leftmargin=*]

        \item

        CodeBeautify is an online code beautifier which allows you to beautify your source code:
        \url{http://codebeautify.org/}.

        \item

        A universal code formatter, written in Dart: \url{https://pub.dartlang.org/packages/unifmt}.

        \item

        Google-java-format is a program that reformats Java source code to comply with Google Java
        Style: \url{https://github.com/google/google-java-format}.

        \item

        CodeFormatter is a Sublime Text 2/3 plugin that supports format (beautify) source code.
        \url{https://github.com/akalongman/sublimetext-codeformatter} and
        \url{https://github.com/aukaost/SublimePrettyYAML}

        \item

        UniversalIndentGUI offers a live preview for setting the parameters of nearly any indenter.
        You change the value of a parameter and directly see how your reformatted code will look
        like. Save your beauty looking code or create an anywhere usable batch/shell script to
        reformat whole directories or just one file even out of the editor of your choice that
        supports external tool calls: \url{http://universalindent.sourceforge.net/} and
        \url{https://github.com/danblakemore/universal-indent-gui}.

    \end{enumerate}
    \end{bluebox}


\subsubsection{Listagens}

    Algumas bibliotecas existentes, e potencialmente utilizadas como `syntect` para o auxílio na
    construção do produto de software:

    \begin{bluebox}
    \begin{enumerate}[leftmargin=*,parsep=0pt]

        \item \url{https://github.com/jbeder/yaml-cpp}
        \item \url{https://github.com/trishume/syntect}
        \item \url{https://github.com/onqtam/doctest}
        \item \url{https://github.com/c42f/tinyformat}
        \item \url{https://github.com/limetext/lime}
        \item \url{https://forum.sublimetext.com/t/disassembling-sublime-text/24824}

    \end{enumerate}
    \end{bluebox}

    Segue-se uma lista básica de formatters/beautifiers acessado no endereço
    \lword{\url{http://www.softpanorama.org/Utilities/beautifiers.shtml}} em março/2017:

    \medskip
    \begin{sloppypar}
    \begin{bluebox}\RaggedRight
    \begin{enumerate}[leftmargin=*,parsep=0pt]

        \item CB210.ZIP - C Beautifier 2.10 - polish C source code (19,406 bytes, 06/22/92)
        \item CL121.ZIP - Codelister 1.21 - print C code with stats (51,110 bytes, 01/10/94)

        \item CPC200.ZIP - CodePrint for C/C++ 2.00 is a full-featured command line driven source
        code reformatter and pretty printer for C++ and C; over 20 customization features including
        auto-indent, adjustable tab spacing, indent styles, flow lines, comment alignment, and line
        editing for consistent white space (140,605 bytes, 01/26/96)

        \item CSCOP120.ZIP - c-scope 1.20 analyzes C source code and produces various reports
        (48,505 bytes, 06/30/95)

        \item HTML : \url{http://www.digital-mines.com/htb/}
        \item HTML : \url{http://www.datacomm.ch/mwoog/software/perl/beautifier.html}
        \item HTML : \url{http://www.watson-net.com/free/perl/s_fhtml.asp}
        \item SQL : \url{http://www.netbula.com/products/sqlb}
        \item Oracle PLSQL : \url{http://www.revealnet.com}
        \item GPL \url{http://www.geocities.com/~starkville/vancbj.html}
        \item GPL \url{http://kevinkelley.mystarband.net/java/dent.html}
        \item Free \url{http://www.tiobe.com/jacobe.htm}
        \item Free \url{http://www.mmsindia.com/JPretty.html}
        \item Free \url{http://members.magnet.at/johann.langhofer/products/jxbeauty/overview.html} (has JBuilder support)
        \item Free \url{http://www.semdesigns.com/Products/Formatters/JavaFormatter.html}
        \item Commercial \$24.99 \url{http://smartbeautify.com}
        \item Commercial \$129 \url{http://www.jindent.com}
        \item Google \url{http://directory.google.com/Top/Computers/Programming/Languages/Java/Development_Tools/Code_Beautifiers/?tc=1}
        \item Java, SQL, HTML, C++ : \url{http://www.semdesigns.com/Products/DMS/DMSToolkit.html}
        \item Java JIndent \url{http://home.wtal.de/software-solutions/jindent}
        \item Java Pat \url{http://javaregex.com/cgi-bin/pat/jbeaut.asp}
        \item Java JStyle \url{http://www.redrival.com/greenrd/java/jstyle}
        \item Java JPrettyPrinter \url{http://www.epoch.com.tw/download/ms/java/java.htm}
        \item Java JxBeauty \url{http://members.nextra.at/johann.langhofer/download/jxbeauty} and the JxBeauty Home
        \item Java beautify percolator
        \item Java list \url{http://www.java.about.com/compute/java/library/weekly/aa102499.htm}
        \item Java html present VasJava2HTML
        \item Java code colorifier and beautifier \url{http://www.mycgiserver.com/~lisali/jccb}
        \item Perl : \url{http://www.consultix-inc.com/www.consultix-inc.com/talk.htm}
        \item Perl : \url{http://www.consultix-inc.com/www.consultix-inc.com/perl_beautifier.html}
        \item Fortran beautifier : \url{http://www.aeem.iastate.edu/Fortran/tools.html}

        \item C++ : BCPP site is at \url{http://dickey.his.com/bcpp/bcpp.html} or at \url{http://www.clark.net/pub/dickey}.
        BCPP ftp site is at \url{ftp://dickey.his.com/bcpp/bcpp.tar.gz}

        \item C++ : \url{http://www.consultix-inc.com/c++b.html}
        \item C : \url{http://www.chips.navy.mil/oasys/c/} and mirror at Oasys
        \item C++, C, Java, Oracle Pro-C Beautifier \url{http://www.geocities.com/~starkville/main.html}

        \item C++, C beautifier \url{http://users.erols.com/astronaut/vim/ccb-1.07.tar.gz} and site at
        \url{http://users.erols.com/astronaut/vim/#vimlinks_src}

        \item GC! GreatCode! is a powerful C/C++ source code beautifier Windows 95/98/NT/2000
        \url{http://perso.club-internet.fr/cbeaudet}

        \item C++ beautifier `SourceStyler' \url{https://web.archive.org/web/20061205061102/http://ochresoftware.com/}
        \item JavaScript : \url{http://jsbeautifier.org/}

    \end{enumerate}
    \end{bluebox}
    \end{sloppypar}


\subsubsection{Trabalhos Correlatos}

    Após a busca do que há de publicações científicas sobre o assunto e entra-se alguns trabalhos na
    área específica e similar aos trabalhos feitos pelor formatadores de códigos (Beautifiers).
    Nessa modalidade de trabalho, pode-se confundir-se com artigos que tratam sobre o `Prettyprint`,
    que trata-se de colorir o texto e exibir-lo ao usuário. O que não é o que se busca nesse
    trabalho, mas sim fazer alterações no texto sobre a forma como ele é estruturado, apresentado ao
    usuário e salvo em disco. Seguem as seguintes publicações:

    % How to add `parsep` to `itemsep` and set `parsep` to 0pt, when declaring my list?
    % https://tex.stackexchange.com/questions/366904/how-to-add-parsep-to-itemsep-and-set-parsep-to-0pt-when-declaring-my-list
    \begin{sloppypar}
    \begin{bluebox}\RaggedRight
    \begin{enumerate}[leftmargin=*,parsep=0pt]

    \item \url{https://www.researchgate.net/publication/228540036_An_industrial_application_of_context-sensitive_formatting}

    \item \url{http://www.suodenjoki.dk/us/archive/2010/cpp-checkstyle.htm}

    \item \url{http://www.basicinputoutput.com/2014/08/uncrustify-your-bios.html}

    \item \url{http://prettyprinter.de/}

    \item \url{https://github.com/ryanmaxwell/UncrustifyX}

    \item \url{http://www.softpanorama.org/Utilities/beautifiers.shtml}

    \item Understanding the Syntax Parsing
    \url{https://forum.sublimetext.com/t/understanding-the-syntax-parsing/28569}

    "So, part of what I've been working on is a code beautifier that, more or less, aligns and
    indents the code properly based on scanning through the source document."
    ...
    "It hasn't escaped my notice that this is to some degree exactly what the syntax file is doing."

    \item

    {\bfseries Towards a universal code formatter through machine learning:}
    In this paper, we solve the formatter construction problem using a novel approach, one that
    automatically derives formatters for any given language without intervention from a language
    expert. We introduce a code formatter called CODEBUFF that uses machine learning to abstract
    formatting rules from a representative corpus, using a carefully designed feature set. Our
    experiments on Java, SQL, and ANTLR grammars show that CODEBUFF is efficient, has excellent
    accuracy, and is grammar invariant for a given language. It also generalizes to a 4th language
    tested during manuscript preparation.
    \begin{enumerate}[nolistsep,topsep=0pt,label=$\star$]
        \item \url{http://dl.acm.org/citation.cfm?id=2997383}
        \item \url{http://homepages.cwi.nl/~jurgenv/papers/SLE16.pdf}
    \end{enumerate}

    \item \url{https://www.google.com/search?q=universal+source+code+formatter}
    \begin{enumerate}[nolistsep,topsep=0pt,label=$\star$]
        \item \url{https://www.google.com/search?q=universal+source+code+beautifier}
    \end{enumerate}

    \item \url{http://en.wikipedia.org/wiki/Indent_style}
    \begin{enumerate}[nolistsep,topsep=0pt,label=$\star$]
        \item \url{https://en.wikipedia.org/wiki/Programming_style}
        \item \url{https://en.wikipedia.org/wiki/Scope_(computer_science)}
    \end{enumerate}

    \item \url{http://wiki.c2.com/?CodingStyle}
    \begin{enumerate}[nolistsep,topsep=0pt,label=$\star$]
        \item \url{https://github.com/google/code-prettify}
        \item \url{https://github.com/uncrustify/uncrustify}
    \end{enumerate}

    \item \url{https://en.wikipedia.org/wiki/Prettyprint}
    \begin{enumerate}[nolistsep,topsep=0pt,label=$\star$]
        \item \url{https://www.researchgate.net/search.Search.html?query=formatting%20source%20code&type=publication}
        \item \url{https://www.researchgate.net/search.Search.html?query=pretty%20print%20source%20code&type=publication}
    \end{enumerate}

    \item \url{https://github.com/gchpaco/gopprint}
    \begin{enumerate}[nolistsep,topsep=0pt,label=$\star$]
        \item \url{http://dl.acm.org.sci-hub.io/citation.cfm?id=357115}
        \item \url{https://www.cs.indiana.edu/~sabry/papers/yield-pp.pdf}
    \end{enumerate}

    \item \url{http://www.worldcat.org/title/beautiful-code-a-customizable-code-beautifier-for-java/oclc/56564674}
    \begin{enumerate}[nolistsep,topsep=0pt,label=$\star$]
        \item \url{https://www.researchgate.net/publication/34736049_Beautiful_code_a_customizable_code_beautifier_for_Java}
        \item \url{https://vufind.carli.illinois.edu/vf-ncc/Record/ncc_118189/Holdings}
    \end{enumerate}

    \item \url{https://www.researchgate.net/publication/4283921_Smart_Formatter_Learning_Coding_Style_from_Existing_Source_Code}
    \begin{enumerate}[nolistsep,topsep=0pt,label=$\star$]
        \item \url{http://www.ing.unisannio.it/mdipenta/index.html}
        \item \url{https://github.com/iain/rspec-smart-formatter}
    \end{enumerate}

    \item \url{https://www.researchgate.net/publication/2543984_Source_Code_Files_as_Structured_Documents}
    \begin{enumerate}[nolistsep,topsep=0pt,label=$\star$]
        \item \url{https://en.wikipedia.org/wiki/SrcML}
    \end{enumerate}

    \item \url{https://www.researchgate.net/publication/228540036_An_industrial_application_of_context-sensitive_formatting}
    \begin{enumerate}[nolistsep,topsep=0pt,label=$\star$]
        \item \url{https://www.researchgate.net/publication/234809222_Program_indentation_and_comprehensibility}
    \end{enumerate}

    \end{enumerate}
    \end{bluebox}
    \end{sloppypar}


\subsubsection{Obfuscators}

    Aqui encontra-se o lado oposto dessas ferramentas, Source Code Obfuscators, que servem para
    destruir o visual do código. Usualmente utilizado para dificultar a leitura por outras pessoas
    ou ainda reduzir o tamanho de códigos de linguagens scripting que devem ser carregadas/baixadas
    por navegadores de internet, assim diminuindo o tráfego de internet e salvando/economizando
    largura de banda para download:

    \begin{sloppypar}
    \begin{bluebox}\RaggedRight
    \begin{enumerate}[leftmargin=*,parsep=0pt]

    \item \url{https://en.wikipedia.org/wiki/Obfuscation_(software)}

    \item \url{http://www.semdesigns.com/Products/Obfuscators/index.html}

    \end{enumerate}
    \end{bluebox}
    \end{sloppypar}

