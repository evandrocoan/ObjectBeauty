
%
% Licensing
%
%   Copyright 2017 @ Evandro Coan
%
%  This program is free software; you can redistribute it and/or modify it
%  under the terms of the GNU General Public License as published by the
%  Free Software Foundation; either version 3 of the License, or ( at
%  your option ) any later version.
%
%  This program is distributed in the hope that it will be useful, but
%  WITHOUT ANY WARRANTY; without even the implied warranty of
%  MERCHANTABILITY or FITNESS FOR A PARTICULAR PURPOSE. See the GNU
%  General Public License for more details.
%
%  You should have received a copy of the GNU General Public License
%  along with this program.  If not, see <http://www.gnu.org/licenses/>.
%

%
% Simple Sectioned Essay Template - LaTeX Template
%
% This template has been downloaded from:
% http://www.latextemplates.com
%
% `proposal.tex`
% Based on http://www.latextemplates.com/template/simple-sectioned-essay
%

%----------------------------------------------------------------------------------------
%   PACKAGES AND OTHER DOCUMENT CONFIGURATIONS
%----------------------------------------------------------------------------------------

% Sets this document font size to 12pt and set is type to article
\documentclass[12pt]{article}

% https://tex.stackexchange.com/questions/664/why-should-i-use-usepackaget1fontenc
\usepackage[T1]{fontenc}

% Sets this document language usage to Portuguese
\usepackage[brazil]{babel}

% Required to change the page size to A4
% \usepackage{geometry}
%
% Sets this document output papel to A4 within 2 centimeters margin
% \geometry{a4paper}
%
% Set the page size to be A4 as opposed to the default US Letter
\usepackage[a4paper, margin=2cm]{geometry}

% Sets the character set of this document to UTF-8
\usepackage[utf8]{inputenc}

% For web links and paths with \path{..} and \url{https://www.python.org/downloads/}
\usepackage{hyperref}

% To use the font Times New Roman, instead of the default LaTeX font
% \usepackage{mathptmx}
%
% more up-to-date than 'mathptmx'
\usepackage{newtxtext,newtxmath}

% For the new command \latex
\usepackage{xspace}

% https://tex.stackexchange.com/questions/10377/texttt-overfull-hbox-problem
% https://tex.stackexchange.com/questions/66052/should-i-load-microtype-with-pdflatex
\usepackage{microtype}

% Indent the first section paragraphs
% https://tex.stackexchange.com/q/39227/119062
\usepackage{indentfirst}

% \lettrine{O}{nce} upon a time...
% \lettrine[findent=2pt]{\fbox{\textbf{T}}}{ }his thesis deals with...
%
% https://tex.stackexchange.com/questions/164298/starting-a-paragraph-with-a-big-letter
\usepackage{lettrine}

% Required for including pictures, resizebox
\usepackage{graphicx}

% Allows putting an [H] in \begin{figure} to specify the exact location of the figure
\usepackage{float}

% Allows in-line images such as the example fish picture
\usepackage{wrapfig}

% Specifies the directory where pictures are stored
\graphicspath{{Pictures/}}

% How to automatically force latex to not justify the text when it is not wise?
% https://tex.stackexchange.com/questions/365801/how-to-automatically-force-latex-to-not-justify-the-text-when-it-is-not-wise
\usepackage{array,ragged2e}

% Use its macro adjustwidth* to extend tables out of outer text border.
% https://tex.stackexchange.com/questions/366155/how-to-write-a-table-a-little-larger-than-the-paragraphs-with-centered-columns
\usepackage[strict]{changepage}

% No spacing between enumerated items with \usepackage{enumerate}
% https://tex.stackexchange.com/questions/119919/no-spacing-between-enumerated-items-with-usepackageenumerate
\usepackage[shortlabels]{enumitem}

\usepackage{tabularx}
\usepackage{multirow}


%
% New Macros
%

% Writing code in latex document. Usage: \begin & \end {lstlisting}
% http://stackoverflow.com/questions/3175105/writing-code-in-latex-document
\usepackage{listings}

% Incompatible color definition when using tikz with color package
% https://tex.stackexchange.com/questions/150369/incompatible-color-definition-when-using-tikz-with-color-package
\usepackage{xcolor}

\definecolor{dkgreen}{rgb}{0,0.6,0}
\definecolor{gray}{rgb}{0.5,0.5,0.5}
\definecolor{mauve}{rgb}{0.58,0,0.82}

\lstset{frame=,
  language=Java,
  aboveskip=3mm,
  belowskip=3mm,
  showstringspaces=false,
  columns=flexible,
  basicstyle={\small\ttfamily},
  numbers=left,
  numberstyle=\color{gray},
  keywordstyle=\color{blue},
  commentstyle=\color{dkgreen},
  stringstyle=\color{mauve},
  breaklines=true,
  breakatwhitespace=true,
  tabsize=3
}

% Change background color for text block
% https://tex.stackexchange.com/questions/238294/change-background-color-for-text-block
\usepackage{framed}
\usepackage[most]{tcolorbox}
\definecolor{shadecolor}{RGB}{219, 229, 241}
\newtcolorbox{myquote}{
colback=shadecolor,
grow to right by=-2mm,
grow to left by=-2mm,
boxrule=0pt,
boxsep=0pt,
breakable
}

% Automatically put a `\medskip` spacing between paragraphs
% https://tex.stackexchange.com/q/365976/119062
\edef\restoreparindent{\parindent=\the\parindent\relax}
\usepackage{parskip}
\restoreparindent

% Uncomment to remove all indentation from paragraphs
%\setlength\parindent{0pt}

% How could the `\everypar` justification statement be used?
% https://tex.stackexchange.com/questions/365818/how-could-the-everypar-justification-statement-be-used
\newbox\linebox \newbox\snapbox
\def\eatlines{
  \setbox\linebox\lastbox % check the last line
  \ifvoid\linebox
  \else % if it’s not empty
    \unskip\unpenalty % take whatever is
    {\eatlines} % above it;
    \setbox\snapbox\hbox{\unhcopy\linebox}
    \ifdim\wd\snapbox<.98\wd\linebox
       \box\snapbox % take the one or the other,
    \else \box\linebox \fi
  \fi
}

% How could the `\everypar` justification statement be used?
% https://tex.stackexchange.com/questions/365818/how-could-the-everypar-justification-statement-be-used
\everypar={\setbox0=\lastbox \par
   \vbox\bgroup \everypar={}\def\par{\endgraf\eatlines\egroup}}

% Creates a new environment which can be used as:
%
% \begin{foo}
%   Text...
%
%   Text ...
% \end{foo}
%
% https://tex.stackexchange.com/questions/62333/push-long-words-in-a-new-line
\newenvironment{foo}
{\par
\hyphenpenalty=10000
\exhyphenpenalty=10000
}
{\par}

% Underfull \hbox in bibliography
% https://tex.stackexchange.com/questions/10924/underfull-hbox-in-bibliography
\apptocmd{\thebibliography}{\raggedright}{}{}


%
% New commands
%

% Line spacing
\linespread{1.5}

% Allow to push long words on new lines when they do not fit entirely on the current line.
% https://tex.stackexchange.com/questions/62333/push-long-words-in-a-new-line
\newcommand\lword[1]{\leavevmode\nobreak\hskip0pt plus\linewidth\penalty50\hskip0pt plus-\linewidth\nobreak{#1}}

% Write the word LaTeX nicely.
\newcommand{\latex}{\LaTeX\xspace}

% Create a bold title all in upper case.
\newcommand{\Title}[1]{\textbf{\MakeUppercase{#1}}}


%----------------------------------------------------------------------------------------
%   DOCUMENT CONTENTS
%----------------------------------------------------------------------------------------

\begin{document}

% pdfTeX warning (ext4): destination with the same identifier (nam e{page.1}) has been already used, duplicate ignored
% https://tex.stackexchange.com/questions/18924/pdftex-warning-ext4-destination-with-the-same-identifier-nam-epage-1-has
\hypersetup{pageanchor=false}




% Author name
%
% What do \makeatletter and \makeatother do?
% https://tex.stackexchange.com/questions/8351/what-do-makeatletter-and-makeatother-do
%
% How can I use @author, @date, and @title after maketitle?
% https://tex.stackexchange.com/questions/27710/how-can-i-use-author-date-and-title-after-maketitle
\makeatletter
\author{Evandro Coan}\let\Author\@author
% \author{Someone}          \let\Author\@author
% \date{Somewhen}           \let\Date\@date
\def\Advisor{Thiago Ângelo Gelaim}
\def\Supervisor{Ricardo Azambuja Silveira}
\makeatother



\begin{titlingpage}

    \center

    \Title{Universidade Federal de Santa Catarina - UFSC}

    \Title{Ciência Da Computação}

    \vspace*{\stretch{1}}

    \Author

    \vspace*{\stretch{2}}

    \Title{Boas Práticas de Programação \& Estilo}

    \bigskip
    Ferramentas Universais de Programação\\[3cm]

    \begin{flushright}

        \begin{minipage}{0.518\textwidth}

            Proposta de Trabalho de Conclusão de Curso,
            a ser submetido ao Curso de Ciência da Computação
            para a obtenção do Grau de Bacharel em Ciência da \lword{Computação}.

            \medskip
            {\bfseries Orientador:} \hfill \Advisor

            \medskip
            {\bfseries Professor Responsável:} \hfill \Supervisor

        \end{minipage}

    \end{flushright}

    \vspace*{\stretch{3}}

    Florianópolis, \today.

\end{titlingpage}










\begin{abstract}

    Faz-se um estudo sobre o que é, para que servem os Beautifiers, assim como abordagens sobre o
    que são boas práticas de programação e por que devemos segui-las para um boa eficiência ao
    escrever códigos nas mais diversas linguagens de programação. Os softwares formatadores de
    código fonte atuais, também conhecidos como Beautifiers, são limitados a um conjunto similar, ou
    mesmo à uma única linguagem, e além de muitos, serem limitados ao que eles podem fazer por você
    ao processar/formatar o código \cite{Terence}.

    \medskip

    Portanto espera-se o final do trabalho, conhecer-se quais são as ferramentas que existem e quais
    delas são as melhores que podem ser utilizadas para o auxilio do programador durante a escrita
    de códigos das mais diversas linguagens de programação. Além de proposta de uma nova ferramenta
    com o intuído de centralizar em uma único programa o abordagem das mais diversas linguagens de
    programação.

    \medskip

    \textbf{Palavras-chave:}
    source, code, formatter, beautifier, prettyprint, universal, reuse, blocks, object, oriented,
    programming, structured, parsing, parse, regular, expression, regex, C, C++,  grammar, Turing,
    machine, automata, lexer, syntax, sublime, Java, Rust, shell, script, obfuscators, learning,
    syntec, teamicide, concensus, indent, settings.

\end{abstract}






\include{summary}


% The \phantomsection command is needed to create a link to a place in the document that is not a
% figure, equation, table, section, subsection, chapter, etc.
%
% When do I need to invoke \phantomsection?
% https://tex.stackexchange.com/questions/44088/when-do-i-need-to-invoke-phantomsection
\phantomsection


% Is it possible to keep my translation together with original text?
% https://tex.stackexchange.com/questions/5076/is-it-possible-to-keep-my-translation-together-with-original-text
\chapter{\lang{Introduction}{Introdução}}

The work first part will be based on research in
articles, books, theses, dissertations, trusted authors websites,
and through new demonstrated evidences based on arguments
in the monograph evolution.
Also, present results after building a new tool
which proposes a solution for the problems presented and detailed.

In this proposal last chapter which lies on the part
called `\nameref{sec:software_implementation}', which holds the implementation of a
tool for code `Beautifying'.



\section{Context}

Questions like ``What are good programming practices?'' Or ``Why are these
practices are good?''Are not easy to answer. But each programmer learns to
write their codes in a certain way, with certain features like using 4 or 8
spaces to indent lines, always leave a blank line before each control
structure as if or for statements, and alike rules.
\cite{naturalCodingConventions}

But entering the universe of good practices, there are many things for
discoursing. Nonetheless, in this work is presented the implementation of
tool called `Object Beautifier', which specifically dedicates on how to
perform the best layout/display of programming code on the computer screen,
so that maximize and facilitate the understanding of same
\cite{automaticSynthesis}.
Therefore, allowing the programmer to disperse
more time and efforts thinking about its coding algorithms problem,
other than trying to decipher the information that is presented
to it on the screen through infinit different code layouts
\cite{usingVersionControlData}.

Within this work\s area, we need to also think long and hard about how to
share the programming code of the programmers among you. Now, the problem of
human diversity, like all big scientific questions -- how do you explain
something like that -- It can be broken down into sub-questions. It happens
many times, which is a good practice for a `Programmer A', is not the same
to another `Programmer B'. For example, imagine some code where a programmer
decided to put before each `if' statement, a blank line. It is therefore
expected that whenever we see a blank line we can potentially find a
matching `if', which can be considered a quite useful pattern matching as
empty line may call better your attention. \cite{aPrettyGoodFormatting}

But again this is something heavily dependent of what each one learning
through their life time. Imagine another programmer do not liked this rule,
and when he was writing your code involving an `if', he did not put such
blank line another programmer is expecting. So when the first programmer
start reading its the code and look for `if', he will be expecting for blank
lines before its if\s. But will lose some time searching until realize
another programmer does not put them, or perhaps he forgot to insert them.
\cite{quantifyingProgramComprehension}

These differences are due to the diversity of ways we learn programming,
i.e., to the ways we are used to doing coding, as much as the abilities and
objectives of every programmer developed. Hence, nowadays it becomes a big
problem because we increasingly need more and more programmers working
together developing several and diverse computing systems. Where the latter
is due to the fact of the complexity of computer systems growing
increasingly, therefore over requiring programmers working and sharing their
codes and ideas. \cite{howProgrammersRead}

Moreover besides only worrying about how the code is displayed on their
computer screen, we need to worry about on how it will be saved in the file
system on its `plain-text' mode. Since for code sharing, it is vital for you
to use a versioning control system\footnote{\url{http://www.codeservedcold.com/version-control-importance/}}
which enable project manager\s and
programmers themselves, take control of their code changes
\cite{redesignOfGit}. It does allow to easily perform the tracking of code
changes \cite{gettingProductive} and
allow you to better understand what each programmer is doing
every time he formalizes a change in the code through a `commit', as in
`git' systems for example. \cite{usingSourceControl}

\begin{citacao}
I'd say there are two main reasons to enforce a single code format in a project. First has
to do with version control: with everybody formatting the code identically, all changes in
the files are guaranteed to be meaningful. No more just adding or removing a space here or
there, let alone reformatting an entire file as a `side effect' of actually changing just a
line or two. \cite{Geukens}
\end{citacao}

That is because while working with a versioning system like `git', we need
to keep the code among a single style or which we may call a `good practice'
set as standard for everybody, due the fact of letting each programmer to
write as he pleases, there will be plenty of noise on the code review and we
are figuring out what actually each programmer did \cite{quitDiffCalculating}.
Hence, if every programmer re-writes the history making changes
like inserting new lines
before each if, we end up with too much noise and focus of a versioning
system is to look at only those changes that are significant to the code,
such as the creation of new functions and not the addition of new blank
lines. \cite{findingRegressionsInProjects}

Talking about the last ideia pointed out, we could also think about an
approach to creating a new version control system which focuses only on
significant changes to the code, while reviewing code changes. However, this
approach could not be ideal, as for example, it would allow programmers to
start tedious wars of unproductive code adjustments. For example, imagine
how it would be for your every day and have to go through your code
re-adding new lines before each one of your beloved if\s, just because some
night shift programmer\footnote{\url{https://blog.codinghorror.com/who-wrote-this-crap/}}
had just removed them?



\section{Research Goals}

Beforehand due the scope limitation for a Graduation Thesis,
we should only think about a basic, simple,
and yet reusable core of features.

\begin{enumerate}
    \item A Software Product with a great Object Orientation and possibilities of extension of features,
    decent research on the state of the art.

    \item Ranking all code formatting classes (beautifying) applicable.
    Including a study on what does is source beautifying,
    how to do such and why.

    \item Establish relationships between good programming practices and efficiency in programming,
    in addition to a new tool to support programmers in order to automate the long and diverse
    programming process in teams of developers with different programming `best practices'.

    \item Define, determine and classify which one are good programming practices and
    perform an in-depth study on the good practices on visual layout area,
    also known as code `Beautifying'.

    \item The definition of a flow pattern of development allowing teams of
    developers with different programming best practices,
    to work without intervene with each other up to start wars of `best good practices'.

    \item Discourse on the variety of existing tools for the support of good programming practices,
    with a comparative analysis between them,
    determining their weaknesses and strengths.
\end{enumerate}



\section{Implementation Goals}

Propose a unique tool that allowing several and distinct
programming `best practices' being implemented in several programming
languages, which can be configured and set accordingly to their wishes,
from a single software working well behaved across all programming languages.

Moreover, explain the differences for other softwares and the benefits
of a unique tool, instead of several heavily different ones.

From this point, a sketch is presented on the problem, solutions,
information as for why to want make such software, or even why do we want to
beautifying things:

\begin{enumerate}[leftmargin=*]
    \item There are many different tools, sometimes paid, and difficult to
          complete. \cite{universalCodeFormatter}

    \item Many programming languages exist, so always having Beautifier
          software for each of them is very laborious
          \cite{universalCodeFormatter}. But the approach to a Universal
          Beautifier proposed in this work, would allow easily new languages to be
          added, being completely different from previous ones, or alike. And in
          case of similarities between them, it is enough to reuse their
          configuration structures already implemented.

    \item Looking for a Beautifier for each one of them because programmers
          currently work daily with several of these languages, and they are not
          similar. So you need to configure several beautifiers to do the
          formatting. This is a problem because only a few beautifiers are more
          complete, and every time you need to make a change in the formatting
          style, you must manually propagate the same change over several
          different program configuration files, which is bad because it takes the
          user a lot of time to learn how to handle many different types of
          settings. \cite{Schweitzer}

    \item In the case of ideal Beautifier, a change in your styling is
          propagated to all languages. And if you want to leave some language out
          of it, you just need to remove it from the list on which the
          configuration block applies to, and `a)' leave it out so no change is
          applied to. Or `b)' create a new block including only the block within
          the desired settings.

\end{enumerate}

The difference from this proposal to remaining formatting tools,
is the tradeoff between end\hyp{}users and developers responsibilities.
Most tools rarely expose to end\hyp{}users their language syntax specification,
in contrast,
this proposal completely exposes the language to the end\hyp{}user as simple plain\hyp{}text,
not requiring the tool to know any language syntax neither semantics.
Moreover,
with no syntax knowledge required,
the tool be can used with any languages their user wishes to.



\section{Related Works}


\url{http://editorconfig.org/}





\section{Objetivos}

    O objeto neste trabalho de TCC proposto aqui não é inicialmente suportar todas as regras de
    formatação de todas as linguagens de programação, mas a criação de uma estrutura básica inicial
    e robusta que sejam capaz de ser desenvolvida a ponto de ser facilmente expandida, tanto na
    adição de novos módulos de processamento no programa escrito, tanto pelo usuário final na
    escrita dos arquivos de programação.



\subsection{Objetivos Gerais}

    \begin{enumerate}

        \item

        Escrever o programa em C++ ou afins, para permitir também que a formação/beautifying seja
        (em trabalhos futuros/talvez nesse) dinâmico, isto é, na medida que você digita o texto, ele
        é formatado para você. Assim você pode focar mais em escrever o código, ao invés que se
        preocupar com o espaçamento, alinhamento, parenteses, linhas novas, e o que mais que seja.

        \item

        Utilizar o Framework `doctest` para escrita dos Testes de Unidade. Pois após procurar e
        testar alguns frameworks para testes de unidade em C++, entrou-se este como servindo muito
        bem as requisitos do projecto. Ele causa baixíssimo incremento no tempo de compilação e
        permite que os testes possam ser escritos no mesmo arquivo onde encontram-se o código do
        programa, sem que eles sejam compilados.

        \item

        Utilizar uma versão/algoritmo multi-core, assim cada uma das regras pode ser processada em
        paralelo e sobre o mesmo source code original. Essa parte é bastante complexa de ser escrita
        por que as regras entre si podem gerar conflitos sobre o que elas estão fazendo. Assim para
        resolver esse problema, fazer com que cada regra processada gere um objeto de mudanças que
        essa regra está propondo. Assim no final do processamento de todas as regras, será realizado
        um fusão das mudanças que cada uma decidiu realizer, e caso duas regras queriam mudar o
        mesmo pedaço/trecho de código, será lançada um exceção e uma nova classe de mudanças/regra
        deve estar disponível para resolver esse conflito. Caso não exista, ambas as mudanças são
        descartadas e somente as mudanças sem conflitos são refletidas no código.

    \end{enumerate}









\section{Método de pesquisa}

\hspace*{5mm}

        O trabalho...









\section{Cronograma}

   \begin{tabularx}{\linewidth}{|X|*{11}{c|}}
        \hline
        \multicolumn{1}{|c|}{\multirow{2}{*}{Etapas}} & \multicolumn{11}{|c|}{Meses}\\ \cline{2-12}

        & fev & mar & abr & mai & jun & jul & ago & set & out & nov & dez  \\ \hline

        Estudo da fundamentação teórica necessária
        &  x  &  x  &  x  &     &     &     &     &     &     &     &     \\ \hline

        Estudo da implementação de agentes na educação
        &     &  x  &  x  &     &     &     &     &     &     &     &     \\ \hline

        Estudo da ferramento Unity
        &     &     &  x  &  x  &  x  &  x  &  x  &     &     &     &     \\ \hline

        Escrita de um esboço do TCC
        &     &     &     &     &  x  &  x  &  x  &  x  &     &     &     \\ \hline

       Estudo da ferramenta Middler VR
        &     &     &     &     &     &  x  &  x  &     &     &     &     \\ \hline

        Finalização da escrita do TCC
        &     &     &     &     &     &     &     &     &  x  &     &     \\ \hline

        Ajustes finais no texto do TCC
        &     &     &     &     &     &     &     &     &  x  &  x  &     \\ \hline

        Defesa do TCC
        &     &     &     &     &     &     &     &     &     &     &  x  \\ \hline

    \end{tabularx}



\section{Custos}

    \begin{tabular}{l l l l}

        \hline

        Item                    &   Quantidade  &   Valor Unitário (R\$)    &   Valor Total (R\$) \\

        \hline

        Impressão               &   200         &   0,10                    &   20,00             \\

        Programador             &   1           &   1500,00                 &   1500,00           \\
        Notebook                &   1           &   2500,00                 &   2500,00           \\
        Desktop                 &   1           &   3000,00                 &   3000,00           \\
        Reserva de contingência &   1           &   4000,00                 &   4000,00           \\

        \hline

        Total                   &               &                           &   11020,00

    \end{tabular}



\section{Recursos Humanos}
    \begin{tabular}{l l}
        \hline
        Nome                            & Função                  \\
        \hline
        Luiz Filipe Moresco da Silva    & Autor                   \\
        Ricardo Azambuja Silveira       & Orientador              \\
        Renato Cislaghi                 & Coordenador de Projetos \\
        Thiago Ângelo Gelaim            & Coorientador            \\
        \hline
    \end{tabular}
    \\



\section{Comunicação}
    \begin{tabular}{l l l l}
        \hline
        O quê  & De quem & Para Quem & Como \\
        \hline
        Proposta de TCC         & Autor     & Renato Cislaghi   & Site de projetos \\
        Relatório de TCC I      & Autor     & Renato Cislaghi   & Site de projetos \\
        Prévia do TCC, em TCC I & Autor     & Banca             & E-mail \\
        Defesa do TCC           & Autor     & Banca             & Pessoalmente \\
        Reunião de Orientação   & Orientadores  & Autor         & Pessoalmente \\
        \hline
    \end{tabular}


\section{Riscos}

    % https://tex.stackexchange.com/questions/366156/how-to-change-the-left-padding-for-one-latex-tables-cell
    % https://tex.stackexchange.com/questions/366155/how-to-write-a-table-a-little-larger-than-the-paragraphs-with-centered-columns
    %
    \begin{adjustwidth}{-0.5\marginparwidth}{-0.5\marginparwidth}
    \small
    \begin{tabularx}{\linewidth}
    {|
        *1{                 >{\RaggedRight\arraybackslash\hsize=1.1\hsize }X       |} % Riscos
        *3{@{\hspace{3.0pt}}>{\Centering\arraybackslash                   }p{0.9cm}|} % Probabilidade, Impacto, Prioridade
        *2{                 >{\RaggedRight\arraybackslash\hsize=0.95\hsize}X       |} % Resposta, Prevenção
    }

    \hline Riscos  & Pro\-ba\-bi\-li\-da\-de & Im\-pac\-to & Prio\-ri\-da\-de & Es\-tra\-té\-gia de res\-pos\-ta & Ações de pre\-ven\-ção \\ \hline

    % Row 1
    % Riscos
    \hline Problemas com perda de dados &
    % Probabilidade
    Baixa &
    % Impacto
    Alto &
    % Prioridade
    Alta &
    % Estratégia de resposta
    Uso do backup &
    % Ações de prevenção
    Backup periódicos \\ \hline

    % Row 2
    % Riscos
    \hline Alteração do cronograma ou descontinuidade do projeto onde recebo uma bolsa &
    % Probabilidade
    \rlap{Média} &
    % Impacto
    Alto &
    % Prioridade
    Alta &
    % Estratégia de resposta
    Redefinição da data de entrega do trabalho &
    % Ações de prevenção
    Monitoramento contínuo das informações obtidas com superiores imediatos \\ \hline

    \hline \end{tabularx}
    \end{adjustwidth}






\bibliographystyle{abbrv}
\bibliography{refs}


\end{document}
