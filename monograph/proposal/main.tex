% proposal.tex
% Based on http://www.latextemplates.com/template/simple-sectioned-essay
\documentclass[12pt]{article}

\usepackage[utf8]{inputenc}
\usepackage[T1]{fontenc}
\usepackage[a4paper, margin=2cm]{geometry}
\usepackage[brazil]{babel}


\usepackage{graphicx} % resizebox
\usepackage{multirow}
\usepackage{tabularx}
\usepackage{hyperref}
\usepackage{cite}

\usepackage{mathptmx}
\linespread{1.2} % Line spacing

\newcommand{\Title}[1]{\textbf{\MakeUppercase{#1}}}

\begin{document}

\begin{titlepage}

    \center

    \Title{Universidade Federal de Santa Catarina - UFSC}

    \Title{Ciência Da Computação}

    \vspace*{\stretch{1}}

    Luiz Filipe Moresco da Silva

    \vspace*{\stretch{2}}

    \Title{Agentes e Realidade Virtual: Uma abordagem de ensino}\\[3cm]

    \begin{flushright}

        \begin{minipage}{0.518\textwidth}

            Proposta de Trabalho de Conclusão de Curso,
            a ser submetido ao Curso de Ciência da Computação
            para a obtenção do Grau de
            Bacharel em Ciência da Computação.

            \medskip
            Orientador: Prof. Dr. Ricardo Azambuja Silveira
            Coorientador: Thiago Ângelo Gelaim

        \end{minipage}

    \end{flushright}

    \vspace*{\stretch{3}}

    Florianópolis, \today.

\end{titlepage}


\begin{abstract}

    Com o avanço da tecnologia em vários setores que presenciamos, cada vez mais
    está sendo pesquisado sobre como incluir a tecnologia em escolas, para motivar e facilitar o
    aprendizado dos alunos. Um dos temas pesquisados atualmente é como ensinar um dado conteúdo para uma
    turma de alunos, de forma que, seja possível encontrar as dificuldades dos mesmos e nivelar a turma
    sempre que necessário. A realidade virtual existe há mais de vinte anos, mas era conhecida e viável
    apenas para empresas e instituições de ensino, que tinham computadores com desempenho considerável
    para a tal tecnologia \cite{vr_historia}. Com o avanço da tecnologia nos últimos anos, a realidade
    virtual vem sendo usufruída também por usuários finais, sendo possível explora-la para criar
    aplicações atrativas e facilitar o ensino, principalmente para crianças.

    \medskip

    Como a realidade virtual é novidade, se torna fácil cativar a atenção de indivíduos, lhe mostrando
    mais detalhe palpáveis e passando o conhecimento, que em muitos casos se torna difícil por se
    distraírem facilmente. Então, nesse trabalho será apresentado como unir realidade virtual e agentes
    com foco no ensino de indivíduos. Os agentes irão guiar o aprendizado do individuo utilizando
    realidade virtual para atrair o mesmo, mostrando o quão divertido pode ser estudar os mais diversos
    assuntos. Com a união das tecnologias que estão em alta, será possível mostrar para os indivíduos
    novos cenários e/ou aplicações durante a explicação de um determinado assunto, assim diminuindo
    abstrações, que em muitos casos é necessário concretizar ideias para melhor entendimento.

    \textbf{Palavras-chave:} agentes, realidade virtual, educação

\end{abstract}

\newpage
\tableofcontents
\newpage

\section{Introdução}

    \hspace*{5mm}

    Realidade virtual é uma tecnologia de interface avançada entre um usuário e um
    software, com o objetivo dessa recriar ao máximo a sensação de realidade para um indivíduo,
    levando-o a adotar essa interação como uma de suas realidades temporais \cite{livro_IA_BU}. Essa
    interação é realizada em tempo real, com o uso de técnicas e de equipamentos computacionais que
    ajudam na ampliação do sentimento de presença do usuário. Em suma, uma realidade ficcional, contudo
    através de relações intelectuais, a compreendemos como sendo muito próxima do universo real que
    conhecemos \cite{livro_IA_BU}. Para usufruir da realidade virtual, é necessário ter um dispositivo,
    onde um indivíduo o veste, e entra em imersão. Um dispositivo muito conhecido, é o óculos Rift que
    em conjunto com um smartphone que possui giroscópio, já é possível ter uma nova experiência. Existe
    também o CAVE (\textit{Cave Automatic Virtual Environment}), que é um sala onde são projetados
    gráficos em 3 dimensões, em suas paredes, e os usuários utilizam óculos 3D para poder visualizar-
    los, podendo então, explorar e interagir com objetos, desta forma mergulhando em um mundo
    virtual.\cite{caverna}

    \medskip
    Realidade aumentada é a sobreposição de objetos e ambientes virtuais com o ambiente físico, através
    de algum dispositivo tecnológico. Com ofato dos objetos virtuais serem trazidos para o espaço físico
    do usuário (por sobreposição) permitiu interações tangíveis mais fáceis e naturais, sem o uso de
    equipamentos especiais. Por isso, a realidade aumentada vem sendo considerada uma possibilidade
    concreta de vir a ser a próxima geração de interface popular, ainda está engatinhando, mas já existe
    muitas aplicações, na medicina por exemplo, é usada para auxiliar no ensino \cite{aplicacoes_VR}. É
    possível ser usada nas mais variadas aplicações em espaços internos e externos
    \cite{realidadevirtual_e_realidadeaumentada}. Uma aplicação comum, são os aplicativos móveis
    desenvolvidos com a plataforma Vuforia, que é uma das plataformas utilizadas para desenvolvimento de
    qualquer tipo de aplicação de realidade aumentada, independentemente do dispositivo ou meio pelo
    qual a ferramenta será executada \cite{realidadevirtual_e_realidadeaumentada}, por exemplo, permite
    que os aplicativos móveis totalmente desenvolvidos para iOS e Android desfrutem da realidade
    aumentada.\cite{site_vuforia}

    \medskip
    A definição de inteligencia artificial segue duas linhas de pensamento entre os autores de
    livros didáticos. Segundo Haugeland, Bellman entre outros, inteligencia artificial é o 'estudo
    do processo de pensamento e raciocínio', onde `sistemas pensam como seres humanos' ou até mesmo
    `automatização de atividades que associamos ao pensamento humano'. Seguindo ponto de vista de
    Kurzweil, inteligencia artificial é `estudo do comportamento', ou seja, 'sistemas atuam/se
    comportam como seres humanos' e/ou `sistemas que atuam racionalmente' \cite{livro_IA_BU}.
    Inteligencia Artificial, pode ser utilizada em vários cenários, como no trânsito, mostrando qual
    é o melhor horário para ir até um determinado local, no clima, informando a temperatura de
    próximas horas, e também, na educação, podendo auxiliar indivíduos em seu aprendizado. Em
    inteligencia artificial existe os agentes, que podem fazer essas tarefas, tanto o trabalho de
    percepção, quanto o trabalho de ação, que em muitos casos, é gerar um alerta.

    \medskip
    Um agente em inteligencia artificial, é tudo que que pode ser considerado capaz de perceber seu
    ambiente por meio de sensores, e de agir sobre esse ambiente por intermédio de atuadores
    \cite{livro_IA_BU}. Um agente por exemplo, pode ter como sensor, uma câmera, um medidor de
    luminosidade, um medidor de umidade, entre outros. Já como atuadores, pode ser um ou mais motores,
    um braço mecânico etc. O funcionamento básico de um agente é, perceber algum comportamento através
    dos seus sensores, interpretar as percepções, e agir de alguma forma, ligando um lampada e/ou
    disparando um alarme, por exemplo. A ação executada pelo agente irá depender das percepções obtida
    e do conhecimento que o agente tem em relação as mesmas, então, quanto mais informações o agente
    conseguir obter, maior a chance do mesmo agir corretamente.

    \medskip
    Um dos paradigmas de desenvolvimento de sistema multi-agente é conhecido como BDI, esse paradigma
    possibilita modelar o conhecimento baseado em estados mentais, semelhante ao raciocínio prático. O
    raciocínio descrito através da lógica formal aplicada na descrição de estados mentais, é a base
    para ações que selecionam um conjunto de desejos em função da crença dos agentes e como esses
    desejos concretos produzidos, como resultado do passo anterior, podem ser atingidos empregando
    meios que o agente dispõe. Os estados mentais no modelo BDI podem ser definidos como:
    \cite{livro_IA_BU}

    \begin{itemize}

        \item Crenças: representam as características do ambiente, segundo o ponto de vista do
        agente, as quais são atualizadas apropriadamente após a percepção de cada ação;

        \item Desejos: contêm informação sobre os objetivos a serem atingidos, bem como as
        prioridades e os custos associados com os vários objetivos;

        \item Intenções representam o atual plano de ação escolhido para atingir objetivos. Os
        estados mentais capturam o componente deliberativo dos agentes e determinam o seu comportamento.

    \end{itemize}

    A junção de agentes e realidade virtual, pode-se criar novos produtos de auxilio há indivíduos.
    Os agentes farão o trabalho de guiar o indivíduo, lhe mostrando exercícios e definições de um
    determinado assunto conforme seu nível de conhecimento, e a realidade virtual, proporcionando um
    nova experiência, onde é possível detalhar a explicação do corte de cônicas estudado em
    geometria analítica, por exemplo.

\newpage
\section{Objetivos}

\subsection{Objetivo geral}

   \hspace*{5mm}Com um alto índice de dificuldades por parte de crianças, jovens e adultos em
    aprender um determinado conteúdo em sua jornada escolar, esse trabalho tem como objetivo auxiliar no
    aprendizado de qualquer indivíduo, mapeando suas dificuldade e o guiando até chegar em seus
    objetivos. Através de uma aplicação, o indivíduo será guiado conforme seu conhecimento prévio
    perante há um dado conteúdo e sua progressão ao longo de seu estudo.

\subsection{Objetivos específicos}
    \hspace*{5mm}

    É necessário dividir por dificuldade cada objetivo, pois alguns deles, terá uma implementação
    mais crua, pois o tempo para desenvolvimento do trabalho é limitado.

    \begin{itemize}
        \item Alta complexidade, onde definir o conhecimento do aluno, não apenas aplicando uma formula.
        \subitem Determinar um dado exercício como fácil, médio ou difícil
        \subitem Determinar o nível de entendimento de um individuo em relação ao um conteúdo
        \item Média complexidade, pois é o desenvolvimento de uma aplicação usando agentes e realidade virtual
        \subitem Implementar software com suporte a realidade virtual
        \subitem Implementar agentes que para guiar o individuo em um dado conteúdo
    \end{itemize}

\newpage
\section{Cronograma}

   \begin{tabularx}{\linewidth}{|X|*{11}{c|}}
        \hline
        \multicolumn{1}{|c|}{\multirow{2}{*}{Etapas}} & \multicolumn{11}{|c|}{Meses}\\ \cline{2-12}
        & fev & mar & abr & mai & jun & jul & ago & set & out & nov & dez   \\ \hline

        Estudo da fundamentação teórica necessária
        &  x  &  x  &  x   &     &      &      &      &      &      &     &      \\ \hline

        Estudo da implementação de agentes na educação
        &     &  x  &  x  &     &      &      &      &      &      &     &     \\ \hline

        Estudo da ferramento Unity
        &     &    &  x  &  x  &  x  &  x  &  x   &     &     &     &   \\ \hline

        Escrita de um esboço do TCC
        &     &     &     &     &  x & x  &  x  &   x   &    &    &     \\ \hline

       Estudo da ferramenta Middler VR
        &     &    &     &     &     &  x   &  x   &   &    &     &      \\ \hline

        Finalização da escrita do TCC
        &     &    &     &     &     &     &     &    &   x  &     &     \\ \hline

        Ajustes finais no texto do TCC
         &     &    &     &     &     &     &     &    &    x  &  x  &   \\ \hline

        Defesa do TCC
        &   &   &   &   &   &   &  &  &   &   &  x  \\ \hline

    \end{tabularx}

\newpage
\section{Método de pesquisa}

\hspace*{5mm}

        O trabalho em questão tem como objetivo melhorar o ensino de indivíduos usando as
        tecnologias atuais, para proporcionar aos indivíduos um conhecimento sólido, guiando os
        mesmos, com o intuito de eliminar e/ou evitar lacunas no aprendizado. Usando realidade
        virtual, é possível ser implementado muitas aplicações, tanto como aplicações focadas, onde
        o indivíduo está usando a mesma para aprender um determinado conteúdo, uma aplicação
        descontraída, onde o indivíduo tem o intuito de apenas se divertir, mas acaba agregando
        conhecimentos, ou uma aplicação mista, onde o indivíduo está usando a aplicação para
        aprender, mas a aplicação traz consigo um ambiente de diversão e inovação. Todas os tipos de
        aplicações, tem suporte há agentes, que irão guiar o indivíduo, testando seu conhecimento
        para remover possíveis lacunas de entendimento do conteúdo.

        \medskip
        Para melhor entendimento, será usado o estudo de geometria, para demostrar algumas possíveis
        aplicações, relacionando-as com os tipos de aplicações definidos anteriormente.

        \begin{itemize}

            \item Aplicação focada

            \subitem O indivíduo é instruído com definições e exemplos.
            O sistema de ensino é o tradicional, definições e exercícios.

            \subitem O indivíduo é instruído com definições, exemplos e figuras,
            tanto figuras abstratas quando da cidade onde o indivíduo mora, por exemplo.

            \item Aplicação descontraída
            \subitem O indivíduo se depara com figuras, mini games como quebra-cabeça, histórias etc.
            Mostra levemente assuntos relacionados há um dado conteúdo.

            \item Aplicação mista
            \subitem O indivíduo tem como foco o aprendizado, e o método de ensino é focado em
            conteúdos práticos e descontraídos. Definições de um dado conteúdo fazem parte das regras
            de um jogo, por exemplo.

        \end{itemize}


        O estudo da ferramenta Unity 3D será uma das prioridades, pois como o foco do trabalho é
        leve inovação para o ensino, é preciso ultrapassar os limites da criatividade e escolher as
        melhores ferramentas do mercado para o desenvolvimento da aplicação. Unity 3D, é um motor de
        jogo 3D proprietário e uma IDE criado pela Unity Technologies. Unity é similar ao Blender,
        Virtools ou Torque Game Engine, em relação a sua forma primária de autoria de jogos: a sua
        interface gráfica. O motor cresceu a partir de uma adição de um suporte para a plataforma
        Mac OS X e depois se tornou um motor multi-plataforma. O Unity na grande maioria das vezes é
        usado na criação jogos de browser, em geral, aplicativos para smartphone e em conjunto com
        outras ferramentas, aplicações de realidade virtual e realidade aumentada. Com o unity,
        podemos simular um ambiente 3D, explorando ambientes do dia a dia, como por exemplo, mostrar
        para o indivíduo formas geométricas a partir de uma foto que o mesmo tirou. Com auxilio de
        outras ferramentas, como o Middle VR, assim é possível transformar o projeto desenvolvido no
        Unity 3D, em uma aplicação de realidade virtual, onde o indivíduo pode usar o óculos Rift
        para desfrutar da aplicação.

        \medskip
        Middle VR é um plugin para o Unit 3D, assim sendo possível adicionar interação há aplicação
        que um indivíduo está visualizando no óculos Rift. Middle VR suporta muitas funcionalidades,
        mas uma delas é a interação 3D, onde o indívíduo pode navegar e/ou manipular objetos que
        estão imerso na aplicação.

        \medskip
        Em relação aos agentes, será necessário estudar técnicas de avaliação de indivíduo, levando
        em consideração o quão difícil é um dado conteúdo, o tempo utilizado pelo indivíduo para
        resolver um dado exercício e a conclusão que o indivíduo chegou. Para cada processo citado
        anteriormente, terá um agente, e para coordenar todos os agentes (agentes escravos), terá um
        agente mestre, que irá decidir qual ação executar a partir dos dados recebidos pelos agentes
        escravos. A ação do agente mestre implicará no próximo exercício que o indivíduo terá que
        resolver. O Agente mestre, irá decidir um novo exercício dentre as seguintes categorias:

        \begin{itemize}
            \item Do mesmo assunto, mas com a mesma dificuldade.
            \item Do mesmo assunto, mas com dificuldade superior ao anterior.
            \item Do mesmo assunto, mas com dificuldade inferior ao anterior.
            \item De assunto diferente, pois o indivíduo já domina o assunto anterior.
        \end{itemize}

        Para modelar a sequência de exercícios, será usado a ideia de ``fluxo em redes'', que nada
        mais é do que um grafo. O grafo a ser usado será um grafo dirigido e sem laços que possui
        exatamente uma raiz e uma anti-raiz.

\newpage
\section{Custos}
    \begin{tabular}{l l l l}
        \hline
        Item                    &   Quantidade  &   Valor Unitário (R\$)    &   Valor Total (R\$) \\
        \hline
        Impressão               &   200         &   0,10                    &   20,00             \\

        Programador             &   1           &   1500,00                 &   1500,00           \\
        Notebook                &   1           &   2500,00                 &   2500,00           \\
        Desktop                 &   1           &   3000,00                 &   3000,00           \\
        Reserva de contingência &   1           &   4000,00                 &   4000,00           \\
        \hline
        Total                   &               &                           &   11020,00
    \end{tabular}

\newpage
\section{Recursos Humanos}
    \begin{tabular}{l l}
        \hline
        Nome                            & Função                  \\
        \hline
        Luiz Filipe Moresco da Silva    & Autor                   \\
        Ricardo Azambuja Silveira       & Orientador              \\
        Renato Cislaghi                 & Coordenador de Projetos \\
        Thiago Ângelo Gelaim            & Coorientador            \\
        \hline
    \end{tabular}
    \\
    \\

\newpage
\section{Comunicação}
    \begin{tabular}{l l l l}
        \hline
        O quê  & De quem & Para Quem & Como \\
        \hline
        Proposta de TCC         & Autor     & Renato Cislaghi   & Site de projetos \\
        Relatório de TCC I      & Autor     & Renato Cislaghi   & Site de projetos \\
        Prévia do TCC, em TCC I & Autor     & Banca             & E-mail \\
        Defesa do TCC           & Autor     & Banca             & Pessoalmente \\
        Reunião de Orientação   & Orientadores  & Autor         & Pessoalmente \\
        \hline
    \end{tabular}

\newpage
\section{Riscos}
    \begin{tabular}{|p{2.5cm}|p{2.5cm}|p{2.5cm}|p{2.5cm}|p{2.5cm}|p{2.5cm}|}
        \hline
        Riscos                       & Probabilidade & Impacto & Prioridade & Resposta           & Prevenção                     \\
        \hline
        \hline
        Problemas com notebook       & Baixa         & Médio   & Alta       & Usar desktop        & Manutenção preventiva        \\
        \hline
        Problemas com o desktop      & Baixa         & Alto    & Alta       & Usar o labUFSC      & Manutenção preventiva        \\
        \hline
        Problemas com perda de dados & Baixa         & Alto    & Alta       & Uso do backup       & Backup periódicos            \\
        \hline
        Problemas de Saúde           & Baixa         & Alto    & Alta       & Tratamento adequado & Cuidados diários apropriados \\
        \hline
    \end{tabular}
\newpage

\bibliographystyle{abbrv}
\bibliography{refs}

\end{document}
