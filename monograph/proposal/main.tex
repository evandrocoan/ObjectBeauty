
%
% Licensing
%
%   Copyright 2017 @ Evandro Coan
%
%  This program is free software; you can redistribute it and/or modify it
%  under the terms of the GNU General Public License as published by the
%  Free Software Foundation; either version 3 of the License, or ( at
%  your option ) any later version.
%
%  This program is distributed in the hope that it will be useful, but
%  WITHOUT ANY WARRANTY; without even the implied warranty of
%  MERCHANTABILITY or FITNESS FOR A PARTICULAR PURPOSE. See the GNU
%  General Public License for more details.
%
%  You should have received a copy of the GNU General Public License
%  along with this program.  If not, see <http://www.gnu.org/licenses/>.
%

%
% Simple Sectioned Essay Template - LaTeX Template
%
% This template has been downloaded from:
% http://www.latextemplates.com
%
% `proposal.tex`
% Based on http://www.latextemplates.com/template/simple-sectioned-essay
%

%----------------------------------------------------------------------------------------
%   PACKAGES AND OTHER DOCUMENT CONFIGURATIONS
%----------------------------------------------------------------------------------------

% Sets this document font size to 12pt and set is type to article
\documentclass[12pt]{article}

% https://tex.stackexchange.com/questions/664/why-should-i-use-usepackaget1fontenc
\usepackage[T1]{fontenc}

% Sets this document language usage to Portuguese
\usepackage[brazil]{babel}

% Required to change the page size to A4
% \usepackage{geometry}
%
% Sets this document output papel to A4 within 2 centimeters margin
% \geometry{a4paper}
%
% Set the page size to be A4 as opposed to the default US Letter
\usepackage[a4paper, margin=2cm]{geometry}

% Sets the character set of this document to UTF-8
\usepackage[utf8]{inputenc}

% To use the font Times New Roman, instead of the default LaTeX font
% \usepackage{mathptmx}
%
% more up-to-date than 'mathptmx'
\usepackage{newtxtext,newtxmath}

% For web links with \url{https://www.python.org/downloads/}
\usepackage{hyperref}

% For the new command \latex
\usepackage{xspace}

% https://tex.stackexchange.com/questions/10377/texttt-overfull-hbox-problem
% https://tex.stackexchange.com/questions/66052/should-i-load-microtype-with-pdflatex
\usepackage{microtype}

% Indent the first section paragraphs, https://tex.stackexchange.com/q/39227/119062
\usepackage{indentfirst}

% \lettrine{O}{nce} upon a time...
% \lettrine[findent=2pt]{\fbox{\textbf{T}}}{ }his thesis deals with...
%
% https://tex.stackexchange.com/questions/164298/starting-a-paragraph-with-a-big-letter
\usepackage{lettrine}

% Required for including pictures, resizebox
\usepackage{graphicx}

% Allows putting an [H] in \begin{figure} to specify the exact location of the figure
\usepackage{float}

% Allows in-line images such as the example fish picture
\usepackage{wrapfig}

% Specifies the directory where pictures are stored
\graphicspath{{Pictures/}}

% How to automatically force latex to not justify the text when it is not wise?
% https://tex.stackexchange.com/questions/365801/how-to-automatically-force-latex-to-not-justify-the-text-when-it-is-not-wise
\usepackage{array,ragged2e}

% Use its macro adjustwidth* to extend tables out of outer text border.
% https://tex.stackexchange.com/questions/366155/how-to-write-a-table-a-little-larger-than-the-paragraphs-with-centered-columns
\usepackage[strict]{changepage}

\usepackage{tabularx}
\usepackage{multirow}


%
% New Macros
%

% Automatically put a `\medskip` spacing between paragraphs
% https://tex.stackexchange.com/q/365976/119062
\edef\restoreparindent{\parindent=\the\parindent\relax}
\usepackage{parskip}
\restoreparindent

% Uncomment to remove all indentation from paragraphs
%\setlength\parindent{0pt}

% How could the `\everypar` justification statement be used?
% https://tex.stackexchange.com/questions/365818/how-could-the-everypar-justification-statement-be-used
\newbox\linebox \newbox\snapbox
\def\eatlines{
  \setbox\linebox\lastbox % check the last line
  \ifvoid\linebox
  \else % if it’s not empty
    \unskip\unpenalty % take whatever is
    {\eatlines} % above it;
    \setbox\snapbox\hbox{\unhcopy\linebox}
    \ifdim\wd\snapbox<.98\wd\linebox
       \box\snapbox % take the one or the other,
    \else \box\linebox \fi
  \fi}

% How could the `\everypar` justification statement be used?
% https://tex.stackexchange.com/questions/365818/how-could-the-everypar-justification-statement-be-used
\everypar={\setbox0=\lastbox \par
   \vbox\bgroup \everypar={}\def\par{\endgraf\eatlines\egroup}}

% Creates a new environment which can be used as:
%
% \begin{foo}
%   Text...
%
%   Text ...
% \end{foo}
%
% https://tex.stackexchange.com/questions/62333/push-long-words-in-a-new-line
\newenvironment{foo}
{\par
\hyphenpenalty=10000
\exhyphenpenalty=10000
}
{\par}


%
% New commands
%

% Line spacing
\linespread{1.5}

% Allow to push long words on new lines when they do not fit entirely on the current line.
% https://tex.stackexchange.com/questions/62333/push-long-words-in-a-new-line
\newcommand\lword[1]{\leavevmode\nobreak\hskip0pt plus\linewidth\penalty50\hskip0pt plus-\linewidth\nobreak{#1}}

% Write the word LaTeX nicely.
\newcommand{\latex}{\LaTeX\xspace}

% Create a bold title all in upper case.
\newcommand{\Title}[1]{\textbf{\MakeUppercase{#1}}}


%----------------------------------------------------------------------------------------
%   DOCUMENT CONTENTS
%----------------------------------------------------------------------------------------

\begin{document}

\begin{titlepage}

    \center

    \Title{Universidade Federal de Santa Catarina - UFSC}

    \Title{Ciência Da Computação}

    \vspace*{\stretch{1}}

    Evandro Coan

    \vspace*{\stretch{2}}

    \Title{Universal Source Code Beautifier}

    \bigskip
    Um Estudo Sobre Formatação de códigos\\[3cm]

    \begin{flushright}

        \begin{minipage}{0.518\textwidth}

            Proposta de Trabalho de Conclusão de Curso,
            a ser submetido ao Curso de Ciência da Computação
            para a obtenção do Grau de Bacharel em Ciência da Computação.

            \medskip
            Orientador: Professor ...

            \medskip
            Coorientador: ...

        \end{minipage}

    \end{flushright}

    \vspace*{\stretch{3}}

    Florianópolis, \today.

\end{titlepage}


\begin{abstract}

    Os softwares formatadores de código fonte atuais, também conhecidos como Beautifiers, são
    limitados a um conjunto similar, ou mesmo à uma única linguagem, e além de muitos, serem
    limitados ao que eles podem fazer por você ao processar/formatar o código \cite{Terence}.
    Portanto este trabalho tem como objetivo criar um formatador (software único) de fácil
    configuração e expansão capaz de abranger todas as linguagens de programação que existem,
    baseado em um uso específico de expressões regulares.

    \medskip
    A metodologia abordada será de não ter a necessidade de ter-se conhecimento da sintaxe das
    linguagens de programação que se irão fazer o parsing. Isso porque trataremos elas como texto
    comum, e será o usuário final que fará a configuração das transformações que serão aplicados no
    texto, dando liberdade de facilmente se configurar várias linguagens de programação (senão
    todas), aproveitando o fato de que muitas deles compartilham estruturas semelhantes senão
    idênticas.

    \medskip
    Como resultado espera-se ter um Beautifier Universal capaz de abranger todas as linguagens que
    existem, senão que seja facilmente extensível para abrange-las. Os pontos positivos dessa
    abordagem são a reusabilidade de componentes entre as linguagens. Por exemplo, `if/for/while`'s
    em C++ e Java são da mesma estrutura. Assim temos que escrever somente uma vez a especificação
    para um componente da linguagem.

    \bigskip
    \bigskip
    \textbf{Palavras-chave:}
    source, code, formatter, beautifier, prettyprint, universal, reuse, blocks, object, oriented,
    programming, structured, parsing, parse, regular, expression, regex, C, C++,  grammar, Turing,
    machine, automata, lexer, syntax, sublime, Java, Rust, shell, script, obfuscators, learning,
    syntec, teamicide, concensus, indent, settings.

\end{abstract}

\newpage
\tableofcontents
\newpage

\section{Introdução}

    \hspace*{5mm}

    Realidade ...

\newpage
\section{Objetivos}

\subsection{Objetivo geral}

   \hspace*{5mm}Com um alto índice de dificuldades por parte de crianças, jovens e adultos em
    aprender um determinado conteúdo em sua jornada escolar, esse trabalho tem como objetivo auxiliar no
    aprendizado de qualquer indivíduo, mapeando suas dificuldade e o guiando até chegar em seus
    objetivos. Através de uma aplicação, o indivíduo será guiado conforme seu conhecimento prévio
    perante há um dado conteúdo e sua progressão ao longo de seu estudo.

\subsection{Objetivos específicos}
    \hspace*{5mm}

    É necessário dividir por dificuldade cada objetivo, pois alguns deles, terá uma implementação
    mais crua, pois o tempo para desenvolvimento do trabalho é limitado.

    \begin{itemize}
        \item Alta complexidade, onde definir o conhecimento do aluno, não apenas aplicando uma formula.
        \subitem Determinar um dado exercício como fácil, médio ou difícil
        \subitem Determinar o nível de entendimento de um individuo em relação ao um conteúdo
        \item Média complexidade, pois é o desenvolvimento de uma aplicação usando agentes e realidade virtual
        \subitem Implementar software com suporte a realidade virtual
        \subitem Implementar agentes que para guiar o individuo em um dado conteúdo
    \end{itemize}

\newpage
\section{Cronograma}

   \begin{tabularx}{\linewidth}{|X|*{11}{c|}}
        \hline
        \multicolumn{1}{|c|}{\multirow{2}{*}{Etapas}} & \multicolumn{11}{|c|}{Meses}\\ \cline{2-12}

        & fev & mar & abr & mai & jun & jul & ago & set & out & nov & dez  \\ \hline

        Estudo da fundamentação teórica necessária
        &  x  &  x  &  x  &     &     &     &     &     &     &     &     \\ \hline

        Estudo da implementação de agentes na educação
        &     &  x  &  x  &     &     &     &     &     &     &     &     \\ \hline

        Estudo da ferramento Unity
        &     &     &  x  &  x  &  x  &  x  &  x  &     &     &     &     \\ \hline

        Escrita de um esboço do TCC
        &     &     &     &     &  x  &  x  &  x  &  x  &     &     &     \\ \hline

       Estudo da ferramenta Middler VR
        &     &     &     &     &     &  x  &  x  &     &     &     &     \\ \hline

        Finalização da escrita do TCC
        &     &     &     &     &     &     &     &     &  x  &     &     \\ \hline

        Ajustes finais no texto do TCC
        &     &     &     &     &     &     &     &     &  x  &  x  &     \\ \hline

        Defesa do TCC
        &     &     &     &     &     &     &     &     &     &     &  x  \\ \hline

    \end{tabularx}

\newpage
\section{Método de pesquisa}

\hspace*{5mm}

        O trabalho...


\newpage
\section{Custos}
    \begin{tabular}{l l l l}
        \hline
        Item                    &   Quantidade  &   Valor Unitário (R\$)    &   Valor Total (R\$) \\
        \hline
        Impressão               &   200         &   0,10                    &   20,00             \\

        Programador             &   1           &   1500,00                 &   1500,00           \\
        Notebook                &   1           &   2500,00                 &   2500,00           \\
        Desktop                 &   1           &   3000,00                 &   3000,00           \\
        Reserva de contingência &   1           &   4000,00                 &   4000,00           \\
        \hline
        Total                   &               &                           &   11020,00
    \end{tabular}

\newpage
\section{Recursos Humanos}
    \begin{tabular}{l l}
        \hline
        Nome                            & Função                  \\
        \hline
        Luiz Filipe Moresco da Silva    & Autor                   \\
        Ricardo Azambuja Silveira       & Orientador              \\
        Renato Cislaghi                 & Coordenador de Projetos \\
        Thiago Ângelo Gelaim            & Coorientador            \\
        \hline
    \end{tabular}
    \\

\newpage
\section{Comunicação}
    \begin{tabular}{l l l l}
        \hline
        O quê  & De quem & Para Quem & Como \\
        \hline
        Proposta de TCC         & Autor     & Renato Cislaghi   & Site de projetos \\
        Relatório de TCC I      & Autor     & Renato Cislaghi   & Site de projetos \\
        Prévia do TCC, em TCC I & Autor     & Banca             & E-mail \\
        Defesa do TCC           & Autor     & Banca             & Pessoalmente \\
        Reunião de Orientação   & Orientadores  & Autor         & Pessoalmente \\
        \hline
    \end{tabular}

\newpage
\section{Riscos}

    % https://tex.stackexchange.com/questions/366156/how-to-change-the-left-padding-for-one-latex-tables-cell
    % https://tex.stackexchange.com/questions/366155/how-to-write-a-table-a-little-larger-than-the-paragraphs-with-centered-columns
    %
    \begin{adjustwidth}{-0.5\marginparwidth}{-0.5\marginparwidth}
    \small
    \begin{tabularx}{\linewidth}
    {|
        *1{                 >{\RaggedRight\arraybackslash\hsize=1.1\hsize }X       |} % Riscos
        *3{@{\hspace{3.0pt}}>{\Centering\arraybackslash                   }p{0.9cm}|} % Probabilidade, Impacto, Prioridade
        *2{                 >{\RaggedRight\arraybackslash\hsize=0.95\hsize}X       |} % Resposta, Prevenção
    }

    \hline Riscos  & Pro\-ba\-bi\-li\-da\-de & Im\-pac\-to & Prio\-ri\-da\-de & Es\-tra\-té\-gia de res\-pos\-ta & Ações de pre\-ven\-ção \\ \hline

    % Row 1
    % Riscos
    \hline Problemas com perda de dados &
    % Probabilidade
    Baixa &
    % Impacto
    Alto &
    % Prioridade
    Alta &
    % Estratégia de resposta
    Uso do backup &
    % Ações de prevenção
    Backup periódicos \\ \hline

    % Row 2
    % Riscos
    \hline Alteração do cronograma ou descontinuidade do projeto onde recebo uma bolsa &
    % Probabilidade
    \rlap{Média} &
    % Impacto
    Alto &
    % Prioridade
    Alta &
    % Estratégia de resposta
    Redefinição da data de entrega do trabalho &
    % Ações de prevenção
    Monitoramento contínuo das informações obtidas com superiores imediatos \\ \hline

    \hline \end{tabularx}
    \end{adjustwidth}

\newpage

\bibliographystyle{abbrv}
\bibliography{refs}

\end{document}
