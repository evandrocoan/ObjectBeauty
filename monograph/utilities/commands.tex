


%
% New Macros
%

% How could the `\everypar` justification statement be used?
% https://tex.stackexchange.com/questions/365818/how-could-the-everypar-justification-statement-be-used
\newbox\linebox \newbox\snapbox
\def\eatlines{
  \setbox\linebox\lastbox % check the last line
  \ifvoid\linebox
  \else % if it’s not empty
    \unskip\unpenalty % take whatever is
    {\eatlines} % above it;
    \setbox\snapbox\hbox{\unhcopy\linebox}
    \ifdim\wd\snapbox<.98\wd\linebox
       \box\snapbox % take the one or the other,
    \else \box\linebox \fi
  \fi
}


% How could the `\everypar` justification statement be used?
% https://tex.stackexchange.com/questions/365818/how-could-the-everypar-justification-statement-be-used
\everypar={\setbox0=\lastbox \par
   \vbox\bgroup \everypar={}\def\par{\endgraf\eatlines\egroup}}


% Creates a new environment which can be used as:
%
% \begin{foo}
%   Text...
%
%   Text ...
% \end{foo}
%
% https://tex.stackexchange.com/questions/62333/push-long-words-in-a-new-line
\newenvironment{foo}
{\par
\hyphenpenalty=10000
\exhyphenpenalty=10000
}
{\par}


% How to break long URLs using common hyphenation but adding a line feed indicator?
%
% some text \brkurl{http://www.example.com/this/directory/here}
%
% https://tex.stackexchange.com/questions/69824/how-to-break-long-urls-using-common-hyphenation-but-adding-a-line-feed-indicator
\def\addurlspace#1{%
\ifx\relax#1%
\else
\ifx/#1\space\fi
\ifx.#1\space\fi
#1%
\ifx/#1\space\fi
\ifx.#1\space\fi
\expandafter\addurlspace
\fi}

\makeatletter

\@namedef{OT1-zwidthchar}{255}
\@namedef{T1-zwidthchar}{"17}

\def\brkurl#1{%
\edef\savedhchar{\the\hyphenchar\font}%
\global\setbox1\hbox{}%
\setbox0=\vbox{\hsize=2pt\rightskip=0pt plus 1fill
\hfuzz\maxdimen
\tracinglostchars0
\overfullrule0pt
\hyphenchar\font=\csname \f@encoding-zwidthchar\endcsname
\noindent \hskip0pt \addurlspace #1\relax
\par
\loop
\setbox4 \lastbox
\ifvoid4 \else
\global\setbox1\hbox{\unhbox4\unskip\unskip\discretionary{\hbox{\rlap{$\leftarrow$}}}{}{}\unhbox1}%
\unskip
\unskip
\unpenalty
\unskip
\repeat
}%
\unhbox1
\hyphenchar\font\savedhchar
\relax}

\makeatother


% Change background color for text block
% https://tex.stackexchange.com/questions/238294/change-background-color-for-text-block
\usepackage{framed}
\usepackage[most]{tcolorbox}
\definecolor{shadecolor}{RGB}{219, 229, 241}
\newtcolorbox{myquote}{
colback=shadecolor,
grow to right by=-2mm,
grow to left by=-2mm,
boxrule=0pt,
boxsep=0pt,
breakable,
}

% Make first row of table all bold
%
% Usage:
% 1. Add `B` on the borders and `^` before each column definition.
% 2. `\rowstyle{\bfseries}` before the row you want to bold.
%
% Example:
% \begin{tabularx}{\linewidth}
% {|
%     *1{                 >{\RaggedRight\arraybackslash\hsize=1.1\hsize }BX       |} % Riscos
%     *3{@{\hspace{3.0pt}}>{\Centering\arraybackslash                   }^p{0.9cm}|} % Probabilidade, Impacto, Prioridade
%     *2{                 >{\RaggedRight\arraybackslash\hsize=0.95\hsize}^X       |} % Resposta, Prevenção
% }
%
% \hline
%
% \rowstyle{\bfseries}
% Riscos  & 1 & 2 & 3 & Estratégia de resposta & Ações de prevenção \\ \hline
%
%
% https://tex.stackexchange.com/questions/4811/make-first-row-of-table-all-bold
\usepackage{array}
\newcolumntype{B}{>{\global\let\currentrowstyle\relax}}
\newcolumntype{^}{>{\currentrowstyle}}
\newcommand{\rowstyle}[1]{\gdef\currentrowstyle{#1}%
  #1\ignorespaces
}



%
% New commands
%

% Allow to push long words on new lines when they do not fit entirely on the current line.
% https://tex.stackexchange.com/questions/62333/push-long-words-in-a-new-line
\newcommand\lword[1]{\leavevmode\nobreak\hskip0pt plus\linewidth\penalty50\hskip0pt plus-\linewidth\nobreak{#1}}
\newcommand\lurl[1]{\leavevmode\nobreak\hskip0pt plus\linewidth\penalty50\hskip0pt plus-\linewidth\nobreak{\url{#1}}}


% For the new command \latex
\usepackage{xspace}

% Write the word LaTeX nicely.
\newcommand{\latex}{\LaTeX\xspace}


% Create a bold title all in upper case.
\newcommand{\Title}[1]{\textbf{\MakeUppercase{#1}}}





