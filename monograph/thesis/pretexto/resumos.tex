

% Is it possible to keep my translation together with original text?
% https://tex.stackexchange.com/questions/5076/is-it-possible-to-keep-my-translation-together-with-original-text
\begin{comment}

    Segundo a \citeonline[3.1-3.2]{NBR6028:2003}, o resumo deve ressaltar o
    objetivo, o método, os resultados e as conclusões do documento. A ordem e a extensão
    destes itens dependem do tipo de resumo (informativo ou indicativo) e do
    tratamento que cada item recebe no documento original. O resumo deve ser
    precedido da referência do documento, com exceção do resumo inserido no
    próprio documento. (\ldots) As palavras-chave devem figurar logo abaixo do
    resumo, antecedidas da expressão Palavras-chave:, separadas entre si por
    ponto e finalizadas também por ponto.

    ufscthesis: O texto do resumo deve ser digitado, em um único bloco, sem espaço de parágrafo. O resumo deve
    ser significativo, composto de uma sequência de frases concisas, afirmativas e não de uma
    enumeração de tópicos. Não deve conter citações. Deve usar o verbo na voz passiva. Abaixo do
    resumo, deve-se informar as palavras-chave (palavras ou expressões significativas retiradas do
    texto) ou, termos retirados de thesaurus da área. \showfont

\end{comment}


% Changing babel package inside a single chapter
% https://tex.stackexchange.com/questions/20987/changing-babel-package-inside-a-single-chapter
%
% Multiple-language document - babel - selectlanguage vs begin/end{otherlanguage}
% https://tex.stackexchange.com/questions/36526/multiple-language-document-babel-selectlanguage-vs-begin-endotherlanguage
\begin{otherlanguage*}{brazil}
\begin{resumo}

    Faz-se um estudo sobre o que é, para que servem os Beautifiers, assim como abordagens sobre o
    que são boas práticas de programação e por que devemos segui-las para um boa eficiência ao
    escrever códigos nas mais diversas linguagens de programação. Os softwares formatadores de
    código fonte atuais, também conhecidos como Beautifiers, são limitados a um conjunto similar, ou
    mesmo à uma única linguagem, e além de muitos, serem limitados ao que eles podem fazer por você
    ao processar/formatar o código.

    Portanto espera-se o final do trabalho, conhecer-se quais são as ferramentas que existem e quais
    delas são as melhores que podem ser utilizadas para o auxilio do programador durante a escrita
    de códigos das mais diversas linguagens de programação. Além de proposta de uma nova ferramenta
    com o intuído de centralizar em uma único programa o abordagem das mais diversas linguagens de
    programação.

    \imprimirpalavraschave{Palavras-chaves}{ fonte. código. formatador. embelezante. prettyprint.
    universal. reuso. blocos. objeto. orientado. programação. estruturada. análise. analisador.
    regular. expressão. regex. c. c++. gramática. turing. máquina. autômatos. lexer. sintaxe.
    sublime. java. rust. shell. roteiro. ofuscadores. aprendizado. syntec. teamicide. consenso.
    recuar. configurações. }

\end{resumo}
\end{otherlanguage*}


% resumo em inglês
\begin{otherlanguage*}{english}
\begin{resumo}[Abstract]

    A study is made of what Beautifiers are for, as well as approaches to what are good programming
    practices and why we should follow them for good efficiency when writing code in the most
    diverse programming languages. Current source code formatting software, also known as
    Beautifiers, is limited to a similar set, or even a single language, and, in addition to many,
    be limited to what they can do for you when processing / formatting the code.

    Therefore, it is expected at the end of the work, to know what tools exist and which of them are
    the best that can be used to help the programmer while writing codes of the most diverse
    programming languages. Besides the proposal of a new tool with the intuition of centralizing in
    a single program the approach of the most diverse programming languages.

    \imprimirpalavraschave{Keywords}{ source. code. formatter. beautifier. prettyprint.
    universal. reuse. blocks. object. oriented. programming. structured. parsing. parse. regular.
    expression. regex. C. C++.  grammar. turing. machine. automata. lexer. syntax. sublime. Java.
    Rust. shell. script. obfuscators. learning. syntec. teamicide. concensus. indent. settings. }

\end{resumo}
\end{otherlanguage*}




% % resumo em francês
% \begin{resumo}[Résumé]
%   \begin{otherlanguage*}{french}
%       Il s'agit d'un résumé en français.

%       \imprimirpalavraschave{Mots-clés}{latex. abntex. publication de textes.}
%   \end{otherlanguage*}
% \end{resumo}


% % resumo em espanhol
% \begin{resumo}[Resumen]
%   \begin{otherlanguage*}{spanish}
%       Este es el resumen en español.

%       \imprimirpalavraschave{Palabras clave}{latex. abntex. publicación de textos.}
%   \end{otherlanguage*}
% \end{resumo}



