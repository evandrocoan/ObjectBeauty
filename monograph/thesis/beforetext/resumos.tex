

\newcommand{\imprimirbrazilabstract}{%
    \cleardoublepage\phantomsection
    \addtotextpreliminarycontent{Resumo em Português}
    \begin{otherlanguage*}{brazil}
    \begin{resumo}[Resumo]

        Faz-se um estudo sobre o que é,
        para que servem os formatadores de código,
        assim como abordagens utilizadas nas mais variadas ferramentas formatação de código e por que devemos utiliza-las
        para ter-se uma boa eficiência ao escrever códigos nas mais diversas linguagens de programação.
        Os softwares formatadores de código fonte atuais,
        também conhecidos como Beautifiers,
        são limitados a um conjunto similar,
        ou mesmo à uma única linguagem de programação,
        e além de muitos serem limitados no que eles podem fazer por você ao formatar seus programas.

        Portanto,
        espera-se ao final deste trabalho,
        conhecer-se quais são as ferramentas de formatação que existem e quais delas são as melhores que podem ser utilizadas
        para o auxilio do programador durante a escrita de códigos nas mais diversas linguagens de programação.
        Além da proposta de uma nova ferramenta com o intuído de centralizar em um único
        programa a formatação de código das mais diversas linguagens de programação.

        \imprimirpalavraschave{Palavras-chaves}{\begin{inparaitem}[]\palavraschaveportugues\end{inparaitem}}

    \end{resumo}
    \end{otherlanguage*}
}


\newcommand{\imprimirenglishabstract}{%
    % https://tex.stackexchange.com/questions/20987/changing-babel-package-inside-a-single-chapter
    % https://tex.stackexchange.com/questions/36526/multiple-language-document-babel-selectlanguage-vs-begin-endotherlanguage
    \cleardoublepage\phantomsection
    \addtotextpreliminarycontent{English's Abstract}
    \begin{otherlanguage*}{english}
    \begin{resumo}[Abstract]

        A study about nowadays used source code formatters,
        and as well,
        the programming algorithms used in most source code formatting tools,
        and why we should use source code formatters,
        for efficient coding in the most diverse programming languages.
        Cutting edge source code formatting softwares,
        also known as Source Code Beautifiers,
        are limited to a common set,
        or even to a single programming language,
        and many formatters are limited in what they can do.

        Therefore,
        it is expected at the end of this work,
        to know which source code formatting tools there are and which ones
        are the best to assist programmers in their daily work,
        at the most diverse programming languages.
        Besides,
        this work proposes a new tool with focus on centering in a single program,
        all source code formatting for all programming languages.

        \imprimirpalavraschave{Keywords}{\begin{inparaitem}[]\palavraschaveingles\end{inparaitem}}

    \end{resumo}
    \end{otherlanguage*}
}


% \newcommand{\imprimirfrenchabstract}{%
%     \addtotextpreliminarycontent{Français Résumé}
%     \begin{resumo}[Résumé]
%       \begin{otherlanguage*}{french}
%           Il s'agit d'un résumé en français.

%           \imprimirpalavraschave{Mots-clés}{latex. abntex. publication de textes.}
%       \end{otherlanguage*}
%     \end{resumo}
% }


% \newcommand{\imprimirspanishabstract}{%
%     \addtotextpreliminarycontent{Español Resumen}
%     \begin{resumo}[Resumen]
%       \begin{otherlanguage*}{spanish}
%           Este es el resumen en español.

%           \imprimirpalavraschave{Palabras clave}{latex. abntex. publicación de textos.}
%       \end{otherlanguage*}
%     \end{resumo}
% }


\makeatletter


\ifenglish
    \@ifundefined{imprimirbrazilabstract}{}{\imprimirbrazilabstract}

    % https://tex.stackexchange.com/questions/11707/how-to-force-output-to-a-left-or-right-page
    % https://tex.stackexchange.com/questions/132966/do-not-display-chapter-title-in-memoir-class
    \cleardoublepage\phantomsection
    \pretextualchapter{Resumo Expandido}
    \addtotextpreliminarycontent{Resumo Expandido}

    \begin{otherlanguage*}{brazil}
        \setlength{\parskip}{0.2cm}
        \setlength{\parindent}{0.0cm}

        \section*{Introdução}
        O resumo expandido é previsto na Resolução Normativa nº 95/CUn/2017, Art. 55, § 2, de 4 de
        abril de 2017, e exigido para teses e dissertações escritas em idiomas estrangeiros (com
        exceção dos cursos pertinentes ao estudo de idiomas estrangeiros – Programa de Pós-Graduação
        em Estudos da Tradução e Programa de Pós-Graduação em Inglês: Estudos Linguísticos e
        Literários).

        O resumo expandido é considerado um elemento pré-textual e deverá ser incluído no trabalho
        após o resumo e antes do abstract. Deverá iniciar em página impar (no anverso de uma folha)
        continuando no verso da folha. O texto deverá seguir o formato A5, com margens espelhadas:
        superior 2,0 cm, inferior 1,5 cm, interna 2,5 cm e externa 1,5. Deve ser empregada a fonte
        Time New Roman.  Todo o texto deve ser digitado em tamanho 10,5. O espaçamento entre as
        linhas deverá ser simples. A expressão “resumo expandido” deve seguir a mesma tipografia das
        demais sessões primárias do trabalho.

        O texto do resumo expandido deve ser redigido em português e conter as seguintes seções (ver
        modelo): Introdução, Objetivos, Metodologia, Resultados e Discussão e Considerações Finais.
        Deve apresentar no mínimo duas (02) e, no máximo, cinco (05) páginas contendo a mesma
        formatação em A5 do resumo e do abstract, bem como palavras-chave.

        \section*{Objetivos}
        Lorem ipsum dolor sit amet, consectetur adipiscing elit. Phasellus vitae dolor lacus. Ut
        accumsan vitae felis nec porttitor. Integer interdum fringilla feugiat. Nullam pulvinar sit
        amet tellus eget maximus. Donec sit amet magna eget justo semper fermentum vel eget velit.
        In iaculis imperdiet mauris, ac ornare libero placerat non. Nulla libero lectus, ullamcorper
        ac ornare eget, pulvinar ac nulla.

        Curabitur vestibulum non nisl eget sagittis. Proin
        gravida lacus id eros bibendum interdum. Mauris ullamcorper elementum tortor sed consequat.
        Integer tempus, est a lobortis vehicula, nisi mi fringilla augue, non semper leo metus in
        quam. Etiam in leo maximus, pulvinar mi eget, vehicula risus. Donec sed dui semper, dictum
        eros at, suscipit felis.

        \section*{Metodologia}
        Quisque efficitur dolor in lectus dapibus elementum. Nam ultrices blandit consectetur.
        Nullam ultricies sit amet odio quis placerat. Aenean eget est elit. Maecenas et nulla dolor.
        Orci varius natoque penatibus et magnis dis parturient montes, nascetur ridiculus mus. In
        pulvinar velit sed mi sagittis ornare. Aenean rutrum suscipit egestas. Phasellus pharetra
        eget ex in volutpat. Quisque eu arcu nunc. Vivamus arcu ligula, pharetra at rhoncus sit
        amet, pulvinar sed eros.

        Sed porta ipsum ipsum, et fermentum magna volutpat sed. Vivamus
        pharetra facilisis orci, sit amet luctus nisl pretium id. Sed consequat, arcu et congue
        pulvinar, risus enim aliquet purus, eget venenatis libero leo sit amet metus. Maecenas vitae
        elit sapien. Fusce mollis libero et gravida placerat. Proin ut quam quis justo aliquam
        dictum. Donec volutpat convallis suscipit. Vivamus metus nisl, placerat ac enim vitae,
        tempus ultricies odio.

        \section*{Resultados e Discussão}
        Nulla porta auctor vestibulum. Sed
        consectetur lacus molestie iaculis ullamcorper. Proin porta posuere massa a lacinia. Nunc a
        lacinia orci, non vehicula ante. Vestibulum ipsum velit, congue et neque aliquam, imperdiet
        ornare augue. Donec et congue sapien. Pellentesque consequat consectetur neque ut varius. In
        aliquam ex quis ante venenatis dapibus. Vivamus et imperdiet urna. Vestibulum quis nibh
        magna. In a congue lectus, eu sodales nunc. Suspendisse id.

        \section*{Considerações Finais}
        Lorem ipsum dolor sit amet, consectetur adipiscing elit. Phasellus vitae dolor lacus. Ut
        accumsan vitae felis nec porttitor. Integer interdum fringilla feugiat. Nullam pulvinar sit
        amet tellus eget maximus. Donec sit amet magna eget justo semper fermentum vel eget velit.
        In iaculis imperdiet mauris, ac ornare libero placerat non. Nulla libero lectus, ullamcorper
        ac ornare eget, pulvinar ac nulla. Curabitur vestibulum non nisl eget sagittis. Proin
        gravida lacus id eros bibendum interdum. Mauris ullamcorper elementum tortor sed consequat.
        Integer tempus, est a lobortis vehicula, nisi mi fringilla augue, non semper leo metus in
        quam. Etiam in leo maximus, pulvinar mi eget, vehicula risus. Donec sed dui semper, dictum
        eros at, suscipit felis.

        \imprimirpalavraschave{Palavras-chaves}{\begin{inparaitem}[]\palavraschaveportugues\end{inparaitem}}

    \end{otherlanguage*}

    \@ifundefined{imprimirenglishabstract}{}{\imprimirenglishabstract}

\else
    \@ifundefined{imprimirbrazilabstract}{}{\imprimirbrazilabstract}
    \@ifundefined{imprimirenglishabstract}{}{\imprimirenglishabstract}
\fi

\@ifundefined{imprimirfrenchabstract}{}{\imprimirfrenchabstract}
\@ifundefined{imprimirspanishabstract}{}{\imprimirspanishabstract}


\makeatother

