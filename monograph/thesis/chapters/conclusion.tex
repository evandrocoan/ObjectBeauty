

\chapter[]{\lang{Conclusion}{Conclusão}}

\lang{%
    The difference from this proposal to remaining formatting tools,
    is the tradeoff between end\hyp{}users and developers responsibilities.
    Most tools rarely expose to end\hyp{}users their language syntax specification,
    in contrast,
    this proposal completely exposes the language to the end\hyp{}user as simple plain\hyp{}text,
    not requiring the tool to know any language syntax neither semantics.
    Moreover,
    with no syntax knowledge required,
    the tool be can used with any languages their user wishes to.

\begin{enumerate}[leftmargin=*]
    \item
        There are many different tools, sometimes paid, and difficult to
        complete. \cite{universalCodeFormatter}
    \item
        Many programming languages exist, so always having Beautifier
        software for each of them is very laborious
        \cite{universalCodeFormatter}. But the approach to a Universal
        Beautifier proposed in this work, would allow easily new languages to be
        added, being completely different from previous ones, or alike. And in
        case of similarities between them, it is enough to reuse their
        configuration structures already implemented.
    \item
        Looking for a Beautifier for each one of them because programmers
        currently work daily with several of these languages, and they are not
        similar. So you need to configure several beautifiers to do the
        formatting. This is a problem because only a few beautifiers are more
        complete, and every time you need to make a change in the formatting
        style, you must manually propagate the same change over several
        different program configuration files, which is bad because it takes the
        user a lot of time to learn how to handle many different types of
        settings. \cite{universalIndentGUI}
    \item
        In the case of ideal Beautifier, a change in your styling is
        propagated to all languages. And if you want to leave some language out
        of it, you just need to remove it from the list on which the
        configuration block applies to, and `a)' leave it out so no change is
        applied to. Or `b)' create a new block including only the block within
        the desired settings.
\end{enumerate}
}{%
    A diferença desta nova proposta de ferramenta de formatação de código~=fonte para as demais é a troca de responsabilidades entre usuários finais da ferramenta e
    os desenvolvedores desta ferramenta e
    das gramáticas para os usuários finais.
    Formatadores de código~=fonte somente expõe controladamente quais são as mudanças que podem ocorrer ao formatar o código~=fonte.
    A maior parte das ferramentas raramente permite que usuários finais o controle total das mudanças no código~=fonte,
    i.e.,
    enquanto que utilizando a ferramenta desenvolvida neste trabalho,
    é facilmente possível escrever regras de formatação que quebrem a sintaxe e
    semântica da linguagem sendo formatada.
    Entretanto,
    utilizando~=se as ferramentas usuais de formatação,
    é impedi~=se que configurações de formatação do usuário final quebrem o código~=fonte da linguagem sendo formatada,
    a não ser em casos de \textit{bugs} na ferramenta de formatação.

    Esta diferença dá~=se por que as ferramentas de formatação em geral,
    tentam reconstruir a árvore de sintaxe da linguagem a ser formatação.
    Já no caso da ferramenta desenvolvida neste trabalho,
    isso não é um requisito.
    O usuário pode escolher entre simplesmente especificar a gramática da linguagem o mínimo necessários para atingir somente as suas necessidades de formatação.
    Mas ao mesmo tempo,
    ele também pode realizar a especificação completa de toda a sintaxe da sua linguagem.
    Entretanto,
    isso ainda não será o suficiente para cobrir os aspectos semânticos das linguagens.
    Neste caso,
    o usuário precisará conhecer quais são as regras semânticas da linguagem que ele está realizando a formatação,
    e configurar o formatador para que ele não quebre nenhuma das regras semânticas da linguagem.

    Para usuários que não possuem conhecimentos sobre semântica das linguagens que querem realizar a formatação,
    a ferramenta desenvolvida neste trabalho não pode ser utilizada.
    Neste caso,
    devem ser utilizados as demais ferramentas de formatação que são mais especificas para cada linguagem e
    possuem inerentemente conhecimentos específicos da sintaxe e
    semântica das linguagens.
}
