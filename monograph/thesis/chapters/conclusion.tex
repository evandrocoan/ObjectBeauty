

\chapter[]{\lang{Conclusion}{Conclusão}}

    The difference from this proposal to remaining formatting tools,
    is the tradeoff between end\hyp{}users and developers responsibilities.
    Most tools rarely expose to end\hyp{}users their language syntax specification,
    in contrast,
    this proposal completely exposes the language to the end\hyp{}user as simple plain\hyp{}text,
    not requiring the tool to know any language syntax neither semantics.
    Moreover,
    with no syntax knowledge required,
    the tool be can used with any languages their user wishes to.

\begin{enumerate}[leftmargin=*]
    \item
        There are many different tools, sometimes paid, and difficult to
        complete. \cite{universalCodeFormatter}
    \item
        Many programming languages exist, so always having Beautifier
        software for each of them is very laborious
        \cite{universalCodeFormatter}. But the approach to a Universal
        Beautifier proposed in this work, would allow easily new languages to be
        added, being completely different from previous ones, or alike. And in
        case of similarities between them, it is enough to reuse their
        configuration structures already implemented.
    \item
        Looking for a Beautifier for each one of them because programmers
        currently work daily with several of these languages, and they are not
        similar. So you need to configure several beautifiers to do the
        formatting. This is a problem because only a few beautifiers are more
        complete, and every time you need to make a change in the formatting
        style, you must manually propagate the same change over several
        different program configuration files, which is bad because it takes the
        user a lot of time to learn how to handle many different types of
        settings. \cite{universalIndentGUI}
    \item
        In the case of ideal Beautifier, a change in your styling is
        propagated to all languages. And if you want to leave some language out
        of it, you just need to remove it from the list on which the
        configuration block applies to, and `a)' leave it out so no change is
        applied to. Or `b)' create a new block including only the block within
        the desired settings.
\end{enumerate}

