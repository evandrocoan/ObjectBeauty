
% The \phantomsection command is needed to create a link to a place in the document that is not a
% figure, equation, table, section, subsection, chapter, etc.
%
% When do I need to invoke \phantomsection?
% https://tex.stackexchange.com/questions/44088/when-do-i-need-to-invoke-phantomsection
\phantomsection

% ---
% Conclusão (outro exemplo de capítulo sem numeração e presente no sumário)
% ---
\chapter[]{\lang{Conclusion}{Conclusão}}


1.2      Rascunho da Monografia para TCC1

1.2.1    * Introdução       Uma breve descrição dos objetivos e da
           justificativa para realização do projeto. Uma breve explicação de
           qual a necessidade e importância deste projeto no escopo da
           Ciência da Computação. Os objetivo gerais e específicos buscados
           após a conclusão e análise deste projeto.

1.2.2    * Bibliográfia     Fazer uma análise do estado da arte em relação
         ao que existe hoje em dia de cunho científico, comentando sobre os
         diversos trabalhos na área de beautifying.

1.2.3    * Classificações   Escrever sobre quais são os tipos possíveis de
         beautifying. Quais são as técnicas mais eficientes, para quais
         linguagens ele se aplicam.

Cronograma:

Id      Atividade                               Data início    Data fim
1.2.2.a Escrever sobre os Formatadores Atuais   31/07/2017     30/08/2017
1.2.2.b Escrever sobre as Pesquisas Atuais      01/09/2017     15/09/2017
1.2.3.a Escrever as Classes de Beautifying      16/09/2017     30/09/2017
1.2.3.b Escrever os Tipos de Beautifying        01/10/2017     30/10/2017
1.2.4.a Fazer a Revisão do Texto Escrito        31/10/2017     20/11/2017

Coloca ChannelManager no tópico e boas praticas, e comenta sobre o modelo de
fork e canais.

Inclui a IA para reconhecer o formatação nos módulos de beautifying. Ela eh
uma heurística, que cada bloco implementa e faz ele gerar um arquivo de
configuração que representa a atual formatação do código (aqui esta o
verdadeiro desafio do trabalho, pesquise trabalhos correlatos).

Inclui sobre a implementação  do semantic linefeed implementado na seção do
linefeed.

Somente incluí somente o que é mais importante para entender o trabalho, não
queira mostrar tudo o que você fez. Uma trabalho extensivo não é necessário,
basta somente apontar como uma referencia que inclua o que você fez.

Mas não esqueça de incluir como foi implementado, i.e., os diagramas UML, se
o sistema é extensível, as bibliotecas utilizadas, como os testes foram
feitos e os resultados deles.

Coloca no capítulo de motivação a seção de trabalhos relacionados. Trabalhos
relacionados com beautifying e com as boas práticas de programação (code clean,
GOF, DEITEL (forminhas das boas práticas)). E deixa claro qual é o problema
que se está resolvendo.

Cria uma capitulo de comparação com os trabalhos relacionados, tanto a parte
teórica (boas práticas), quanto a parte prática (beautifier). Complexidade do
algoritmo do beautifier e o que esse trabalho tem de diferente dos outros.

Coloque evidencias de que funciona o formatador, de boas práticas, mas
práticas e críticas.

Fazer um texto mais didático com exemplos, para os leitores leigos.


recebi alguns comentários dos professores da banca sobre o seu TCC que
gostaria de repassar para voce.

Os dois professores se manifestaram a respeito da formatação que voce
usou. Como eu já havia comentado com voce, eles pedem que voce adote o
formato oficial da BU quando fizer o texto do TCC II no semestre que
vem, pois isso é obrigatório.

Reconhecem o mérito de voce fazer o texto em ingles mas comentam
também sobre a necessidade de fazer uma boa revisão no texto para
corrigir erros gramaticais. Talvez fosse o caso de voce buscar algum
profissional para isso, na versão final do texto

Quanto a estrutura, pedem que voce seja mais claro e conciso nos
objetivos específicos do tcc (há duas seções de objetivos) e o texto
está mais com estilo de um tutorial do que de um TCC.

No resumo, o objetivo também fica dúbio. Diz que vai estudar e depois
também propor uma linguagem. Acho que se a ideia é propor algo, focar
nisso e o resto é para apoio e suporte para fins de comparação

A revisão de literatura está um tanto confusa. Na seção da proposta
tem algumas partes que deveriam estar na fundamentação.

De minha parte, eu concordo com as observações e vamos agora trabalhar
em melhorar a organização do texto. Mas é importante que voce
transcreva o texto para o template da BU.



