

\chapter{Draft for implementation}

This chapter will not be presented on the monograph definitive version.
It only is a collection of articles,
tools and websites which should be considered useful for this work, i.e., ideas and references.

1.2      Rascunho da Monografia para TCC1

1.2.1    * Introdução       Uma breve descrição dos objetivos e da
           justificativa para realização do projeto. Uma breve explicação de
           qual a necessidade e importância deste projeto no escopo da
           Ciência da Computação. Os objetivo gerais e específicos buscados
           após a conclusão e análise deste projeto.

1.2.2    * Bibliográfia     Fazer uma análise do estado da arte em relação
         ao que existe hoje em dia de cunho científico, comentando sobre os
         diversos trabalhos na área de beautifying.

1.2.3    * Classificações   Escrever sobre quais são os tipos possíveis de
         beautifying. Quais são as técnicas mais eficientes, para quais
         linguagens ele se aplicam.

Cronograma:

Id      Atividade                               Data início    Data fim
1.2.2.a Escrever sobre os Formatadores Atuais   31/07/2017     30/08/2017
1.2.2.b Escrever sobre as Pesquisas Atuais      01/09/2017     15/09/2017
1.2.3.a Escrever as Classes de Beautifying      16/09/2017     30/09/2017
1.2.3.b Escrever os Tipos de Beautifying        01/10/2017     30/10/2017
1.2.4.a Fazer a Revisão do Texto Escrito        31/10/2017     20/11/2017

Coloca ChannelManager no tópico e boas praticas, e comenta sobre o modelo de
fork e canais.

Inclui a IA para reconhecer o formatação nos módulos de beautifying. Ela eh
uma heurística, que cada bloco implementa e faz ele gerar um arquivo de
configuração que representa a atual formatação do código (aqui esta o
verdadeiro desafio do trabalho, pesquise trabalhos correlatos).

Inclui sobre a implementação  do semantic linefeed implementado na seção do
linefeed.

Somente incluí somente o que é mais importante para entender o trabalho, não
queira mostrar tudo o que você fez. Uma trabalho extensivo não é necessário,
basta somente apontar como uma referencia que inclua o que você fez.

Mas não esqueça de incluir como foi implementado, i.e., os diagramas UML, se
o sistema é extensível, as bibliotecas utilizadas, como os testes foram
feitos e os resultados deles.

Coloca no capítulo de motivação a seção de trabalhos relacionados. Trabalhos
relacionados com beautifying e com as boas práticas de programação (code clean,
GOF, DEITEL (forminhas das boas práticas)). E deixa claro qual é o problema
que se está resolvendo.

Cria uma capitulo de comparação com os trabalhos relacionados, tanto a parte
teórica (boas práticas), quanto a parte prática (beautifier). Complexidade do
algoritmo do beautifier e o que esse trabalho tem de diferente dos outros.

Coloque evidencias de que funciona o formatador, de boas práticas, mas
práticas e críticas.

Fazer um texto mais didático com exemplos, para os leitores leigos.


recebi alguns comentários dos professores da banca sobre o seu TCC que
gostaria de repassar para voce.

Os dois professores se manifestaram a respeito da formatação que voce
usou. Como eu já havia comentado com voce, eles pedem que voce adote o
formato oficial da BU quando fizer o texto do TCC II no semestre que
vem, pois isso é obrigatório.

Reconhecem o mérito de voce fazer o texto em ingles mas comentam
também sobre a necessidade de fazer uma boa revisão no texto para
corrigir erros gramaticais. Talvez fosse o caso de voce buscar algum
profissional para isso, na versão final do texto

Quanto a estrutura, pedem que voce seja mais claro e conciso nos
objetivos específicos do tcc (há duas seções de objetivos) e o texto
está mais com estilo de um tutorial do que de um TCC.

No resumo, o objetivo também fica dúbio. Diz que vai estudar e depois
também propor uma linguagem. Acho que se a ideia é propor algo, focar
nisso e o resto é para apoio e suporte para fins de comparação

A revisão de literatura está um tanto confusa. Na seção da proposta
tem algumas partes que deveriam estar na fundamentação.

De minha parte, eu concordo com as observações e vamos agora trabalhar
em melhorar a organização do texto. Mas é importante que voce
transcreva o texto para o template da BU.


\medskip
\begin{bluebox}
\begin{enumerate}[leftmargin=*]

    \item Implement tabstops with white space align. The solution - move
    tabstops to fit the text between them and align them with matching tabstops
    on adjacent lines. \url{http://nickgravgaard.com/elastic-tabstops/}
    \url{https://forum.sublimetext.com/t/elastic-tabs/128}

\end{enumerate}
\end{bluebox}

Some existing libraries,
and to be potentially used as `syntect` for assistance in building the software product:

\begin{bluebox}
\begin{enumerate}[leftmargin=*,parsep=0pt]

    \item \url{https://github.com/jbeder/yaml-cpp}
    \item \url{https://github.com/trishume/syntect}
    \item \url{https://github.com/onqtam/doctest}
    \item \url{https://github.com/c42f/tinyformat}
    \item \url{https://github.com/limetext/lime}
    \item \url{https://forum.sublimetext.com/t/disassembling-sublime-text/24824}

\end{enumerate}
\end{bluebox}

The following is a basic list of formatters/beautifiers accessed at
\lword{\url{http://www.softpanorama.org/Utilities/beautifiers.shtml}} on march/2017:

\medskip
\begin{sloppypar}
\begin{bluebox}\RaggedRight
\begin{enumerate}[leftmargin=*,parsep=0pt]

    \item CB210.ZIP - C Beautifier 2.10 - polish C source code (19,406 bytes, 06/22/92)
    \item CL121.ZIP - Codelister 1.21 - print C code with stats (51,110 bytes, 01/10/94)

    \item CPC200.ZIP - CodePrint for C/C++ 2.00 is a full-featured command line driven source
    code reformatter and pretty printer for C++ and C; over 20 customization features including
    auto-indent, adjustable tab spacing, indent styles, flow lines, comment alignment, and line
    editing for consistent white space (140,605 bytes, 01/26/96)

    \item CSCOP120.ZIP - c-scope 1.20 analyzes C source code and produces various reports
    (48,505 bytes, 06/30/95)

    \item HTML : \url{http://www.digital-mines.com/htb/}
    \item HTML : \url{http://www.datacomm.ch/mwoog/software/perl/beautifier.html}
    \item HTML : \url{http://www.watson-net.com/free/perl/s_fhtml.asp}
    \item SQL : \url{http://www.netbula.com/products/sqlb}
    \item Oracle PLSQL : \url{http://www.revealnet.com}
    \item GPL \url{http://www.geocities.com/~starkville/vancbj.html}
    \item GPL \url{http://kevinkelley.mystarband.net/java/dent.html}
    \item Free \url{http://www.tiobe.com/jacobe.htm}
    \item Free \url{http://www.mmsindia.com/JPretty.html}
    \item Free \url{http://members.magnet.at/johann.langhofer/products/jxbeauty/overview.html} (has JBuilder support)
    \item Free \url{http://www.semdesigns.com/Products/Formatters/JavaFormatter.html}
    \item Commercial \$24.99 \url{http://smartbeautify.com}
    \item Commercial \$129 \url{http://www.jindent.com}
    \item Google \url{http://directory.google.com/Top/Computers/Programming/Languages/Java/Development_Tools/Code_Beautifiers/?tc=1}
    \item Java, SQL, HTML, C++ : \url{http://www.semdesigns.com/Products/DMS/DMSToolkit.html}
    \item Java JIndent \url{http://home.wtal.de/software-solutions/jindent}
    \item Java Pat \url{http://javaregex.com/cgi-bin/pat/jbeaut.asp}
    \item Java JStyle \url{http://www.redrival.com/greenrd/java/jstyle}
    \item Java JPrettyPrinter \url{http://www.epoch.com.tw/download/ms/java/java.htm}
    \item Java JxBeauty \url{http://members.nextra.at/johann.langhofer/download/jxbeauty} and the JxBeauty Home
    \item Java beautify percolator
    \item Java list \url{http://www.java.about.com/compute/java/library/weekly/aa102499.htm}
    \item Java html present VasJava2HTML
    \item Java code colorifier and beautifier \url{http://www.mycgiserver.com/~lisali/jccb}
    \item Perl : \url{http://www.consultix-inc.com/www.consultix-inc.com/talk.htm}
    \item Perl : \url{http://www.consultix-inc.com/www.consultix-inc.com/perl_beautifier.html}
    \item Fortran beautifier : \url{http://www.aeem.iastate.edu/Fortran/tools.html}

    \item C++ : BCPP site is at \url{http://dickey.his.com/bcpp/bcpp.html} or at \url{http://www.clark.net/pub/dickey}.
    BCPP ftp site is at \url{ftp://dickey.his.com/bcpp/bcpp.tar.gz}

    \item C++ : \url{http://www.consultix-inc.com/c++b.html}
    \item C : \url{http://www.chips.navy.mil/oasys/c/} and mirror at Oasys
    \item C++, C, Java, Oracle Pro-C Beautifier \url{http://www.geocities.com/~starkville/main.html}

    \item C++, C beautifier \url{http://users.erols.com/astronaut/vim/ccb-1.07.tar.gz} and site at
    \url{http://users.erols.com/astronaut/vim/#vimlinks_src}

    \item GC! GreatCode! is a powerful C/C++ source code beautifier Windows 95/98/NT/2000
    \url{http://perso.club-internet.fr/cbeaudet}

    \item C++ beautifier `SourceStyler' \url{https://web.archive.org/web/20061205061102/http://ochresoftware.com/}
    \item JavaScript : \url{http://jsbeautifier.org/}

\end{enumerate}
\end{bluebox}
\end{sloppypar}


\section{Related Programs for Beautifying}

After the search of the scientific publications on the subject,
it was found some works in the specific area similar to the works done by code formatters (Beautifiers).
However,
source code beautifying is ambiguous definition which can be easily confused with `Prettyprinting',
which is about coloring the text and displaying it to the user.
`Prettyprint' is not is sought in this work implementation,
but rather make changes in the text on how it is structured,
presented to the user and saved on the file system.

Following we may find some references and publications for source code beautifying,
which should be used futurely by this monograph study:

% How to add `parsep` to `itemsep` and set `parsep` to 0pt, when declaring my list?
% https://tex.stackexchange.com/questions/366904/how-to-add-parsep-to-itemsep-and-set-parsep-to-0pt-when-declaring-my-list
\begin{sloppypar}
\begin{bluebox}\RaggedRight
\begin{enumerate}[leftmargin=*,parsep=0pt]

    \item CodeBeautify is an online code beautifier which allows you to beautify
    your source code: \url{http://codebeautify.org/}.

    \item A universal code formatter, written in Dart:
    \url{https://pub.dartlang.org/packages/unifmt}.

    \item Google-java-format is a program that reformats Java source code to
    comply with Google Java Style:
    \url{https://github.com/google/google-java-format}.

    \item CodeFormatter is a Sublime Text 2/3 plugin that supports format
    (beautify) source code.
    \url{https://github.com/akalongman/sublimetext-codeformatter} and
    \url{https://github.com/aukaost/SublimePrettyYAML}

    \item UniversalIndentGUI offers a live preview for setting the parameters of
    nearly any indenter. You change the value of a parameter and directly see
    how your reformatted code will look like. Save your beauty looking code or
    create an anywhere usable batch/shell script to reformat whole directories
    or just one file even out of the editor of your choice that supports
    external tool calls: \url{http://universalindent.sourceforge.net/} and
    \url{https://github.com/danblakemore/universal-indent-gui}.

    \item Language-agnostic pretty-printing through machine learning (uh, like,
    is this possible? YES, apparently). By Terence Parr (primary developer),
    Fangzhou (Morgan) Zhang (help with initial development), Jurgen Vinju
    (co-author of academic paper, help with empirical results and algorithm
    discussions). \url{https://github.com/antlr/codebuff}

    \item To every developer in this world, the closest thing to their heart is
    the text editor of their choice. Over the last few years many new text
    editors has come into the market in both free and paid model, but
    unfortunately not all of them were able to make a real dent on the developer
    community. I remember in my college days we uses to use Notepad++ as our
    beloved text editor, as at that point of time it was one of the popular and
    free text editor with a lot of features for coding. But as time goes on, the
    entire development community started to lean towards sublime text since it’s
    launch.
    \url{https://www.isaumya.com/sublime-text-vs-atom-which-one-i-prefer-most-and-why/}

    \item As a developer, your code editor is one of the most important parts of
    your setup. It can save your wrists and fingers from repetitive strain
    injuries. It can save your eyes from going blind after a coding marathon.
    \url{https://hackernoon.com/virtualstudio-code-the-editor-i-didnt-think-i-needed-16970c8356d5}

    \item VS Code is an Editor while VS is an IDE.
    \url{https://stackoverflow.com/questions/30527522/what-are-the-differences-between-visual-studio-code-and-visual-studio}

    \item What is the difference between VS Code and VS Community?
    Visual Studio Code is a streamlined code editor with support for development operations like
    debugging, task running and version control. It aims to provide just the tools a developer needs
    for a quick code-build-debug cycle and leaves more complex workflows to fuller featured IDEs.
    For more details about the goals of VS Code, see Why VS Code.
    \url{https://code.visualstudio.com/docs/supporting/faq#_licensing}

    \item Reg Replace is a plugin for Sublime Text 2 that allows the creating of commands consisting of
    sequences of find and replace instructions.
    \url{https://forum.sublimetext.com/t/regreplace-plugin/3810}

    \item The main reason I moved was that I find that it’s much slower, the simple things like opening a
    new window for a project should be instantaneous and sadly it’s far from it. As I've said before
    it's all about personal preference, I've gone back to Sublime but Adam for example is sticking
    with it...
    \url{http://engageinteractive.co.uk/blog/atom-review}

    \item \citeonline{aPrettyGoodFormatting}

    \item \url{https://www.researchgate.net/publication/228540036_An_industrial_application_of_context-sensitive_formatting}

    \item \url{http://www.suodenjoki.dk/us/archive/2010/cpp-checkstyle.htm}

    \item \url{http://www.basicinputoutput.com/2014/08/uncrustify-your-bios.html}

    \item \url{http://prettyprinter.de/}

    \item \url{https://github.com/ryanmaxwell/UncrustifyX}

    \item \url{http://www.softpanorama.org/Utilities/beautifiers.shtml}

    \item Understanding the Syntax Parsing
    \url{https://forum.sublimetext.com/t/understanding-the-syntax-parsing/28569}

    "So, part of what I've been working on is a code beautifier that, more or less, aligns and
    indents the code properly based on scanning through the source document."
    ...
    "It hasn't escaped my notice that this is to some degree exactly what the syntax file is doing."

    \item

    {\bfseries Towards a universal code formatter through machine learning:}
    In this paper, we solve the formatter construction problem using a novel approach, one that
    automatically derives formatters for any given language without intervention from a language
    expert. We introduce a code formatter called CODEBUFF that uses machine learning to abstract
    formatting rules from a representative corpus, using a carefully designed feature set. Our
    experiments on Java, SQL, and ANTLR grammars show that CODEBUFF is efficient, has excellent
    accuracy, and is grammar invariant for a given language. It also generalizes to a 4th language
    tested during manuscript preparation.
    \begin{enumerate}[nolistsep,topsep=0pt,label=$\star$]
        \item \url{http://dl.acm.org/citation.cfm?id=2997383}
        \item \url{http://homepages.cwi.nl/~jurgenv/papers/SLE16.pdf}
    \end{enumerate}

    \item \url{https://www.google.com/search?q=universal+source+code+formatter}
    \begin{enumerate}[nolistsep,topsep=0pt,label=$\star$]
        \item \url{https://www.google.com/search?q=universal+source+code+beautifier}
    \end{enumerate}

    \item \url{http://en.wikipedia.org/wiki/Indent_style}
    \begin{enumerate}[nolistsep,topsep=0pt,label=$\star$]
        \item \url{https://en.wikipedia.org/wiki/Programming_style}
        \item \url{https://en.wikipedia.org/wiki/Scope_(computer_science)}
    \end{enumerate}

    \item \url{http://wiki.c2.com/?CodingStyle}
    \begin{enumerate}[nolistsep,topsep=0pt,label=$\star$]
        \item \url{https://github.com/google/code-prettify}
        \item \url{https://github.com/uncrustify/uncrustify}
    \end{enumerate}

    \item \url{https://en.wikipedia.org/wiki/Prettyprint}
    \begin{enumerate}[nolistsep,topsep=0pt,label=$\star$]
        \item \url{https://www.researchgate.net/search.Search.html?query=formatting%20source%20code&type=publication}
        \item \url{https://www.researchgate.net/search.Search.html?query=pretty%20print%20source%20code&type=publication}
    \end{enumerate}

    \item \url{https://github.com/gchpaco/gopprint}
    \begin{enumerate}[nolistsep,topsep=0pt,label=$\star$]
        \item \url{http://dl.acm.org.sci-hub.io/citation.cfm?id=357115}
        \item \url{https://www.cs.indiana.edu/~sabry/papers/yield-pp.pdf}
    \end{enumerate}

    \item \url{http://www.worldcat.org/title/beautiful-code-a-customizable-code-beautifier-for-java/oclc/56564674}
    \begin{enumerate}[nolistsep,topsep=0pt,label=$\star$]
        \item \url{https://www.researchgate.net/publication/34736049_Beautiful_code_a_customizable_code_beautifier_for_Java}
        \item \url{https://vufind.carli.illinois.edu/vf-ncc/Record/ncc_118189/Holdings}
    \end{enumerate}

    \item \url{https://www.researchgate.net/publication/4283921_Smart_Formatter_Learning_Coding_Style_from_Existing_Source_Code}
    \begin{enumerate}[nolistsep,topsep=0pt,label=$\star$]
        \item \url{http://www.ing.unisannio.it/mdipenta/index.html}
        \item \url{https://github.com/iain/rspec-smart-formatter}
    \end{enumerate}

    \item \url{https://www.researchgate.net/publication/2543984_Source_Code_Files_as_Structured_Documents}
    \begin{enumerate}[nolistsep,topsep=0pt,label=$\star$]
        \item \url{https://en.wikipedia.org/wiki/SrcML}
    \end{enumerate}

    \item \url{https://www.researchgate.net/publication/228540036_An_industrial_application_of_context-sensitive_formatting}
    \begin{enumerate}[nolistsep,topsep=0pt,label=$\star$]
        \item \url{https://www.researchgate.net/publication/234809222_Program_indentation_and_comprehensibility}
    \end{enumerate}

\end{enumerate}
\end{bluebox}
\end{sloppypar}



