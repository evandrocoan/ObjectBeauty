

\chapter{Deletar este Capítulo (Rascunho)}

Somente incluí somente o que é mais importante para entender o trabalho, não
queira mostrar tudo o que você fez. Uma trabalho extensivo não é necessário,
basta somente apontar como uma referencia que inclua o que você fez.

Mas não esqueça de incluir como foi implementado, i.e., os diagramas UML, se
o sistema é extensível, as bibliotecas utilizadas, como os testes foram
feitos e os resultados deles.

Coloca no capítulo de motivação a seção de trabalhos relacionados. Trabalhos
relacionados com beautifying e com as boas práticas de programação (code clean,
GOF, DEITEL (forminhas das boas práticas)). E deixa claro qual é o problema
que se está resolvendo.

Cria uma capitulo de comparação com os trabalhos relacionados, tanto a parte
teórica (boas práticas), quanto a parte prática (beautifier). Complexidade do
algoritmo do beautifier e o que esse trabalho tem de diferente dos outros.

Coloque evidencias de que funciona o formatador, de boas práticas, mas
práticas e críticas.

Fazer um texto mais didático com exemplos, para os leitores leigos.

No dia da apresentação venha também com o texto escrito e tenha ele aberto durante a apresentação


recebi alguns comentários dos professores da banca sobre o seu TCC que
gostaria de repassar para voce.

Os dois professores se manifestaram a respeito da formatação que voce
usou. Como eu já havia comentado com voce, eles pedem que voce adote o
formato oficial da BU quando fizer o texto do TCC II no semestre que
vem, pois isso é obrigatório.

Reconhecem o mérito de voce fazer o texto em ingles mas comentam
também sobre a necessidade de fazer uma boa revisão no texto para
corrigir erros gramaticais. Talvez fosse o caso de voce buscar algum
profissional para isso, na versão final do texto

Quanto a estrutura, pedem que voce seja mais claro e conciso nos
objetivos específicos do tcc (há duas seções de objetivos) e o texto
está mais com estilo de um tutorial do que de um TCC.

No resumo, o objetivo também fica dúbio. Diz que vai estudar e depois
também propor uma linguagem. Acho que se a ideia é propor algo, focar
nisso e o resto é para apoio e suporte para fins de comparação

A revisão de literatura está um tanto confusa. Na seção da proposta
tem algumas partes que deveriam estar na fundamentação.

De minha parte, eu concordo com as observações e vamos agora trabalhar
em melhorar a organização do texto. Mas é importante que voce
transcreva o texto para o template da BU.
