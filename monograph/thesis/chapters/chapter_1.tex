

% The \phantomsection command is needed to create a link to a place in the document that is not a
% figure, equation, table, section, subsection, chapter, etc.
%
% When do I need to invoke \phantomsection?
% https://tex.stackexchange.com/questions/44088/when-do-i-need-to-invoke-phantomsection
\cleardoublepage
\phantomsection


% Is it possible to keep my translation together with original text?
% https://tex.stackexchange.com/questions/5076/is-it-possible-to-keep-my-translation-together-with-original-text
\chapter{\chooselang{Introduction}{Introdução}}


\chooselang
{
    Questions like ``What are good programming practices?'' Or ``Why these practices
    are they good?'', are not easy to answer. But every programmer learns to write their codes
    in a certain way, with certain features like using 4 or 8 spaces to
    indentation of lines, always leave a blank line before each control structure as
    if for example, for \textquotesingle, and the like.

    But entering the universe of good practices, there are many things about discoursing. Thus in this work
    specifically work out how to perform the best layout / display of
    computer screen, so that it maximizes and facilitates its understanding.
    Therefore allowing the programmer to disperse more tempe thinking about the problem, than
    try to decipher the information that is presented to him on the screen.

    Within this area of work, you also need to think very carefully about how to share the
    programming codes among programmers. This is why the problem of
    diversity of good programming practice. It happens because often, what is a
    good practice for one `programmer A ', it is not for the other` programmer B'. For example, imagine a
    code where a programmer decided to put before each `if ', a blank line. So it is
    expected that whenever we see a blank line we can potentially find
    `if '. However imagine that another programmer did not like this rule and when he went to write
    his code that involved an `if ', he did not put such a blank line as the other
    programmer had been putting. So when the first programmer reads the code and looks for
    `if'es, it will be waiting for blank lines. But you're going to spend some time looking
    realize that the other programmer has not put them on.

    These differences are due to the diversity of means of learning programming, as much as
    to the tastes, aptitudes and goals of each programmer. So today it becomes a great
    problem as we increasingly need more and more programmers working together between
    themselves, developing the most diverse computational systems. Where the latter is due to the fact that
    complexity of computing systems is increasing, so it is
    more programmers work and share codes.

    So besides just worrying about how the code is displayed on the computer screen, we
    we need to worry about how it will be saved in the file system. Since the
    code, it is vital to use a versioning system to allow
    projects and programmers themselves, have control of code changes. What I allowed and
    can easily track change and enable you to better understand
    that each programmer is doing each time he formalizes a change in the code through
    a `commit ', as in` git` systems, for example.

    This is why when working on a versioning system like `git 'we need to keep the
    code between a single style or good practice defined as standard, due to the fact that
    letting each programmer write as he wants, we will have a lot of noise during the review of
    code and we are determining what the programmer did / wrote, if each programmer re-writes the
    making historical changes like putting new lines before each if. So we will have noise for
    that the focus of a versioning system is to look only at changes that are significant
    to the code, such as creating new functions and not adding new blank lines.

    On the last point, we can also think about an approach to the creation of a new
    version that only focuses on the significant changes to the code, during the moment of the
    review. However, this approach is not ideal because, for example, it gives
    programmers engage in daunting and unproductive wars of code adjustments. For example,
    imagine how it would be every day that you wake up and start working, you have to go through
    code by placing new lines before each of the if \textquotesingle s because the programmer's
    night shift had just removed them?

\section{Goals}

    Establish relationships between good programming practices and efficiency in programming, as well as
    new tool to support the programmer in order to automate the long and diverse process
    of programming in teams of developers with different good programming practices.


\subsection{Specific objectives}

    \begin{enumerate}

        \item

        A study on universal programming tools, allowing
        software, be programmed in all programming languages. So explain the
        differences for other software and why you want a unique tool, instead
        of several.

        \item

        Define, study, determine and classify what are good programming practices and perform
        an in-depth study on good practice in the area of visual
        also like `Beautifying '.

        \item

        A study on the most diverse tools available to support good
        programming, as well as a comparative analysis between them, determining their weaknesses and
        Strong points.

        \item

        The definition of a development flox pattern that allows teams of programmers
        with different good programming practices, work on themselves without intervening and start wars
        good practice.

        \item

        To propose a unique tool that allows several different good programming practices to be
        implemented in the most diverse programming languages and that they can be configured
        and defined to the liking of the programmers who use it.

    \end{enumerate}


\section{Search method}

    The work will be based on research in articles, books, theses, dissertations, authors' websites
    reliable, and through new evidence demonstrated and based on arguments in the course of
    of the evolution of the monograph. It will also present the results of the construction of a
    new tool that proposes the solution of one of the problems presented and explained.

    The last chapter of this proposal is in the topic \autoref{sec:implementation}
    a series of links and references that are pre-selected and can be used in the
    final construction of this work. Notes that in themselves, the parts of the last section will be gradually
    moved to the first part of the text where theoretical research is found, in the course of which its
    information is incorporated into the written work.

    So at the end of the first part of this work that will take place at the conclusion of the discipline
    Course Completion Work 1, there will be only information
    implementation of the proposed tool, which will be implemented in the second part of the monograph
    called \nameref{sec:implementation}, which will be developed at the end of the
    discipline of Course Completion Work 2.
}



{
    Perguntas como ``O que são boas práticas de programação?'' ou ainda ``O por quê estas práticas
    são boas?'', não são fáceis de responder. Mas cada programador aprende a escrever seus códigos
    em uma determinada maneira, com determinadas características como utilizar 4 ou 8 espaços para
    indentação de linhas, sempre deixar uma linha em branco antes de cada estrutura de controle como
    if\textquotesingle s, for\textquotesingle s, e afins.

    Mas entrando o universo de boas práticas, há muitos coisas sobre discorrer. Assim neste trabalho
    especificamente trabalhá-se sobre como realizar a melhor disposição/exibição do código de
    programação na tela do computador, de modo que maximize e facilite o entendimento do mesmo.
    Portanto permitindo que o programador dispersa mais tempe pensando sobre o problema, do que
    tentar decifrar a informação que é apresentado para ele na tela.

    Dentro desta área de trabalho, precisa-se também pensar muito bem sobre como compartilhar os
    códigos de programação dos programadores entre si. Isso por que entra agora o problema da
    diversidade de boas práticas de programação. Ela acontece por que muitas vezes, aquilo que é uma
    boa prática para um `programador A', não é para o outro `programador B'. Por exemplo, imagine um
    código onde um programador decidiu colocar antes de cada `if', uma linha em branco. Portanto é
    de se esperar que sempre que vemos uma linha em branco nos podemos potencialmente encontrar um
    `if'. Entretanto imagine que outro programador não gostou dessa regra e quando ele foi escrever
    seu código que envolvia um `if', ele não colocou a essa tal linha em branco que o outro
    programador vinha colocando. Então quando o primeiro programador for ler o código e procurar por
    `if'es, ele vai estar esperando por linhas em branco. Mas vai perder algum tempo procurando até
    perceber que o outro programador não as colocou.

    Essas diferenças dão-se devido a diversidade de meios de se aprender programação, tanto quanto
    aos gostos, aptidões e objetivos de cada programador. Assim hoje em dia isso torna-se um grande
    problema por que cada vez mais precisamos de mais e mais programadores trabalhem juntos entre
    si, desenvolvendo os mais diversos sistemas computações. Onde este último deve-se ao fato de que
    a complexidade dos sistemas computacionais cresce cada vez mais, portanto requer-se que mais e
    mais programadores trabalhem e compartilhem códigos.

    Então além de nos preocupar-mos somente como o código é exibido na tela do computador, nós
    precisamos nos preocupar sobre como ele será salvo no sistema de arquivos. Já que ao
    compartilhar o código, é vital o uso de um sistema de versionamento para permitir a gerências de
    projetos e os programadores em si, terem o controle de mudanças do código. O que permiti e
    facilmente possa realizar o rastreamento de mudanças e permitir que se possa entender melhor o
    que cada programador está fazendo a cada vez que ele formaliza um mudança no código através de
    uma `commit', como no sistemas `git` por exemplo.

    Isso por que quando trabalhos em um sistema de versionamento como `git' precisamos manter o
    código dentre um único estilo ou boa prática definida como padrão, devido ao fato de que se
    deixar-mos cada programador escrever como ele quiser, teremos muito ruído durante a revisão do
    código e estamos determinando o que o programador fez/escreveu, se cada programador re-escreve o
    histórico fazendo alterações como colocar linhas novas antes de cada if. Assim teremos ruído por
    que o foco de um sistema de versionamento é olhar somente as mudanças que são significativas
    para o código, como a criação de novas funções e não a adição de novas linhas em branco.

    Sobre o último ponto, podemos pensar também sobre uma abordagem da criação de um novo sistema de
    versão que foque somente nas mudanças significativas para o código, durante o momento da
    revisão. Entretanto essa abordagem não é ideal por que, por exemplo, ela dá margem para que
    programadores entrem em guerras tediantes e não produtivas de ajustes de código. Por exemplo,
    imagine o quão seria todo dia que você acorda e começa a trabalhar, você tem que passar pelo
    código colocando linhas novas antes de cada um dos if\textquotesingle s por que o programador do
    turno da noite tinha acabado de remover eles?


\section{Objetivos}

    Estabelecer relações entre boas práticas de programação e eficiência em programar, além de uma
    nova ferramenta ao apoio do programador com o intuito de automatizar o longo e diverso processo
    de programação em equipes de desenvolvedores com distintas boas práticas de programação.


\subsection{Objetivos específicos}

    \begin{enumerate}

        \item

        Um estudo sobre ferramentas universais de programação, que permitam que a partir de um único
        software, seja programado em todas as linguagens de programação. Assim explicar as
        diferenças para os outros softwares e os porquês de querer-se uma ferramenta única, ao invés
        de diversas.

        \item

        Definir, estudar, determinar e classificar o que são boas práticas de programação e realizar
        um estudo aprofundado sobre a as boas práticas da área de disposição visual, conhecidas
        também como `Beautifying'.

        \item

        Um estudo sobre as mais diversas ferramentas existentes para o apoio de boas práticas de
        programação, além de uma análise comparativa entre elas, determinando suas fraquezas e
        pontos fortes.

        \item

        A definição de um padrão de floxo de desenvolvimento que permita equipes de programadores
        com distintas boas práticas de programação, trabalhem em si sem intervir e iniciar guerras
        de boas práticas.

        \item

        Propor uma ferramenta única que permita diversas e distintas boas práticas de programação serem
        implementadas nas mais diversas linguagens de programação e que elas possam ser configuradas
        e definidas ao gosto dos programadores que a usa.

    \end{enumerate}


\section{Método de pesquisa}

    O trabalho será baseado em pesquisas em artigos, livros, teses, dissertações, sites de autores
    confiáveis, e por meio de novas provas demonstradas e baseadas através de argumentos no decorrer
    da evolução da monografia. Também sera apresentado os resultados decorridos da construção de uma
    nova ferramenta que proprõe a solução de um dos problemas apresentados e explicados.

    No último capítulo desta proposta encontra-se no tópico \autoref{sec:implementation} encontra-se
    uma série de links e referências que forma pré-selecionadas e poderão ser utilizadas na
    construção final deste trabalho. Notes que em si, as partes da última seção serão gradativamente
    movida para primeira parte do texto onde encontra-se pesquisa teórica, no decorrer que suas
    informações correlacionadas são incorporadas no trabalho escrito.

    Assim no final da primeira parte desta obra que dará-se no final da conclusão da disciplina
    intitulada de Trabalho de Conclusão de Curso 1, restarão somente as informações destinadas a
    implementação da ferramenta proposta, que serão implementadas na segunda parte da monografia
    denominada \nameref{sec:implementation}, que será desenvolvida no final da conclusão da
    disciplina de Trabalho de Conclusão de Curso 2.
}


