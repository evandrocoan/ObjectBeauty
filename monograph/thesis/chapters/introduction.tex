

% The \phantomsection command is needed to create a link to a place in the document that is not a
% figure, equation, table, section, subsection, chapter, etc.
%
% When do I need to invoke \phantomsection?
% https://tex.stackexchange.com/questions/44088/when-do-i-need-to-invoke-phantomsection
\phantomsection


% Is it possible to keep my translation together with original text?
% https://tex.stackexchange.com/questions/5076/is-it-possible-to-keep-my-translation-together-with-original-text
\chapter{\lang{Introduction}{Introdução}}

    The work first part will be based on research in
    articles, books, theses, dissertations, trusted authors websites,
    and through new demonstrated evidences based on arguments
    in the monograph evolution road.
    Also, present results after building a new tool
    which proposes a solution for the problems presented and detailed.

    In this proposal last chapter which lies on the part
    called `\nameref{sec:software_implementation}', which holds the implementation of a
    tool for code `Beautifying'.



    \section{Context}

    Questions like ``What are good programming practices?'' Or ``Why are these
    practices are good?''Are not easy to answer. But each programmer learns to
    write their codes in a certain way, with certain features like using 4 or 8
    spaces to indent lines, always leave a blank line before each control
    structure as if or for statements, and alike rules.
    \cite{naturalCodingConventions}

    But entering the universe of good practices, there are many things for
    discoursing. Nonetheless, in this work is presented the implementation of
    tool called `Object Beautifier', which specifically dedicates on how to
    perform the best layout/display of programming code on the computer screen,
    so that maximize and facilitate the understanding of same
    \cite{automaticSynthesis}.
    Therefore, allowing the programmer to disperse
    more time and efforts thinking about its coding algorithms problem,
    other than trying to decipher the information that is presented
    to it on the screen through infinit different code layouts
    \cite{usingVersionControlData}.

    Within this work\s area, we need to also think long and hard about how to
    share the programming code of the programmers among you. Now, the problem of
    human diversity, like all big scientific questions -- how do you explain
    something like that -- It can be broken down into sub-questions. It happens
    many times, which is a good practice for a `Programmer A', is not the same
    to another `Programmer B'. For example, imagine some code where a programmer
    decided to put before each `if' statement, a blank line. It is therefore
    expected that whenever we see a blank line we can potentially find a
    matching `if', which can be considered a quite useful pattern matching as
    empty line may call better your attention. \cite{aPrettyGoodFormatting}

    But again this is something heavily dependent of what each one learning
    through their life time. Imagine another programmer do not liked this rule,
    and when he was writing your code involving an `if', he did not put such
    blank line another programmer is expecting. So when the first programmer
    start reading its the code and look for `if', he will be expecting for blank
    lines before its if\s. But will lose some time searching until realize
    another programmer does not put them, or perhaps he forgot to insert them.
    \cite{quantifyingProgramComprehension}

    These differences are due to the diversity of ways we learn programming,
    i.e., to the ways we are used to doing coding, as much as the abilities and
    objectives of every programmer developed. Hence, nowadays it becomes a big
    problem because we increasingly need more and more programmers working
    together developing several and diverse computing systems. Where the latter
    is due to the fact of the complexity of computer systems growing
    increasingly, therefore over requiring programmers working and sharing their
    codes and ideas. \cite{howProgrammersRead}

    Moreover besides only worrying about how the code is displayed on their
    computer screen, we need to worry about on how it will be saved in the file
    system on its `plain-text' mode. Since for code sharing, it is vital for you
    to use a versioning control system\footnote{\url{http://www.codeservedcold.com/version-control-importance/}}
    which enable project manager\s and
    programmers themselves, take control of their code changes
    \cite{redesignOfGit}. It does allow to easily perform the tracking of code
    changes \cite{gettingProductive} and
    allow you to better understand what each programmer is doing
    every time he formalizes a change in the code through a `commit', as in
    `git' systems for example. \cite{usingSourceControl}

    \begin{citacao}
    I'd say there are two main reasons to enforce a single code format in a project. First has
    to do with version control: with everybody formatting the code identically, all changes in
    the files are guaranteed to be meaningful. No more just adding or removing a space here or
    there, let alone reformatting an entire file as a `side effect' of actually changing just a
    line or two. \cite{Geukens}
    \end{citacao}

    That is because while working with a versioning system like `git', we need
    to keep the code among a single style or which we may call a `good practice'
    set as standard for everybody, due the fact of letting each programmer to
    write as he pleases, there will be plenty of noise on the code review and we
    are figuring out what actually each programmer did \cite{quitDiffCalculating}.
    Hence, if every programmer re-writes the history making changes
    like inserting new lines
    before each if, we end up with too much noise and focus of a versioning
    system is to look at only those changes that are significant to the code,
    such as the creation of new functions and not the addition of new blank
    lines. \cite{findingRegressionsInProjects}

    Talking about the last ideia pointed out, we could also think about an
    approach to creating a new version control system which focuses only on
    significant changes to the code, while reviewing code changes. However, this
    approach could not be ideal, as for example, it would allow programmers to
    start tedious wars of unproductive code adjustments. For example, imagine
    how it would be for your every day and have to go through your code
    re-adding new lines before each one of your beloved if\s, just because some
    night shift programmer\footnote{\url{https://blog.codinghorror.com/who-wrote-this-crap/}}
    had just removed them?



    \section{Research Goals}

    Beforehand due the scope limitation for a Graduation Thesis,
    we should only think about a basic, simple,
    and yet reusable core of features.

    \begin{enumerate}
        \item A Software Product with a great Object Orientation and possibilities of extension of features,
        decent research on the state of the art.

        \item Ranking all code formatting classes (beautifying) applicable.
        Including a study on what does is source beautifying,
        how to do such and why.

        \item Establish relationships between good programming practices and efficiency in programming,
        in addition to a new tool to support programmers in order to automate the long and diverse
        programming process in teams of developers with different programming `best practices'.

        \item Define, determine and classify which one are good programming practices and
        perform an in-depth study on the good practices on visual layout area,
        also known as code `Beautifying'.

        \item The definition of a flow pattern of development allowing teams of
        developers with different programming best practices,
        to work without intervene with each other up to start wars of `best good practices'.

        \item Discourse on the variety of existing tools for the support of good programming practices,
        with a comparative analysis between them,
        determining their weaknesses and strengths.
    \end{enumerate}



    \section{Implementation Goals}

    Propose a unique tool that allowing several and distinct
    programming `best practices' being implemented in several programming
    languages, which can be configured and set accordingly to their wishes,
    from a single software working well behaved across all programming languages.

    Moreover, explain the differences for other softwares and the benefits
    of a unique tool, instead of several heavily different ones.

    From this point, a sketch is presented on the problem, solutions,
    information as for why to want make such software, or even why do we want to
    beautifying things:

    \begin{enumerate}[leftmargin=*]
        \item There are many different tools, sometimes paid, and difficult to
              complete. \cite{universalCodeFormatter}

        \item Many programming languages exist, so always having Beautifier
              software for each of them is very laborious
              \cite{universalCodeFormatter}. But the approach to a Universal
              Beautifier proposed in this work, would allow easily new languages to be
              added, being completely different from previous ones, or alike. And in
              case of similarities between them, it is enough to reuse their
              configuration structures already implemented.

        \item Looking for a Beautifier for each one of them because programmers
              currently work daily with several of these languages, and they are not
              similar. So you need to configure several beautifiers to do the
              formatting. This is a problem because only a few beautifiers are more
              complete, and every time you need to make a change in the formatting
              style, you must manually propagate the same change over several
              different program configuration files, which is bad because it takes the
              user a lot of time to learn how to handle many different types of
              settings. \cite{Schweitzer}

        \item In the case of ideal Beautifier, a change in your styling is
              propagated to all languages. And if you want to leave some language out
              of it, you just need to remove it from the list on which the
              configuration block applies to, and `a)' leave it out so no change is
              applied to. Or `b)' create a new block including only the block within
              the desired settings.

    \end{enumerate}



% 1.2      Rascunho da Monografia para TCC1
%
% 1.2.1    * Introdução       Uma breve descrição dos objetivos e da
%            justificativa para realização do projeto. Uma breve explicação de
%            qual a necessidade e importância deste projeto no escopo da
%            Ciência da Computação. Os objetivo gerais e específicos buscados
%            após a conclusão e análise deste projeto.
%
% 1.2.2    * Bibliográfia     Fazer uma análise do estado da arte em relação
%          ao que existe hoje em dia de cunho científico, comentando sobre os
%          diversos trabalhos na área de beautifying.
%
% 1.2.3    * Classificações   Escrever sobre quais são os tipos possíveis de
%          beautifying. Quais são as técnicas mais eficientes, para quais
%          linguagens ele se aplicam.

% Cronograma:
%
% Id      Atividade                               Data início    Data fim
% 1.2.2.a Escrever sobre os Formatadores Atuais   31/07/2017     30/08/2017
% 1.2.2.b Escrever sobre as Pesquisas Atuais      01/09/2017     15/09/2017
% 1.2.3.a Escrever as Classes de Beautifying      16/09/2017     30/09/2017
% 1.2.3.b Escrever os Tipos de Beautifying        01/10/2017     30/10/2017
% 1.2.4.a Fazer a Revisão do Texto Escrito        31/10/2017     20/11/2017

% Coloca ChannelManager no tópico e boas praticas, e comenta sobre o modelo de
% fork e canais.
%
% Inclui a IA para reconhecer o formatação nos módulos de beautifying. Ela eh
% uma heurística, que cada bloco implementa e faz ele gerar um arquivo de
% configuração que representa a atual formatação do código (aqui esta o
% verdadeiro desafio do trabalho, pesquise trabalhos correlatos).
%
% Inclui sobre a implementação  do semantic linefeed implementado na seção do
% linefeed.
%
% Somente incluí somente o que é mais importante para entender o trabalho, não
% queira mostrar tudo o que você fez. Uma trabalho extensivo não é necessário,
% basta somente apontar como uma referencia que inclua o que você fez.
%
% Mas não esqueça de incluir como foi implementado, i.e., os diagramas UML, se
% o sistema é extensível, as bibliotecas utilizadas, como os testes foram
% feitos e os resultados deles.
%
% Coloca no capítulo de motivação a seção de trabalhos relacionados. Trabalhos
% relacionados com beautifying e com as boas práticas de programação (code clean,
% GOF, DEITEL (forminhas das boas práticas)). E deixa claro qual é o problema
% que se está resolvendo.
%
% Cria uma capitulo de comparação com os trabalhos relacionados, tanto a parte
% teórica (boas práticas), quanto a parte prática (beautifier). Complexidade do
% algoritmo do beautifier e o que esse trabalho tem de diferente dos outros.
%
% Coloque evidencias de que funciona o formatador, de boas práticas, mas
% práticas e críticas.
%
% Fazer um texto mais didático com exemplos, para os leitores leigos.


    \section{Related Works}

    The problem on Integration Tests to find which tests must to be re\hyp{}run
    after some code has been changed. There is no way of determine which
    tests must to re\hyp{}run after some code has changed without knowing the
    language syntax and semantics. TODO put reference to this

    \url{http://editorconfig.org/}

