

\chapter{\lang{Introduction}{Introdução}}

\lang{%
    The work first part will be based on research in
    articles, books, theses, dissertations, trusted authors websites,
    and through new demonstrated evidences based on arguments
    in the monograph evolution.
    Also, present results after building a new tool
    which proposes a solution for the problems presented and detailed.
}{%
    A \hyperref[sec:primeira_parte]{primeira parte} deste trabalho será baseado em artigos,
    livros, outras dissertações,
    sites confiáveis e por novas evidências demonstradas no desenvolvimento deste trabalho.
}%
\lang{%
    In this proposal last chapter which lies on the part
    called `\nameref{sec:segunda_parte}', which holds the implementation of a
    tool for code `Beautifying'.
}{%
    E finalmente,
    na \hyperref[sec:segunda_parte]{segunda parte} deste trabalho,
    será apresentada a sugestão e implementação de uma ferramenta.
}


\section{\lang{Context}{Contextualização}}

\lang{%
    Questions like ``What are good programming practices?'' Or ``Why are these
    practices are good?''Are not easy to answer. But each programmer learns to
    write their codes in a certain way, with certain features like using 4 or 8
    spaces to indent lines, always leave a blank line before each control
    structure as if or for statements, and alike rules.
    \cite{naturalCodingConventions}
}{%
    Perguntas como ``O quê são boas práticas de programação'' não são fáceis de responder.
    Cada programador aprende a escrever seu códigos de uma determinada maneira,
    seja utilizando 2, 4,
    ou 8 espaços para criar blocos de indentação,
    ou sempre deixar uma linha em branco antes de estruturas de controle como if\s,
    while\s, etc. \cite{naturalCodingConventions}
}

\lang{%
    But entering the universe of good practices, there are many things for
    discoursing. Nonetheless, in this work is presented the implementation of
    tool called `Object Beautifier', which specifically dedicates on how to
    perform the best layout/display of programming code on the computer screen,
    so that maximize and facilitate the understanding of same
    \cite{automaticSynthesis}.
    Therefore, allowing the programmer to disperse
    more time and efforts thinking about its coding algorithms problem,
    other than trying to decipher the information that is presented
    to it on the screen through infinit different code layouts
    \cite{usingVersionControlData}.
}{%
    Assim,
    este trabalho será focado nas ferramentas disposição ou organização do código,
    também conhecidos como ``Source Code Beautifiers'' ou ``Formatadores de Código''.
    Com isso,
    espera\hyp{}se que o desenvolvimento de códigos demande menos esforços,
    pois o desenvolvedor pode forcar seus esforços pensando mais sobre o problema que está sendo resolvido,
    ao contrário de tentar decifrar o que está escrito em sua frente por meio de layouts extravagantes ou incomuns.
    \cite{usingVersionControlData}
}

\lang{%
    Within this work\s area, we need to also think long and hard about how to
    share the programming code of the programmers among you. Now, the problem of
    human diversity, like all big scientific questions -- how do you explain
    something like that -- It can be broken down into sub\hyp{}questions. It happens
    many times, which is a good practice for a `Programmer A', is not the same
    to another `Programmer B'. For example, imagine some code where a programmer
    decided to put before each `if' statement, a blank line. It is therefore
    expected that whenever we see a blank line we can potentially find a
    matching `if', which can be considered a quite useful pattern matching as
    empty line may call better your attention. \cite{aPrettyGoodFormatting}
}{%
    Programação de computadores usualmente não é uma tarefa solitária.
    Muitos projetos podem sempre ser escritos pela mesma pessoa durante toda a vida desse projeto mas,
    o quão grande pode ser um projeto que uma pessoa pode fazer sozinha?
    Poderia uma pessoa escrever sozinha todo o sistema de controle de tráfego aéreo do Estados Unidos?
    Uma vez que precisa\hyp{}se escrever sistemas computacionais maiores,
    % TODO: cite reference to `escrever sistemas computacionais maiores`
    tende\hyp{}se a colocar vários programadores,
    engenheiros de software, gerentes de projeto, etc,
    para que se possa escrever o sistema computacional requerido
    entro do prazo que dado projeto computacional possui.
    Com a criação de sistemas computacionais cada vez mais complexos,
    torna\hyp{}se um problema juntar em um único projeto diversos programadores,
    pois cada um com suas características pessoais distintas,
    entende melhor código escrito de acordo com seu costume,
    já que pode saber melhor como localizar e
    entender os seus elementos \cite{howProgrammersRead}.

    Então,
    quando diferentes programadores são juntos nesses grandes projetos,
    como essas pessoas todas poderiam trabalhar juntas,
    mesmo elas não tendo o mesmo background?
    Assim,
    uma barreira extra para cada um desses programadores ao desenvolvimento do projeto será
    aprender as diferenças de trabalho que cada um desses desenvolvedores possuem entre sí.
    Uma das diferenças é a maneira no qual cada programador organiza
    estrutura de seu código no requisito de formatação.
    Por exemplo,
    imagine que um certo `Programador A' tem o costume de deixar uma linha
    em branco antes de cada estrutura de controle do tipo `if',
    `for',
    etc.
    Já um outro `Programador B',
    possui o costume oposto,
    ele jamais deixa uma linha em branco antes dessas estruturas de controle.
    Então,
    o quê pode dar de errado quando ambos os programadores `A' e
    `B' trabalham num mesmo projeto?

    Existem algumas possibilidades,
    uma delas poderia ser que a metade do código escrita pelo `Programador A' estará de um jeito.
    Já a outra metade escrita pelo `Programador B' estará de outro.
    Então,
    que tipos de problema isso poderia causar?
    Uma vez que o código não está mais todo escrito sobre um mesmo padrão,
    certas heurísticas perdem possuem sua taxa de acerto reduzidas ao fazer o reconhecimento de padrões.
    Um exemplo de padrão poderia ser rapidamente identificar onde pode estar uma estrutura de controle
    como `if' ou `for' de acordo com a presença ou não de uma linha em branco antes dela,
    já que a presença de uma linha em branco faz um destacamento maior sendo
    reconhecida como um separador de blocos de código distintos,
    \cite{aPrettyGoodFormatting}
}

\lang{%
    But again this is something heavily dependent of what each one learning
    through their life time. Imagine another programmer do not liked this rule,
    and when he was writing your code involving an `if', he did not put such
    blank line another programmer is expecting. So when the first programmer
    start reading its the code and look for `if', he will be expecting for blank
    lines before its if\s. But will lose some time searching until realize
    another programmer does not put them, or perhaps he forgot to insert them.
    \cite{quantifyingProgramComprehension}
}{%
    E por mais uma vez,
    estas heurísticas são fortemente dependentes do aprendizado de cada programador durante sua vida.
    Imagine um segundo programador que não esta habituado com a heurística de outro,
    e toda vez que ele for ler o código deste terceiro programador,
    ela estará perdido para poder se localizar rapidamente dentro do código.
    E portanto,
    o que isso significa?
    Que ele terá que atentamente ler o código escrito por este terceiro programador,
    uma vez que a sua organização difere dos padrões no qual ele sabe como
    rapidamente se localizar. \cite{quantifyingProgramComprehension}
}

\lang{%
    These differences are due to the diversity of ways we learn programming,
    i.e., to the ways we are used to doing coding, as much as the abilities and
    objectives of every programmer developed. Hence, nowadays it becomes a big
    problem because we increasingly need more and more programmers working
    together developing several and diverse computing systems. Where the latter
    is due to the fact of the complexity of computer systems growing
    increasingly, therefore over requiring programmers working and sharing their
    codes and ideas. \cite{howProgrammersRead}
}{%
    As diferenças entres os diversos estilos de programação entre os programadores deve\hyp{}se à diversos fatores.
    \begin{enumerate}
        \item Alguns programadores simplesmente fazem de um determinado jeito
        por que foi assim que ele aprenderam de seus tutores,
        e nunca preocuparam\hyp{}se em questionar se poderia ser feito de outra maneira.
        Uma caso onde isso pode acontecer é devido se ao cansaço que algumas linguagens de
        programação como \latex causam por sua sintaxe incomum e extraordinária.
        Uma vez que você já está cansado de ver tantos errors,
        você simplesmente para de questionar como o texto poderia ser formado e aceita o que
        você encontra escrito e funcionando como parte da sintaxe da linguagem.

        \item Outros podem ter determinadas características na formatação da escrita
        de seus códigos devido a alguma característica emocional ou física.
        Emocionalmente você pode ``não gostar'' de alguma característica por que sofreu
        algum trauma ou simplesmente esta muito acostumado a ter a ter
        visualmente todos os elementos muito próximos um dos outros na tela.
        Assim,
        mesmo tal característica não ser nada fácil para que outras pessoas entendam o seu código,
        ela pode ser muito fácil de ser compreendida por você,
        que a muitos anos já está habituado a realizar a escrita de seus códigos desta determinada forma e
        portanto é muito hábil e
        muito rapidamente consegue entender o que escreve.

        \item Uma característica física que pode determinar a formatação
        de código de alguém pode ser um teclado defeituoso.
        Caso determinada tecla não funcione adequadamente e
        a pessoa não se dê ao trabalho de comprar um teclado novo,
        a pessoa pode então adquirir um novo hábito,
        e uma vez que ela aprenda este habito,
        ela provavelmente continuará com ele mesmo depois de comprar um teclado novo.
        Por exemplo,
        suponha que a sua tecla `TAB' não funcione.
        Assim,
        ao invés de somente pressionar `TAB' uma vez para indentar seu código,
        você ganha habito de pressionar a tecla de espaços 4 vezes.
        Portanto,
        nasce um motivo para alguém utilizar espaços ao invés `TAB's para realizar a indentação de seu código.
    \end{enumerate}
}

\lang{%
    Moreover besides only worrying about how the code is displayed on their
    computer screen, we need to worry about on how it will be saved in the file
    system on its `plain\hyp{}text' mode. Since for code sharing, it is vital for you
    to use a versioning control system\footnote{\url{http://www.codeservedcold.com/version-control-importance/}}
    which enable project manager\s and
    programmers themselves, take control of their code changes
    \cite{redesignOfGit}. It does allow to easily perform the tracking of code
    changes \cite{gettingProductive} and
    allow you to better understand what each programmer is doing
    every time he formalizes a change in the code through a `commit', as in
    `git' systems for example. \cite{usingSourceControl}
}{%
    Indo além de como programadores leem código enquanto estão escrevendo ele,
    também tem\hyp{}se que preocupar como o código será salvo no sistema de arquivos,
    em seu formato de texto simples.
    Já que para compartilhar código e
    trabalhar em times de forma eficiente,
    é essential utilizar\hyp{}se um sistema de
    versionamento\footnote{\url{http://www.codeservedcold.com/version-control-importance/}}
    que permita aos gerentes de projetos e
    os próprios programadores,
    ter o controle de suas mudanças no código \cite{redesignOfGit}.
    Tal ferramenta deve permitir facilmente realizar o rastreamento das mudanças \cite{gettingProductive} e
    assim,
    permitir que você entenda melhor o que cada programador está fazendo,
    todas as vezes que eles formalizam uma nova alteração no código\hyp{}fonte,
    por exemplo,
    através de uma `commit' como conhecidas em sistemas `git' \cite{usingSourceControl}.
}

\begin{citacao}
    Eu diria que há duas razões principais para impor um formato de código único em um projeto.
    Primeiro,
    tem a ver com controle de versão:
    com todo mundo formatando o código de forma idêntica,
    todas as alterações nos arquivos têm a garantia de terem algum significado.
    Nada mais de coisas como adicionar ou remover um espaço aqui ou ali aleatoriamente,
    muito menos reformatar um arquivo inteiro como um `efeito
    colateral' de alterar apenas uma linha ou duas.
    \cite[our translation]{Geukens} \footnote{\englishword{I'd say there
    are two main reasons to enforce a single code format in a project.
    First has to do with version control:
    with everybody formatting the code identically,
    all changes in the files are guaranteed to be meaningful.
    No more just adding or removing a space here or there,
    let alone reformatting an entire file as a `side effect' of actually changing just a line or two.
    }}
\end{citacao}

\lang{%
    That is because while working with a versioning system like `git', we need
    to keep the code among a single style or which we may call a `good practice'
    set as standard for everybody, due the fact of letting each programmer to
    write as he pleases, there will be plenty of noise on the code review and we
    are figuring out what actually each programmer did \cite{quitDiffCalculating}.
    Hence, if every programmer re\hyp{}writes the history making changes
    like inserting new lines
    before each if, we end up with too much noise and focus of a versioning
    system is to look at only those changes that are significant to the code,
    such as the creation of new functions and not the addition of new blank
    lines. \cite{findingRegressionsInProjects}
}{%
    Estas conclusões devem\hyp{}se por que ao trabalhar com um sistema de versionamento como `git',
    recomenda\hyp{}se manter todo o código sobre um único stilo,
    ou aquilo que conhece\hyp{}se ``Boas Práticas'' colocado como um padrão a todos.
    Segue\hyp{}se isso devido aos problemas gerado ao permitir
    que cada programador escreva como lhe agrada melhor.
    Se cada um escreve como bem entende,
    irá existir ruídos de formação em abundância espalhados por tudo o código\hyp{}fonte,
    o quê complicará a revisão de código,
    dificultando entender o que cada programador fez \cite{quitDiffCalculating}.
    Portando,
    caso todos os programadores decidam re~=escrever o histórico,
    fazendo mudanças como inserindo novas linhas antes de cada `if':
    \begin{enumerate}
        \item Eles irão disperdiçar um tempo que eles poderiam estar utilizando para escrever código.
        \item O histórico git irá conter ruídos desnecessários,
        pois o foco de um sistema de versionamento são conter mudanças se sejam significantes,
        como criações de novas funções,
        correção de bugs, etc. \cite{findingRegressionsInProjects}
    \end{enumerate}
}

\lang{%
    Talking about the last ideia pointed out, we could also think about an
    approach to creating a new version control system which focuses only on
    significant changes to the code, while reviewing code changes. However, this
    approach could not be ideal, as for example, it would allow programmers to
    start tedious wars of unproductive code adjustments. For example, imagine
    how it would be for your every day and have to go through your code
    re\hyp{}adding new lines before each one of your beloved if\s, just because some
    night shift programmer\footnote{\url{https://blog.codinghorror.com/who-wrote-this-crap/}}
    had just removed them?
}{%
    Sobre a última ideia apontada,
    poderíamos também pensar em uma abordagem para criar um novo sistema de controle de versão,
    que foca apenas mudanças significativas no código enquanto faz-se a revisão das mudanças no código.
    No entanto,
    essa abordagem poderia não ser ideal,
    pois, por exemplo,
    permitiria que os programadores iniciassem guerras tediosas de ajustes de código improdutivos.
    Por exemplo,
    imagine como seria se todo os dias você tivesse que passar pelo código
    re~=adicionando novas linhas antes de cada um dos seus amados if\s,
    só porque algum programador
    do turno da
    noite\footnote{\url{https://blog.codinghorror.com/who-wrote-this-crap/}}
    tinha acabado de
    removê-los?
}


\section{\lang{Goals}{Objetivos}}

\lang{%
    Beforehand due the scope limitation for a Graduation Thesis,
    we should only think about a basic, simple,
    and yet reusable core of features.
}{%
    Devido ao trabalho tratar-se de uma tese de graduação,
    limita~=se seu escopo de trabalho a somente realizara criação de um núcleo reduzido de recursos.
}%
\lang{%
\begin{enumerate}
    \item
        A Software Product with a great Object Orientation and possibilities of extension of features,
        decent research on the state of the art.
    \item
        Ranking all code formatting classes (beautifying) applicable.
        Including a study on what does is source beautifying,
        how to do such and why.
    \item
        Establish relationships between good programming practices and efficiency in programming,
        in addition to a new tool to support programmers in order to automate the long and diverse
        programming process in teams of developers with different programming `best practices'.
    \item
        Define, determine and classify which one are good programming practices and
        perform an in\hyp{}depth study on the good practices on visual layout area,
        also known as code `Beautifying'.
    \item
        The definition of a flow pattern of development allowing teams of
        developers with different programming best practices,
        to work without intervene with each other up to start wars of `best good practices'.
    \item
        Discourse on the variety of existing tools for the support of good programming practices,
        with a comparative analysis between them,
        determining their weaknesses and strengths.
\end{enumerate}
}{%
    Enumerar e
    explicar as ferramentas de formatação de código~=fonte e
    realizar o desenvolvimento de uma nova ferramenta de formatação de código~=fonte que seja extensível.
}

\subsection{Objetivos Específicos}

\lang{%
    Propose a unique tool that allowing several and distinct
    programming `best practices' being implemented in several programming
    languages, which can be configured and set accordingly to their wishes,
    from a single software working well behaved across all programming languages.
}{%
    Propor uma ferramenta que permita várias e
    diferentes linguagens de programação sejam formatadas utilizando o mesmo conjunto de configurações.
    Utilizando esta única ferramenta,
    novas linguagens podem ser configuradas e
    adicionadas de acordo com a vontade de seus usuários.
}

\lang{%
    Moreover, explain the differences for other softwares and the benefits
    of a unique tool, instead of several heavily different ones.
    From this point, a sketch is presented on the problem, solutions,
    information as for why to want make such software, or even why do we want to
    beautifying things:
}{%
    E explicar como esta nova ferramenta difere das demais já existentes,
    e qual as vantagens de ter uma ferramenta para formatar todas as linguagens de programação,
    ao contrário de ter uma ferramenta para cada linguagem.
}


\section{ESTRUTURAÇÃO DO TEXTO}

No capítulo 2,
será mostrado um revisão da literatura sobre os conceitos básicos.

TODO.

