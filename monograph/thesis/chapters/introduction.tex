

% The \phantomsection command is needed to create a link to a place in the document that is not a
% figure, equation, table, section, subsection, chapter, etc.
%
% When do I need to invoke \phantomsection?
% https://tex.stackexchange.com/questions/44088/when-do-i-need-to-invoke-phantomsection
\phantomsection


% Is it possible to keep my translation together with original text?
% https://tex.stackexchange.com/questions/5076/is-it-possible-to-keep-my-translation-together-with-original-text
\chapter{\lang{Introduction}{Introdução}}


\begin{englishtext}

    \section{Context}

    Questions like ``What are good programming practices?'' Or ``Why are these
    practices are good?''Are not easy to answer. But each programmer learns to
    write their codes in a certain way, with certain features like using 4 or 8
    spaces to indent lines, always leave a blank line before each control
    structure as if or for statements, and alike rules.
    \cite{naturalCodingConventions}

    But entering the universe of good practices, there are many things for
    discoursing. Nonetheless, in this work is presented the implementation of
    tool called `Object Beautifier', which specifically dedicates on how to
    perform the best layout/display of programming code on the computer screen,
    so that maximize and facilitate the understanding of same
    \cite{automaticSynthesis}.
    Therefore, allowing the programmer to disperse
    more time and efforts thinking about its coding algorithms problem,
    other than trying to decipher the information that is presented
    to it on the screen through infinit different code layouts
    \cite{usingVersionControlData}.

    Within this work\s area, we need to also think long and hard about how to
    share the programming code of the programmers among you. Now, the problem of
    human diversity, like all big scientific questions -- how do you explain
    something like that -- It can be broken down into sub-questions. It happens
    many times, which is a good practice for a `Programmer A', is not the same
    to another `Programmer B'. For example, imagine some code where a programmer
    decided to put before each `if' statement, a blank line. It is therefore
    expected that whenever we see a blank line we can potentially find a
    matching `if', which can be considered a quite useful pattern matching as
    empty line may call better your attention. \cite{aPrettyGoodFormatting}

    But again this is something heavily dependent of what each one learning
    through their life time. Imagine another programmer do not liked this rule,
    and when he was writing your code involving an `if', he did not put such
    blank line another programmer is expecting. So when the first programmer
    start reading its the code and look for `if', he will be expecting for blank
    lines before its if\s. But will lose some time searching until realize
    another programmer does not put them, or perhaps he forgot to insert them.
    \cite{quantifyingProgramComprehension}

    These differences are due to the diversity of ways we learn programming,
    i.e., to the ways we are used to doing coding, as much as the abilities and
    objectives of every programmer developed. Hence, nowadays it becomes a big
    problem because we increasingly need more and more programmers working
    together developing several and diverse computing systems. Where the latter
    is due to the fact of the complexity of computer systems growing
    increasingly, therefore over requiring programmers working and sharing their
    codes and ideas. \cite{howProgrammersRead}

    Moreover besides only worrying about how the code is displayed on their
    computer screen, we need to worry about on how it will be saved in the file
    system on its `plain-text' mode. Since for code sharing, it is vital for you
    to use a versioning control system\footnote{\url{http://www.codeservedcold.com/version-control-importance/}}
    which enable project manager\s and
    programmers themselves, take control of their code changes
    \cite{redesignOfGit}. It does allow to easily perform the tracking of code
    changes \cite{gettingProductive} and
    allow you to better understand what each programmer is doing
    every time he formalizes a change in the code through a `commit', as in
    `git' systems for example. \cite{usingSourceControl}

    That is because while working with a versioning system like `git', we need
    to keep the code among a single style or which we may call a `good practice'
    set as standard for everybody, due the fact of letting each programmer to
    write as he pleases, there will be plenty of noise on the code review and we
    are figuring out what actually each programmer did \cite{quitDiffCalculating}.
    Hence, if every programmer re-writes the history making changes
    like inserting new lines
    before each if, we end up with too much noise and focus of a versioning
    system is to look at only those changes that are significant to the code,
    such as the creation of new functions and not the addition of new blank
    lines. \cite{findingRegressionsInProjects}

    Talking about the last ideia pointed out, we could also think about an
    approach to creating a new version control system which focuses only on
    significant changes to the code, while reviewing code changes. However, this
    approach could not be ideal, as for example, it would allow programmers to
    start tedious wars of unproductive code adjustments. For example, imagine
    how it would be for your every day and have to go through your code
    re-adding new lines before each one of your beloved if\s, just because some
    night shift programmer\footnote{\url{https://blog.codinghorror.com/who-wrote-this-crap/}}
    had just removed them?



    \section{Goals}

    The work first part will be based on research in
    articles, books, theses, dissertations, trusted authors websites,
    and through new demonstrated evidences based on arguments
    in the monograph evolution road.
    Also, present results after building a new tool
    which proposes a solution for the problems presented and detailed.

    In this proposal last chapter which lies on the part
    called `\nameref{sec:implementation}', which holds the implementation of a
    tool for code `Beautifying'.


    \subsection{Specific Goals}

    \begin{enumerate}

        \item Establish relationships between good programming practices and efficiency in
        programming, in addition to a new tool to support programmers in order to
        automate the long and diverse programming process in teams of developers
        with different programming `best practices'.

        \item Define, determine and classify which one are good programming
        practices and perform an in-depth study on the good practices on visual
        layout area, also known as code `Beautifying'.

        \item The definition of a flow pattern of development allowing teams of
        developers with different programming best practices, to work without
        intervene with each other up to start wars of `best good practices'.

        \item Discourse on the variety of existing tools for the support of good
        programming practices, with a comparative analysis between them,
        determining their weaknesses and strengths.

    \end{enumerate}



    \section{Implementation Goals}

    Propose a unique tool that allowing several and distinct
    programming `best practices' being implemented in several programming
    languages, which can be configured and set accordingly to their wishes,
    from a single software working well behaved across all programming languages.

    Moreover, explain the differences for other softwares and the benefits
    of a unique tool, instead of several heavily different ones.


\begin{comment}
    1.2      Rascunho da Monografia para TCC1

    1.2.1    * Introdução       Uma breve descrição dos objetivos e da
               justificativa para realização do projeto. Uma breve explicação de
               qual a necessidade e importância deste projeto no escopo da
               Ciência da Computação. Os objetivo gerais e específicos buscados
               após a conclusão e análise deste projeto.

    1.2.2    * Bibliográfia     Fazer uma análise do estado da arte em relação
             ao que existe hoje em dia de cunho científico, comentando sobre os
             diversos trabalhos na área de beautifying.

    1.2.3    * Classificações   Escrever sobre quais são os tipos possíveis de
             beautifying. Quais são as técnicas mais eficientes, para quais
             linguagens ele se aplicam.

    Cronograma:

    Id      Atividade                               Data início    Data fim
    1.2.2.a Escrever sobre os Formatadores Atuais   31/07/2017     30/08/2017
    1.2.2.b Escrever sobre as Pesquisas Atuais      01/09/2017     15/09/2017
    1.2.3.a Escrever as Classes de Beautifying      16/09/2017     30/09/2017
    1.2.3.b Escrever os Tipos de Beautifying        01/10/2017     30/10/2017
    1.2.4.a Fazer a Revisão do Texto Escrito        31/10/2017     20/11/2017
\end{comment}



\end{englishtext}


% % Portuguese
% \lang{}{

    Perguntas como ``O que são boas práticas de programação?'' ou ainda ``O por
    quê estas práticas são boas?'', não são fáceis de responder. Mas cada
    programador aprende a escrever seus códigos em uma determinada maneira, com
    determinadas características como utilizar 4 ou 8 espaços para indentação de
    linhas, sempre deixar uma linha em branco antes de cada estrutura de
    controle como if\s, for\s, e afins.

    Mas entrando o universo de boas práticas, há muitos coisas sobre discorrer.
    Assim neste trabalho especificamente trabalhá-se sobre como realizar a
    melhor disposição/exibição do código de programação na tela do computador,
    de modo que maximize e facilite o entendimento do mesmo. Portanto permitindo
    que o programador dispersa mais tempe pensando sobre o problema, do que
    tentar decifrar a informação que é apresentado para ele na tela.

    Dentro desta área de trabalho, precisa-se também pensar muito bem sobre como
    compartilhar os códigos de programação dos programadores entre si. Isso por
    que entra agora o problema da diversidade de boas práticas de programação.
    Ela acontece por que muitas vezes, aquilo que é uma boa prática para um
    `programador A', não é para o outro `programador B'. Por exemplo, imagine um
    código onde um programador decidiu colocar antes de cada `if', uma linha em
    branco. Portanto é de se esperar que sempre que vemos uma linha em branco
    nos podemos potencialmente encontrar um `if'. Entretanto imagine que outro
    programador não gostou dessa regra e quando ele foi escrever seu código que
    envolvia um `if', ele não colocou a essa tal linha em branco que o outro
    programador vinha colocando. Então quando o primeiro programador for ler o
    código e procurar por `if'es, ele vai estar esperando por linhas em branco.
    Mas vai perder algum tempo procurando até perceber que o outro programador
    não as colocou.

    Essas diferenças dão-se devido a diversidade de meios de se aprender
    programação, tanto quanto aos gostos, aptidões e objetivos de cada
    programador. Assim hoje em dia isso torna-se um grande problema por que cada
    vez mais precisamos de mais e mais programadores trabalhem juntos entre si,
    desenvolvendo os mais diversos sistemas computações. Onde este último
    deve-se ao fato de que a complexidade dos sistemas computacionais cresce
    cada vez mais, portanto requer-se que mais e mais programadores trabalhem e
    compartilhem códigos.

    Então além de nos preocupar-mos somente como o código é exibido na tela do
    computador, nós precisamos nos preocupar sobre como ele será salvo no
    sistema de arquivos. Já que ao compartilhar o código, é vital o uso de um
    sistema de versionamento para permitir a gerências de projetos e os
    programadores em si, terem o controle de mudanças do código. O que permiti e
    facilmente possa realizar o rastreamento de mudanças e permitir que se possa
    entender melhor o que cada programador está fazendo a cada vez que ele
    formaliza um mudança no código através de uma `commit', como no sistemas
    `git` por exemplo.

    Isso por que quando trabalhos em um sistema de versionamento como `git'
    precisamos manter o código dentre um único estilo ou boa prática definida
    como padrão, devido ao fato de que se deixar-mos cada programador escrever
    como ele quiser, teremos muito ruído durante a revisão do código e estamos
    determinando o que o programador fez/escreveu, se cada programador
    re-escreve o histórico fazendo alterações como colocar linhas novas antes de
    cada if. Assim teremos ruído por que o foco de um sistema de versionamento é
    olhar somente as mudanças que são significativas para o código, como a
    criação de novas funções e não a adição de novas linhas em branco.

    Sobre o último ponto, podemos pensar também sobre uma abordagem da criação
    de um novo sistema de versão que foque somente nas mudanças significativas
    para o código, durante o momento da revisão. Entretanto essa abordagem não é
    ideal por que, por exemplo, ela dá margem para que programadores entrem em
    guerras tediantes e não produtivas de ajustes de código. Por exemplo,
    imagine o quão seria todo dia que você acorda e começa a trabalhar, você tem
    que passar pelo código colocando linhas novas antes de cada um dos if\s por
    que o programador do turno da noite tinha acabado de remover eles?


\section{Objetivos}

    Estabelecer relações entre boas práticas de programação e eficiência em
    programar, além de uma nova ferramenta ao apoio do programador com o intuito
    de automatizar o longo e diverso processo de programação em equipes de
    desenvolvedores com distintas boas práticas de programação.


\subsection{Objetivos específicos}

    \begin{enumerate}

        \item

        Um estudo sobre ferramentas universais de programação, que permitam que
        a partir de um único software, seja programado em todas as linguagens de
        programação. Assim explicar as diferenças para os outros softwares e os
        porquês de querer-se uma ferramenta única, ao invés de diversas.

        \item

        Definir, estudar, determinar e classificar o que são boas práticas de
        programação e realizar um estudo aprofundado sobre a as boas práticas da
        área de disposição visual, conhecidas também como `Beautifying'.

        \item

        Um estudo sobre as mais diversas ferramentas existentes para o apoio de
        boas práticas de programação, além de uma análise comparativa entre
        elas, determinando suas fraquezas e pontos fortes.

        \item

        A definição de um padrão de floxo de desenvolvimento que permita equipes
        de programadores com distintas boas práticas de programação, trabalhem
        em si sem intervir e iniciar guerras de boas práticas.

        \item

        Propor uma ferramenta única que permita diversas e distintas boas
        práticas de programação serem implementadas nas mais diversas linguagens
        de programação e que elas possam ser configuradas e definidas ao gosto
        dos programadores que a usa.

    \end{enumerate}


\section{Método de pesquisa}

    O trabalho será baseado em pesquisas em artigos, livros, teses,
    dissertações, sites de autores confiáveis, e por meio de novas provas
    demonstradas e baseadas através de argumentos no decorrer da evolução da
    monografia. Também sera apresentado os resultados decorridos da construção
    de uma nova ferramenta que proprõe a solução de um dos problemas
    apresentados e explicados.

    No último capítulo desta proposta encontra-se no tópico
    \autoref{sec:implementation} encontra-se uma série de links e referências
    que forma pré-selecionadas e poderão ser utilizadas na construção final
    deste trabalho. Notes que em si, as partes da última seção serão
    gradativamente movida para primeira parte do texto onde encontra-se pesquisa
    teórica, no decorrer que suas informações correlacionadas são incorporadas
    no trabalho escrito.

    Assim no final da primeira parte desta obra que dará-se no final da
    conclusão da disciplina intitulada de Trabalho de Conclusão de Curso 1,
    restarão somente as informações destinadas a implementação da ferramenta
    proposta, que serão implementadas na segunda parte da monografia denominada
    \nameref{sec:implementation}, que será desenvolvida no final da conclusão da
    disciplina de Trabalho de Conclusão de Curso 2.


}


