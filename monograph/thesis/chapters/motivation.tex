

% The \phantomsection command is needed to create a link to a place in the document that is not a
% figure, equation, table, section, subsection, chapter, etc.
%
% When do I need to invoke \phantomsection?
% https://tex.stackexchange.com/questions/44088/when-do-i-need-to-invoke-phantomsection
\phantomsection


% Is it possible to keep my translation together with original text?
% https://tex.stackexchange.com/questions/5076/is-it-possible-to-keep-my-translation-together-with-original-text
\chapter{\lang{Motivation}{Motivação}}


\begin{englishtext}

    \section{Application fields for code beautifiers}

    \citeonline{annotationAssistant},
    \citeonline{codePlagiarismDetection},
    \citeonline{softwarePortfolio},
    \citeonline{legacyAssets},
    \citeonline{massMaintenance},
    \citeonline{prettyPrinting},
    \citeonline{architectureFormatting},
    \citeonline{independentFramework},
    \citeonline{programIndentation},
    \citeonline{industrialApplication},
    \citeonline{toolsForProjectManagement},
    \citeonline{codeClassification},
    \citeonline{codeScanningPatterns},
    \citeonline{debuggingIntoExamples},
    \citeonline{programUnderstanding},
    \citeonline{documentingAndSharingKnowledge},
    \citeonline{autofoldingForSourceCode},
    \citeonline{learningSupportSystem},
    \citeonline{syntaxHighlightingInfluencing},
    \citeonline{improvingCodeReadability},
    \citeonline{howNovicesRead},
    \citeonline{theRoleOfMethodChains},
    \citeonline{codeComprehensionComparedToOO},
    \citeonline{enhancingLegacySoftwareSystemAnalysis},
    \citeonline{moldableCodeEditor},
    \citeonline{blindAndSightedProgrammers},



    \section{What does coding is?}

    Coding is like writing and reading a book for the large people, you like it
    to look beautifully. Or at least do you expect such when you buy a book, for
    example, to learn programming for you first time. You expect: % Reference to
    % book writing style/formatting articles

    \begin{enumerate}
        \item Things to be well organized, so you do not get lost.

        \item The colors to be properly placed, so you do not get distracted
           from the main content.

        \item The spacing between paragraphs, words, chapters, sections
           subsections, etc, to be well adjusted. Not everything cluttered in
           only one file, line, function, class, or whatsoever so.
    \end{enumerate}



    \section{Spaces and Tabs}

    The problem is that I will certainly not notice when I paste something
    indented with spaces instead of tabs. This is problem because for some file
    types as `.sublime-settings' files (or a Makefile), which has the setting
    `translate\_tabs\_to\_spaces' set to false, so I would expect to all
    `.sublime-settings' files to be indented with tabs, not spaces.
    \cite{tabsAndSpacesConversion}

    The setting `translate\_tabs\_to\_spaces' set to false works fine until I paste
    something on a setting's files which is indented with spaces, instead of
    tabs. This is a problem because as I am over git versioning, I can easily
    create files with mixed tabs and spaces on the history, and some day later
    Sublime Text will fix the indentation to tabs, which will cause noise on the
    git history due the tab/space conversion war. % Cite reference to the war

    I think this can have a performance problem as when I am pasting something
    big on Sublime Text. Then to perform the conversion on the would not be
    easily possible and it should be performed afterwards by the user. Now
    Sublime Text should warn the user when he is pasting something indented with
    spaces instead of tabs in a file which is expected to be indented with tabs?

    The detection of whether the contents of the clipboard should not be
    expensive as we should just check some lines (which would not cover the
    cases where there is already mixed indentation on the clipboard contents).
    But there would be a performance problem when pasting something with very
    big first lines. On this case a threshold should stop Sublime Text from
    looking forward and ceases the detection as inconclusive for this paste and
    just paste like it is.



    \subsection{Computer Assisted Programming}

    Your computer should help you with with these unforeseen tasks. Why should I
    spend my precious time checking whether I am actually copying something
    space indented, when I am actually coping something tab indented?

    Therefore, how to do such a thing on this 21\q{}st century? Perhaps we
    should sit and cry while waiting for some greater force to come and rescue
    us. Or may be you should stop crying and actually do something about other
    than keep waiting for you mommy to come and save you from the darkness
    growing behind you back leading you to endless unsleepy nights fixing your
    code just because everything just went wrong.



    \section{The Upper Stream}

    TODO.

    Do related and small commits, i.e., one change per commit.

    What is upstream?

    \url{https://help.github.com/articles/configuring-a-remote-for-a-fork/}

    \url{https://stackoverflow.com/questions/2739376/definition-of-downstream-and-upstream}

    What is origin?

    Is there a downstream?

    \citeonline{redesignOfGit}


    \subsection{How to keep up with the upstream}

    TODO.



    \section{Common Tasks}

    So you are developing a software which is under version control, however to
    deploy your tests, you need to copy some big folders into the deployment or
    testing system. Then how do you do it?

    Copying and pasting them probably the most straight forward idea, which is
    nice if you are going to it only a few times in a life time like two or
    three. However if you are going to do it move than these limits,
    please don\q t do that. It is bad for the planet and is worsening your
    health for nothing other than more headaches.

    As a promptly good computer user, at this point you already have some tool,
    either graphical or by command line which can help you easily and fastly
    setup the folder\q s. Easily like:

    \begin{enumerate}
        \item You open the tool
        \item Click on the new button
        \item Name your sync task as `My cuttie'
        \item Copy and paste there source and destine addresses
        \item Hit the `sync now' button
    \end{enumerate}



    \section{The rsync side}

    Doing everything out of the box by a graphical interface seems not
    practical. Command Line Interfaces (CLI) are simpler to be built and allows
    their programmers to saver their efforts in actually writing the tool
    instead of designing a reasonable Graphical User Interface (GUI)
    \cite{quantificationOfInterface}.

    GUI interfaces are awesome but for their proper usage, which is mostly
    defined by their aim public. Non-computer programmers, perhaps even novice
    programmers, cannot easily deal with command lines, but experienced
    programmers should be able to get great advantage from it usage.
    \cite{commandLineInterface}.

    Following we may see an example about the simpleness of a shell script,
    which runs several commands to accomplish a clean build of the testing
    environment:

    \begin{lstlisting}[caption={rebuild\_workspace.sh}]
    #!/bin/sh
    printf "$(date)\nRemoving folders...\n"

    rm -rf "Installed Packages"
    rm -rf "Lib"
    rm -rf "Local"
    rm -rf "Packages"

    printf "Unzipping files...\n"
    unzip -q "Packages.zip"

    mkdir -p "./Deployment/Code A"
    mkdir -p "./Deployment/Code B"

    printf "Syncing folders...\n"
    rsync -r \
         "/cygdrive/d/Development/Environment/Code A/" \
         "/cygdrive/c/Test/Deployment/Code A/"

    rsync -r \
         "/cygdrive/d/Development/Environment/Code B/" \
         "/cygdrive/c/Test/Deployment/Code B/"
    \end{lstlisting}
    \vspace*{-4mm}

    On preceding example, the `rebuild\_workspace.sh' script is located on the
    testing folder `/cygdrive/c/Test', then when calling it we get some folders
    removed, a file unpacked on the current folder, and our code synced from the
    versioning system directly to testing environment. You can read more about
    `rsync' utility on \citeonline{synchronizingFolders}.



    \section{How to Write Good Code}

    If you ask around, you will probably listen you need to write code to learn
    how to write good code. This seems to make sense like training for the next
    Olympics. As long as you train very hard, you get results.

    However, the same way as for gymnasium training there is the right or
    perhaps nice way to do it and the wrong way, for computer programming is the
    same.


    \section{Good Coding Practices}


    \subsection{Variable Naming}

    \cite{theImpactOfIdentifierStyle},
    \cite{womenAndMen},

    Variables naming are extremely important. That is it, a dot. By the article
    \citeonline{analysisOfCodeReading} there is the understanding about tracking
    techniques to improve code comprehension by closing the program code with
    natural language, i.e., as long as your code looks like natural language
    text, better would be your understanding of the code.

    \begin{itemize}

    \item Global identifiers must start with `g\_' prefix.

    \item Global identifiers which are constant must be written in ALLCAPS.

    \item Boolean variables must be prefix with `is\_' even when they are in
    global scope, meaning double prefixed as `g\_is\_'.

    \end{itemize}

    All other cases not cited must not be prefixed. Now we can continue the most
    infamous wars about Hungarian notation versus ?.

    Nonetheless, looking over this list we can quickly pop up about different
    programming languages whereas such rules are applied and not. The most
    notorious cases are about Java and C.

    Java programming language has no such idealism as exhaustively prefixing as
    `g\_' as explicitly there is not such concept. Although a more hawk-eyed
    reader may spot about the Java classes atributes being the global variables.
    And in fact they are so. But within their scope limited to the classes body,
    and perhaps the instance objects which may or may not provide access to
    them.

    Regardless, Java also suffers from the same evil as remaining programming
    languages with the common global variable concept. Here we enter on the
    war about the evil global variable usage. However, keeping such bias against
    global variable usage is not correct. Each one of the cases, either passing
    the variables by reference, copy, or sharing globally is a correct approach.

    TODO, reference the reader to proper reading and stop this discussion.


    \subsection{Hungarian Notation}

    TODO.


    \subsection{?. Notation}

    TODO.

\end{englishtext}


% % Portuguese
% \lang{}{

    Perguntas como ``O que são boas práticas de programação?'' ou ainda ``O por
    quê estas práticas são boas?'', não são fáceis de responder. Mas cada
    programador aprende a escrever seus códigos em uma determinada maneira, com
    determinadas características como utilizar 4 ou 8 espaços para indentação de
    linhas, sempre deixar uma linha em branco antes de cada estrutura de
    controle como if\s, for\s, e afins.

    Mas entrando o universo de boas práticas, há muitos coisas sobre discorrer.
    Assim neste trabalho especificamente trabalhá-se sobre como realizar a
    melhor disposição/exibição do código de programação na tela do computador,
    de modo que maximize e facilite o entendimento do mesmo. Portanto permitindo
    que o programador dispersa mais tempe pensando sobre o problema, do que
    tentar decifrar a informação que é apresentado para ele na tela.

    Dentro desta área de trabalho, precisa-se também pensar muito bem sobre como
    compartilhar os códigos de programação dos programadores entre si. Isso por
    que entra agora o problema da diversidade de boas práticas de programação.
    Ela acontece por que muitas vezes, aquilo que é uma boa prática para um
    `programador A', não é para o outro `programador B'. Por exemplo, imagine um
    código onde um programador decidiu colocar antes de cada `if', uma linha em
    branco. Portanto é de se esperar que sempre que vemos uma linha em branco
    nos podemos potencialmente encontrar um `if'. Entretanto imagine que outro
    programador não gostou dessa regra e quando ele foi escrever seu código que
    envolvia um `if', ele não colocou a essa tal linha em branco que o outro
    programador vinha colocando. Então quando o primeiro programador for ler o
    código e procurar por `if'es, ele vai estar esperando por linhas em branco.
    Mas vai perder algum tempo procurando até perceber que o outro programador
    não as colocou.

    Essas diferenças dão-se devido a diversidade de meios de se aprender
    programação, tanto quanto aos gostos, aptidões e objetivos de cada
    programador. Assim hoje em dia isso torna-se um grande problema por que cada
    vez mais precisamos de mais e mais programadores trabalhem juntos entre si,
    desenvolvendo os mais diversos sistemas computações. Onde este último
    deve-se ao fato de que a complexidade dos sistemas computacionais cresce
    cada vez mais, portanto requer-se que mais e mais programadores trabalhem e
    compartilhem códigos.

    Então além de nos preocupar-mos somente como o código é exibido na tela do
    computador, nós precisamos nos preocupar sobre como ele será salvo no
    sistema de arquivos. Já que ao compartilhar o código, é vital o uso de um
    sistema de versionamento para permitir a gerências de projetos e os
    programadores em si, terem o controle de mudanças do código. O que permiti e
    facilmente possa realizar o rastreamento de mudanças e permitir que se possa
    entender melhor o que cada programador está fazendo a cada vez que ele
    formaliza um mudança no código através de uma `commit', como no sistemas
    `git` por exemplo.

    Isso por que quando trabalhos em um sistema de versionamento como `git'
    precisamos manter o código dentre um único estilo ou boa prática definida
    como padrão, devido ao fato de que se deixar-mos cada programador escrever
    como ele quiser, teremos muito ruído durante a revisão do código e estamos
    determinando o que o programador fez/escreveu, se cada programador
    re-escreve o histórico fazendo alterações como colocar linhas novas antes de
    cada if. Assim teremos ruído por que o foco de um sistema de versionamento é
    olhar somente as mudanças que são significativas para o código, como a
    criação de novas funções e não a adição de novas linhas em branco.

    Sobre o último ponto, podemos pensar também sobre uma abordagem da criação
    de um novo sistema de versão que foque somente nas mudanças significativas
    para o código, durante o momento da revisão. Entretanto essa abordagem não é
    ideal por que, por exemplo, ela dá margem para que programadores entrem em
    guerras tediantes e não produtivas de ajustes de código. Por exemplo,
    imagine o quão seria todo dia que você acorda e começa a trabalhar, você tem
    que passar pelo código colocando linhas novas antes de cada um dos if\s por
    que o programador do turno da noite tinha acabado de remover eles?


\section{Objetivos}

    Estabelecer relações entre boas práticas de programação e eficiência em
    programar, além de uma nova ferramenta ao apoio do programador com o intuito
    de automatizar o longo e diverso processo de programação em equipes de
    desenvolvedores com distintas boas práticas de programação.


\subsection{Objetivos específicos}

    \begin{enumerate}

        \item

        Um estudo sobre ferramentas universais de programação, que permitam que
        a partir de um único software, seja programado em todas as linguagens de
        programação. Assim explicar as diferenças para os outros softwares e os
        porquês de querer-se uma ferramenta única, ao invés de diversas.

        \item

        Definir, estudar, determinar e classificar o que são boas práticas de
        programação e realizar um estudo aprofundado sobre a as boas práticas da
        área de disposição visual, conhecidas também como `Beautifying'.

        \item

        Um estudo sobre as mais diversas ferramentas existentes para o apoio de
        boas práticas de programação, além de uma análise comparativa entre
        elas, determinando suas fraquezas e pontos fortes.

        \item

        A definição de um padrão de floxo de desenvolvimento que permita equipes
        de programadores com distintas boas práticas de programação, trabalhem
        em si sem intervir e iniciar guerras de boas práticas.

        \item

        Propor uma ferramenta única que permita diversas e distintas boas
        práticas de programação serem implementadas nas mais diversas linguagens
        de programação e que elas possam ser configuradas e definidas ao gosto
        dos programadores que a usa.

    \end{enumerate}


\section{Método de pesquisa}

    O trabalho será baseado em pesquisas em artigos, livros, teses,
    dissertações, sites de autores confiáveis, e por meio de novas provas
    demonstradas e baseadas através de argumentos no decorrer da evolução da
    monografia. Também sera apresentado os resultados decorridos da construção
    de uma nova ferramenta que proprõe a solução de um dos problemas
    apresentados e explicados.

    No último capítulo desta proposta encontra-se no tópico
    \autoref{sec:implementation} encontra-se uma série de links e referências
    que forma pré-selecionadas e poderão ser utilizadas na construção final
    deste trabalho. Notes que em si, as partes da última seção serão
    gradativamente movida para primeira parte do texto onde encontra-se pesquisa
    teórica, no decorrer que suas informações correlacionadas são incorporadas
    no trabalho escrito.

    Assim no final da primeira parte desta obra que dará-se no final da
    conclusão da disciplina intitulada de Trabalho de Conclusão de Curso 1,
    restarão somente as informações destinadas a implementação da ferramenta
    proposta, que serão implementadas na segunda parte da monografia denominada
    \nameref{sec:implementation}, que será desenvolvida no final da conclusão da
    disciplina de Trabalho de Conclusão de Curso 2.


}


