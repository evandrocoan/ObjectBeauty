

% The \phantomsection command is needed to create a link to a place in the document that is not a
% figure, equation, table, section, subsection, chapter, etc.
%
% When do I need to invoke \phantomsection?
% https://tex.stackexchange.com/questions/44088/when-do-i-need-to-invoke-phantomsection
\phantomsection


% Is it possible to keep my translation together with original text?
% https://tex.stackexchange.com/questions/5076/is-it-possible-to-keep-my-translation-together-with-original-text
\chapter{\lang{Motivation}{Motivação}}


    \section{What does coding is?}

    Coding is like writing and reading a book for the large people, you like it
    to look beautifully. Or at least do you expect such when you buy a book, for
    example, to learn programming for you first time \cite{howNovicesRead}.
    You expect:

    \begin{enumerate}
        \item Things to be well organized,
              so you do not get lost.
        \item The colors to be properly placed,
              so you do not get distracted from the main content.
        \item The spacing between paragraphs, words, chapters, sections
              subsections, etc, to be well adjusted.
              Not everything cluttered in only one file,
              line, function, class, or whatsoever so.
    \end{enumerate}



    \section{Computer Assisted Programming}

    Your computer should help you with with these unforeseen tasks.
    Why should I spend my precious time checking whether I am actually copying
    something space indented,
    when I am actually coping something tab indented?

    Therefore, how to do such a thing on this 21\q{}st century?
    Perhaps we should sit and cry while waiting for some greater force to come
    and rescue us. Or may be you should stop crying and actually do something
    about other than keep waiting for you mommy to come and save you from the
    darkness growing behind you back leading you to endless unsleepy nights
    fixing your code just because everything just went wrong.



    \section{Code Beautifying}

    A robust Code Beautifier can get a lot more complicated just with the basic
    definitions of formatting (\nameref{source_code_beautifiers})
    applied over each language own characteristics
    as for example:

    \begin{enumerate}[nosep,nolistsep]
        \item Add spaces before if\s name as in `if(var)' versus `if (var)'
        \item Add spaces inside if\s as in `if(var)' versus `if( var )'
        \item Add spaces before for\s name as in `for(var)' versus `for (var)'
        \item Add spaces inside for\s as in `for(var)' versus `for( var )'
        \item ...
    \end{enumerate}

    As may be noticed, the list may became quite big,
    and if fact such big list of rules has been implemented.
    Looking over the Beautifier called
    `Uncrustify'\footnote{\url{https://github.com/uncrustify/uncrustify}},
    we can find about 500 settings with specifications like these above.

    The problem about is, even if you go through all these settings,
    which will take you quite some time,
    you still only configuring a few languages closely related.
    On this case, C, C++, Java, Pawn, etc.
    For all other languages you still need to find out another source code
    formatter tool, which will be certainly
    limited\footnote{\url{https://stackoverflow.com/questions/31438377/how-can-i-get-eclipse-to-wrap-lines-after-a-period-instead-of-before}}
    and still need to configure all over again.


