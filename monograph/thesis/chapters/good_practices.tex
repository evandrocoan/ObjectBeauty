

% The \phantomsection command is needed to create a link to a place in the document that is not a
% figure, equation, table, section, subsection, chapter, etc.
%
% When do I need to invoke \phantomsection?
% https://tex.stackexchange.com/questions/44088/when-do-i-need-to-invoke-phantomsection
\phantomsection


% Is it possible to keep my translation together with original text?
% https://tex.stackexchange.com/questions/5076/is-it-possible-to-keep-my-translation-together-with-original-text
\chapter{\lang{Good Practices}{Boas Práticas}}


    \section{The Upper Stream}

    TODO.

    Do related and small commits, i.e., one change per commit.

    What is upstream?

    \url{https://help.github.com/articles/configuring-a-remote-for-a-fork/}

    \url{https://stackoverflow.com/questions/2739376/definition-of-downstream-and-upstream}

    What is origin?

    Is there a downstream?

    \citeonline{redesignOfGit}


    \subsection{How to keep up with the upstream}

    TODO.



    \section{Common Tasks}

    So you are developing a software which is under version control, however to
    deploy your tests, you need to copy some big folders into the deployment or
    testing system. Then how do you do it?

    Copying and pasting them probably the most straight forward
    idea\footnote{\url{https://blog.codinghorror.com/the-two-types-of-programmers/}},
    which is
    nice if you are going to it only a few times in a life time like two or
    three. However if you are going to do it move than these limits,
    please don\q t do that. It is bad for the planet and is worsening your
    health for nothing other than more headaches.

    As a promptly good computer user, at this point you already have some tool,
    either graphical or by command line which can help you easily and fastly
    setup the folder\q s. Easily like:

    \begin{enumerate}
        \item You open the tool
        \item Click on the new button
        \item Name your sync task as `My cuttie'
        \item Copy and paste there source and destine addresses
        \item Hit the `sync now' button
    \end{enumerate}



    \section{The rsync side}

    Doing everything out of the box by a graphical interface seems not
    practical. Command Line Interfaces (CLI) are simpler to be built and allows
    their programmers to saver their efforts in actually writing the tool
    instead of designing a reasonable Graphical User Interface (GUI)
    \cite{quantificationOfInterface}.

    GUI interfaces are awesome but for their proper usage, which is mostly
    defined by their aim public. Non-computer programmers, perhaps even novice
    programmers, cannot easily deal with command lines, but experienced
    programmers should be able to get great advantage from it usage.
    \cite{commandLineInterface}.

    Following we may see an example about the simpleness of a shell script,
    which runs several commands to accomplish a clean build of the testing
    environment:

    \begin{lstlisting}[caption={rebuild\_workspace.sh}]
    #!/bin/sh
    printf "$(date)\nRemoving folders...\n"

    rm -rf "Installed Packages"
    rm -rf "Lib"
    rm -rf "Local"
    rm -rf "Packages"

    printf "Unzipping files...\n"
    unzip -q "Packages.zip"

    mkdir -p "./Deployment/Code A"
    mkdir -p "./Deployment/Code B"

    printf "Syncing folders...\n"
    rsync -r \
         "/cygdrive/d/Development/Environment/Code A/" \
         "/cygdrive/c/Test/Deployment/Code A/"

    rsync -r \
         "/cygdrive/d/Development/Environment/Code B/" \
         "/cygdrive/c/Test/Deployment/Code B/"
    \end{lstlisting}
    \vspace*{-4mm}

    On preceding example, the `rebuild\_workspace.sh' script is located on the
    testing folder `/cygdrive/c/Test', then when calling it we get some folders
    removed, a file unpacked on the current folder, and our code synced from the
    versioning system directly to testing environment. You can read more about
    `rsync' utility on \citeonline{synchronizingFolders}.



    \section{Good Coding Practices}


    \subsection{Variable Naming}

    \cite{theImpactOfIdentifierStyle},
    \cite{womenAndMen},

    Variables naming are extremely important. That is it, a dot. By the article
    \citeonline{analysisOfCodeReading} there is the understanding about tracking
    techniques to improve code comprehension by closing the program code with
    natural language, i.e., as long as your code looks like natural language
    text, better would be your understanding of the code.

    \begin{itemize}
        \item Global identifiers must start with `g\_' prefix.
        \item Global identifiers which are constant must be written in ALLCAPS.
        \item Boolean variables must be prefix with `is\_' even when they are in
        global scope, meaning double prefixed as `g\_is\_'.
    \end{itemize}

    All other cases not cited must not be prefixed. Now we can continue the most
    infamous wars about Hungarian notation versus ?.

    Nonetheless, looking over this list we can quickly pop up about different
    programming languages whereas such rules are applied and not. The most
    notorious cases are about Java and C.

    Java programming language has no such idealism as exhaustively prefixing as
    `g\_' as explicitly there is not such concept. Although a more hawk-eyed
    reader may spot about the Java classes atributes being the global variables.
    And in fact they are so. But within their scope limited to the classes body,
    and perhaps the instance objects which may or may not provide access to
    them.

    Regardless, Java also suffers from the same evil as remaining programming
    languages with the common global variable concept. Here we enter on the
    war about the evil global variable usage. However, keeping such bias against
    global variable usage is not correct. Each one of the cases, either passing
    the variables by reference, copy, or sharing globally is a correct approach.
    \footnote{\url{https://blog.codinghorror.com/code-smells/}}


    \subsection{Hungarian Notation}

    TODO.


    \subsection{Other Notations}

    TODO.


    \subsection{Spaces and Tabs}

    Either use spaces, or tabs
    \footnote{\url{https://blog.codinghorror.com/whitespace-the-silent-killer/}}.
    However while using tabs,
    you may end up with mixed file with tabs and spaces \cite{tabsAndSpacesConversion}.



    \section{Continuous Integration}

    On a distributed version control system,
    continuous integration would be continuously running the system tests
    as long new code is integrated into main system
    from the distributed clients \cite{continuousIntegration}.


    \subsection{Test Driven Development}


    \footnote{\url{https://blog.codinghorror.com/check-in-early-check-in-often/}}

    \cite{aspectOriented}


