
%
% Simple Sectioned Essay Template - LaTeX Template
%
% This template has been downloaded from:
% http://www.latextemplates.com
%
% `proposal.tex`
% Based on
%
% 1. https://github.com/royertiago/tcc
% 2. https://github.com/evandrocoan/abntex2-ufsc
% 3. http://www.latextemplates.com/template/simple-sectioned-essay


%----------------------------------------------------------------------------------------
%   PACKAGES AND OTHER DOCUMENT CONFIGURATIONS
%----------------------------------------------------------------------------------------

% Applying options to already loaded package
% https://tex.stackexchange.com/questions/124049/applying-options-to-already-loaded-package
\PassOptionsToPackage{backref,colorlinks,linkcolor=link_color,citecolor=dkgreen}{hyperref}
\PassOptionsToPackage{strict}{changepage}
\PassOptionsToPackage{shortlabels}{enumitem}

% Rhe ABNT class hard codes the language to English. It needs to be edited to change it.
\documentclass[
    % -- opções da classe memoir --
    10pt,               % tamanho da fonte
    openright,          % capítulos começam em pág ímpar (insere página vazia caso preciso)
    twoside,            % para impressão em recto e verso. Oposto a oneside
    a5paper,            % tamanho do papel.
    % -- opções da classe abntex2 --
    % chapter=TITLE,     % títulos de capítulos convertidos em letras maiúsculas
    % section=TITLE,     % títulos de seções convertidos em letras maiúsculas
    % -- opções do pacote babel, idioma adicional para hifenização --
    brazil,
    english
    ]{abntevandro/tex/latex/abntevandro/abntevandro}

% ABNT Citations rules
\usepackage[alf]{abntevandro/tex/latex/abntevandro/abntevandrocite}
\usepackage{ufscthesis/abntex2-ufsc}


%----------------------------------------------------------------------------------------
%   File settings
%----------------------------------------------------------------------------------------

% Load General Packages




% Incompatible color definition when using tikz with color package
% https://tex.stackexchange.com/questions/150369/incompatible-color-definition-when-using-tikz-with-color-package
\usepackage{xcolor}

\definecolor{dkgreen}{rgb}{0,0.6,0}
\definecolor{gray}{rgb}{0.5,0.5,0.5}
\definecolor{mauve}{rgb}{0.58,0,0.82}

\definecolor{link_color}{RGB}{26,13,178}


% For web links and paths with \path{..} and \url{https://www.python.org/downloads/}
%
% https://tex.stackexchange.com/questions/3033/forcing-linebreaks-in-url
\PassOptionsToPackage{hyphens}{url}
\usepackage[backref,colorlinks,linkcolor=link_color]{hyperref}

% How to fix URL overfull & underfull on emumeration?
% % https://tex.stackexchange.com/questions/366803/how-to-fix-url-overfull-underfull-on-emumeration
%
% Forcing linebreaks in \url
% https://tex.stackexchange.com/questions/3033/forcing-linebreaks-in-url/10401
\usepackage{url}
\makeatletter
\g@addto@macro{\UrlBreaks}{\UrlOrds}
\makeatother

% \lettrine{O}{nce} upon a time...
% \lettrine[findent=2pt]{\fbox{\textbf{T}}}{ }his thesis deals with...
%
% https://tex.stackexchange.com/questions/164298/starting-a-paragraph-with-a-big-letter
\usepackage{lettrine}

% Required for including pictures, resizebox
\usepackage{graphicx}

% Allows putting an [H] in \begin{figure} to specify the exact location of the figure
\usepackage{float}

% Allows in-line images such as the example fish picture
\usepackage{wrapfig}

% How to automatically force latex to not justify the text when it is not wise?
% https://tex.stackexchange.com/questions/365801/how-to-automatically-force-latex-to-not-justify-the-text-when-it-is-not-wise
\usepackage{array,ragged2e}

% Use its macro adjustwidth* to extend tables out of outer text border.
% https://tex.stackexchange.com/questions/366155/how-to-write-a-table-a-little-larger-than-the-paragraphs-with-centered-columns
\usepackage[strict]{changepage}

% No spacing between enumerated items with \usepackage{enumerate}
% https://tex.stackexchange.com/questions/119919/no-spacing-between-enumerated-items-with-usepackageenumerate
\usepackage[shortlabels]{enumitem}

\usepackage{tabularx}
\usepackage{multirow}








%
% New Macros
%

% Automatically put a `\medskip` spacing between paragraphs
% https://tex.stackexchange.com/q/365976/119062
\edef\restoreparindent{\parindent=\the\parindent\relax}
\usepackage{parskip}
\restoreparindent

% Uncomment to remove all indentation from paragraphs
%\setlength\parindent{0pt}

% How could the `\everypar` justification statement be used?
% https://tex.stackexchange.com/questions/365818/how-could-the-everypar-justification-statement-be-used
\newbox\linebox \newbox\snapbox
\def\eatlines{
  \setbox\linebox\lastbox % check the last line
  \ifvoid\linebox
  \else % if it’s not empty
    \unskip\unpenalty % take whatever is
    {\eatlines} % above it;
    \setbox\snapbox\hbox{\unhcopy\linebox}
    \ifdim\wd\snapbox<.98\wd\linebox
       \box\snapbox % take the one or the other,
    \else \box\linebox \fi
  \fi
}

% How could the `\everypar` justification statement be used?
% https://tex.stackexchange.com/questions/365818/how-could-the-everypar-justification-statement-be-used
\everypar={\setbox0=\lastbox \par
   \vbox\bgroup \everypar={}\def\par{\endgraf\eatlines\egroup}}

% Creates a new environment which can be used as:
%
% \begin{foo}
%   Text...
%
%   Text ...
% \end{foo}
%
% https://tex.stackexchange.com/questions/62333/push-long-words-in-a-new-line
\newenvironment{foo}
{\par
\hyphenpenalty=10000
\exhyphenpenalty=10000
}
{\par}



%
% New commands
%

% Allow to push long words on new lines when they do not fit entirely on the current line.
% https://tex.stackexchange.com/questions/62333/push-long-words-in-a-new-line
\newcommand\lword[1]{\leavevmode\nobreak\hskip0pt plus\linewidth\penalty50\hskip0pt plus-\linewidth\nobreak{#1}}


% For the new command \latex
\usepackage{xspace}

% Write the word LaTeX nicely.
\newcommand{\latex}{\LaTeX\xspace}


% Create a bold title all in upper case.
\newcommand{\Title}[1]{\textbf{\MakeUppercase{#1}}}









% Writing code in latex document. Usage: \begin & \end {lstlisting}
% http://stackoverflow.com/questions/3175105/writing-code-in-latex-document
\usepackage{listings}

% How to insert code with accents with listings?
% https://tex.stackexchange.com/questions/30512/how-to-insert-code-with-accents-with-listings
\usepackage{listingsutf8}

% Incompatible color definition when using tikz with color package
% https://tex.stackexchange.com/questions/150369/incompatible-color-definition-when-using-tikz-with-color-package
\usepackage{xcolor}

\definecolor{dkgreen}{rgb}{0,0.6,0}
\definecolor{gray}{rgb}{0.5,0.5,0.5}
\definecolor{mauve}{rgb}{0.58,0,0.82}

\lstset{frame=,
  language=Java,
  aboveskip=3mm,
  belowskip=3mm,
  showstringspaces=false,
  columns=flexible,
  basicstyle={\small\ttfamily},
  numbers=left,
  numberstyle=\color{gray},
  keywordstyle=\color{blue},
  commentstyle=\color{dkgreen},
  stringstyle=\color{mauve},
  breaklines=true,
  breakatwhitespace=true,
  tabsize=3
}

% Defining `lstset` parameters for multiple languages & How can I highlight YAML code in a pretty way with listings?
%
% Usage \begin{lstlisting}[style=yaml_style] ... \end{lstlisting}
%
% https://tex.stackexchange.com/questions/45711/defining-lstset-parameters-for-multiple-languages
% https://tex.stackexchange.com/questions/152829/how-can-i-highlight-yaml-code-in-a-pretty-way-with-listings
\newcommand\YAMLcolonstyle{\color{red}}
\newcommand\YAMLkeystyle{\color{black}}
\newcommand\YAMLvaluestyle{\color{blue}}
\newcommand\ProcessThreeDashes{\llap{\color{cyan}\mdseries-{-}-}}

\lstdefinestyle{yaml_style}{
  frame=,
  aboveskip=3mm,
  belowskip=3mm,
  showstringspaces=false,
  columns=flexible,
  numbers=left,
  numberstyle=\color{gray},
  breaklines=true,
  breakatwhitespace=true,
  tabsize=2,
  keywords={true,false,null,y,n},
  keywordstyle=\color{darkgray},
  basicstyle=\YAMLkeystyle,                                 % assuming a key comes first
  sensitive=false,
  comment=[l]{\#},
  morecomment=[s]{/*}{*/},
  commentstyle=\color{purple}\ttfamily,
  stringstyle=\YAMLvaluestyle\ttfamily,
  moredelim=[l][\color{orange}]{\&},
  moredelim=[l][\color{magenta}]{*},
  moredelim=**[il][\YAMLcolonstyle{:}\YAMLvaluestyle]{:},   % switch to value style at :
  morestring=[b]',
  morestring=[b]",
  literate = {---}{{\ProcessThreeDashes}}3
             {>}{{\textcolor{red}\textgreater}}1
             {|}{{\textcolor{red}\textbar}}1
             {\ -\ }{{\mdseries\ -\ }}3,
  inputencoding=utf8, % Listings in Latex with UTF-8 (or at least german umlauts)
  extendedchars=true, % http://stackoverflow.com/questions/1116266/listings-in-latex-with-utf-8-or-at-least-german-umlauts
  literate=%
  {é}{{\'{e}}}1
  {è}{{\`{e}}}1
  {ê}{{\^{e}}}1
  {ë}{{\¨{e}}}1
  {É}{{\'{E}}}1
  {Ê}{{\^{E}}}1
  {û}{{\^{u}}}1
  {ù}{{\`{u}}}1
  {ú}{{\'{u}}}1
  {â}{{\^{a}}}1
  {à}{{\`{a}}}1
  {á}{{\'{a}}}1
  {ã}{{\~{a}}}1
  {Á}{{\'{A}}}1
  {Â}{{\^{A}}}1
  {Ã}{{\~{A}}}1
  {ç}{{\c{c}}}1
  {Ç}{{\c{C}}}1
  {õ}{{\~{o}}}1
  {ó}{{\'{o}}}1
  {ô}{{\^{o}}}1
  {Õ}{{\~{O}}}1
  {Ó}{{\'{O}}}1
  {Ô}{{\^{O}}}1
  {î}{{\^{i}}}1
  {Î}{{\^{I}}}1
  {í}{{\'{i}}}1
  {Í}{{\~{Í}}}1
}



% \input{utilities/ufsc}

% To use the font Times New Roman, instead of the default LaTeX font
% \usepackage{mathptmx}
%
% more up-to-date than 'mathptmx'
\usepackage{newtxtext}
\usepackage{newtxmath}

% Always use it as should improve full justification
% https://tex.stackexchange.com/questions/10377/texttt-overfull-hbox-problem
% https://tex.stackexchange.com/questions/66052/should-i-load-microtype-with-pdflatex
\usepackage{microtype}

% Indent the first section paragraphs
% https://tex.stackexchange.com/q/39227/119062
\usepackage{indentfirst}

% Specifies the directory where pictures are stored
\graphicspath{{pictures/}}



%----------------------------------------------------------------------------------------
%   Bad Boxes settings
%----------------------------------------------------------------------------------------

% Bad formatting using URLs in bibtex
% https://tex.stackexchange.com/questions/22888/bad-formatting-using-urls-in-bibtex
\usepackage{etoolbox}

% Underfull \hbox in bibliography
% https://tex.stackexchange.com/questions/10924/underfull-hbox-in-bibliography
\apptocmd{\thebibliography}{\raggedright}{}{}

% How to avoid overfull error with url package?
% See also the `\usepackage{url}` declarationon the file `basic.tex`.
% Set this to 2mu or 3mu if URL start troubling again.
% https://tex.stackexchange.com/questions/261776/how-to-avoid-overfull-error-with-url-package
\Urlmuskip=0mu plus 1mu

% Automatically put a `\medskip` spacing between paragraphs
% https://tex.stackexchange.com/questions/365976/how-to-stop-the-package-usepackageparskip-disabling-the-paragraph-indentation
% \edef\restoreparindent{\parindent=\the\parindent\relax}
% \usepackage{parskip}
% \restoreparindent
%
% Uncomment to remove all indentation from paragraphs
%\setlength\parindent{0pt}
%
% How to restore the parskip skips between list items?
% https://tex.stackexchange.com/questions/366848/how-to-restore-the-parskip-skips-between-list-items
\parskip=0.5\baselineskip \advance\parskip by 0pt plus 2pt



%----------------------------------------------------------------------------------------
%   DOCUMENT CONTENTS
%----------------------------------------------------------------------------------------

\begin{document}

% pdfTeX warning (ext4): destination with the same identifier (nam e{page.1}) has been already used, duplicate ignored
% https://tex.stackexchange.com/questions/18924/pdftex-warning-ext4-destination-with-the-same-identifier-nam-epage-1-has
% \hypersetup{pageanchor=false}

% Author name
%
% What do \makeatletter and \makeatother do?
% https://tex.stackexchange.com/questions/8351/what-do-makeatletter-and-makeatother-do
%
% How can I use @author, @date, and @title after maketitle?
% https://tex.stackexchange.com/questions/27710/how-can-i-use-author-date-and-title-after-maketitle
\makeatletter
\author{Evandro Coan}\let\Author\@author
% \author{Someone}          \let\Author\@author
% \date{Somewhen}           \let\Date\@date
\def\Advisor{Thiago Ângelo Gelaim}
\def\Supervisor{Ricardo Azambuja Silveira}
\makeatother




% Author name
%
% What do \makeatletter and \makeatother do?
% https://tex.stackexchange.com/questions/8351/what-do-makeatletter-and-makeatother-do
%
% How can I use @author, @date, and @title after maketitle?
% https://tex.stackexchange.com/questions/27710/how-can-i-use-author-date-and-title-after-maketitle
\makeatletter
\author{Evandro Coan}\let\Author\@author
% \author{Someone}          \let\Author\@author
% \date{Somewhen}           \let\Date\@date
\def\Advisor{Thiago Ângelo Gelaim}
\def\Supervisor{Ricardo Azambuja Silveira}
\makeatother



\begin{titlingpage}

    \center

    \Title{Universidade Federal de Santa Catarina - UFSC}

    \Title{Ciência Da Computação}

    \vspace*{\stretch{1}}

    \Author

    \vspace*{\stretch{2}}

    \Title{Boas Práticas de Programação \& Estilo}

    \bigskip
    Ferramentas Universais de Programação\\[3cm]

    \begin{flushright}

        \begin{minipage}{0.518\textwidth}

            Proposta de Trabalho de Conclusão de Curso,
            a ser submetido ao Curso de Ciência da Computação
            para a obtenção do Grau de Bacharel em Ciência da \lword{Computação}.

            \medskip
            {\bfseries Orientador:} \hfill \Advisor

            \medskip
            {\bfseries Professor Responsável:} \hfill \Supervisor

        \end{minipage}

    \end{flushright}

    \vspace*{\stretch{3}}

    Florianópolis, \today.

\end{titlingpage}










\begin{abstract}

    Faz-se um estudo sobre o que é, para que servem os Beautifiers, assim como abordagens sobre o
    que são boas práticas de programação e por que devemos segui-las para um boa eficiência ao
    escrever códigos nas mais diversas linguagens de programação. Os softwares formatadores de
    código fonte atuais, também conhecidos como Beautifiers, são limitados a um conjunto similar, ou
    mesmo à uma única linguagem, e além de muitos, serem limitados ao que eles podem fazer por você
    ao processar/formatar o código \cite{Terence}.

    \medskip

    Portanto espera-se o final do trabalho, conhecer-se quais são as ferramentas que existem e quais
    delas são as melhores que podem ser utilizadas para o auxilio do programador durante a escrita
    de códigos das mais diversas linguagens de programação. Além de proposta de uma nova ferramenta
    com o intuído de centralizar em uma único programa o abordagem das mais diversas linguagens de
    programação.

    \medskip

    \textbf{Palavras-chave:}
    source, code, formatter, beautifier, prettyprint, universal, reuse, blocks, object, oriented,
    programming, structured, parsing, parse, regular, expression, regex, C, C++,  grammar, Turing,
    machine, automata, lexer, syntax, sublime, Java, Rust, shell, script, obfuscators, learning,
    syntec, teamicide, concensus, indent, settings.

\end{abstract}






\include{summary}

% To automatically put a [Go To Top] on each section
\addGoToSummary



% The \phantomsection command is needed to create a link to a place in the document that is not a
% figure, equation, table, section, subsection, chapter, etc.
%
% When do I need to invoke \phantomsection?
% https://tex.stackexchange.com/questions/44088/when-do-i-need-to-invoke-phantomsection
\phantomsection


% Is it possible to keep my translation together with original text?
% https://tex.stackexchange.com/questions/5076/is-it-possible-to-keep-my-translation-together-with-original-text
\chapter{\lang{Introduction}{Introdução}}

The work first part will be based on research in
articles, books, theses, dissertations, trusted authors websites,
and through new demonstrated evidences based on arguments
in the monograph evolution.
Also, present results after building a new tool
which proposes a solution for the problems presented and detailed.

In this proposal last chapter which lies on the part
called `\nameref{sec:software_implementation}', which holds the implementation of a
tool for code `Beautifying'.



\section{Context}

Questions like ``What are good programming practices?'' Or ``Why are these
practices are good?''Are not easy to answer. But each programmer learns to
write their codes in a certain way, with certain features like using 4 or 8
spaces to indent lines, always leave a blank line before each control
structure as if or for statements, and alike rules.
\cite{naturalCodingConventions}

But entering the universe of good practices, there are many things for
discoursing. Nonetheless, in this work is presented the implementation of
tool called `Object Beautifier', which specifically dedicates on how to
perform the best layout/display of programming code on the computer screen,
so that maximize and facilitate the understanding of same
\cite{automaticSynthesis}.
Therefore, allowing the programmer to disperse
more time and efforts thinking about its coding algorithms problem,
other than trying to decipher the information that is presented
to it on the screen through infinit different code layouts
\cite{usingVersionControlData}.

Within this work\s area, we need to also think long and hard about how to
share the programming code of the programmers among you. Now, the problem of
human diversity, like all big scientific questions -- how do you explain
something like that -- It can be broken down into sub-questions. It happens
many times, which is a good practice for a `Programmer A', is not the same
to another `Programmer B'. For example, imagine some code where a programmer
decided to put before each `if' statement, a blank line. It is therefore
expected that whenever we see a blank line we can potentially find a
matching `if', which can be considered a quite useful pattern matching as
empty line may call better your attention. \cite{aPrettyGoodFormatting}

But again this is something heavily dependent of what each one learning
through their life time. Imagine another programmer do not liked this rule,
and when he was writing your code involving an `if', he did not put such
blank line another programmer is expecting. So when the first programmer
start reading its the code and look for `if', he will be expecting for blank
lines before its if\s. But will lose some time searching until realize
another programmer does not put them, or perhaps he forgot to insert them.
\cite{quantifyingProgramComprehension}

These differences are due to the diversity of ways we learn programming,
i.e., to the ways we are used to doing coding, as much as the abilities and
objectives of every programmer developed. Hence, nowadays it becomes a big
problem because we increasingly need more and more programmers working
together developing several and diverse computing systems. Where the latter
is due to the fact of the complexity of computer systems growing
increasingly, therefore over requiring programmers working and sharing their
codes and ideas. \cite{howProgrammersRead}

Moreover besides only worrying about how the code is displayed on their
computer screen, we need to worry about on how it will be saved in the file
system on its `plain-text' mode. Since for code sharing, it is vital for you
to use a versioning control system\footnote{\url{http://www.codeservedcold.com/version-control-importance/}}
which enable project manager\s and
programmers themselves, take control of their code changes
\cite{redesignOfGit}. It does allow to easily perform the tracking of code
changes \cite{gettingProductive} and
allow you to better understand what each programmer is doing
every time he formalizes a change in the code through a `commit', as in
`git' systems for example. \cite{usingSourceControl}

\begin{citacao}
I'd say there are two main reasons to enforce a single code format in a project. First has
to do with version control: with everybody formatting the code identically, all changes in
the files are guaranteed to be meaningful. No more just adding or removing a space here or
there, let alone reformatting an entire file as a `side effect' of actually changing just a
line or two. \cite{Geukens}
\end{citacao}

That is because while working with a versioning system like `git', we need
to keep the code among a single style or which we may call a `good practice'
set as standard for everybody, due the fact of letting each programmer to
write as he pleases, there will be plenty of noise on the code review and we
are figuring out what actually each programmer did \cite{quitDiffCalculating}.
Hence, if every programmer re-writes the history making changes
like inserting new lines
before each if, we end up with too much noise and focus of a versioning
system is to look at only those changes that are significant to the code,
such as the creation of new functions and not the addition of new blank
lines. \cite{findingRegressionsInProjects}

Talking about the last ideia pointed out, we could also think about an
approach to creating a new version control system which focuses only on
significant changes to the code, while reviewing code changes. However, this
approach could not be ideal, as for example, it would allow programmers to
start tedious wars of unproductive code adjustments. For example, imagine
how it would be for your every day and have to go through your code
re-adding new lines before each one of your beloved if\s, just because some
night shift programmer\footnote{\url{https://blog.codinghorror.com/who-wrote-this-crap/}}
had just removed them?



\section{Research Goals}

Beforehand due the scope limitation for a Graduation Thesis,
we should only think about a basic, simple,
and yet reusable core of features.

\begin{enumerate}
    \item A Software Product with a great Object Orientation and possibilities of extension of features,
    decent research on the state of the art.

    \item Ranking all code formatting classes (beautifying) applicable.
    Including a study on what does is source beautifying,
    how to do such and why.

    \item Establish relationships between good programming practices and efficiency in programming,
    in addition to a new tool to support programmers in order to automate the long and diverse
    programming process in teams of developers with different programming `best practices'.

    \item Define, determine and classify which one are good programming practices and
    perform an in-depth study on the good practices on visual layout area,
    also known as code `Beautifying'.

    \item The definition of a flow pattern of development allowing teams of
    developers with different programming best practices,
    to work without intervene with each other up to start wars of `best good practices'.

    \item Discourse on the variety of existing tools for the support of good programming practices,
    with a comparative analysis between them,
    determining their weaknesses and strengths.
\end{enumerate}



\section{Implementation Goals}

Propose a unique tool that allowing several and distinct
programming `best practices' being implemented in several programming
languages, which can be configured and set accordingly to their wishes,
from a single software working well behaved across all programming languages.

Moreover, explain the differences for other softwares and the benefits
of a unique tool, instead of several heavily different ones.

From this point, a sketch is presented on the problem, solutions,
information as for why to want make such software, or even why do we want to
beautifying things:

\begin{enumerate}[leftmargin=*]
    \item There are many different tools, sometimes paid, and difficult to
          complete. \cite{universalCodeFormatter}

    \item Many programming languages exist, so always having Beautifier
          software for each of them is very laborious
          \cite{universalCodeFormatter}. But the approach to a Universal
          Beautifier proposed in this work, would allow easily new languages to be
          added, being completely different from previous ones, or alike. And in
          case of similarities between them, it is enough to reuse their
          configuration structures already implemented.

    \item Looking for a Beautifier for each one of them because programmers
          currently work daily with several of these languages, and they are not
          similar. So you need to configure several beautifiers to do the
          formatting. This is a problem because only a few beautifiers are more
          complete, and every time you need to make a change in the formatting
          style, you must manually propagate the same change over several
          different program configuration files, which is bad because it takes the
          user a lot of time to learn how to handle many different types of
          settings. \cite{Schweitzer}

    \item In the case of ideal Beautifier, a change in your styling is
          propagated to all languages. And if you want to leave some language out
          of it, you just need to remove it from the list on which the
          configuration block applies to, and `a)' leave it out so no change is
          applied to. Or `b)' create a new block including only the block within
          the desired settings.

\end{enumerate}

The difference from this proposal to remaining formatting tools,
is the tradeoff between end\hyp{}users and developers responsibilities.
Most tools rarely expose to end\hyp{}users their language syntax specification,
in contrast,
this proposal completely exposes the language to the end\hyp{}user as simple plain\hyp{}text,
not requiring the tool to know any language syntax neither semantics.
Moreover,
with no syntax knowledge required,
the tool be can used with any languages their user wishes to.



\section{Related Works}


\url{http://editorconfig.org/}


% 


\section{Cronograma}

   \begin{tabularx}{\linewidth}{|X|*{11}{c|}}
        \hline
        \multicolumn{1}{|c|}{\multirow{2}{*}{Etapas}} & \multicolumn{11}{|c|}{Meses}\\ \cline{2-12}

        & fev & mar & abr & mai & jun & jul & ago & set & out & nov & dez  \\ \hline

        Estudo da fundamentação teórica necessária
        &  x  &  x  &  x  &     &     &     &     &     &     &     &     \\ \hline

        Estudo da implementação de agentes na educação
        &     &  x  &  x  &     &     &     &     &     &     &     &     \\ \hline

        Estudo da ferramento Unity
        &     &     &  x  &  x  &  x  &  x  &  x  &     &     &     &     \\ \hline

        Escrita de um esboço do TCC
        &     &     &     &     &  x  &  x  &  x  &  x  &     &     &     \\ \hline

       Estudo da ferramenta Middler VR
        &     &     &     &     &     &  x  &  x  &     &     &     &     \\ \hline

        Finalização da escrita do TCC
        &     &     &     &     &     &     &     &     &  x  &     &     \\ \hline

        Ajustes finais no texto do TCC
        &     &     &     &     &     &     &     &     &  x  &  x  &     \\ \hline

        Defesa do TCC
        &     &     &     &     &     &     &     &     &     &     &  x  \\ \hline

    \end{tabularx}



\section{Custos}

    \begin{tabular}{l l l l}

        \hline

        Item                    &   Quantidade  &   Valor Unitário (R\$)    &   Valor Total (R\$) \\

        \hline

        Impressão               &   200         &   0,10                    &   20,00             \\

        Programador             &   1           &   1500,00                 &   1500,00           \\
        Notebook                &   1           &   2500,00                 &   2500,00           \\
        Desktop                 &   1           &   3000,00                 &   3000,00           \\
        Reserva de contingência &   1           &   4000,00                 &   4000,00           \\

        \hline

        Total                   &               &                           &   11020,00

    \end{tabular}



\section{Recursos Humanos}
    \begin{tabular}{l l}
        \hline
        Nome                            & Função                  \\
        \hline
        Luiz Filipe Moresco da Silva    & Autor                   \\
        Ricardo Azambuja Silveira       & Orientador              \\
        Renato Cislaghi                 & Coordenador de Projetos \\
        Thiago Ângelo Gelaim            & Coorientador            \\
        \hline
    \end{tabular}
    \\



\section{Comunicação}
    \begin{tabular}{l l l l}
        \hline
        O quê  & De quem & Para Quem & Como \\
        \hline
        Proposta de TCC         & Autor     & Renato Cislaghi   & Site de projetos \\
        Relatório de TCC I      & Autor     & Renato Cislaghi   & Site de projetos \\
        Prévia do TCC, em TCC I & Autor     & Banca             & E-mail \\
        Defesa do TCC           & Autor     & Banca             & Pessoalmente \\
        Reunião de Orientação   & Orientadores  & Autor         & Pessoalmente \\
        \hline
    \end{tabular}


\section{Riscos}

    % https://tex.stackexchange.com/questions/366156/how-to-change-the-left-padding-for-one-latex-tables-cell
    % https://tex.stackexchange.com/questions/366155/how-to-write-a-table-a-little-larger-than-the-paragraphs-with-centered-columns
    %
    \begin{adjustwidth}{-0.5\marginparwidth}{-0.5\marginparwidth}
    \small
    \begin{tabularx}{\linewidth}
    {|
        *1{                 >{\RaggedRight\arraybackslash\hsize=1.1\hsize }X       |} % Riscos
        *3{@{\hspace{3.0pt}}>{\Centering\arraybackslash                   }p{0.9cm}|} % Probabilidade, Impacto, Prioridade
        *2{                 >{\RaggedRight\arraybackslash\hsize=0.95\hsize}X       |} % Resposta, Prevenção
    }

    \hline Riscos  & Pro\-ba\-bi\-li\-da\-de & Im\-pac\-to & Prio\-ri\-da\-de & Es\-tra\-té\-gia de res\-pos\-ta & Ações de pre\-ven\-ção \\ \hline

    % Row 1
    % Riscos
    \hline Problemas com perda de dados &
    % Probabilidade
    Baixa &
    % Impacto
    Alto &
    % Prioridade
    Alta &
    % Estratégia de resposta
    Uso do backup &
    % Ações de prevenção
    Backup periódicos \\ \hline

    % Row 2
    % Riscos
    \hline Alteração do cronograma ou descontinuidade do projeto onde recebo uma bolsa &
    % Probabilidade
    \rlap{Média} &
    % Impacto
    Alto &
    % Prioridade
    Alta &
    % Estratégia de resposta
    Redefinição da data de entrega do trabalho &
    % Ações de prevenção
    Monitoramento contínuo das informações obtidas com superiores imediatos \\ \hline

    \hline \end{tabularx}
    \end{adjustwidth}




% 



\section{Implementação}

    Este trabalho tem como objetivo criar um formatador (software único) de fácil configuração e
    expansão capaz de abranger todas as linguagens de programação que existem, baseado em um uso
    específico de expressões regulares.

    A metodologia abordada será de não ter a necessidade de ter-se conhecimento da sintaxe das
    linguagens de programação que se irão fazer o parsing. Isso porque trataremos elas como texto
    comum, e será o usuário final que fará a configuração das transformações que serão aplicados no
    texto, dando liberdade de facilmente se configurar várias linguagens de programação (senão
    todas), aproveitando o fato de que muitas deles compartilham estruturas semelhantes senão
    idênticas.

    Como resultado espera-se ter um Beautifier Universal capaz de abranger todas as linguagens que
    existem, senão que seja facilmente extensível para abrange-las. Os pontos positivos dessa
    abordagem são a reusabilidade de componentes entre as linguagens. Por exemplo, `if/for/while's
    em C++ e Java são da mesma estrutura. Assim temos que escrever somente uma vez a especificação
    para um componente da linguagem.

    A ideia de um software, que em certa extensão pode continuar um ramo do Trabalho de Conclusão de
    Curso do aluno `Lucas Boppre Niehues', orientado do Professor `Olinto José Varela Furtado'
    defendido em 2013/1, com o título: `Estudo e Criação de um Editor de Código Estruturado'. Donde
    durante a leitura de seu TCC, encontra-se o seguinte trecho que faz ligação com uma das
    propostas deste trabalho, no capítulo: `8.1.2 Separação de formato de exibição e de saída':

    \medskip
    \begin{myquote}
    ``As formas que o código é exibido ao usuário e que ele é salvo em disco são controladas
    por arquivos de configuração distintos. O arquivo ``theme.ini'' contém, entre outras
    configurações, informações de como serializar a árvore sintática.''
    \end{myquote}

    \vspace{-5mm}
    ...
    \begin{myquote}
    ``A configuração de formato de saída é dada da mesma forma, mas em um arquivo
    separado, chamado ``output\_format.ini''. A decisão desta separação foi em vista de equipes
    de programadores que queiram utilizar uma convenção única para os arquivos salvos,
    mas manter a exibição a escolha de cada um. Assim os integrantes desta equipe podem
    compartilhar os seus arquivos ``output\_format.ini'' enquanto personalizam o arquivo
    ``theme.ini'' a seu gosto.''
    \end{myquote}

    Com base nisso, pode-se pensar na escrita de plugins para editores de texto/IDEs comuns como
    Sublime Text. Assim ao carregar o arquivo do disco, este plugin chama o formatter e faz a
    formatação de acordo com as configurações de exibição para o usuário. Após isso, quando o
    usuário for salvar o arquivo, o arquivo com a formatação original é devolvido.

    Para auxilar nesse processo, um módulo de autoconfiguração é de grande ajuda. Ele detecta como o
    source code está formatado e cria arquivos de configuração para ele. Assim ao salvar o arquivo,
    automaticamente ele é salvo no formato que ele foi lido. Então temos o mesmo beneficio de
    editores estruturados, como proposto trabalho de `Lucas Boppre Niehues'. De inicio podemos
    pensar com os seguinte objetivo/ideia para um TCC:

    \medskip
    \begin{myquote}
    \begin{enumerate}[nolistsep]
        \item Criar um formatador de fácil configuração e expansão para todas as linguagens de
              programação existem e que irão existir.
    \end{enumerate}
    \end{myquote}



\subsection{Problema}

    O problema proposto a se resolver é criar um Beautifier Universal. Os softwares atuais são
    limitados a um conjunto similar, ou mesmo à uma única linguagem, e além de muitos, serem
    limitados ao que eles podem fazer por você ao processar/formatar o código \cite{Terence}.

    Logo abaixo há algumas regras de formatação básica encontrados no serviço online
    \url{http://prettyprinter.de/} acessado em março/2017:

    \medskip
    \begin{myquote}
    \begin{enumerate}[nolistsep]
        \item Add new lines after ``\{'' and before ``\}''
        \item Add new lines before ``\{''
        \item Remove empty lines
        \item Add comment lines before function
        \item Add new lines after ``;''
        \item Add new lines after ``\}''
        \item Remove new lines
        \item Reduce whitespace
        \item Put the code again in the input box above after submit
    \end{enumerate}
    \end{myquote}

    A partir deste ponto, apresenta-se um esboço sobre o problema, soluções, informações como
    porquês de se querer fazer um software assim, ou ainda de querer-se o beautifying:

    \begin{enumerate}[leftmargin=*]

        \item

        Motivação: Existem muitas ferramentas distintas, por vezes pagas, e dificilmente completas
        \cite{Terence}.

        \item

        Muitas linguagens de programação existem, assim sempre ter fazer um software Beautifier para
        cada uma delas é muito trabalhoso \cite{Terence}. Mas a abordagem para um Beautifier
        Universal proposta nesse trabalho, permite que facilmente novas linguagens sejam
        adicionadas, sendo elas completamente diferentes das anteriores, ou similares. No caso de
        similaridades, basta reutilizar as estruturas de configuração das linguagens já existentes.

        \item

        Preocupa-se de fazer um Beautifier para cada uma delas por que programadores atualmente
        trabalham diariamente com varias dessas linguagens, e elas não são similares. Assim precisa-
        se configurar vários beautifiers para fazer a formatação. Isso é um problema por que,
        somente alguns beautifiers são mais completos, e toda vez que precisa-se fazer uma alteração
        no estilo de formatação, precisa-se propagar manualmente a mesma mudança ao longo de vários
        arquivos de configuração de programas distintos, o que é ruim pois toma ao usuário muito
        tempo de aprender a lidar com várias e muito diferentes tipos de configurações
        \cite{Schweitzer}.

        \item

        No caso do Beautifier que propõem-se, uma mudança no estilo é propagada para todas as
        linguagens. E caso queira-se deixar alguma linguagem fora da regra, basta remover ela da
        lista ao qual esse bloco da configuração se aplica, e `a)' deixar ela de fora assim nenhuma
        mudança é aplicada a ela. Ou `b)' criar um novo bloco que inclua somente ela com a
        configuração desejada.

        \item

        A seguir, temos algumas frases sobre o assunto:

        \begin{myquote}
        % \setlength{\itemindent}{5pt}
        ``One of absolute worst, worst methods of teamicide for software developers is to engage
        in these kinds of passive-aggressive formatting wars. I know because I've been there.
        They destroy peer relationships, and depending on the type of formatting, can also damage
        your ability to effectively compare revisions in source control, which is really scary.
        I can't even imagine how bad it would get if the lead was guilty of this behavior. That's
        leading by example, all right. Bad example.'', \cite{Atwood}.
        \end{myquote}
        \vspace{-5mm}
        ...
        \begin{myquote}
        ``So yes, absurd as it may sound, fighting over whitespace characters and other seemingly
        trivial issues of code layout is actually justified. Within reason of course -- when done
        openly, in a fair and concensus building way, and without stabbing your teammates in the
        face along the way.'', \cite{Atwood}.
        \end{myquote}

        \begin{myquote}``
        I'd say there are two main reasons to enforce a single code format in a project. First has
        to do with version control: with everybody formatting the code identically, all changes in
        the files are guaranteed to be meaningful. No more just adding or removing a space here or
        there, let alone reformatting an entire file as a `side effect' of actually changing just a
        line or two.'', \cite{Geukens}.
        \end{myquote}

    \end{enumerate}



\subsection{Objetivos}

    O objeto neste trabalho de TCC proposto aqui não é inicialmente suportar todas as regras de
    formatação de todas as linguagens de programação, mas a criação de uma estrutura básica inicial
    e robusta que sejam capaz de ser desenvolvida a ponto de ser facilmente expandida, tanto na
    adição de novos módulos de processamento no programa escrito, tanto pelo usuário final na
    escrita dos arquivos de programação.

    A teoria da técnica empregada é muito simples, mas diferente das atuais por que é atribuído ao
    usuário final a responsabilidade de dizer onde será realizado o beautifying do modulo que está
    se configurando. Esse é o preço a pagar para permitir a criação de um Beautifier Universal.
    Quando diz-se fácil configuração, refire-se a não necessidade de recorrer a programação ´C++',
    i.e., alterar o código fonte do programa para permitir/especificar onde devem ser realizadas as
    alterações de beautifying.


\subsubsection{Objetivos Gerais}

    \begin{enumerate}[leftmargin=*]

        \item

        Escrever o programa em C++ ou afins, para permitir também que a formação/beautifying seja
        (em trabalhos futuros/talvez nesse) dinâmico, isto é, na medida que você digita o texto, ele
        é formatado para você. Assim você pode focar mais em escrever o código, ao invés que se
        preocupar com o espaçamento, alinhamento, parenteses, linhas novas, e o que mais que seja.

        \item

        Utilizar o Framework `doctest` para escrita dos Testes de Unidade. Pois após procurar e
        testar alguns frameworks para testes de unidade em C++, entrou-se este como servindo muito
        bem as requisitos do projecto. Ele causa baixíssimo incremento no tempo de compilação e
        permite que os testes possam ser escritos no mesmo arquivo onde encontram-se o código do
        programa, sem que eles sejam compilados.

        \item

        Utilizar uma versão/algoritmo multi-core, então cada uma das regras pode ser processada em
        paralelo e sobre o mesmo source code original. Essa parte é bastante complexa de ser escrita
        por que as regras entre si podem gerar conflitos sobre o que elas estão fazendo. Para
        resolver esse problema, fazer com que cada regra processada gere um objeto de mudanças que
        essa regra está propondo. No final do processamento de todas as regras, será realizado um
        fusão das mudanças que cada uma decidiu realizer, e caso duas regras queriam mudar o mesmo
        pedaço/trecho de código, será lançada um exceção e uma nova classe de mudanças/regra deve
        estar disponível para resolver esse conflito. Caso não exista, ambas as mudanças são
        descartadas e somente as mudanças sem conflitos são refletidas no código.

    \end{enumerate}


\subsubsection{Objetivos Específicos}


    \begin{enumerate}[leftmargin=*]

        \item

        Um Produto de Software com uma ótima orientação a objetos e possibilidades de extensão das
        funcionalidades.

        \item

        Classificar todas classes e tipos de formatações (beautifying) de código aplicáveis com
        facilidade. Acredito que esse seja uma das partes a serem escritas e entregues na
        monografia. Um estudo sobre o que é beautifying, como fazer e por que fazer.

        \item

        Implementação de um núcleo funcional e de uma pesquisa decente sobre o estado da arte. Um
        dos pontos difíceis seria a marcação dos escopos, mas isso já é implementado pelo núcleo do
        editor Sublime Text, assim provado como possível de ser feito.

        \item

        Inicialmente devido a limitação de tempo em 1 ano e meio para um TCC, podemos pensar somente
        um núcleo básico, simples, reutilizável e que talvez possa ajudar no contexto da linguagem
        que vocês desenvolvem.

    \end{enumerate}


\subsubsection{Trabalhos Futuros}

    O número de recursos/funcionalidades e estratégias de otimizações para serem implementadas, e
    etc, são imensas. Mas esses trabalhos podem ser muito mais para frente depois da entrega do TCC.
    Hoje o controle de espaços em chamadas de funções, declarações de classes, comentários e etc,
    são mais tranquilos de se entender e pensar. Entretanto no requisito e ajuste de indentação,
    inserção/remoção de parenteses redundantes, etc ainda falta estudo sobre como deve ser
    implementado isso.

    Contudo essa especificação por parte do usuário é limitado a linguagens Livres de Contexto
    (máquinas de pilha). Assim caso as especificações de escope precisarem ser feitas em termos de
    linguagens Sensíveis ao Contexto ou ainda Recursivamente Enumeráveis, vai ser preciso tratar
    esses elementos diretamente em C++ (máquina de turing).

    Entretanto não consegue-se pensar facilmente em casos em que precise mais do que tratadores
    Livres de Contexto para realizar a especificação de quais partes do código deve ser necessário
    formatar. Sublime Text faz uso dessa técnica para o Highlight dos códigos das mais diversas
    linguagens e acredita-se que tenha um bom resultado.





\subsection{Método de pesquisa}

    A vantagem nesta abordagem é não ter a necessidade de ter-se conhecimento da sintaxe das
    linguagens de programação que se irão fazer o parsing. Isso porque trataremos elas como texto
    comum, e será o usuário final que fará a configuração das transformações que serão aplicados no
    texto, dando liberdade de facilmente se configurar várias linguagens de programação,
    aproveitando o fato de que muitas deles compartilham estruturas semelhantes senão idênticas.

    A literatura/programas atuais são dependentes de linguagem de programação. Minha proposta é
    fazer este processo independente de linguagem, mas de dialetos como este exemplo tirado do PDF
    em anexo a este e-mail `Initial check list tasks to do.pdf':

    \begin{lstlisting}
    // This is the name used to reference this scope around the settings files.
    Scope Name:
    %c++_like_block_comment

    // This set on which languages this block should be included. Setting it
    // to empty will allow it to be parsed for any languages.
    Language Inclusion:
    Java, C++, Pawn

    // Defines a expression which will map the beginning of a exclusion block.
    Scope Start:
    /\*\*

    // Defines a expression which will map the ending of a exclusion block.
    Scope End:
    \\\*
    \end{lstlisting}
    \vspace*{-4mm}

    A abordagem acima é uma abordagem ingênua, portanto somente brevemente ilustrativa. O real motor
    para o software é baseado em expressões regulares e um pilha de contextos. Esta ideia foi
    inicialmente desenvolvida pelo editor de texto `Sublime Text' \cite{Skinner}. Este editor
    utiliza essa estrutura de blocos para fazer a sintaxe highlighting do códigos das linguagens
    através de expressões regulares alocação de contextos/escopos. Essa mesma abordagem pode ser
    utilizada pelo usuário para definir em quais regiões uma Máquina de Turing (linguagens C++/Rust)
    devem fazer/propor as alterações no código.


\subsubsection{Pontos}

    Os pontos positivos dessa abordagem para um formatador de código são a reusabilidade de
    componentes entre as linguagens pelo usuário final da aplicação ao invés do programador, o que
    torna este software muito mais genérico e abre a possibilidades de maior sucesso para a criação
    definitiva de um formatador Universal de códigos das linguagens de programação, quaisquer sejam
    elas. Por exemplo, `if/for/while'\textquotesingle s em linguagens de programação como C++ e Java
    são da mesma estrutura. Assim temos que escrever somente uma vez a especificação para um
    componente da linguagem sem recorrer a programação de do código do programa. Isso tem a vantagem
    de por der ser configurado pelo usuário final ao invés do programador, assim fica mais simples
    de configurar e expandir o conjunto de linguagens disponíveis ao processamento/beautifying.

    Softwares existentes e similares:

    \medskip
    \begin{myquote}
    \begin{enumerate}[leftmargin=*]

        \item

        CodeBeautify is an online code beautifier which allows you to beautify your source code:
        \url{http://codebeautify.org/}.

        \item

        A universal code formatter, written in Dart: \url{https://pub.dartlang.org/packages/unifmt}.

        \item

        Google-java-format is a program that reformats Java source code to comply with Google Java
        Style: \url{https://github.com/google/google-java-format}.

        \item

        CodeFormatter is a Sublime Text 2/3 plugin that supports format (beautify) source code.
        \url{https://github.com/akalongman/sublimetext-codeformatter} and
        \url{https://github.com/aukaost/SublimePrettyYAML}

        \item

        UniversalIndentGUI offers a live preview for setting the parameters of nearly any indenter.
        You change the value of a parameter and directly see how your reformatted code will look
        like. Save your beauty looking code or create an anywhere usable batch/shell script to
        reformat whole directories or just one file even out of the editor of your choice that
        supports external tool calls: \url{http://universalindent.sourceforge.net/} and
        \url{https://github.com/danblakemore/universal-indent-gui}.

    \end{enumerate}
    \end{myquote}


\subsubsection{Listagens}

    Algumas bibliotecas existentes, e potencialmente utilizadas como `syntect` para o auxílio na
    construção do produto de software:

    \begin{myquote}
    \begin{enumerate}[leftmargin=*,parsep=0pt]

        \item \url{https://github.com/jbeder/yaml-cpp}
        \item \url{https://github.com/trishume/syntect}
        \item \url{https://github.com/onqtam/doctest}
        \item \url{https://github.com/c42f/tinyformat}
        \item \url{https://github.com/limetext/lime}
        \item \url{https://forum.sublimetext.com/t/disassembling-sublime-text/24824}

    \end{enumerate}
    \end{myquote}

    Segue-se uma lista básica de formatters/beautifiers acessado no endereço
    \lword{\url{http://www.softpanorama.org/Utilities/beautifiers.shtml}} em março/2017:

    \medskip
    \begin{sloppypar}
    \begin{myquote}\RaggedRight
    \begin{enumerate}[leftmargin=*,parsep=0pt]

        \item CB210.ZIP - C Beautifier 2.10 - polish C source code (19,406 bytes, 06/22/92)
        \item CL121.ZIP - Codelister 1.21 - print C code with stats (51,110 bytes, 01/10/94)

        \item CPC200.ZIP - CodePrint for C/C++ 2.00 is a full-featured command line driven source
        code reformatter and pretty printer for C++ and C; over 20 customization features including
        auto-indent, adjustable tab spacing, indent styles, flow lines, comment alignment, and line
        editing for consistent white space (140,605 bytes, 01/26/96)

        \item CSCOP120.ZIP - c-scope 1.20 analyzes C source code and produces various reports
        (48,505 bytes, 06/30/95)

        \item HTML : \url{http://www.digital-mines.com/htb/}
        \item HTML : \url{http://www.datacomm.ch/mwoog/software/perl/beautifier.html}
        \item HTML : \url{http://www.watson-net.com/free/perl/s_fhtml.asp}
        \item SQL : \url{http://www.netbula.com/products/sqlb}
        \item Oracle PLSQL : \url{http://www.revealnet.com}
        \item GPL \url{http://www.geocities.com/~starkville/vancbj.html}
        \item GPL \url{http://kevinkelley.mystarband.net/java/dent.html}
        \item Free \url{http://www.tiobe.com/jacobe.htm}
        \item Free \url{http://www.mmsindia.com/JPretty.html}
        \item Free \url{http://members.magnet.at/johann.langhofer/products/jxbeauty/overview.html} (has JBuilder support)
        \item Free \url{http://www.semdesigns.com/Products/Formatters/JavaFormatter.html}
        \item Commercial \$24.99 \url{http://smartbeautify.com}
        \item Commercial \$129 \url{http://www.jindent.com}
        \item Google \url{http://directory.google.com/Top/Computers/Programming/Languages/Java/Development_Tools/Code_Beautifiers/?tc=1}
        \item Java, SQL, HTML, C++ : \url{http://www.semdesigns.com/Products/DMS/DMSToolkit.html}
        \item Java JIndent \url{http://home.wtal.de/software-solutions/jindent}
        \item Java Pat \url{http://javaregex.com/cgi-bin/pat/jbeaut.asp}
        \item Java JStyle \url{http://www.redrival.com/greenrd/java/jstyle}
        \item Java JPrettyPrinter \url{http://www.epoch.com.tw/download/ms/java/java.htm}
        \item Java JxBeauty \url{http://members.nextra.at/johann.langhofer/download/jxbeauty} and the JxBeauty Home
        \item Java beautify percolator
        \item Java list \url{http://www.java.about.com/compute/java/library/weekly/aa102499.htm}
        \item Java html present VasJava2HTML
        \item Java code colorifier and beautifier \url{http://www.mycgiserver.com/~lisali/jccb}
        \item Perl : \url{http://www.consultix-inc.com/www.consultix-inc.com/talk.htm}
        \item Perl : \url{http://www.consultix-inc.com/www.consultix-inc.com/perl_beautifier.html}
        \item Fortran beautifier : \url{http://www.aeem.iastate.edu/Fortran/tools.html}

        \item C++ : BCPP site is at \url{http://dickey.his.com/bcpp/bcpp.html} or at \url{http://www.clark.net/pub/dickey}.
        BCPP ftp site is at \url{ftp://dickey.his.com/bcpp/bcpp.tar.gz}

        \item C++ : \url{http://www.consultix-inc.com/c++b.html}
        \item C : \url{http://www.chips.navy.mil/oasys/c/} and mirror at Oasys
        \item C++, C, Java, Oracle Pro-C Beautifier \url{http://www.geocities.com/~starkville/main.html}

        \item C++, C beautifier \url{http://users.erols.com/astronaut/vim/ccb-1.07.tar.gz} and site at
        \url{http://users.erols.com/astronaut/vim/#vimlinks_src}

        \item GC! GreatCode! is a powerful C/C++ source code beautifier Windows 95/98/NT/2000
        \url{http://perso.club-internet.fr/cbeaudet}

        \item C++ beautifier `SourceStyler' \url{https://web.archive.org/web/20061205061102/http://ochresoftware.com/}
        \item JavaScript : \url{http://jsbeautifier.org/}

    \end{enumerate}
    \end{myquote}
    \end{sloppypar}


\subsubsection{Trabalhos Correlatos}

    Após a busca do que há de publicações científicas sobre o assunto e entra-se alguns trabalhos na
    área específica e similar aos trabalhos feitos pelor formatadores de códigos (Beautifiers).
    Nessa modalidade de trabalho, pode-se confundir-se com artigos que tratam sobre o `Prettyprint`,
    que trata-se de colorir o texto e exibir-lo ao usuário. O que não é o que se busca nesse
    trabalho, mas sim fazer alterações no texto sobre a forma como ele é estruturado, apresentado ao
    usuário e salvo em disco. Seguem as seguintes publicações:

    % How to add `parsep` to `itemsep` and set `parsep` to 0pt, when declaring my list?
    % https://tex.stackexchange.com/questions/366904/how-to-add-parsep-to-itemsep-and-set-parsep-to-0pt-when-declaring-my-list
    \begin{sloppypar}
    \begin{myquote}\RaggedRight
    \begin{enumerate}[leftmargin=*,parsep=0pt]

    \item \url{https://www.researchgate.net/publication/228540036_An_industrial_application_of_context-sensitive_formatting}

    \item \url{http://www.suodenjoki.dk/us/archive/2010/cpp-checkstyle.htm}

    \item \url{http://www.basicinputoutput.com/2014/08/uncrustify-your-bios.html}

    \item \url{http://prettyprinter.de/}

    \item \url{http://www.softpanorama.org/Utilities/beautifiers.shtml}

    \item

    {\bfseries Towards a universal code formatter through machine learning:}
    In this paper, we solve the formatter construction problem using a novel approach, one that
    automatically derives formatters for any given language without intervention from a language
    expert. We introduce a code formatter called CODEBUFF that uses machine learning to abstract
    formatting rules from a representative corpus, using a carefully designed feature set. Our
    experiments on Java, SQL, and ANTLR grammars show that CODEBUFF is efficient, has excellent
    accuracy, and is grammar invariant for a given language. It also generalizes to a 4th language
    tested during manuscript preparation.
    \begin{enumerate}[nolistsep,topsep=0pt,label=$\star$]
        \item \url{http://dl.acm.org/citation.cfm?id=2997383}
        \item \url{http://homepages.cwi.nl/~jurgenv/papers/SLE16.pdf}
    \end{enumerate}

    \item \url{https://www.google.com/search?q=universal+source+code+formatter}
    \begin{enumerate}[nolistsep,topsep=0pt,label=$\star$]
        \item \url{https://www.google.com/search?q=universal+source+code+beautifier}
    \end{enumerate}

    \item \url{http://en.wikipedia.org/wiki/Indent_style}
    \begin{enumerate}[nolistsep,topsep=0pt,label=$\star$]
        \item \url{https://en.wikipedia.org/wiki/Programming_style}
        \item \url{https://en.wikipedia.org/wiki/Scope_(computer_science)}
    \end{enumerate}

    \item \url{http://wiki.c2.com/?CodingStyle}
    \begin{enumerate}[nolistsep,topsep=0pt,label=$\star$]
        \item \url{https://github.com/google/code-prettify}
        \item \url{https://github.com/uncrustify/uncrustify}
    \end{enumerate}

    \item \url{https://en.wikipedia.org/wiki/Prettyprint}
    \begin{enumerate}[nolistsep,topsep=0pt,label=$\star$]
        \item \url{https://www.researchgate.net/search.Search.html?query=formatting%20source%20code&type=publication}
        \item \url{https://www.researchgate.net/search.Search.html?query=pretty%20print%20source%20code&type=publication}
    \end{enumerate}

    \item \url{https://github.com/gchpaco/gopprint}
    \begin{enumerate}[nolistsep,topsep=0pt,label=$\star$]
        \item \url{http://dl.acm.org.sci-hub.io/citation.cfm?id=357115}
        \item \url{https://www.cs.indiana.edu/~sabry/papers/yield-pp.pdf}
    \end{enumerate}

    \item \url{http://www.worldcat.org/title/beautiful-code-a-customizable-code-beautifier-for-java/oclc/56564674}
    \begin{enumerate}[nolistsep,topsep=0pt,label=$\star$]
        \item \url{https://www.researchgate.net/publication/34736049_Beautiful_code_a_customizable_code_beautifier_for_Java}
        \item \url{https://vufind.carli.illinois.edu/vf-ncc/Record/ncc_118189/Holdings}
    \end{enumerate}

    \item \url{https://www.researchgate.net/publication/4283921_Smart_Formatter_Learning_Coding_Style_from_Existing_Source_Code}
    \begin{enumerate}[nolistsep,topsep=0pt,label=$\star$]
        \item \url{http://www.ing.unisannio.it/mdipenta/index.html}
        \item \url{https://github.com/iain/rspec-smart-formatter}
    \end{enumerate}

    \item \url{https://www.researchgate.net/publication/2543984_Source_Code_Files_as_Structured_Documents}
    \begin{enumerate}[nolistsep,topsep=0pt,label=$\star$]
        \item \url{https://en.wikipedia.org/wiki/SrcML}
    \end{enumerate}

    \item \url{https://www.researchgate.net/publication/228540036_An_industrial_application_of_context-sensitive_formatting}
    \begin{enumerate}[nolistsep,topsep=0pt,label=$\star$]
        \item \url{https://www.researchgate.net/publication/234809222_Program_indentation_and_comprehensibility}
    \end{enumerate}

    \end{enumerate}
    \end{myquote}
    \end{sloppypar}


\subsubsection{Obfuscators}

    Aqui encontra-se o lado oposto dessas ferramentas, Source Code Obfuscators, que servem para
    destruir o visual do código. Usualmente utilizado para dificultar a leitura por outras pessoas
    ou ainda reduzir o tamanho de códigos de linguagens scripting que devem ser carregadas/baixadas
    por navegadores de internet, assim diminuindo o tráfego de internet e salvando/economizando
    largura de banda para download:

    \begin{sloppypar}
    \begin{myquote}\RaggedRight
    \begin{enumerate}[leftmargin=*,parsep=0pt]

    \item \url{https://en.wikipedia.org/wiki/Obfuscation_(software)}

    \item \url{http://www.semdesigns.com/Products/Obfuscators/index.html}

    \end{enumerate}
    \end{myquote}
    \end{sloppypar}




\bibliography{refs}


\end{document}
