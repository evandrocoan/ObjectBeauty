
%
% Simple Sectioned Essay Template - LaTeX Template
%
% This template has been downloaded from:
% http://www.latextemplates.com
%
% `proposal.tex`
% Based on
%
% 1. https://github.com/royertiago/tcc
% 2. https://github.com/evandrocoan/abntex2-ufsc
% 3. http://www.latextemplates.com/template/simple-sectioned-essay


%----------------------------------------------------------------------------------------
%   PACKAGES AND OTHER DOCUMENT CONFIGURATIONS
%----------------------------------------------------------------------------------------

% Applying options to already loaded package
% https://tex.stackexchange.com/questions/124049/applying-options-to-already-loaded-package
\PassOptionsToPackage{backref,colorlinks,linkcolor=link_color,citecolor=dkgreen}{hyperref}
\PassOptionsToPackage{strict}{changepage}
\PassOptionsToPackage{shortlabels}{enumitem}

% Rhe ABNT class hard codes the language to English. It needs to be edited to change it.
\documentclass[
    % -- opções da classe memoir --
    10pt,               % tamanho da fonte
    openright,          % capítulos começam em pág ímpar (insere página vazia caso preciso)
    twoside,            % para impressão em recto e verso. Oposto a oneside
    a5paper,            % tamanho do papel.
    % -- opções da classe abntex2 --
    % chapter=TITLE,     % títulos de capítulos convertidos em letras maiúsculas
    % section=TITLE,     % títulos de seções convertidos em letras maiúsculas
    % -- opções do pacote babel, idioma adicional para hifenização --
    brazil,
    english
    ]{abntevandro/tex/latex/abntevandro/abntevandro}

% ABNT Citations rules
\usepackage[alf]{abntevandro/tex/latex/abntevandro/abntevandrocite}
\usepackage{ufscthesis/abntex2-ufsc}


%----------------------------------------------------------------------------------------
%   File settings
%----------------------------------------------------------------------------------------

% Load General Packages




% Incompatible color definition when using tikz with color package
% https://tex.stackexchange.com/questions/150369/incompatible-color-definition-when-using-tikz-with-color-package
\usepackage{xcolor}

\definecolor{dkgreen}{rgb}{0,0.6,0}
\definecolor{gray}{rgb}{0.5,0.5,0.5}
\definecolor{mauve}{rgb}{0.58,0,0.82}

\definecolor{link_color}{RGB}{26,13,178}


% For web links and paths with \path{..} and \url{https://www.python.org/downloads/}
%
% https://tex.stackexchange.com/questions/3033/forcing-linebreaks-in-url
% ftp://tug.ctan.org/pub/tex-archive/macros/latex/contrib/hyperref/doc/options.pdf
\PassOptionsToPackage{hyphens}{url}
\usepackage[backref,colorlinks,linkcolor=link_color,citecolor=dkgreen]{hyperref}

% How to fix URL overfull & underfull on emumeration?
% % https://tex.stackexchange.com/questions/366803/how-to-fix-url-overfull-underfull-on-emumeration
%
% Forcing linebreaks in \url
% https://tex.stackexchange.com/questions/3033/forcing-linebreaks-in-url/10401
\usepackage{url}
\makeatletter
\g@addto@macro{\UrlBreaks}{\UrlOrds}
\makeatother

% \lettrine{O}{nce} upon a time...
% \lettrine[findent=2pt]{\fbox{\textbf{T}}}{ }his thesis deals with...
%
% https://tex.stackexchange.com/questions/164298/starting-a-paragraph-with-a-big-letter
\usepackage{lettrine}

% Required for including pictures, resizebox
\usepackage{graphicx}

% Allows putting an [H] in \begin{figure} to specify the exact location of the figure
\usepackage{float}

% Allows in-line images such as the example fish picture
\usepackage{wrapfig}

% How to automatically force latex to not justify the text when it is not wise?
% https://tex.stackexchange.com/questions/365801/how-to-automatically-force-latex-to-not-justify-the-text-when-it-is-not-wise
\usepackage{array,ragged2e}

% Use its macro adjustwidth* to extend tables out of outer text border.
% https://tex.stackexchange.com/questions/366155/how-to-write-a-table-a-little-larger-than-the-paragraphs-with-centered-columns
\usepackage[strict]{changepage}

% No spacing between enumerated items with \usepackage{enumerate}
% https://tex.stackexchange.com/questions/119919/no-spacing-between-enumerated-items-with-usepackageenumerate
\usepackage[shortlabels]{enumitem}

\usepackage{tabularx}
\usepackage{multirow}








%
% New Macros
%

% How could the `\everypar` justification statement be used?
% https://tex.stackexchange.com/questions/365818/how-could-the-everypar-justification-statement-be-used
\newbox\linebox \newbox\snapbox
\def\eatlines{
  \setbox\linebox\lastbox % check the last line
  \ifvoid\linebox
  \else % if it’s not empty
    \unskip\unpenalty % take whatever is
    {\eatlines} % above it;
    \setbox\snapbox\hbox{\unhcopy\linebox}
    \ifdim\wd\snapbox<.98\wd\linebox
       \box\snapbox % take the one or the other,
    \else \box\linebox \fi
  \fi
}


% How could the `\everypar` justification statement be used?
% https://tex.stackexchange.com/questions/365818/how-could-the-everypar-justification-statement-be-used
\everypar={\setbox0=\lastbox \par
   \vbox\bgroup \everypar={}\def\par{\endgraf\eatlines\egroup}}


% Creates a new environment which can be used as:
%
% \begin{foo}
%   Text...
%
%   Text ...
% \end{foo}
%
% https://tex.stackexchange.com/questions/62333/push-long-words-in-a-new-line
\newenvironment{foo}
{\par
\hyphenpenalty=10000
\exhyphenpenalty=10000
}
{\par}


% How to break long URLs using common hyphenation but adding a line feed indicator?
%
% some text \brkurl{http://www.example.com/this/directory/here}
%
% https://tex.stackexchange.com/questions/69824/how-to-break-long-urls-using-common-hyphenation-but-adding-a-line-feed-indicator
\def\addurlspace#1{%
\ifx\relax#1%
\else
\ifx/#1\space\fi
\ifx.#1\space\fi
#1%
\ifx/#1\space\fi
\ifx.#1\space\fi
\expandafter\addurlspace
\fi}

\makeatletter

\@namedef{OT1-zwidthchar}{255}
\@namedef{T1-zwidthchar}{"17}

\def\brkurl#1{%
\edef\savedhchar{\the\hyphenchar\font}%
\global\setbox1\hbox{}%
\setbox0=\vbox{\hsize=2pt\rightskip=0pt plus 1fill
\hfuzz\maxdimen
\tracinglostchars0
\overfullrule0pt
\hyphenchar\font=\csname \f@encoding-zwidthchar\endcsname
\noindent \hskip0pt \addurlspace #1\relax
\par
\loop
\setbox4 \lastbox
\ifvoid4 \else
\global\setbox1\hbox{\unhbox4\unskip\unskip\discretionary{\hbox{\rlap{$\leftarrow$}}}{}{}\unhbox1}%
\unskip
\unskip
\unpenalty
\unskip
\repeat
}%
\unhbox1
\hyphenchar\font\savedhchar
\relax}

\makeatother


% Change background color for text block
% https://tex.stackexchange.com/questions/238294/change-background-color-for-text-block
\usepackage{framed}
\usepackage[most]{tcolorbox}
\definecolor{shadecolor}{RGB}{219, 229, 241}
\newtcolorbox{myquote}{
colback=shadecolor,
grow to right by=-2mm,
grow to left by=-2mm,
boxrule=0pt,
boxsep=0pt,
breakable,
}

% Make first row of table all bold
%
% Usage:
% 1. Add `B` on the borders and `^` before each column definition.
% 2. `\rowstyle{\bfseries}` before the row you want to bold.
%
% Example:
% \begin{tabularx}{\linewidth}
% {|
%     *1{                 >{\RaggedRight\arraybackslash\hsize=1.1\hsize }BX       |} % Riscos
%     *3{@{\hspace{3.0pt}}>{\Centering\arraybackslash                   }^p{0.9cm}|} % Probabilidade, Impacto, Prioridade
%     *2{                 >{\RaggedRight\arraybackslash\hsize=0.95\hsize}^X       |} % Resposta, Prevenção
% }
%
% \hline
%
% \rowstyle{\bfseries}
% Riscos  & 1 & 2 & 3 & Estratégia de resposta & Ações de prevenção \\ \hline
%
%
% https://tex.stackexchange.com/questions/4811/make-first-row-of-table-all-bold
\usepackage{array}
\newcolumntype{B}{>{\global\let\currentrowstyle\relax}}
\newcolumntype{^}{>{\currentrowstyle}}
\newcommand{\rowstyle}[1]{\gdef\currentrowstyle{#1}%
  #1\ignorespaces
}



%
% New commands
%

% Allow to push long words on new lines when they do not fit entirely on the current line.
% https://tex.stackexchange.com/questions/62333/push-long-words-in-a-new-line
\newcommand\lword[1]{\leavevmode\nobreak\hskip0pt plus\linewidth\penalty50\hskip0pt plus-\linewidth\nobreak{#1}}
\newcommand\lurl[1]{\leavevmode\nobreak\hskip0pt plus\linewidth\penalty50\hskip0pt plus-\linewidth\nobreak{\url{#1}}}


% For the new command \latex
\usepackage{xspace}

% Write the word LaTeX nicely.
\newcommand{\latex}{\LaTeX\xspace}


% Create a bold title all in upper case.
\newcommand{\Title}[1]{\textbf{\MakeUppercase{#1}}}








% Writing code in latex document. Usage: \begin & \end {lstlisting}
% http://stackoverflow.com/questions/3175105/writing-code-in-latex-document
\usepackage{listings}

% How to insert code with accents with listings?
% https://tex.stackexchange.com/questions/30512/how-to-insert-code-with-accents-with-listings
\usepackage{listingsutf8}

% Incompatible color definition when using tikz with color package
% https://tex.stackexchange.com/questions/150369/incompatible-color-definition-when-using-tikz-with-color-package
\usepackage{xcolor}

\definecolor{dkgreen}{rgb}{0,0.6,0}
\definecolor{gray}{rgb}{0.5,0.5,0.5}
\definecolor{mauve}{rgb}{0.58,0,0.82}

\lstset{frame=,
  language=Java,
  aboveskip=3mm,
  belowskip=3mm,
  showstringspaces=false,
  columns=flexible,
  basicstyle={\small\ttfamily},
  numbers=left,
  numberstyle=\color{gray},
  keywordstyle=\color{blue},
  commentstyle=\color{dkgreen},
  stringstyle=\color{mauve},
  breaklines=true,
  breakatwhitespace=true,
  tabsize=3
}

% Defining `lstset` parameters for multiple languages & How can I highlight YAML code in a pretty way with listings?
%
% Usage \begin{lstlisting}[style=yaml_style] ... \end{lstlisting}
%
% https://tex.stackexchange.com/questions/45711/defining-lstset-parameters-for-multiple-languages
% https://tex.stackexchange.com/questions/152829/how-can-i-highlight-yaml-code-in-a-pretty-way-with-listings
\newcommand\YAMLcolonstyle{\color{red}}
\newcommand\YAMLkeystyle{\color{black}}
\newcommand\YAMLvaluestyle{\color{blue}}
\newcommand\ProcessThreeDashes{\llap{\color{cyan}\mdseries-{-}-}}

\lstdefinestyle{yaml_style}{
  frame=,
  aboveskip=3mm,
  belowskip=3mm,
  showstringspaces=false,
  columns=flexible,
  numbers=left,
  numberstyle=\color{gray},
  breaklines=true,
  breakatwhitespace=true,
  tabsize=2,
  keywords={true,false,null,y,n},
  keywordstyle=\color{darkgray},
  basicstyle=\YAMLkeystyle,                                 % assuming a key comes first
  sensitive=false,
  comment=[l]{\#},
  morecomment=[s]{/*}{*/},
  commentstyle=\color{purple}\ttfamily,
  stringstyle=\YAMLvaluestyle\ttfamily,
  moredelim=[l][\color{orange}]{\&},
  moredelim=[l][\color{magenta}]{*},
  moredelim=**[il][\YAMLcolonstyle{:}\YAMLvaluestyle]{:},   % switch to value style at :
  morestring=[b]',
  morestring=[b]",
  literate = {---}{{\ProcessThreeDashes}}3
             {>}{{\textcolor{red}\textgreater}}1
             {|}{{\textcolor{red}\textbar}}1
             {\ -\ }{{\mdseries\ -\ }}3,
  inputencoding=utf8, % Listings in Latex with UTF-8 (or at least german umlauts)
  extendedchars=true, % http://stackoverflow.com/questions/1116266/listings-in-latex-with-utf-8-or-at-least-german-umlauts
  literate=%
  {é}{{\'{e}}}1
  {è}{{\`{e}}}1
  {ê}{{\^{e}}}1
  {ë}{{\¨{e}}}1
  {É}{{\'{E}}}1
  {Ê}{{\^{E}}}1
  {û}{{\^{u}}}1
  {ù}{{\`{u}}}1
  {ú}{{\'{u}}}1
  {â}{{\^{a}}}1
  {à}{{\`{a}}}1
  {á}{{\'{a}}}1
  {ã}{{\~{a}}}1
  {Á}{{\'{A}}}1
  {Â}{{\^{A}}}1
  {Ã}{{\~{A}}}1
  {ç}{{\c{c}}}1
  {Ç}{{\c{C}}}1
  {õ}{{\~{o}}}1
  {ó}{{\'{o}}}1
  {ô}{{\^{o}}}1
  {Õ}{{\~{O}}}1
  {Ó}{{\'{O}}}1
  {Ô}{{\^{O}}}1
  {î}{{\^{i}}}1
  {Î}{{\^{I}}}1
  {í}{{\'{i}}}1
  {Í}{{\~{Í}}}1
}

% Change background color for text block
% https://tex.stackexchange.com/questions/238294/change-background-color-for-text-block
\usepackage{framed}
\usepackage[most]{tcolorbox}
\definecolor{shadecolor}{RGB}{219, 229, 241}
\newtcolorbox{myquote}{
colback=shadecolor,
grow to right by=-2mm,
grow to left by=-2mm,
boxrule=0pt,
boxsep=0pt,
breakable
}



% \input{utilities/ufsc}

% If you want to see the text block frame, you can use the showframe package
\usepackage{showframe}

% To use the font Times New Roman, instead of the default LaTeX font
% \usepackage{mathptmx}
%
% more up-to-date than 'mathptmx'
\usepackage{newtxtext}
\usepackage{newtxmath}

% Always use it as should improve full justification
% https://tex.stackexchange.com/questions/10377/texttt-overfull-hbox-problem
% https://tex.stackexchange.com/questions/66052/should-i-load-microtype-with-pdflatex
\usepackage{microtype}

% Indent the first section paragraphs
% https://tex.stackexchange.com/q/39227/119062
\usepackage{indentfirst}

% Specifies the directory where pictures are stored
\graphicspath{{pictures/}}



%----------------------------------------------------------------------------------------
%   Bad Boxes settings
%----------------------------------------------------------------------------------------

% Bad formatting using URLs in bibtex
% https://tex.stackexchange.com/questions/22888/bad-formatting-using-urls-in-bibtex
\usepackage{etoolbox}

% Underfull \hbox in bibliography
% https://tex.stackexchange.com/questions/10924/underfull-hbox-in-bibliography
\apptocmd{\thebibliography}{\raggedright}{}{}

% How to avoid overfull error with url package?
% See also the `\usepackage{url}` declarationon the file `basic.tex`.
% Set this to 2mu or 3mu if URL start troubling again.
% https://tex.stackexchange.com/questions/261776/how-to-avoid-overfull-error-with-url-package
\Urlmuskip=0mu plus 1mu

% Automatically put a `\medskip` spacing between paragraphs
% https://tex.stackexchange.com/questions/365976/how-to-stop-the-package-usepackageparskip-disabling-the-paragraph-indentation
% \edef\restoreparindent{\parindent=\the\parindent\relax}
% \usepackage{parskip}
% \restoreparindent
%
% Uncomment to remove all indentation from paragraphs
%\setlength\parindent{0pt}
%
% How to restore the parskip skips between list items?
% https://tex.stackexchange.com/questions/366848/how-to-restore-the-parskip-skips-between-list-items
\parskip=0.5\baselineskip \advance\parskip by 0pt plus 2pt



%----------------------------------------------------------------------------------------
%   DOCUMENT CONTENTS
%----------------------------------------------------------------------------------------

\begin{document}

% pdfTeX warning (ext4): destination with the same identifier (nam e{page.1}) has been already used, duplicate ignored
% https://tex.stackexchange.com/questions/18924/pdftex-warning-ext4-destination-with-the-same-identifier-nam-epage-1-has
\hypersetup{pageanchor=false}




\begin{titlepage}

    \center

    \Title{Universidade Federal de Santa Catarina - UFSC}

    \Title{Ciência Da Computação}

    \vspace*{\stretch{1}}

    \Author

    \vspace*{\stretch{2}}

    \Title{Boas Práticas de Programação \& Estilo}

    \bigskip
    Ferramentas Universais de Programação\\[3cm]

    \begin{flushright}

        \begin{minipage}{0.518\textwidth}

            Proposta de Trabalho de Conclusão de Curso,
            a ser submetido ao Curso de Ciência da Computação
            para a obtenção do Grau de Bacharel em Ciência da \lword{Computação}.

            \medskip
            {\bfseries Orientador:} \hfill \Advisor

            \medskip
            {\bfseries Professor Responsável:} \hfill \Supervisor

        \end{minipage}

    \end{flushright}

    \vspace*{\stretch{3}}

    Florianópolis, \today.

\end{titlepage}










\begin{abstract}

    Os softwares formatadores de código fonte atuais, também conhecidos como Beautifiers, são
    limitados a um conjunto similar, ou mesmo à uma única linguagem, e além de muitos, serem
    limitados ao que eles podem fazer por você ao processar/formatar o código \cite{Terence}.
    Portanto este trabalho tem como objetivo criar um formatador (software único) de fácil
    configuração e expansão capaz de abranger todas as linguagens de programação que existem,
    baseado em um uso específico de expressões regulares.

    \medskip
    A metodologia abordada será de não ter a necessidade de ter-se conhecimento da sintaxe das
    linguagens de programação que se irão fazer o parsing. Isso porque trataremos elas como texto
    comum, e será o usuário final que fará a configuração das transformações que serão aplicados no
    texto, dando liberdade de facilmente se configurar várias linguagens de programação (senão
    todas), aproveitando o fato de que muitas deles compartilham estruturas semelhantes senão
    idênticas.

    \medskip
    Como resultado espera-se ter um Beautifier Universal capaz de abranger todas as linguagens que
    existem, senão que seja facilmente extensível para abrange-las. Os pontos positivos dessa
    abordagem são a reusabilidade de componentes entre as linguagens. Por exemplo, `if/for/while's
    em C++ e Java são da mesma estrutura. Assim temos que escrever somente uma vez a especificação
    para um componente da linguagem.

    \bigskip
    \bigskip
    \textbf{Palavras-chave:}
    source, code, formatter, beautifier, prettyprint, universal, reuse, blocks, object, oriented,
    programming, structured, parsing, parse, regular, expression, regex, C, C++,  grammar, Turing,
    machine, automata, lexer, syntax, sublime, Java, Rust, shell, script, obfuscators, learning,
    syntec, teamicide, concensus, indent, settings.

\end{abstract}









\tableofcontents







% warning: destination with same identifier has been already used, duplicate ignored
% https://tex.stackexchange.com/questions/13083/pdftex-warning-destination-with-same-identifier-has-been-already-used-duplicat
\hypersetup{pageanchor=true}

% To automatically put a [Go To Top] on each section
\addGoToSummary



% The \phantomsection command is needed to create a link to a place in the document that is not a
% figure, equation, table, section, subsection, chapter, etc.
%
% When do I need to invoke \phantomsection?
% https://tex.stackexchange.com/questions/44088/when-do-i-need-to-invoke-phantomsection
\phantomsection


% Is it possible to keep my translation together with original text?
% https://tex.stackexchange.com/questions/5076/is-it-possible-to-keep-my-translation-together-with-original-text
\chapter{\lang{Introduction}{Introdução}}


\begin{englishtext}

    % TODO, put reference for this
    Questions like ``What are good programming practices?'' Or ``Why are these
    practices are good?''Are not easy to answer. But each programmer learns to
    write their codes in a certain way, with certain features like using 4 or 8
    spaces to indent lines, always leave a blank line before each control
    structure as if or for statements, and alike rules.

    % TODO, put reference for this
    But entering the universe of good practices, there are many things for
    discoursing. So in this work implementation tool called `Object Beautifier'
    specifically dedicates on how to perform the best layout/display of
    programming code on the computer screen, so that maximize and facilitate the
    understanding of same. Therefore, allowing the programmer to disperse more
    tempe thinking about its coding algorithms problem, other than trying to
    decipher the information that is presented to it on the screen through
    infinit different code layouts.

    % TODO, put reference for this
    Within this work\s area, we need to also think long and hard about how to
    share the programming code of the programmers among you. Now, the problem of
    human diversity, like all big scientific questions -- how do you explain
    something like that -- It can be broken down into sub-questions. It happens
    many times, which is a good practice for a `Programmer A', is not the same
    to another `Programmer B'. For example, imagine some code where a programmer
    decided to put before each `if' statement, a blank line. It is therefore
    expected that whenever we see a blank line we can potentially find a
    matching `if', which can be considered a quite useful pattern matching as
    empty line may call better your attention.

    % TODO, put reference for this
    But again this is something heavily dependent of what each one learning
    through their life time. Imagine another programmer do not liked this rule,
    and when he was writing your code involving an `if', he did not put such
    blank line another programmer is expecting. So when the first programmer
    start reading its the code and look for `if', he will be expecting for blank
    lines before its if\s. But will lose some time searching until realize
    another programmer does not put them, or perhaps he forgot to insert them.

    % TODO, put reference for this
    These differences are due to the diversity of ways we learn programming,
    i.e., to the ways we are used to doing coding, as much as the abilities and
    objectives of every programmer developed. Hence, nowadays it becomes a big
    problem because we increasingly need more and more programmers working
    together developing several and diverse computing systems. Where the latter
    is due to the fact of the complexity of computer systems growing
    increasingly, therefore over requiring programmers working and sharing their
    codes and ideas.

    % TODO, put reference for this
    Moreover besides only worrying about how the code is displayed on their
    computer screen, we need to worry about on how it will be saved in the file
    system on its `plain-text' mode. Since for code sharing, it is vital for you
    to use a versioning control system which enable project manager\s and
    programmers themselves, take control of their code changes. It does allow to
    easily perform the tracking of code changes and allow you to better
    understand what each programmer is doing every time he formalizes a change
    in the code through a `commit', as in `git' systems for example.

    % TODO, put reference for this (ref coding\_horror)
    That is because while working with a versioning system like `git', we need
    to keep the code among a single style or which we may call a `good practice'
    set as standard for everybody, due the fact of letting each programmer to
    write as he pleases, there will be plenty of noise on the code review and we
    are figuring out what actually each programmer did. Hence, if every
    programmer re-writes the history making changes like inserting new lines
    before each if, we end up with too much noise and focus of a versioning
    system is to look at only those changes that are significant to the code,
    such as the creation of new functions and not the addition of new blank
    lines (ref find\_some).

    % TODO, put reference for this
    Talking about the last thing pointed out, we could also think about an
    approach to creating a new version control system which focuses only on
    significant changes to the code, while reviewing code changes. However, this
    approach could not be ideal, as for example, it would allow programmers to
    start tedious wars of unproductive code adjustments. For example, imagine
    how it would be for your every day and have to go through your code
    re-adding new lines before each one of your beloved if\s, just because some
    night shift programmer had just removed them?


    \cite{softwarePortfolio}
    \cite{legacyAssets}
    \cite{massMaintenance}
    \cite{prettyPrinting}
    \cite{architectureFormatting}
    \cite{independentFramework}
    \cite{programIndentation}
    \cite{industrialApplication}

    \section{Goals}

    Establish relationships between good programming practices and efficiency in
    programming, in addition to a new tool to support programmers in order to
    automate the long and diverse programming process in teams of developers
    with different programming `best practices'. \cite{pushdownAutomata}


    \subsection{Specific Goals}

    \begin{enumerate}
        \item A study on universal programming tools, which from a single
        software, to work well behaved across all programming languages.
        Moreover, explain the differences for other softwares and the benefits
        of a unique tool, instead of several heavily different ones.

        \item Define, determine and classify which one are good programming
        practices and perform an in-depth study on the good practices on visual
        layout area, also known as code `Beautifying'.

        \item A study on the variety of existing tools for the support of good
        programming practices, beyond a comparative analysis between them,
        determining their weaknesses and strengths.

        \item The definition of a flow pattern of development allowing teams of
        developers with different programming best practices, to work without
        intervene with each other up to start wars of `best good practices'.

        \item Propose a unique tool that allowing several and distinct
        programming `best practices' being implemented in several programming
        languages, which can be configured and set accordingly to their wishes.
    \end{enumerate}



    \section{Search Method}

    The work will be based on research in articles, books, theses,
    dissertations, trusted authors websites, and through new demonstrated
    evidences based on arguments in the monograph evolution road. Also, present
    results after building a new tool which proposes a solution for the problems
    presented and detailed. \cite{aspectOriented}

    In this proposal last chapter which lies in the topic
    \autoref{sec:implementation}, there is a series of weblinks and references
    preselected and may be used in the release build of this work. Noticing the
    texts of the last section probably will end up gradually moved to the first
    section of the text where there is the theoretical research, while
    correlated research are incorporated in the main written work.
    \cite{aspectOrientationReview}

    Moreover, at the end of the first part of this work, the completion of the
    subject entitled of Course Conclusion Work 1, leaving only the information
    for the implementation of the proposed tool to be implemented in the second
    part of this named thesis on \nameref{sec:implementation}.



    \section{What does coding is?}

    Coding is like writing and reading a book for the large people, you like it
    to look beautifully. Or at least do you expect such when you buy a book, for
    example, to learn programming for you first time. You expect: % Reference to
    % book writing style/formatting articles

    \begin{enumerate}
        \item Things to be well organized, so you do not get lost.

        \item The colors to be properly placed, so you do not get distracted
           from the main content.

        \item The spacing between paragraphs, words, chapters, sections
           subsections, etc, to be well adjusted. Not everything cluttered in
           only one file, line, function, class, or whatsoever so.
    \end{enumerate}



    \section{Spaces and Tabs}

    The problem is that I will certainly not notice when I paste something
    indented with spaces instead of tabs. This is problem because for some file
    types as `.sublime-settings' files (or a Makefile), which has the setting
    `translate\_tabs\_to\_spaces' set to false, so I would expect to all
    `.sublime-settings' files to be indented with tabs, not spaces.
    \cite{tabsAndSpacesConversion}

    The setting `translate\_tabs\_to\_spaces' set to false works fine until I paste
    something on a setting's files which is indented with spaces, instead of
    tabs. This is a problem because as I am over git versioning, I can easily
    create files with mixed tabs and spaces on the history, and some day later
    Sublime Text will fix the indentation to tabs, which will cause noise on the
    git history due the tab/space conversion war. % Cite reference to the war

    I think this can have a performance problem as when I am pasting something
    big on Sublime Text. Then to perform the conversion on the would not be
    easily possible and it should be performed afterwards by the user. Now
    Sublime Text should warn the user when he is pasting something indented with
    spaces instead of tabs in a file which is expected to be indented with tabs?

    The detection of whether the contents of the clipboard should not be
    expensive as we should just check some lines (which would not cover the
    cases where there is already mixed indentation on the clipboard contents).
    But there would be a performance problem when pasting something with very
    big first lines. On this case a threshold should stop Sublime Text from
    looking forward and ceases the detection as inconclusive for this paste and
    just paste like it is.



    \subsection{Computer Assisted Programming}

    Your computer should help you with with these unforeseen tasks. Why should I
    spend my precious time checking whether I am actually copying something
    space indented, when I am actually coping something tab indented?

    Therefore, how to do such a thing on this 21\q{}st century? Perhaps we
    should sit and cry while waiting for some greater force to come and rescue
    us. Or may be you should stop crying and actually do something about other
    than keep waiting for you mommy to come and save you from the darkness
    growing behind you back leading you to endless unsleepy nights fixing your
    code just because everything just went wrong.



    \section{The Upper Stream}

    TODO.


    \subsection{How to keep up with the upstream}

    TODO.



    \section{Common Tasks}

    So you are developing a software which is under version control, however to
    deploy your tests, you need to copy some big folders into the deployment or
    testing system. Then how do you do it?

    Copying and pasting them probably the most straight forward idea, which is
    nice if you are going to it only a few times in a life time like two or
    three. However if you are going to do it move than these limits,
    please don\q t do that. It is bad for the planet and is worsening your
    health for nothing other than more headaches.

    As a promptly good computer user, at this point you already have some tool,
    either graphical or by command line which can help you easily and fastly
    setup the folder\q s. Easily like:

    \begin{enumerate}
        \item You open the tool
        \item Click on the new button
        \item Name your sync task as `My cuttie'
        \item Copy and paste there source and destine addresses
        \item Hit the `sync now' button
    \end{enumerate}



    \section{The rsync side}

    Doing everything out of the box by a graphical interface seems not
    practical. Command Line Interfaces (CLI) are simpler to be built and allows
    their programmers to saver their efforts in actually writing the tool
    instead of designing a reasonable Graphical User Interface (GUI)
    \cite{quantificationOfInterface}.

    GUI interfaces are awesome but for their proper usage, which is mostly
    defined by their aim public. Non-computer programmers, perhaps even novice
    programmers, cannot easily deal with command lines, but experienced
    programmers should be able to get great advantage from it usage.
    \cite{commandLineInterface}.

    Following we may see an example about the simpleness of a shell script,
    which runs several commands to accomplish a clean build of the testing
    environment:

    \begin{lstlisting}[caption={rebuild\_workspace.sh}]
    #!/bin/sh
    printf "$(date)\nRemoving folders...\n"

    rm -rf "Installed Packages"
    rm -rf "Lib"
    rm -rf "Local"
    rm -rf "Packages"

    printf "Unzipping files...\n"
    unzip -q "Packages.zip"

    mkdir -p "./Deployment/Code A"
    mkdir -p "./Deployment/Code B"

    printf "Syncing folders...\n"
    rsync -r \
         "/cygdrive/d/Development/Environment/Code A/" \
         "/cygdrive/c/Test/Deployment/Code A/"

    rsync -r \
         "/cygdrive/d/Development/Environment/Code B/" \
         "/cygdrive/c/Test/Deployment/Code B/"
    \end{lstlisting}
    \vspace*{-4mm}

    On preceding example, the `rebuild\_workspace.sh' script is located on the
    testing folder `/cygdrive/c/Test', then when calling it we get some folders
    removed, a file unpacked on the current folder, and our code synced from the
    versioning system directly to testing environment. You can read more about
    `rsync' utility on \citeonline{synchronizingFolders}.


\end{englishtext}


% Portuguese
\lang{}{

    Perguntas como ``O que são boas práticas de programação?'' ou ainda ``O por
    quê estas práticas são boas?'', não são fáceis de responder. Mas cada
    programador aprende a escrever seus códigos em uma determinada maneira, com
    determinadas características como utilizar 4 ou 8 espaços para indentação de
    linhas, sempre deixar uma linha em branco antes de cada estrutura de
    controle como if\s, for\s, e afins.

    Mas entrando o universo de boas práticas, há muitos coisas sobre discorrer.
    Assim neste trabalho especificamente trabalhá-se sobre como realizar a
    melhor disposição/exibição do código de programação na tela do computador,
    de modo que maximize e facilite o entendimento do mesmo. Portanto permitindo
    que o programador dispersa mais tempe pensando sobre o problema, do que
    tentar decifrar a informação que é apresentado para ele na tela.

    Dentro desta área de trabalho, precisa-se também pensar muito bem sobre como
    compartilhar os códigos de programação dos programadores entre si. Isso por
    que entra agora o problema da diversidade de boas práticas de programação.
    Ela acontece por que muitas vezes, aquilo que é uma boa prática para um
    `programador A', não é para o outro `programador B'. Por exemplo, imagine um
    código onde um programador decidiu colocar antes de cada `if', uma linha em
    branco. Portanto é de se esperar que sempre que vemos uma linha em branco
    nos podemos potencialmente encontrar um `if'. Entretanto imagine que outro
    programador não gostou dessa regra e quando ele foi escrever seu código que
    envolvia um `if', ele não colocou a essa tal linha em branco que o outro
    programador vinha colocando. Então quando o primeiro programador for ler o
    código e procurar por `if'es, ele vai estar esperando por linhas em branco.
    Mas vai perder algum tempo procurando até perceber que o outro programador
    não as colocou.

    Essas diferenças dão-se devido a diversidade de meios de se aprender
    programação, tanto quanto aos gostos, aptidões e objetivos de cada
    programador. Assim hoje em dia isso torna-se um grande problema por que cada
    vez mais precisamos de mais e mais programadores trabalhem juntos entre si,
    desenvolvendo os mais diversos sistemas computações. Onde este último
    deve-se ao fato de que a complexidade dos sistemas computacionais cresce
    cada vez mais, portanto requer-se que mais e mais programadores trabalhem e
    compartilhem códigos.

    Então além de nos preocupar-mos somente como o código é exibido na tela do
    computador, nós precisamos nos preocupar sobre como ele será salvo no
    sistema de arquivos. Já que ao compartilhar o código, é vital o uso de um
    sistema de versionamento para permitir a gerências de projetos e os
    programadores em si, terem o controle de mudanças do código. O que permiti e
    facilmente possa realizar o rastreamento de mudanças e permitir que se possa
    entender melhor o que cada programador está fazendo a cada vez que ele
    formaliza um mudança no código através de uma `commit', como no sistemas
    `git` por exemplo.

    Isso por que quando trabalhos em um sistema de versionamento como `git'
    precisamos manter o código dentre um único estilo ou boa prática definida
    como padrão, devido ao fato de que se deixar-mos cada programador escrever
    como ele quiser, teremos muito ruído durante a revisão do código e estamos
    determinando o que o programador fez/escreveu, se cada programador
    re-escreve o histórico fazendo alterações como colocar linhas novas antes de
    cada if. Assim teremos ruído por que o foco de um sistema de versionamento é
    olhar somente as mudanças que são significativas para o código, como a
    criação de novas funções e não a adição de novas linhas em branco.

    Sobre o último ponto, podemos pensar também sobre uma abordagem da criação
    de um novo sistema de versão que foque somente nas mudanças significativas
    para o código, durante o momento da revisão. Entretanto essa abordagem não é
    ideal por que, por exemplo, ela dá margem para que programadores entrem em
    guerras tediantes e não produtivas de ajustes de código. Por exemplo,
    imagine o quão seria todo dia que você acorda e começa a trabalhar, você tem
    que passar pelo código colocando linhas novas antes de cada um dos if\s por
    que o programador do turno da noite tinha acabado de remover eles?


\section{Objetivos}

    Estabelecer relações entre boas práticas de programação e eficiência em
    programar, além de uma nova ferramenta ao apoio do programador com o intuito
    de automatizar o longo e diverso processo de programação em equipes de
    desenvolvedores com distintas boas práticas de programação.


\subsection{Objetivos específicos}

    \begin{enumerate}

        \item

        Um estudo sobre ferramentas universais de programação, que permitam que
        a partir de um único software, seja programado em todas as linguagens de
        programação. Assim explicar as diferenças para os outros softwares e os
        porquês de querer-se uma ferramenta única, ao invés de diversas.

        \item

        Definir, estudar, determinar e classificar o que são boas práticas de
        programação e realizar um estudo aprofundado sobre a as boas práticas da
        área de disposição visual, conhecidas também como `Beautifying'.

        \item

        Um estudo sobre as mais diversas ferramentas existentes para o apoio de
        boas práticas de programação, além de uma análise comparativa entre
        elas, determinando suas fraquezas e pontos fortes.

        \item

        A definição de um padrão de floxo de desenvolvimento que permita equipes
        de programadores com distintas boas práticas de programação, trabalhem
        em si sem intervir e iniciar guerras de boas práticas.

        \item

        Propor uma ferramenta única que permita diversas e distintas boas
        práticas de programação serem implementadas nas mais diversas linguagens
        de programação e que elas possam ser configuradas e definidas ao gosto
        dos programadores que a usa.

    \end{enumerate}


\section{Método de pesquisa}

    O trabalho será baseado em pesquisas em artigos, livros, teses,
    dissertações, sites de autores confiáveis, e por meio de novas provas
    demonstradas e baseadas através de argumentos no decorrer da evolução da
    monografia. Também sera apresentado os resultados decorridos da construção
    de uma nova ferramenta que proprõe a solução de um dos problemas
    apresentados e explicados.

    No último capítulo desta proposta encontra-se no tópico
    \autoref{sec:implementation} encontra-se uma série de links e referências
    que forma pré-selecionadas e poderão ser utilizadas na construção final
    deste trabalho. Notes que em si, as partes da última seção serão
    gradativamente movida para primeira parte do texto onde encontra-se pesquisa
    teórica, no decorrer que suas informações correlacionadas são incorporadas
    no trabalho escrito.

    Assim no final da primeira parte desta obra que dará-se no final da
    conclusão da disciplina intitulada de Trabalho de Conclusão de Curso 1,
    restarão somente as informações destinadas a implementação da ferramenta
    proposta, que serão implementadas na segunda parte da monografia denominada
    \nameref{sec:implementation}, que será desenvolvida no final da conclusão da
    disciplina de Trabalho de Conclusão de Curso 2.


}






\section{Cronograma}

    \begin{adjustwidth}{-0.5\marginparwidth}{-0.5\marginparwidth}
    \small
    \begin{tabularx}{\linewidth}{|BX|*{11}{^c|}}

        \hline
        \multicolumn{1}{|c|}{\multirow{2}{*}{{\bfseries Etapas}}} & \multicolumn{11}{|c|}{{\bfseries Meses}} \\
        \cline{2-12}

        \rowstyle{\bfseries}
        & ago & set & out & dez & jan & fev & mar & abr & mai & jun & jul   \\ \hline

        Escrita da revisão bibliográfica
        &  x  &  x  &     &     &     &     &     &     &     &     &     \\ \hline

        Classificar todas classes e tipos de formatações
        &     &  x  &  x  &     &     &     &     &     &     &     &     \\ \hline

        Implementação de um núcleo funcional
        &     &     &  x  &  x  &     &     &     &     &     &     &     \\ \hline

        Finalização da escrita do TCC
        &     &     &     &  x  &  x  &     &     &     &     &     &     \\ \hline

        Ajustes finais no texto do TCC
        &     &     &     &     &  x  &  x  &  x  &  x  &     &     &     \\ \hline

        Defesa do TCC
        &     &     &     &     &     &     &     &     &  x  &  x  &     \\ \hline

    \end{tabularx}
    \end{adjustwidth}

    \hfill \cite{Silva}


\section{Custos}

    % How to align a vertical line at the end of the multicolumn in a table?
    % https://tex.stackexchange.com/questions/367075/how-to-align-a-vertical-line-at-the-end-of-the-multicolumn-in-a-table
    \begin{tabular}
    {|
        *1{@{\hspace{3.0pt}}>{ \RaggedRight\arraybackslash\hsize=1.1\hsize }Bp{3.9cm}|} % Item
        *1{@{\hspace{3.0pt}}>{ \RaggedRight\arraybackslash\hsize=1.1\hsize }^p{2.0cm}|} % Quantidade
        *1{@{\hspace{3.0pt}}>{ \RaggedRight\arraybackslash\hsize=1.1\hsize }^p{3.2cm}|} % Valor, Valor
        *1{@{\hspace{3.0pt}}>{ \RaggedRight\arraybackslash\hsize=1.1\hsize }^p{2.8cm}|} % Valor, Valor
    }

        \hline
        \rowstyle{\bfseries}
        Item                    &   Quantidade  &   Valor Unitário (R\$)    &   Valor Total (R\$) \\ \hline
        CD                      &   1           &   5,00                    &   5,00              \\ \hline
        Impressão               &   800         &   0,15                    &   120,00            \\ \hline
        Reserva Gerencial       &   1           &   20,00                   &   20,00             \\ \hline
        Reserva de Contingência &   1           &   20,00                   &   20,00             \\ \hline
        Total                   & \multicolumn{2}{c|@{\hspace{3.0pt}}}{}    &   165,00            \\ \hline

    \end{tabular}

    \medskip
    \hfill \cite{Silva}


\section{Recursos Humanos}

    \begin{tabular}{|Bl|^l|}

        \hline
        \rowstyle{\bfseries}
        Nome                            & Função                  \\ \hline
        \Author                         & Autor                   \\ \hline
        \Advisor                        & Orientador              \\ \hline
        Renato Cislaghi                 & Coordenador de Projetos \\ \hline
        \Supervisor                     & Professor Responsável   \\ \hline

    \end{tabular}

    \medskip
    \hfill \cite{Silva}


\section{Comunicação}

    \begin{tabular}{|Bl|^l|^l|^l|}

        \hline
        \rowstyle{\bfseries}
        O quê  & De quem & Para Quem & Como                                        \\ \hline
        Proposta de TCC         & Autor     & Renato Cislaghi   & Site de projetos \\ \hline
        Relatório de TCC I      & Autor     & Renato Cislaghi   & Site de projetos \\ \hline
        Prévia do TCC, em TCC I & Autor     & Banca             & E-mail           \\ \hline
        Defesa do TCC           & Autor     & Banca             & Pessoalmente     \\ \hline
        Reunião de Orientação   & Orientadores  & Autor         & Pessoalmente     \\ \hline

    \end{tabular}

    \medskip
    \hfill \cite{Silva}


\section{Riscos}

    % https://tex.stackexchange.com/questions/366156/how-to-change-the-left-padding-for-one-latex-tables-cell
    % https://tex.stackexchange.com/questions/366155/how-to-write-a-table-a-little-larger-than-the-paragraphs-with-centered-columns
    %
    \begin{adjustwidth}{-0.5\marginparwidth}{-0.5\marginparwidth}
    \small
    \begin{tabularx}{\linewidth}
    {|
        *1{                 >{\RaggedRight\arraybackslash\hsize=1.1\hsize }BX       |} % Riscos
        *3{@{\hspace{3.0pt}}>{\Centering\arraybackslash                   }^p{0.9cm}|} % Probabilidade, Impacto, Prioridade
        *2{                 >{\RaggedRight\arraybackslash\hsize=0.95\hsize}^X       |} % Resposta, Prevenção
    }

    \hline

    \rowstyle{\bfseries}
    Riscos  & a & b & c & Estratégia de resposta & Ações de prevenção \\ \hline

    % Row 1
    % Riscos
    Problemas com perda de dados &
    % Probabilidade
    Baixa &
    % Impacto
    Alto &
    % Prioridade
    Alta &
    % Estratégia de resposta
    Uso do backup &
    % Ações de prevenção
    Backup periódicos \\ \hline

    % Row 2
    % Riscos
    Alteração do cronograma ou descontinuidade do projeto onde recebo uma bolsa &
    % Probabilidade
    \rlap{Média} &
    % Impacto
    Alto &
    % Prioridade
    Alta &
    % Estratégia de resposta
    Redefinição da data de entrega do trabalho &
    % Ações de prevenção
    Monitoramento contínuo das informações obtidas com superiores imediatos \\ \hline

    \end{tabularx}

    \hfill {\small {\bfseries a}: Probabilidade, {\bfseries b}: Impacto, {\bfseries c}: Prioridade}

    \end{adjustwidth}

    \vspace*{-4mm}
    \hfill\cite{Silva}






% The \phantomsection command is needed to create a link to a place in the document that is not a
% figure, equation, table, section, subsection, chapter, etc.
%
% When do I need to invoke \phantomsection?
% https://tex.stackexchange.com/questions/44088/when-do-i-need-to-invoke-phantomsection
\phantomsection


% Is it possible to keep my translation together with original text?
% https://tex.stackexchange.com/questions/5076/is-it-possible-to-keep-my-translation-together-with-original-text
\chapter{\lang{Implementation}{Implementação}}\label{sec:implementation}


\begin{englishtext}

    This work aims to create a formatter (single software) of easy configuration
    and expansion capable of encompassing the programming languages that exist,
    based on a specific use of regular expressions. The strategy is not to be
    relaying on the syntax of the programming languages parsed. With it in mind,
    we will treat them as a common text, and it will be up to the end user to
    configure the transformations/formatting to be applied in the text, giving
    reasonably more freedom to easily configure several programming languages
    (if not them all), taking advantage of the fact which many of them share
    similar structures if not identical.

    As a result, it is expected to have a Universal Beautifier capable of
    covering existing languages, and which is easily extendable to cover new
    languages. The key points of this approach are the reusability of components
    between languages. For example, `if/for/while'\s in C++ and Java have of the
    same structure, therefore accordingly to this approach we have to write only
    once the specification for that language component.

    Such software ideia may to a certain extent may continue a branch of the of
    the student `Lucas Boppre Niehues'\s graduation thesis, oriented by the
    Professor `Olinto José Varela Furtado', entitled 'Study and Creating a
    Structured Code Editor' defended in 2013/1 \cite{structuredEditorStudy}.
    While reading its work, the following section that links to one of the
    proposals of this work, in the chapter: `8.1.2 Separation of display and
    output format\footnote{Translation of: \brazilword{Separação de formato de
    exibição e de saída}}':

    \begin{citacao}
    The ways that code is displayed to the user and that it is saved to disk are
    controlled by different configuration files. The `theme.ini' file contains,
    among other settings, information on how to serialize the syntax tree.
    \cite[our translation]{structuredEditorStudy} \footnote{\brazilword{As
    formas que o código é exibido ao usuário e que ele é salvo em disco são
    controladas por arquivos de configuração distintos. O arquivo `theme.ini'
    contém, entre outras configurações, informações de como serializar a árvore
    sintática.}}
    \end{citacao}
    \begin{citacao}
    The output format setting is given the same way, but in a separate file,
    called `output \_format.ini'. The decision of this separation was in view of
    teams of programmers who want to use a single convention for the saved
    files, but keep the display the choice of each. So members of this team can
    share their `output\_format.ini 'files while customizing the `theme.ini'
    file to their liking. \cite[our translation]{structuredEditorStudy}
    \footnote{\brazilword{A configuração de formato de saída é dada da mesma
    forma, mas em um arquivo separado, chamado `output\_format.ini'. A decisão
    desta separação foi em vista de equipes de programadores que queiram
    utilizar uma convenção única para os arquivos salvos, mas manter a exibição
    a escolha de cada um. Assim os integrantes desta equipe podem compartilhar
    os seus arquivos `output\_format.ini' enquanto personalizam o arquivo
    `theme.ini' a seu gosto.}}
    \end{citacao}

    Based on this, we may think about writing plugins for common text
    editors/IDEs as Sublime Text. So while loading files from the disk, such
    plugin calls the formatter and does the formatting according to the display
    settings for the user. After that, when the user saves the file, the file
    with the original formatting is returned.

    To assist in this process, an autoconfiguration module is a great help. It
    detects how the source code is formatted and creates configuration files for
    it. So when saving the file, the file is automatically saved in the
    formatting it was originally read from the file system. So we have the same
    benefit of structured editors as the proposed work of
    \textcite{structuredEditorStudy}. At first we can think with the following
    goal/idea for a new automated software tool:

    \medskip
    \begin{bluebox}
    \begin{enumerate}[nolistsep]
        \item Create an easy-to-configure and expandable formatter for the
        existing and coming up programming languages.
    \end{enumerate}
    \end{bluebox}



\subsection{The Problem}

    The problem proposed to solve is the creation of a Universal Beautifier.
    Current softwares are limited to a similar/restricted set, or even a single
    language and in addition, many are limited in what they can do for you when
    processing/formatting/beautifying your source code.
    \cite{universalCodeFormatter}

    Below there are some basic formatting rules for illustration:

    \medskip
    \begin{bluebox}
    \begin{enumerate}[nosep,nolistsep]
        \item Add new lines after `\{' and before `\}
        \item Add new lines before `\{'
        \item Remove empty lines
        \item Add comment lines before function
        \item Add new lines after `;'
        \item Add new lines after `\}'
        \item Remove new lines
        \item Reduce whitespace
        \item Put the code again in the input box above after submit
    \end{enumerate}
    \end{bluebox}
    \vspace{-4mm}\begin{flushright}\textcite{prettyPrinter}\end{flushright}

    From this point, a sketch is presented on the problem, solutions,
    information as for why to want make such software, or even why do we want to
    beautifying things:

    \begin{enumerate}[leftmargin=*]

        \item There are many different tools, sometimes paid, and difficult to
        complete. \cite{universalCodeFormatter}

        \item Many programming languages exist, so always having Beautifier
        software for each of them is very laborious
        \cite{universalCodeFormatter}. But the approach to a Universal
        Beautifier proposed in this work, would allow easily new languages to be
        added, being completely different from previous ones, or alike. And in
        case of similarities between them, it is enough to reuse their
        configuration structures already implemented.

        \item Looking for a Beautifier for each one of them because programmers
        currently work daily with several of these languages, and they are not
        similar. So you need to configure several beautifiers to do the
        formatting. This is a problem because only a few beautifiers are more
        complete, and every time you need to make a change in the formatting
        style, you must manually propagate the same change over several
        different program configuration files, which is bad because it takes the
        user a lot of time to learn how to handle many different types of
        settings. \cite{Schweitzer}

        \item In the case of ideal Beautifier, a change in your styling is
        propagated to all languages. And if you want to leave some language out
        of it, you just need to remove it from the list on which the
        configuration block applies to, and `a)' leave it out so no change is
        applied to. Or `b)' create a new block including only the block within
        the desired settings.

        \item
        \begin{citacao}
        % \setlength{\itemindent}{5pt}
        One of absolute worst, worst methods of teamicide for software developers is to engage
        in these kinds of passive-aggressive formatting wars. I know because I've been there.
        They destroy peer relationships, and depending on the type of formatting, can also damage
        your ability to effectively compare revisions in source control, which is really scary.
        I can't even imagine how bad it would get if the lead was guilty of this behavior. That's
        leading by example, all right. Bad example. \cite{Atwood}
        \end{citacao}

        \item
        \begin{citacao}
        So yes, absurd as it may sound, fighting over whitespace characters and other seemingly
        trivial issues of code layout is actually justified. Within reason of course -- when done
        openly, in a fair and concensus building way, and without stabbing your teammates in the
        face along the way. \cite{Atwood}
        \end{citacao}

        \item
        \begin{citacao}
        I'd say there are two main reasons to enforce a single code format in a project. First has
        to do with version control: with everybody formatting the code identically, all changes in
        the files are guaranteed to be meaningful. No more just adding or removing a space here or
        there, let alone reformatting an entire file as a `side effect' of actually changing just a
        line or two. \cite{Geukens}
        \end{citacao}

    \end{enumerate}



\subsection{Goals}

    The object in the proposed thesis is not initially to support all the
    formatting rules for all programming languages, but the creation of an
    initial and robust basic formatter capable of being developed to the point
    of being easily expanded by programmers with new processing modules and by
    end users writing the formatting configuration files.

    The theory of the technique proposed is quite simple, but different from the
    usual approach because the end user is assigned the responsibility to
    dictate more precisely how the beautifying being configured will be
    performed. Such tradeoff is the price to pay to allow the creation of a
    Universal Beautifier. But notice when it is said, it is supposed to be easy
    to set up, it is meant to not be required to do `C++' programming, i.e.
    change the source code of the program to set/specify where the beautifying
    changes are supposed to be performed.


\subsubsection{General Goals}

    \begin{enumerate}[leftmargin=*]

        \item Write the program in C++ for greater performance, allowing the
        formating/beautifying to be dynamic, i.e., as you type the text, the
        text is formatted for you. So you can focus more on writing the code,
        rather than worrying about spacing, alignment, parentheses, new lines,
        and whatever else.

        \item

        Utilizar o Framework `doctest` para escrita dos Testes de Unidade. Pois após procurar e
        testar alguns frameworks para testes de unidade em C++, entrou-se este como servindo muito
        bem as requisitos do projecto. Ele causa baixíssimo incremento no tempo de compilação e
        permite que os testes possam ser escritos no mesmo arquivo onde encontram-se o código do
        programa, sem que eles sejam compilados.

        \item

        Utilizar uma versão/algoritmo multi-core, então cada uma das regras pode ser processada em
        paralelo e sobre o mesmo source code original. Essa parte é bastante complexa de ser escrita
        por que as regras entre si podem gerar conflitos sobre o que elas estão fazendo. Para
        resolver esse problema, fazer com que cada regra processada gere um objeto de mudanças que
        essa regra está propondo. No final do processamento de todas as regras, será realizado um
        fusão das mudanças que cada uma decidiu realizer, e caso duas regras queriam mudar o mesmo
        pedaço/trecho de código, será lançada um exceção e uma nova classe de mudanças/regra deve
        estar disponível para resolver esse conflito. Caso não exista, ambas as mudanças são
        descartadas e somente as mudanças sem conflitos são refletidas no código.

    \end{enumerate}


\subsubsection{Focused Goals}


    \begin{enumerate}[leftmargin=*]

        \item

        Um Produto de Software com uma ótima orientação a objetos e possibilidades de extensão das
        funcionalidades.

        \item

        Classificar todas classes e tipos de formatações (beautifying) de código aplicáveis com
        facilidade. Uma das partes a serem escritas e entregues na monografia. Um estudo sobre o que
        é beautifying, como fazer e por que fazer.

        \item

        Implementação de um núcleo funcional e de uma pesquisa decente sobre o estado da arte. Um
        dos pontos difíceis seria a marcação dos escopos, mas isso já é implementado pelo núcleo do
        editor Sublime Text, assim provado como possível de ser feito.

        \item

        Inicialmente devido a limitação de tempo em 1 ano e meio para um TCC, podemos pensar somente
        um núcleo básico, simples, reutilizável e que talvez possa ajudar no contexto da linguagem
        que vocês desenvolvem.

    \end{enumerate}


\subsubsection{Future Works}

    O número de recursos/funcionalidades e estratégias de otimizações para serem implementadas, e
    etc, são imensas. Mas esses trabalhos podem ser muito mais para frente depois da entrega do TCC.
    Hoje o controle de espaços em chamadas de funções, declarações de classes, comentários e etc,
    são mais tranquilos de se entender e pensar. Entretanto no requisito e ajuste de indentação,
    inserção/remoção de parenteses redundantes, etc ainda falta estudo sobre como deve ser
    implementado isso.

    Contudo essa especificação por parte do usuário é limitado a linguagens Livres de Contexto
    (máquinas de pilha). Assim caso as especificações de escope precisarem ser feitas em termos de
    linguagens Sensíveis ao Contexto ou ainda Recursivamente Enumeráveis, vai ser preciso tratar
    esses elementos diretamente em C++ (máquina de turing).

    Entretanto não consegue-se pensar facilmente em casos em que precise mais do que tratadores
    Livres de Contexto para realizar a especificação de quais partes do código deve ser necessário
    formatar. Sublime Text faz uso dessa técnica para o Highlight dos códigos das mais diversas
    linguagens e acredita-se que tenha um bom resultado.


\subsection{Research Methods}

    A vantagem nesta abordagem é não ter a necessidade de ter-se conhecimento da sintaxe das
    linguagens de programação que se irão fazer o parsing. Isso porque trataremos elas como texto
    comum, e será o usuário final que fará a configuração das transformações que serão aplicados no
    texto, dando liberdade de facilmente se configurar várias linguagens de programação,
    aproveitando o fato de que muitas deles compartilham estruturas semelhantes senão idênticas.

    A literatura/programas atuais são dependentes de linguagem de programação. Minha proposta é
    fazer este processo independente de linguagem, mas de dialetos como este exemplo tirado do PDF
    em anexo a este e-mail `Initial check list tasks to do.pdf':

    \begin{lstlisting}
    // This is the name used to reference this scope around the settings files.
    Scope Name:
    %c++_like_block_comment

    // This set on which languages this block should be included. Setting it
    // to empty will allow it to be parsed for any languages.
    Language Inclusion:
    Java, C++, Pawn

    // Defines a expression which will map the beginning of a exclusion block.
    Scope Start:
    /\*\*

    // Defines a expression which will map the ending of a exclusion block.
    Scope End:
    \\\*
    \end{lstlisting}
    \vspace*{-4mm}

    A abordagem acima é uma abordagem ingênua, portanto somente brevemente ilustrativa. O real motor
    para o software é baseado em expressões regulares e um pilha de contextos. Esta ideia foi
    inicialmente desenvolvida pelo editor de texto `Sublime Text' \cite{Skinner}. Este editor
    utiliza essa estrutura de blocos para fazer a sintaxe highlighting do códigos das linguagens
    através de expressões regulares alocação de contextos/escopos. Essa mesma abordagem pode ser
    utilizada pelo usuário para definir em quais regiões uma Máquina de Turing (linguagens C++/Rust)
    devem fazer/propor as alterações no código.


\subsubsection{Ideas}

    Os pontos positivos dessa abordagem para um formatador de código são a reusabilidade de
    componentes entre as linguagens pelo usuário final da aplicação ao invés do programador, o que
    torna este software muito mais genérico e abre a possibilidades de maior sucesso para a criação
    definitiva de um formatador Universal de códigos das linguagens de programação, quaisquer sejam
    elas. Por exemplo, `if/for/while'\textquotesingle s em linguagens de programação como C++ e Java
    são da mesma estrutura. Assim temos que escrever somente uma vez a especificação para um
    componente da linguagem sem recorrer a programação de do código do programa. Isso tem a vantagem
    de por der ser configurado pelo usuário final ao invés do programador, assim fica mais simples
    de configurar e expandir o conjunto de linguagens disponíveis ao processamento/beautifying.

    Ideas for implementation:

\medskip
\begin{bluebox}
\begin{enumerate}[leftmargin=*]

    \item Implement tabstops with white space align. The solution - move
    tabstops to fit the text between them and align them with matching tabstops
    on adjacent lines. \url{http://nickgravgaard.com/elastic-tabstops/}
    \url{https://forum.sublimetext.com/t/elastic-tabs/128}

\end{enumerate}
\end{bluebox}


\subsubsection{Listings}

    Algumas bibliotecas existentes, e potencialmente utilizadas como `syntect` para o auxílio na
    construção do produto de software:

    \begin{bluebox}
    \begin{enumerate}[leftmargin=*,parsep=0pt]

        \item \url{https://github.com/jbeder/yaml-cpp}
        \item \url{https://github.com/trishume/syntect}
        \item \url{https://github.com/onqtam/doctest}
        \item \url{https://github.com/c42f/tinyformat}
        \item \url{https://github.com/limetext/lime}
        \item \url{https://forum.sublimetext.com/t/disassembling-sublime-text/24824}

    \end{enumerate}
    \end{bluebox}

    Segue-se uma lista básica de formatters/beautifiers acessado no endereço
    \lword{\url{http://www.softpanorama.org/Utilities/beautifiers.shtml}} em março/2017:

    \medskip
    \begin{sloppypar}
    \begin{bluebox}\RaggedRight
    \begin{enumerate}[leftmargin=*,parsep=0pt]

        \item CB210.ZIP - C Beautifier 2.10 - polish C source code (19,406 bytes, 06/22/92)
        \item CL121.ZIP - Codelister 1.21 - print C code with stats (51,110 bytes, 01/10/94)

        \item CPC200.ZIP - CodePrint for C/C++ 2.00 is a full-featured command line driven source
        code reformatter and pretty printer for C++ and C; over 20 customization features including
        auto-indent, adjustable tab spacing, indent styles, flow lines, comment alignment, and line
        editing for consistent white space (140,605 bytes, 01/26/96)

        \item CSCOP120.ZIP - c-scope 1.20 analyzes C source code and produces various reports
        (48,505 bytes, 06/30/95)

        \item HTML : \url{http://www.digital-mines.com/htb/}
        \item HTML : \url{http://www.datacomm.ch/mwoog/software/perl/beautifier.html}
        \item HTML : \url{http://www.watson-net.com/free/perl/s_fhtml.asp}
        \item SQL : \url{http://www.netbula.com/products/sqlb}
        \item Oracle PLSQL : \url{http://www.revealnet.com}
        \item GPL \url{http://www.geocities.com/~starkville/vancbj.html}
        \item GPL \url{http://kevinkelley.mystarband.net/java/dent.html}
        \item Free \url{http://www.tiobe.com/jacobe.htm}
        \item Free \url{http://www.mmsindia.com/JPretty.html}
        \item Free \url{http://members.magnet.at/johann.langhofer/products/jxbeauty/overview.html} (has JBuilder support)
        \item Free \url{http://www.semdesigns.com/Products/Formatters/JavaFormatter.html}
        \item Commercial \$24.99 \url{http://smartbeautify.com}
        \item Commercial \$129 \url{http://www.jindent.com}
        \item Google \url{http://directory.google.com/Top/Computers/Programming/Languages/Java/Development_Tools/Code_Beautifiers/?tc=1}
        \item Java, SQL, HTML, C++ : \url{http://www.semdesigns.com/Products/DMS/DMSToolkit.html}
        \item Java JIndent \url{http://home.wtal.de/software-solutions/jindent}
        \item Java Pat \url{http://javaregex.com/cgi-bin/pat/jbeaut.asp}
        \item Java JStyle \url{http://www.redrival.com/greenrd/java/jstyle}
        \item Java JPrettyPrinter \url{http://www.epoch.com.tw/download/ms/java/java.htm}
        \item Java JxBeauty \url{http://members.nextra.at/johann.langhofer/download/jxbeauty} and the JxBeauty Home
        \item Java beautify percolator
        \item Java list \url{http://www.java.about.com/compute/java/library/weekly/aa102499.htm}
        \item Java html present VasJava2HTML
        \item Java code colorifier and beautifier \url{http://www.mycgiserver.com/~lisali/jccb}
        \item Perl : \url{http://www.consultix-inc.com/www.consultix-inc.com/talk.htm}
        \item Perl : \url{http://www.consultix-inc.com/www.consultix-inc.com/perl_beautifier.html}
        \item Fortran beautifier : \url{http://www.aeem.iastate.edu/Fortran/tools.html}

        \item C++ : BCPP site is at \url{http://dickey.his.com/bcpp/bcpp.html} or at \url{http://www.clark.net/pub/dickey}.
        BCPP ftp site is at \url{ftp://dickey.his.com/bcpp/bcpp.tar.gz}

        \item C++ : \url{http://www.consultix-inc.com/c++b.html}
        \item C : \url{http://www.chips.navy.mil/oasys/c/} and mirror at Oasys
        \item C++, C, Java, Oracle Pro-C Beautifier \url{http://www.geocities.com/~starkville/main.html}

        \item C++, C beautifier \url{http://users.erols.com/astronaut/vim/ccb-1.07.tar.gz} and site at
        \url{http://users.erols.com/astronaut/vim/#vimlinks_src}

        \item GC! GreatCode! is a powerful C/C++ source code beautifier Windows 95/98/NT/2000
        \url{http://perso.club-internet.fr/cbeaudet}

        \item C++ beautifier `SourceStyler' \url{https://web.archive.org/web/20061205061102/http://ochresoftware.com/}
        \item JavaScript : \url{http://jsbeautifier.org/}

    \end{enumerate}
    \end{bluebox}
    \end{sloppypar}


\subsubsection{Related Works}

    Após a busca do que há de publicações científicas sobre o assunto e entra-se alguns trabalhos na
    área específica e similar aos trabalhos feitos pelor formatadores de códigos (Beautifiers).
    Nessa modalidade de trabalho, pode-se confundir-se com artigos que tratam sobre o `Prettyprint`,
    que trata-se de colorir o texto e exibir-lo ao usuário. O que não é o que se busca nesse
    trabalho, mas sim fazer alterações no texto sobre a forma como ele é estruturado, apresentado ao
    usuário e salvo em disco. Seguem as seguintes publicações:

% How to add `parsep` to `itemsep` and set `parsep` to 0pt, when declaring my list?
% https://tex.stackexchange.com/questions/366904/how-to-add-parsep-to-itemsep-and-set-parsep-to-0pt-when-declaring-my-list
\begin{sloppypar}
\begin{bluebox}\RaggedRight
\begin{enumerate}[leftmargin=*,parsep=0pt]

    \item CodeBeautify is an online code beautifier which allows you to beautify
    your source code: \url{http://codebeautify.org/}.

    \item A universal code formatter, written in Dart:
    \url{https://pub.dartlang.org/packages/unifmt}.

    \item Google-java-format is a program that reformats Java source code to
    comply with Google Java Style:
    \url{https://github.com/google/google-java-format}.

    \item CodeFormatter is a Sublime Text 2/3 plugin that supports format
    (beautify) source code.
    \url{https://github.com/akalongman/sublimetext-codeformatter} and
    \url{https://github.com/aukaost/SublimePrettyYAML}

    \item UniversalIndentGUI offers a live preview for setting the parameters of
    nearly any indenter. You change the value of a parameter and directly see
    how your reformatted code will look like. Save your beauty looking code or
    create an anywhere usable batch/shell script to reformat whole directories
    or just one file even out of the editor of your choice that supports
    external tool calls: \url{http://universalindent.sourceforge.net/} and
    \url{https://github.com/danblakemore/universal-indent-gui}.

    \item Language-agnostic pretty-printing through machine learning (uh, like,
    is this possible? YES, apparently). By Terence Parr (primary developer),
    Fangzhou (Morgan) Zhang (help with initial development), Jurgen Vinju
    (co-author of academic paper, help with empirical results and algorithm
    discussions). \url{https://github.com/antlr/codebuff}

    \item To every developer in this world, the closest thing to their heart is
    the text editor of their choice. Over the last few years many new text
    editors has come into the market in both free and paid model, but
    unfortunately not all of them were able to make a real dent on the developer
    community. I remember in my college days we uses to use Notepad++ as our
    beloved text editor, as at that point of time it was one of the popular and
    free text editor with a lot of features for coding. But as time goes on, the
    entire development community started to lean towards sublime text since it’s
    launch.
    \url{https://www.isaumya.com/sublime-text-vs-atom-which-one-i-prefer-most-and-why/}

    \item As a developer, your code editor is one of the most important parts of
    your setup. It can save your wrists and fingers from repetitive strain
    injuries. It can save your eyes from going blind after a coding marathon.
    \url{https://hackernoon.com/virtualstudio-code-the-editor-i-didnt-think-i-needed-16970c8356d5}

    \item VS Code is an Editor while VS is an IDE.
    \url{https://stackoverflow.com/questions/30527522/what-are-the-differences-between-visual-studio-code-and-visual-studio}

    \item What is the difference between VS Code and VS Community?
    Visual Studio Code is a streamlined code editor with support for development operations like
    debugging, task running and version control. It aims to provide just the tools a developer needs
    for a quick code-build-debug cycle and leaves more complex workflows to fuller featured IDEs.
    For more details about the goals of VS Code, see Why VS Code.
    \url{https://code.visualstudio.com/docs/supporting/faq#_licensing}

    \item Reg Replace is a plugin for Sublime Text 2 that allows the creating of commands consisting of
    sequences of find and replace instructions.
    \url{https://forum.sublimetext.com/t/regreplace-plugin/3810}

    \item The main reason I moved was that I find that it’s much slower, the simple things like opening a
    new window for a project should be instantaneous and sadly it’s far from it. As I've said before
    it's all about personal preference, I've gone back to Sublime but Adam for example is sticking
    with it...
    \url{http://engageinteractive.co.uk/blog/atom-review}

    \item \url{https://www.researchgate.net/publication/228540036_An_industrial_application_of_context-sensitive_formatting}

    \item \url{http://www.suodenjoki.dk/us/archive/2010/cpp-checkstyle.htm}

    \item \url{http://www.basicinputoutput.com/2014/08/uncrustify-your-bios.html}

    \item \url{http://prettyprinter.de/}

    \item \url{https://github.com/ryanmaxwell/UncrustifyX}

    \item \url{http://www.softpanorama.org/Utilities/beautifiers.shtml}

    \item Understanding the Syntax Parsing
    \url{https://forum.sublimetext.com/t/understanding-the-syntax-parsing/28569}

    "So, part of what I've been working on is a code beautifier that, more or less, aligns and
    indents the code properly based on scanning through the source document."
    ...
    "It hasn't escaped my notice that this is to some degree exactly what the syntax file is doing."

    \item

    {\bfseries Towards a universal code formatter through machine learning:}
    In this paper, we solve the formatter construction problem using a novel approach, one that
    automatically derives formatters for any given language without intervention from a language
    expert. We introduce a code formatter called CODEBUFF that uses machine learning to abstract
    formatting rules from a representative corpus, using a carefully designed feature set. Our
    experiments on Java, SQL, and ANTLR grammars show that CODEBUFF is efficient, has excellent
    accuracy, and is grammar invariant for a given language. It also generalizes to a 4th language
    tested during manuscript preparation.
    \begin{enumerate}[nolistsep,topsep=0pt,label=$\star$]
        \item \url{http://dl.acm.org/citation.cfm?id=2997383}
        \item \url{http://homepages.cwi.nl/~jurgenv/papers/SLE16.pdf}
    \end{enumerate}

    \item \url{https://www.google.com/search?q=universal+source+code+formatter}
    \begin{enumerate}[nolistsep,topsep=0pt,label=$\star$]
        \item \url{https://www.google.com/search?q=universal+source+code+beautifier}
    \end{enumerate}

    \item \url{http://en.wikipedia.org/wiki/Indent_style}
    \begin{enumerate}[nolistsep,topsep=0pt,label=$\star$]
        \item \url{https://en.wikipedia.org/wiki/Programming_style}
        \item \url{https://en.wikipedia.org/wiki/Scope_(computer_science)}
    \end{enumerate}

    \item \url{http://wiki.c2.com/?CodingStyle}
    \begin{enumerate}[nolistsep,topsep=0pt,label=$\star$]
        \item \url{https://github.com/google/code-prettify}
        \item \url{https://github.com/uncrustify/uncrustify}
    \end{enumerate}

    \item \url{https://en.wikipedia.org/wiki/Prettyprint}
    \begin{enumerate}[nolistsep,topsep=0pt,label=$\star$]
        \item \url{https://www.researchgate.net/search.Search.html?query=formatting%20source%20code&type=publication}
        \item \url{https://www.researchgate.net/search.Search.html?query=pretty%20print%20source%20code&type=publication}
    \end{enumerate}

    \item \url{https://github.com/gchpaco/gopprint}
    \begin{enumerate}[nolistsep,topsep=0pt,label=$\star$]
        \item \url{http://dl.acm.org.sci-hub.io/citation.cfm?id=357115}
        \item \url{https://www.cs.indiana.edu/~sabry/papers/yield-pp.pdf}
    \end{enumerate}

    \item \url{http://www.worldcat.org/title/beautiful-code-a-customizable-code-beautifier-for-java/oclc/56564674}
    \begin{enumerate}[nolistsep,topsep=0pt,label=$\star$]
        \item \url{https://www.researchgate.net/publication/34736049_Beautiful_code_a_customizable_code_beautifier_for_Java}
        \item \url{https://vufind.carli.illinois.edu/vf-ncc/Record/ncc_118189/Holdings}
    \end{enumerate}

    \item \url{https://www.researchgate.net/publication/4283921_Smart_Formatter_Learning_Coding_Style_from_Existing_Source_Code}
    \begin{enumerate}[nolistsep,topsep=0pt,label=$\star$]
        \item \url{http://www.ing.unisannio.it/mdipenta/index.html}
        \item \url{https://github.com/iain/rspec-smart-formatter}
    \end{enumerate}

    \item \url{https://www.researchgate.net/publication/2543984_Source_Code_Files_as_Structured_Documents}
    \begin{enumerate}[nolistsep,topsep=0pt,label=$\star$]
        \item \url{https://en.wikipedia.org/wiki/SrcML}
    \end{enumerate}

    \item \url{https://www.researchgate.net/publication/228540036_An_industrial_application_of_context-sensitive_formatting}
    \begin{enumerate}[nolistsep,topsep=0pt,label=$\star$]
        \item \url{https://www.researchgate.net/publication/234809222_Program_indentation_and_comprehensibility}
    \end{enumerate}

\end{enumerate}
\end{bluebox}
\end{sloppypar}


\subsubsection{Obfuscators}

    Aqui encontra-se o lado oposto dessas ferramentas, Source Code Obfuscators, que servem para
    destruir o visual do código. Usualmente utilizado para dificultar a leitura por outras pessoas
    ou ainda reduzir o tamanho de códigos de linguagens scripting que devem ser carregadas/baixadas
    por navegadores de internet, assim diminuindo o tráfego de internet e salvando/economizando
    largura de banda para download:

    \begin{sloppypar}
    \begin{bluebox}\RaggedRight
    \begin{enumerate}[leftmargin=*,parsep=0pt]

    \item \url{https://en.wikipedia.org/wiki/Obfuscation_(software)}

    \item \url{http://www.semdesigns.com/Products/Obfuscators/index.html}

    \end{enumerate}
    \end{bluebox}
    \end{sloppypar}

\end{englishtext}


% Portuguese
\lang{}{

    Este trabalho tem como objetivo criar um formatador (software único) de fácil configuração e
    expansão capaz de abranger as linguagens de programação que existem, baseado em um uso
    específico de expressões regulares.

    A metodologia abordada será de não ter a necessidade de ter-se conhecimento da sintaxe das
    linguagens de programação que se irão fazer o parsing. Isso porque trataremos elas como texto
    comum, e será o usuário final que fará a configuração das transformações que serão aplicados no
    texto, dando liberdade de facilmente se configurar várias linguagens de programação (senão
    todas), aproveitando o fato de que muitas deles compartilham estruturas semelhantes senão
    idênticas.

    Como resultado espera-se ter um Beautifier Universal capaz de abranger as linguagens que
    existem, senão que seja facilmente extensível para abrange-las. Os pontos positivos dessa
    abordagem são a reusabilidade de componentes entre as linguagens. Por exemplo, `if/for/while's
    em C++ e Java são da mesma estrutura. Assim temos que escrever somente uma vez a especificação
    para um componente da linguagem.

    A ideia de um software, que em certa extensão pode continuar um ramo do Trabalho de Conclusão de
    Curso do aluno `Lucas Boppre Niehues', orientado do Professor `Olinto José Varela Furtado'
    defendido em 2013/1, com o título: `Estudo e Criação de um Editor de Código Estruturado'. Donde
    durante a leitura de seu TCC, encontra-se o seguinte trecho que faz ligação com uma das
    propostas deste trabalho, no capítulo: `8.1.2 Separação de formato de exibição e de saída':

    \medskip
    \begin{myquote}
    ``As formas que o código é exibido ao usuário e que ele é salvo em disco são controladas
    por arquivos de configuração distintos. O arquivo ``theme.ini'' contém, entre outras
    configurações, informações de como serializar a árvore sintática.''
    \end{myquote}

    \vspace{-5mm}
    ...
    \begin{myquote}
    ``A configuração de formato de saída é dada da mesma forma, mas em um arquivo
    separado, chamado ``output\_format.ini''. A decisão desta separação foi em vista de equipes
    de programadores que queiram utilizar uma convenção única para os arquivos salvos,
    mas manter a exibição a escolha de cada um. Assim os integrantes desta equipe podem
    compartilhar os seus arquivos ``output\_format.ini'' enquanto personalizam o arquivo
    ``theme.ini'' a seu gosto.''
    \end{myquote}

    Com base nisso, pode-se pensar na escrita de plugins para editores de texto/IDEs comuns como
    Sublime Text. Assim ao carregar o arquivo do disco, este plugin chama o formatter e faz a
    formatação de acordo com as configurações de exibição para o usuário. Após isso, quando o
    usuário for salvar o arquivo, o arquivo com a formatação original é devolvido.

    Para auxilar nesse processo, um módulo de autoconfiguração é de grande ajuda. Ele detecta como o
    source code está formatado e cria arquivos de configuração para ele. Assim ao salvar o arquivo,
    automaticamente ele é salvo no formato que ele foi lido. Então temos o mesmo beneficio de
    editores estruturados, como proposto trabalho de `Lucas Boppre Niehues'. De inicio podemos
    pensar com os seguinte objetivo/ideia para um TCC:

    \medskip
    \begin{myquote}
    \begin{enumerate}[nolistsep]
        \item Criar um formatador de fácil configuração e expansão para as linguagens de
              programação que existem e que irão existir.
    \end{enumerate}
    \end{myquote}



\subsection{Problema}

    O problema proposto a se resolver é criar um Beautifier Universal. Os softwares atuais são
    limitados a um conjunto similar, ou mesmo à uma única linguagem, e além de muitos, serem
    limitados ao que eles podem fazer por você ao processar/formatar o código \cite{universalCodeFormatter}.

    Logo abaixo há algumas regras de formatação básica encontrados no serviço online
    \url{http://prettyprinter.de/} acessado em março/2017:

    \medskip
    \begin{myquote}
    \begin{enumerate}[nolistsep]
        \item Add new lines after ``\{'' and before ``\}''
        \item Add new lines before ``\{''
        \item Remove empty lines
        \item Add comment lines before function
        \item Add new lines after ``;''
        \item Add new lines after ``\}''
        \item Remove new lines
        \item Reduce whitespace
        \item Put the code again in the input box above after submit
    \end{enumerate}
    \end{myquote}

    A partir deste ponto, apresenta-se um esboço sobre o problema, soluções, informações como
    porquês de se querer fazer um software assim, ou ainda de querer-se o beautifying:

    \begin{enumerate}[leftmargin=*]

        \item

        Motivação: Existem muitas ferramentas distintas, por vezes pagas, e dificilmente completas
        \cite{universalCodeFormatter}.

        \item

        Muitas linguagens de programação existem, assim sempre ter fazer um software Beautifier para
        cada uma delas é muito trabalhoso \cite{universalCodeFormatter}. Mas a abordagem para um Beautifier
        Universal proposta nesse trabalho, permite que facilmente novas linguagens sejam
        adicionadas, sendo elas completamente diferentes das anteriores, ou similares. No caso de
        similaridades, basta reutilizar as estruturas de configuração das linguagens já existentes.

        \item

        Preocupa-se de fazer um Beautifier para cada uma delas por que programadores atualmente
        trabalham diariamente com varias dessas linguagens, e elas não são similares. Assim precisa-
        se configurar vários beautifiers para fazer a formatação. Isso é um problema por que,
        somente alguns beautifiers são mais completos, e toda vez que precisa-se fazer uma alteração
        no estilo de formatação, precisa-se propagar manualmente a mesma mudança ao longo de vários
        arquivos de configuração de programas distintos, o que é ruim pois toma ao usuário muito
        tempo de aprender a lidar com várias e muito diferentes tipos de configurações
        \cite{Schweitzer}.

        \item

        No caso do Beautifier que propõem-se, uma mudança no estilo é propagada para todas as
        linguagens. E caso queira-se deixar alguma linguagem fora da regra, basta remover ela da
        lista ao qual esse bloco da configuração se aplica, e `a)' deixar ela de fora assim nenhuma
        mudança é aplicada a ela. Ou `b)' criar um novo bloco que inclua somente ela com a
        configuração desejada.

        \item

        A seguir, temos algumas frases sobre o assunto:

        \begin{myquote}
        % \setlength{\itemindent}{5pt}
        ``One of absolute worst, worst methods of teamicide for software developers is to engage
        in these kinds of passive-aggressive formatting wars. I know because I've been there.
        They destroy peer relationships, and depending on the type of formatting, can also damage
        your ability to effectively compare revisions in source control, which is really scary.
        I can't even imagine how bad it would get if the lead was guilty of this behavior. That's
        leading by example, all right. Bad example.'', \cite{Atwood}.
        \end{myquote}
        \vspace{-5mm}
        ...
        \begin{myquote}
        ``So yes, absurd as it may sound, fighting over whitespace characters and other seemingly
        trivial issues of code layout is actually justified. Within reason of course -- when done
        openly, in a fair and concensus building way, and without stabbing your teammates in the
        face along the way.'', \cite{Atwood}.
        \end{myquote}

        \begin{myquote}``
        I'd say there are two main reasons to enforce a single code format in a project. First has
        to do with version control: with everybody formatting the code identically, all changes in
        the files are guaranteed to be meaningful. No more just adding or removing a space here or
        there, let alone reformatting an entire file as a `side effect' of actually changing just a
        line or two.'', \cite{Geukens}.
        \end{myquote}

    \end{enumerate}



\subsection{Objetivos}

    O objeto neste trabalho de TCC proposto aqui não é inicialmente suportar todas as regras de
    formatação de todas as linguagens de programação, mas a criação de uma estrutura básica inicial
    e robusta que sejam capaz de ser desenvolvida a ponto de ser facilmente expandida, tanto na
    adição de novos módulos de processamento no programa escrito, tanto pelo usuário final na
    escrita dos arquivos de programação.

    A teoria da técnica empregada é muito simples, mas diferente das atuais por que é atribuído ao
    usuário final a responsabilidade de dizer onde será realizado o beautifying do modulo que está
    se configurando. Esse é o preço a pagar para permitir a criação de um Beautifier Universal.
    Quando diz-se fácil configuração, refire-se a não necessidade de recorrer a programação ´C++',
    i.e., alterar o código fonte do programa para permitir/especificar onde devem ser realizadas as
    alterações de beautifying.


\subsubsection{Objetivos Gerais}

    \begin{enumerate}[leftmargin=*]

        \item

        Escrever o programa em C++ ou afins, para permitir também que a formação/beautifying seja
        (em trabalhos futuros/talvez nesse) dinâmico, isto é, na medida que você digita o texto, ele
        é formatado para você. Assim você pode focar mais em escrever o código, ao invés que se
        preocupar com o espaçamento, alinhamento, parenteses, linhas novas, e o que mais que seja.

        \item

        Utilizar o Framework `doctest` para escrita dos Testes de Unidade. Pois após procurar e
        testar alguns frameworks para testes de unidade em C++, entrou-se este como servindo muito
        bem as requisitos do projecto. Ele causa baixíssimo incremento no tempo de compilação e
        permite que os testes possam ser escritos no mesmo arquivo onde encontram-se o código do
        programa, sem que eles sejam compilados.

        \item

        Utilizar uma versão/algoritmo multi-core, então cada uma das regras pode ser processada em
        paralelo e sobre o mesmo source code original. Essa parte é bastante complexa de ser escrita
        por que as regras entre si podem gerar conflitos sobre o que elas estão fazendo. Para
        resolver esse problema, fazer com que cada regra processada gere um objeto de mudanças que
        essa regra está propondo. No final do processamento de todas as regras, será realizado um
        fusão das mudanças que cada uma decidiu realizer, e caso duas regras queriam mudar o mesmo
        pedaço/trecho de código, será lançada um exceção e uma nova classe de mudanças/regra deve
        estar disponível para resolver esse conflito. Caso não exista, ambas as mudanças são
        descartadas e somente as mudanças sem conflitos são refletidas no código.

    \end{enumerate}


\subsubsection{Objetivos Específicos}


    \begin{enumerate}[leftmargin=*]

        \item

        Um Produto de Software com uma ótima orientação a objetos e possibilidades de extensão das
        funcionalidades.

        \item

        Classificar todas classes e tipos de formatações (beautifying) de código aplicáveis com
        facilidade. Uma das partes a serem escritas e entregues na monografia. Um estudo sobre o que
        é beautifying, como fazer e por que fazer.

        \item

        Implementação de um núcleo funcional e de uma pesquisa decente sobre o estado da arte. Um
        dos pontos difíceis seria a marcação dos escopos, mas isso já é implementado pelo núcleo do
        editor Sublime Text, assim provado como possível de ser feito.

        \item

        Inicialmente devido a limitação de tempo em 1 ano e meio para um TCC, podemos pensar somente
        um núcleo básico, simples, reutilizável e que talvez possa ajudar no contexto da linguagem
        que vocês desenvolvem.

    \end{enumerate}


\subsubsection{Trabalhos Futuros}

    O número de recursos/funcionalidades e estratégias de otimizações para serem implementadas, e
    etc, são imensas. Mas esses trabalhos podem ser muito mais para frente depois da entrega do TCC.
    Hoje o controle de espaços em chamadas de funções, declarações de classes, comentários e etc,
    são mais tranquilos de se entender e pensar. Entretanto no requisito e ajuste de indentação,
    inserção/remoção de parenteses redundantes, etc ainda falta estudo sobre como deve ser
    implementado isso.

    Contudo essa especificação por parte do usuário é limitado a linguagens Livres de Contexto
    (máquinas de pilha). Assim caso as especificações de escope precisarem ser feitas em termos de
    linguagens Sensíveis ao Contexto ou ainda Recursivamente Enumeráveis, vai ser preciso tratar
    esses elementos diretamente em C++ (máquina de turing).

    Entretanto não consegue-se pensar facilmente em casos em que precise mais do que tratadores
    Livres de Contexto para realizar a especificação de quais partes do código deve ser necessário
    formatar. Sublime Text faz uso dessa técnica para o Highlight dos códigos das mais diversas
    linguagens e acredita-se que tenha um bom resultado.





\subsection{Método de pesquisa}

    A vantagem nesta abordagem é não ter a necessidade de ter-se conhecimento da sintaxe das
    linguagens de programação que se irão fazer o parsing. Isso porque trataremos elas como texto
    comum, e será o usuário final que fará a configuração das transformações que serão aplicados no
    texto, dando liberdade de facilmente se configurar várias linguagens de programação,
    aproveitando o fato de que muitas deles compartilham estruturas semelhantes senão idênticas.

    A literatura/programas atuais são dependentes de linguagem de programação. Minha proposta é
    fazer este processo independente de linguagem, mas de dialetos como este exemplo tirado do PDF
    em anexo a este e-mail `Initial check list tasks to do.pdf':

    \begin{lstlisting}
    // This is the name used to reference this scope around the settings files.
    Scope Name:
    %c++_like_block_comment

    // This set on which languages this block should be included. Setting it
    // to empty will allow it to be parsed for any languages.
    Language Inclusion:
    Java, C++, Pawn

    // Defines a expression which will map the beginning of a exclusion block.
    Scope Start:
    /\*\*

    // Defines a expression which will map the ending of a exclusion block.
    Scope End:
    \\\*
    \end{lstlisting}
    \vspace*{-4mm}

    A abordagem acima é uma abordagem ingênua, portanto somente brevemente ilustrativa. O real motor
    para o software é baseado em expressões regulares e um pilha de contextos. Esta ideia foi
    inicialmente desenvolvida pelo editor de texto `Sublime Text' \cite{Skinner}. Este editor
    utiliza essa estrutura de blocos para fazer a sintaxe highlighting do códigos das linguagens
    através de expressões regulares alocação de contextos/escopos. Essa mesma abordagem pode ser
    utilizada pelo usuário para definir em quais regiões uma Máquina de Turing (linguagens C++/Rust)
    devem fazer/propor as alterações no código.


\subsubsection{Pontos}

    Os pontos positivos dessa abordagem para um formatador de código são a reusabilidade de
    componentes entre as linguagens pelo usuário final da aplicação ao invés do programador, o que
    torna este software muito mais genérico e abre a possibilidades de maior sucesso para a criação
    definitiva de um formatador Universal de códigos das linguagens de programação, quaisquer sejam
    elas. Por exemplo, `if/for/while'\textquotesingle s em linguagens de programação como C++ e Java
    são da mesma estrutura. Assim temos que escrever somente uma vez a especificação para um
    componente da linguagem sem recorrer a programação de do código do programa. Isso tem a vantagem
    de por der ser configurado pelo usuário final ao invés do programador, assim fica mais simples
    de configurar e expandir o conjunto de linguagens disponíveis ao processamento/beautifying.

    Softwares existentes e similares:

    \medskip
    \begin{myquote}
    \begin{enumerate}[leftmargin=*]

        \item

        CodeBeautify is an online code beautifier which allows you to beautify your source code:
        \url{http://codebeautify.org/}.

        \item

        A universal code formatter, written in Dart: \url{https://pub.dartlang.org/packages/unifmt}.

        \item

        Google-java-format is a program that reformats Java source code to comply with Google Java
        Style: \url{https://github.com/google/google-java-format}.

        \item

        CodeFormatter is a Sublime Text 2/3 plugin that supports format (beautify) source code.
        \url{https://github.com/akalongman/sublimetext-codeformatter} and
        \url{https://github.com/aukaost/SublimePrettyYAML}

        \item

        UniversalIndentGUI offers a live preview for setting the parameters of nearly any indenter.
        You change the value of a parameter and directly see how your reformatted code will look
        like. Save your beauty looking code or create an anywhere usable batch/shell script to
        reformat whole directories or just one file even out of the editor of your choice that
        supports external tool calls: \url{http://universalindent.sourceforge.net/} and
        \url{https://github.com/danblakemore/universal-indent-gui}.

    \end{enumerate}
    \end{myquote}


\subsubsection{Listagens}

    Algumas bibliotecas existentes, e potencialmente utilizadas como `syntect` para o auxílio na
    construção do produto de software:

    \begin{myquote}
    \begin{enumerate}[leftmargin=*,parsep=0pt]

        \item \url{https://github.com/jbeder/yaml-cpp}
        \item \url{https://github.com/trishume/syntect}
        \item \url{https://github.com/onqtam/doctest}
        \item \url{https://github.com/c42f/tinyformat}
        \item \url{https://github.com/limetext/lime}
        \item \url{https://forum.sublimetext.com/t/disassembling-sublime-text/24824}

    \end{enumerate}
    \end{myquote}

    Segue-se uma lista básica de formatters/beautifiers acessado no endereço
    \lword{\url{http://www.softpanorama.org/Utilities/beautifiers.shtml}} em março/2017:

    \medskip
    \begin{sloppypar}
    \begin{myquote}\RaggedRight
    \begin{enumerate}[leftmargin=*,parsep=0pt]

        \item CB210.ZIP - C Beautifier 2.10 - polish C source code (19,406 bytes, 06/22/92)
        \item CL121.ZIP - Codelister 1.21 - print C code with stats (51,110 bytes, 01/10/94)

        \item CPC200.ZIP - CodePrint for C/C++ 2.00 is a full-featured command line driven source
        code reformatter and pretty printer for C++ and C; over 20 customization features including
        auto-indent, adjustable tab spacing, indent styles, flow lines, comment alignment, and line
        editing for consistent white space (140,605 bytes, 01/26/96)

        \item CSCOP120.ZIP - c-scope 1.20 analyzes C source code and produces various reports
        (48,505 bytes, 06/30/95)

        \item HTML : \url{http://www.digital-mines.com/htb/}
        \item HTML : \url{http://www.datacomm.ch/mwoog/software/perl/beautifier.html}
        \item HTML : \url{http://www.watson-net.com/free/perl/s_fhtml.asp}
        \item SQL : \url{http://www.netbula.com/products/sqlb}
        \item Oracle PLSQL : \url{http://www.revealnet.com}
        \item GPL \url{http://www.geocities.com/~starkville/vancbj.html}
        \item GPL \url{http://kevinkelley.mystarband.net/java/dent.html}
        \item Free \url{http://www.tiobe.com/jacobe.htm}
        \item Free \url{http://www.mmsindia.com/JPretty.html}
        \item Free \url{http://members.magnet.at/johann.langhofer/products/jxbeauty/overview.html} (has JBuilder support)
        \item Free \url{http://www.semdesigns.com/Products/Formatters/JavaFormatter.html}
        \item Commercial \$24.99 \url{http://smartbeautify.com}
        \item Commercial \$129 \url{http://www.jindent.com}
        \item Google \url{http://directory.google.com/Top/Computers/Programming/Languages/Java/Development_Tools/Code_Beautifiers/?tc=1}
        \item Java, SQL, HTML, C++ : \url{http://www.semdesigns.com/Products/DMS/DMSToolkit.html}
        \item Java JIndent \url{http://home.wtal.de/software-solutions/jindent}
        \item Java Pat \url{http://javaregex.com/cgi-bin/pat/jbeaut.asp}
        \item Java JStyle \url{http://www.redrival.com/greenrd/java/jstyle}
        \item Java JPrettyPrinter \url{http://www.epoch.com.tw/download/ms/java/java.htm}
        \item Java JxBeauty \url{http://members.nextra.at/johann.langhofer/download/jxbeauty} and the JxBeauty Home
        \item Java beautify percolator
        \item Java list \url{http://www.java.about.com/compute/java/library/weekly/aa102499.htm}
        \item Java html present VasJava2HTML
        \item Java code colorifier and beautifier \url{http://www.mycgiserver.com/~lisali/jccb}
        \item Perl : \url{http://www.consultix-inc.com/www.consultix-inc.com/talk.htm}
        \item Perl : \url{http://www.consultix-inc.com/www.consultix-inc.com/perl_beautifier.html}
        \item Fortran beautifier : \url{http://www.aeem.iastate.edu/Fortran/tools.html}

        \item C++ : BCPP site is at \url{http://dickey.his.com/bcpp/bcpp.html} or at \url{http://www.clark.net/pub/dickey}.
        BCPP ftp site is at \url{ftp://dickey.his.com/bcpp/bcpp.tar.gz}

        \item C++ : \url{http://www.consultix-inc.com/c++b.html}
        \item C : \url{http://www.chips.navy.mil/oasys/c/} and mirror at Oasys
        \item C++, C, Java, Oracle Pro-C Beautifier \url{http://www.geocities.com/~starkville/main.html}

        \item C++, C beautifier \url{http://users.erols.com/astronaut/vim/ccb-1.07.tar.gz} and site at
        \url{http://users.erols.com/astronaut/vim/#vimlinks_src}

        \item GC! GreatCode! is a powerful C/C++ source code beautifier Windows 95/98/NT/2000
        \url{http://perso.club-internet.fr/cbeaudet}

        \item C++ beautifier `SourceStyler' \url{https://web.archive.org/web/20061205061102/http://ochresoftware.com/}
        \item JavaScript : \url{http://jsbeautifier.org/}

    \end{enumerate}
    \end{myquote}
    \end{sloppypar}


\subsubsection{Trabalhos Correlatos}

    Após a busca do que há de publicações científicas sobre o assunto e entra-se alguns trabalhos na
    área específica e similar aos trabalhos feitos pelor formatadores de códigos (Beautifiers).
    Nessa modalidade de trabalho, pode-se confundir-se com artigos que tratam sobre o `Prettyprint`,
    que trata-se de colorir o texto e exibir-lo ao usuário. O que não é o que se busca nesse
    trabalho, mas sim fazer alterações no texto sobre a forma como ele é estruturado, apresentado ao
    usuário e salvo em disco. Seguem as seguintes publicações:

    % How to add `parsep` to `itemsep` and set `parsep` to 0pt, when declaring my list?
    % https://tex.stackexchange.com/questions/366904/how-to-add-parsep-to-itemsep-and-set-parsep-to-0pt-when-declaring-my-list
    \begin{sloppypar}
    \begin{myquote}\RaggedRight
    \begin{enumerate}[leftmargin=*,parsep=0pt]

    \item \url{https://www.researchgate.net/publication/228540036_An_industrial_application_of_context-sensitive_formatting}

    \item \url{http://www.suodenjoki.dk/us/archive/2010/cpp-checkstyle.htm}

    \item \url{http://www.basicinputoutput.com/2014/08/uncrustify-your-bios.html}

    \item \url{http://prettyprinter.de/}

    \item \url{https://github.com/ryanmaxwell/UncrustifyX}

    \item \url{http://www.softpanorama.org/Utilities/beautifiers.shtml}

    \item Understanding the Syntax Parsing
    \url{https://forum.sublimetext.com/t/understanding-the-syntax-parsing/28569}

    "So, part of what I've been working on is a code beautifier that, more or less, aligns and
    indents the code properly based on scanning through the source document."
    ...
    "It hasn't escaped my notice that this is to some degree exactly what the syntax file is doing."

    \item

    {\bfseries Towards a universal code formatter through machine learning:}
    In this paper, we solve the formatter construction problem using a novel approach, one that
    automatically derives formatters for any given language without intervention from a language
    expert. We introduce a code formatter called CODEBUFF that uses machine learning to abstract
    formatting rules from a representative corpus, using a carefully designed feature set. Our
    experiments on Java, SQL, and ANTLR grammars show that CODEBUFF is efficient, has excellent
    accuracy, and is grammar invariant for a given language. It also generalizes to a 4th language
    tested during manuscript preparation.
    \begin{enumerate}[nolistsep,topsep=0pt,label=$\star$]
        \item \url{http://dl.acm.org/citation.cfm?id=2997383}
        \item \url{http://homepages.cwi.nl/~jurgenv/papers/SLE16.pdf}
    \end{enumerate}

    \item \url{https://www.google.com/search?q=universal+source+code+formatter}
    \begin{enumerate}[nolistsep,topsep=0pt,label=$\star$]
        \item \url{https://www.google.com/search?q=universal+source+code+beautifier}
    \end{enumerate}

    \item \url{http://en.wikipedia.org/wiki/Indent_style}
    \begin{enumerate}[nolistsep,topsep=0pt,label=$\star$]
        \item \url{https://en.wikipedia.org/wiki/Programming_style}
        \item \url{https://en.wikipedia.org/wiki/Scope_(computer_science)}
    \end{enumerate}

    \item \url{http://wiki.c2.com/?CodingStyle}
    \begin{enumerate}[nolistsep,topsep=0pt,label=$\star$]
        \item \url{https://github.com/google/code-prettify}
        \item \url{https://github.com/uncrustify/uncrustify}
    \end{enumerate}

    \item \url{https://en.wikipedia.org/wiki/Prettyprint}
    \begin{enumerate}[nolistsep,topsep=0pt,label=$\star$]
        \item \url{https://www.researchgate.net/search.Search.html?query=formatting%20source%20code&type=publication}
        \item \url{https://www.researchgate.net/search.Search.html?query=pretty%20print%20source%20code&type=publication}
    \end{enumerate}

    \item \url{https://github.com/gchpaco/gopprint}
    \begin{enumerate}[nolistsep,topsep=0pt,label=$\star$]
        \item \url{http://dl.acm.org.sci-hub.io/citation.cfm?id=357115}
        \item \url{https://www.cs.indiana.edu/~sabry/papers/yield-pp.pdf}
    \end{enumerate}

    \item \url{http://www.worldcat.org/title/beautiful-code-a-customizable-code-beautifier-for-java/oclc/56564674}
    \begin{enumerate}[nolistsep,topsep=0pt,label=$\star$]
        \item \url{https://www.researchgate.net/publication/34736049_Beautiful_code_a_customizable_code_beautifier_for_Java}
        \item \url{https://vufind.carli.illinois.edu/vf-ncc/Record/ncc_118189/Holdings}
    \end{enumerate}

    \item \url{https://www.researchgate.net/publication/4283921_Smart_Formatter_Learning_Coding_Style_from_Existing_Source_Code}
    \begin{enumerate}[nolistsep,topsep=0pt,label=$\star$]
        \item \url{http://www.ing.unisannio.it/mdipenta/index.html}
        \item \url{https://github.com/iain/rspec-smart-formatter}
    \end{enumerate}

    \item \url{https://www.researchgate.net/publication/2543984_Source_Code_Files_as_Structured_Documents}
    \begin{enumerate}[nolistsep,topsep=0pt,label=$\star$]
        \item \url{https://en.wikipedia.org/wiki/SrcML}
    \end{enumerate}

    \item \url{https://www.researchgate.net/publication/228540036_An_industrial_application_of_context-sensitive_formatting}
    \begin{enumerate}[nolistsep,topsep=0pt,label=$\star$]
        \item \url{https://www.researchgate.net/publication/234809222_Program_indentation_and_comprehensibility}
    \end{enumerate}

    \end{enumerate}
    \end{myquote}
    \end{sloppypar}


\subsubsection{Obfuscators}

    Aqui encontra-se o lado oposto dessas ferramentas, Source Code Obfuscators, que servem para
    destruir o visual do código. Usualmente utilizado para dificultar a leitura por outras pessoas
    ou ainda reduzir o tamanho de códigos de linguagens scripting que devem ser carregadas/baixadas
    por navegadores de internet, assim diminuindo o tráfego de internet e salvando/economizando
    largura de banda para download:

    \begin{sloppypar}
    \begin{myquote}\RaggedRight
    \begin{enumerate}[leftmargin=*,parsep=0pt]

    \item \url{https://en.wikipedia.org/wiki/Obfuscation_(software)}

    \item \url{http://www.semdesigns.com/Products/Obfuscators/index.html}

    \end{enumerate}
    \end{myquote}
    \end{sloppypar}

}





\bibliography{refs}


\end{document}
