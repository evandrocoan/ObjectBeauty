
%
% Simple Sectioned Essay Template - LaTeX Template
%
% This template has been downloaded from:
% http://www.latextemplates.com
%
% `proposal.tex`
% Based on
%
% 1. https://github.com/royertiago/tcc
% 2. http://portal.bu.ufsc.br/normalizacao/
% 3. https://github.com/evandrocoan/ufscthesisx
% 4. http://www.latextemplates.com/template/simple-sectioned-essay
%
% Initially translated from Portuguese with help of https://github.com/omegat-org/omegat
% <Computer Assisted Translation of LaTeX document>
% https://tex.stackexchange.com/questions/313732/computer-assisted-translation-of-latex-document
%
% In case a translation back to Portuguese is required, keep both languages toguether.
% <Is it possible to keep my translation together with original text?>
% https://tex.stackexchange.com/questions/5076/is-it-possible-to-keep-my-translation-together-with-original-text
%
% You can build this using the command:
% latexmk -pdf -jobname=output -output-directory=cache -aux-directory=cache -pdflatex="pdflatex -interaction=nonstopmode" -use-make main.tex



%----------------------------------------------------------------------------------------
%   PACKAGES AND OTHER DOCUMENT CONFIGURATIONS
%----------------------------------------------------------------------------------------

% Uncomment the line `\englishtrue` to set the document default language to english.
%
% Is it possible to keep my translation together with original text?
% https://tex.stackexchange.com/questions/5076/is-it-possible-to-keep-my-translation-together-with-original-text
\newif\ifenglish
\englishfalse
\englishtrue

% How to expand \ifthenelse so that it can be used in \parshape?
% https://tex.stackexchange.com/questions/131002/how-to-expand-ifthenelse-so-that-it-can-be-used-in-parshape
\newcommand{\chooselang}[2]{\ifenglish#1\else#2\fi}

\ifenglish
    % How to make \PassOptionsToPackage add the option as the last option?
    % https://tex.stackexchange.com/questions/385895/how-to-make-passoptionstopackage-add-the-option-as-the-last
    \PassOptionsToPackage{brazil,main=english,spanish,french}{babel}
    \newcommand{\swapcontents}[2]{#1 #2}
\else
    \PassOptionsToPackage{main=brazil,english,spanish,french}{babel}
    \newcommand{\swapcontents}[2]{#2 #1}
\fi

% You need to run `pdfTeX` 5 times on the following order: 1. `pdfTeX`, 2. `bibtex`, 3. `pdfTeX` 4.
% `pdfTeX` 5. `pdfTeX` 6. `pdfTeX`, because the bibliography includes a cyclic reference to another
% bibliography, so we need a last pass to fix the bibliography undefined references.
%
% Fix recurring LaTeX Warning
% https://github.com/abntex/abntex2/pull/189
%
% To fix the warning `LaTeX Warning: Label(s) may have changed. Rerun to get cross-references right`,
% open the file `D:\User\Documents\latex\texmfs\install\tex\latex\abntex2\abntex2cite.sty` and
% comment out these two lines:
% 547: % \renewcommand{\bibcite}[2]{%
% 548: %   \@newl@bel{b}{#1}{\hyper@@link[cite]{}{cite.#1}{#2}}}%
\input{ufscthesisx/setup/setup}

% Load the UFSC thesis settings
\usepackage{ufscthesisx/setup/ufscthesisx}

% Load all required basic packages




% Incompatible color definition when using tikz with color package
% https://tex.stackexchange.com/questions/150369/incompatible-color-definition-when-using-tikz-with-color-package
\usepackage{xcolor}

\definecolor{dkgreen}{rgb}{0,0.6,0}
\definecolor{gray}{rgb}{0.5,0.5,0.5}
\definecolor{mauve}{rgb}{0.58,0,0.82}

\definecolor{link_color}{RGB}{26,13,178}


% For web links and paths with \path{..} and \url{https://www.python.org/downloads/}
%
% https://tex.stackexchange.com/questions/3033/forcing-linebreaks-in-url
\PassOptionsToPackage{hyphens}{url}
\usepackage[backref,colorlinks,linkcolor=link_color]{hyperref}

% How to fix URL overfull & underfull on emumeration?
% % https://tex.stackexchange.com/questions/366803/how-to-fix-url-overfull-underfull-on-emumeration
%
% Forcing linebreaks in \url
% https://tex.stackexchange.com/questions/3033/forcing-linebreaks-in-url/10401
\usepackage{url}
\makeatletter
\g@addto@macro{\UrlBreaks}{\UrlOrds}
\makeatother

% \lettrine{O}{nce} upon a time...
% \lettrine[findent=2pt]{\fbox{\textbf{T}}}{ }his thesis deals with...
%
% https://tex.stackexchange.com/questions/164298/starting-a-paragraph-with-a-big-letter
\usepackage{lettrine}

% Required for including pictures, resizebox
\usepackage{graphicx}

% Allows putting an [H] in \begin{figure} to specify the exact location of the figure
\usepackage{float}

% Allows in-line images such as the example fish picture
\usepackage{wrapfig}

% How to automatically force latex to not justify the text when it is not wise?
% https://tex.stackexchange.com/questions/365801/how-to-automatically-force-latex-to-not-justify-the-text-when-it-is-not-wise
\usepackage{array,ragged2e}

% Use its macro adjustwidth* to extend tables out of outer text border.
% https://tex.stackexchange.com/questions/366155/how-to-write-a-table-a-little-larger-than-the-paragraphs-with-centered-columns
\usepackage[strict]{changepage}

% No spacing between enumerated items with \usepackage{enumerate}
% https://tex.stackexchange.com/questions/119919/no-spacing-between-enumerated-items-with-usepackageenumerate
\usepackage[shortlabels]{enumitem}

\usepackage{tabularx}
\usepackage{multirow}








%
% New Macros
%

% Automatically put a `\medskip` spacing between paragraphs
% https://tex.stackexchange.com/q/365976/119062
\edef\restoreparindent{\parindent=\the\parindent\relax}
\usepackage{parskip}
\restoreparindent

% Uncomment to remove all indentation from paragraphs
%\setlength\parindent{0pt}

% How could the `\everypar` justification statement be used?
% https://tex.stackexchange.com/questions/365818/how-could-the-everypar-justification-statement-be-used
\newbox\linebox \newbox\snapbox
\def\eatlines{
  \setbox\linebox\lastbox % check the last line
  \ifvoid\linebox
  \else % if it’s not empty
    \unskip\unpenalty % take whatever is
    {\eatlines} % above it;
    \setbox\snapbox\hbox{\unhcopy\linebox}
    \ifdim\wd\snapbox<.98\wd\linebox
       \box\snapbox % take the one or the other,
    \else \box\linebox \fi
  \fi
}

% How could the `\everypar` justification statement be used?
% https://tex.stackexchange.com/questions/365818/how-could-the-everypar-justification-statement-be-used
\everypar={\setbox0=\lastbox \par
   \vbox\bgroup \everypar={}\def\par{\endgraf\eatlines\egroup}}

% Creates a new environment which can be used as:
%
% \begin{foo}
%   Text...
%
%   Text ...
% \end{foo}
%
% https://tex.stackexchange.com/questions/62333/push-long-words-in-a-new-line
\newenvironment{foo}
{\par
\hyphenpenalty=10000
\exhyphenpenalty=10000
}
{\par}



%
% New commands
%

% Allow to push long words on new lines when they do not fit entirely on the current line.
% https://tex.stackexchange.com/questions/62333/push-long-words-in-a-new-line
\newcommand\lword[1]{\leavevmode\nobreak\hskip0pt plus\linewidth\penalty50\hskip0pt plus-\linewidth\nobreak{#1}}


% For the new command \latex
\usepackage{xspace}

% Write the word LaTeX nicely.
\newcommand{\latex}{\LaTeX\xspace}


% Create a bold title all in upper case.
\newcommand{\Title}[1]{\textbf{\MakeUppercase{#1}}}








% Bad boxes settings and programming environments
\input{ufscthesisx/utilities/badboxes}


% Writing code in latex document. Usage: \begin & \end {lstlisting}
% http://stackoverflow.com/questions/3175105/writing-code-in-latex-document
\usepackage{listings}

% How to insert code with accents with listings?
% https://tex.stackexchange.com/questions/30512/how-to-insert-code-with-accents-with-listings
\usepackage{listingsutf8}

% Incompatible color definition when using tikz with color package
% https://tex.stackexchange.com/questions/150369/incompatible-color-definition-when-using-tikz-with-color-package
\usepackage{xcolor}

\definecolor{dkgreen}{rgb}{0,0.6,0}
\definecolor{gray}{rgb}{0.5,0.5,0.5}
\definecolor{mauve}{rgb}{0.58,0,0.82}

\lstset{frame=,
  language=Java,
  aboveskip=3mm,
  belowskip=3mm,
  showstringspaces=false,
  columns=flexible,
  basicstyle={\small\ttfamily},
  numbers=left,
  numberstyle=\color{gray},
  keywordstyle=\color{blue},
  commentstyle=\color{dkgreen},
  stringstyle=\color{mauve},
  breaklines=true,
  breakatwhitespace=true,
  tabsize=3
}

% Defining `lstset` parameters for multiple languages & How can I highlight YAML code in a pretty way with listings?
%
% Usage \begin{lstlisting}[style=yaml_style] ... \end{lstlisting}
%
% https://tex.stackexchange.com/questions/45711/defining-lstset-parameters-for-multiple-languages
% https://tex.stackexchange.com/questions/152829/how-can-i-highlight-yaml-code-in-a-pretty-way-with-listings
\newcommand\YAMLcolonstyle{\color{red}}
\newcommand\YAMLkeystyle{\color{black}}
\newcommand\YAMLvaluestyle{\color{blue}}
\newcommand\ProcessThreeDashes{\llap{\color{cyan}\mdseries-{-}-}}

\lstdefinestyle{yaml_style}{
  frame=,
  aboveskip=3mm,
  belowskip=3mm,
  showstringspaces=false,
  columns=flexible,
  numbers=left,
  numberstyle=\color{gray},
  breaklines=true,
  breakatwhitespace=true,
  tabsize=2,
  keywords={true,false,null,y,n},
  keywordstyle=\color{darkgray},
  basicstyle=\YAMLkeystyle,                                 % assuming a key comes first
  sensitive=false,
  comment=[l]{\#},
  morecomment=[s]{/*}{*/},
  commentstyle=\color{purple}\ttfamily,
  stringstyle=\YAMLvaluestyle\ttfamily,
  moredelim=[l][\color{orange}]{\&},
  moredelim=[l][\color{magenta}]{*},
  moredelim=**[il][\YAMLcolonstyle{:}\YAMLvaluestyle]{:},   % switch to value style at :
  morestring=[b]',
  morestring=[b]",
  literate = {---}{{\ProcessThreeDashes}}3
             {>}{{\textcolor{red}\textgreater}}1
             {|}{{\textcolor{red}\textbar}}1
             {\ -\ }{{\mdseries\ -\ }}3,
  inputencoding=utf8, % Listings in Latex with UTF-8 (or at least german umlauts)
  extendedchars=true, % http://stackoverflow.com/questions/1116266/listings-in-latex-with-utf-8-or-at-least-german-umlauts
  literate=%
  {é}{{\'{e}}}1
  {è}{{\`{e}}}1
  {ê}{{\^{e}}}1
  {ë}{{\¨{e}}}1
  {É}{{\'{E}}}1
  {Ê}{{\^{E}}}1
  {û}{{\^{u}}}1
  {ù}{{\`{u}}}1
  {ú}{{\'{u}}}1
  {â}{{\^{a}}}1
  {à}{{\`{a}}}1
  {á}{{\'{a}}}1
  {ã}{{\~{a}}}1
  {Á}{{\'{A}}}1
  {Â}{{\^{A}}}1
  {Ã}{{\~{A}}}1
  {ç}{{\c{c}}}1
  {Ç}{{\c{C}}}1
  {õ}{{\~{o}}}1
  {ó}{{\'{o}}}1
  {ô}{{\^{o}}}1
  {Õ}{{\~{O}}}1
  {Ó}{{\'{O}}}1
  {Ô}{{\^{O}}}1
  {î}{{\^{i}}}1
  {Î}{{\^{I}}}1
  {í}{{\'{i}}}1
  {Í}{{\~{Í}}}1
}




% Input a empty list of commands when on debug mode
\input{ufscthesisx/utilities/commands_list}



%----------------------------------------------------------------------------------------
%   File settings
%----------------------------------------------------------------------------------------

% Print page margins of a document
% https://tex.stackexchange.com/questions/14508/print-page-margins-of-a-document
% \usepackage[showframe,pass]{geometry}

% To use the font Times New Roman, instead of the default LaTeX font
% more up-to-date than '\usepackage{mathptmx}'
\usepackage{newtxtext}
\usepackage{newtxmath}

% Always use it as should improve full justification
% https://tex.stackexchange.com/questions/10377/texttt-overfull-hbox-problem
% https://tex.stackexchange.com/questions/66052/should-i-load-microtype-with-pdflatex
\usepackage{microtype}

% Thesis settings


% Thesis settings
\newcommand{\brazil}[1]{\foreignlanguage{brazil}{#1}}
\newcommand{\english}[1]{\foreignlanguage{english}{#1}}

\newcommand{\s}[0]{\textquotesingle{}s{ }}

% What is the difference between \def and \newcommand?
% https://tex.stackexchange.com/questions/655/what-is-the-difference-between-def-and-newcommand
\def\mytextpreliminarylistname{\chooselang{Brief Table of Contents}{Breve Sumário}}

% How to manually set where a word is split?
% https://tex.stackexchange.com/questions/182569/how-to-manually-set-where-a-word-is-split
\hyphenation{Ge-la-im}

% Informações de dados para CAPA e FOLHA DE ROSTO
\titulo
{%
    \chooselang
    {Good Programming Practices \& Style}
    {Boas Práticas de Programação \& Estilo}
}
\subtitulo
{%
    \chooselang
    {Universal Programming Tools}
    {Ferramentas Universais de Programação}
}

\data{\today}
\autor{\brazil{Evandro Coan}}
\local{\chooselang{\brazil{Florianópolis, Santa Catarina} -- Brazil}{Florianópolis, Santa Catarina -- Brasil}}

\biblioteca{\chooselang{University Library}{Biblioteca Universitária}}
\orientador{\chooselang{Prof. PhD. Ricardo Azambuja Silveira}{Prof. Dr. Ricardo Azambuja Silveira}}
\coorientador{\chooselang{M.S. Thiago Ângelo Gelaim}{M.S. Thiago Ângelo Gelaim}}

\instituicaosigla{UFSC}
\instituicao{\chooselang{Federal University of \brazil{Santa Catarina}}{Universidade Federal de Santa Catarina}}
\tipotrabalho{\chooselang{Bachelor's Thesis}{Tese de Graduação}}

\area{\chooselang{Formal Languages}{Linguagens Formais}}
\formacao{\chooselang{Bachelor of Science Degree in Computer Science}{Grau de Bacharel em Ciência da Computação}}
\programa{\chooselang{Undergraduate Program in Computer Science}{Trabalho de Conclusão de Curso}}
\centro{\chooselang{Department of Informatics and Statistics}{Departamento de Informática e Estatística}}

% O preambulo deve conter tipo do trabalho, objetivo, nome da instituição e a área de concentração.
\preambulo
{%
    \chooselang
    {Thesis submitted to the~\imprimirprograma~of the~\imprimirinstituicao~to obtain the~\imprimirformacao.}
    {Tese submetida ao \imprimirprograma da \imprimirinstituicao para a obtenção do Título de \imprimirformacao.}
}

% Keywords
\newcommand{\palavraschaveingles}
{%
    \item source. \item code. \item formatter. \item beautifier. \item prettyprint. \item universal.
    \item reuse. \item blocks. \item object. \item oriented. \item programming. \item structured.
    \item parsing. \item parse. \item regular. \item expression. \item regex. \item C. \item C++.
    \item grammar. \item Turing. \item machine. \item automata. \item lexer. \item syntax. \item
    Sublime. \item Java. \item Rust. \item Shell. \item script. \item obfuscators. \item learning.
    \item syntec. \item teamicide. \item concensus. \item indentação. \item settings.
}
\newcommand{\palavraschaveportugues}
{%
    \item fonte. \item código. \item formatador. \item embelezante. \item prettyprint. \item
    universal. \item reuso. \item blocos. \item objeto. \item orientado. \item programação. \item
    estruturada. \item análise. \item analisador. \item regular. \item expressão. \item regex. \item
    C. \item C++. \item gramática. \item Turing. \item máquina. \item autômatos. \item lexer. \item
    sintaxe. \item Sublime. \item Java. \item Rust. \item Shell. \item roteiro. \item ofuscadores.
    \item aprendizado. \item Syntec. \item teamicide. \item consenso. \item indentation. \item
    configurações.
}

% Remove the colon appended to theses variables, allowing us to use other separators
\addto\captionsbrazil
{
    \renewcommand{\orientadorname}{Orientador}
    \renewcommand{\coorientadorname}{Coorientador}
}

% Create caption English translations as the sections headers
% https://tex.stackexchange.com/questions/8564/what-is-the-right-way-to-redefine-macros-defined-by-babel
\addto\captionsenglish
{
    %% adjusts names from abnTeX2
    \renewcommand{\folhaderostoname}{Title page}
    \renewcommand{\epigraphname}{Epigraph}
    \renewcommand{\dedicatorianame}{Dedication}
    \renewcommand{\errataname}{Errata sheet}
    \renewcommand{\agradecimentosname}{Acknowledgements}
    \renewcommand{\anexoname}{ANNEX}
    \renewcommand{\anexosname}{Annex}
    \renewcommand{\apendicename}{APPENDIX}
    \renewcommand{\apendicesname}{Appendix}
    \renewcommand{\orientadorname}{Supervisor}
    \renewcommand{\coorientadorname}{Co\hyp{}supervisor}
    \renewcommand{\folhadeaprovacaoname}{Approval}
    \renewcommand{\resumoname}{Abstract}
    \renewcommand{\listadesiglasname}{List of abbreviations and acronyms}
    \renewcommand{\listadesimbolosname}{List of symbols}
    \renewcommand{\fontename}{Source}
    \renewcommand{\notaname}{Note}
    %% adjusts names used by \autoref
    \renewcommand{\pageautorefname}{page}
    \renewcommand{\sectionautorefname}{section}
    \renewcommand{\subsectionautorefname}{subsection}
    \renewcommand{\subsubsectionautorefname}{subsubsection}
    \renewcommand{\paragraphautorefname}{subsubsubsection}
}

% Source Code Settings in Document
\makeatletter
\@ifpackageloaded{listings}
{
\ifenglish
    % These default values are already in English
\else
    % Listing -> Codigo fonte
    \renewcommand{\lstlistingname}{Código--fonte}

    % List of Listings -> Lista de códigos-fonte
    \renewcommand{\lstlistlistingname}{Lista de códigos--fonte}

    % Calculate the size of the header
    \calculatelisteningsheader
\fi
}{}
\makeatother


% Backref package settings, pages with citations in bibliography
\makeatletter
\@ifpackageloaded{backref}
{
\ifenglish
    % Used without the backref hyperpageref option
    \renewcommand{\backrefpagesname}{Cited on page(s):~}

    % Default text before page number
    \renewcommand{\backref}{}

    % Sets the text of the citation
    \renewcommand*{\backrefalt}[4]
    {
        \ifcase #1
            No citation in the text.
        \or
            Cited on page #2.
        \else
            Cited #1 times on pages #2.
        \fi
    }
\else
    % Usado sem a opção hyperpageref de backref
    \renewcommand{\backrefpagesname}{Citado na(s) página(s):~}

    % Texto padrão antes do número das páginas
    \renewcommand{\backref}{}

    % Define os textos da citação
    \renewcommand*{\backrefalt}[4]
    {
        \ifcase #1
            Nenhuma citação no texto.
        \or
            Citado na página #2.
        \else
            Citado #1 vezes nas páginas #2.
        \fi
    }
\fi
}{}
\makeatother


% Espaçamentos entre linhas e parágrafos
%
% ifpackageloaded question
% https://tex.stackexchange.com/questions/70212/ifpackageloaded-question
\makeatletter
\@ifclassloaded{memoir}
{
    % Estilo de capítulos, ver classe para maiores detalhes.Veja outros estilos em:
    % http://mirrors.ibiblio.org/CTAN/macros/latex/contrib/memoir/memman.pdf
    \chapterstyle{VZ14}
    \setlength\beforechapskip{0pt}
    \setlength\midchapskip{15pt}
    \setlength\afterchapskip{15pt}

    % O tamanho do parágrafo é dado por:
    \setlength{\parindent}{1.3cm}

    % Controle do espaçamento entre um parágrafo e outro. Tente também
    % \onelineskip
    \setlength{\parskip}{0.2cm}

    % Memoir: Warnings “The material used in the headers is too large” w/ accented titles
    % https://tex.stackexchange.com/questions/387293/how-to-change-the-page-layout-with-memoir
    \setheadfoot{30.0pt}{\footskip}
    \checkandfixthelayout
}{}
\makeatother


% Color settings across the document
\makeatletter
\@ifpackageloaded{xcolor}
{
    % RGB colors in absolute values from 0 to 255 by using `RGB` tag
    \definecolor{darkblue}{RGB}{26,13,178}

    % Definição de cores, RGB colors in percentage notation by using `rgb` tag
    \definecolor{mygreen}{rgb}{0,0.6,0}
    \definecolor{mygray}{rgb}{0.5,0.5,0.5}
    \definecolor{mymauve}{rgb}{0.58,0,0.82}

    % Configurações de aparência do PDF final
    \definecolor{figcolor}{rgb}{1,0.4,0}  % orange
    \definecolor{tabcolor}{rgb}{1,0.4,0}  % orange
    \definecolor{eqncolor}{rgb}{1,0.4,0}  % orange
    \definecolor{linkcolor}{rgb}{1,0.4,0} % orange
    \definecolor{citecolor}{rgb}{1,0.4,0} % orange
    \definecolor{seccolor}{rgb}{0,0,1}    % blue
    \definecolor{abscolor}{rgb}{0,0,1}    % blue
    \definecolor{titlecolor}{rgb}{0,0,1}  % blue
    \definecolor{biocolor}{rgb}{0,0,1}    % blue

    % Alterando o aspecto da cor azul
    \definecolor{blue}{RGB}{41,5,195}

    % Informações do PDF
    \@ifpackageloaded{hyperref}
    {
        \hypersetup
        {
            pdftitle={\@title},
            colorlinks=true, % false: boxed links; true: colored links
            linkcolor=darkblue, % color of internal links
            citecolor=darkgreen, % color of links to bibliography
            filecolor=black, % color of file links
            urlcolor=linkcolor,
            bookmarksdepth=4
        }
        \ifenglish
            \hypersetup
            {
                pdfauthor={Author},
                pdfsubject={Thesis' Abstract},
                pdfcreator={LaTeX with abnTeX2 for UFSC},
                pdfkeywords={abnt}{latex}{UFSC}{abntex2}{thesis},
            }
        \else
            \hypersetup
            {
                pdfauthor={Autores},
                pdfsubject={Resumo da tese},
                pdfcreator={LaTeX com abnTeX2 para UFSC},
                pdfkeywords={abnt}{latex}{UFSC}{abntex2}{tese},
            }
        \fi
    }
}{}
\makeatother


% Fontes das entradas do sumario
\makeatletter
\renewcommand*{\l@chapter}[2]
{%
    \l@chapapp{\uppercase{#1}}{#2}{\cftchaptername}
}
\renewcommand*{\l@section}[2]
{%
    \l@chapapp{\ABNTEXsectionfont\uppercase{#1}}{#2}{\cftsectionname}
}
\makeatother

% Changing the font of the numbers in the ToC in the memoir class
% https://tex.stackexchange.com/questions/14314/changing-the-font-of-the-numbers-in-the-toc-in-the-memoir-class
\renewcommand{\cftpartfont}{\ABNTEXpartfont\color{darkblue}}
\renewcommand{\cftpartpagefont}{\ABNTEXpartfont\color{black}}

\renewcommand{\cftchapterfont}{\ABNTEXchapterfont\color{darkblue}}
\renewcommand{\cftchapterpagefont}{\ABNTEXchapterfont\color{black}}

\renewcommand{\cftsectionfont}{\ABNTEXsectionfont\color{darkblue}}
\renewcommand{\cftsectionpagefont}{\ABNTEXsectionfont\color{black}}

\renewcommand{\cftsubsectionfont}{\ABNTEXsubsectionfont\color{darkblue}}
\renewcommand{\cftsubsectionpagefont}{\ABNTEXsubsectionfont\color{black}}

\renewcommand{\cftsubsubsectionfont}{\ABNTEXsubsubsectionfont\color{darkblue}}
\renewcommand{\cftsubsubsectionpagefont}{\ABNTEXsubsubsectionfont\color{black}}

\renewcommand{\cftparagraphfont}{\ABNTEXsubsubsubsectionfont\color{darkblue}}
\renewcommand{\cftparagraphpagefont}{\ABNTEXsubsubsubsectionfont\color{black}}




% When writing a large document, it is sometimes useful to work on selected sections of the document
% to speed up compilation time: https://en.wikibooks.org/wiki/TeX/includeonly
%
% \includeonly{pretexto/agradecimentos}
% \includeonly{pretexto/epigrafe}
% \includeonly{pretexto/fichacatalografica}
% \includeonly{pretexto/folhadeaprovacao}
% \includeonly{pretexto/resumos}
% \includeonly{pretexto/siglas}
% \includeonly{pretexto/simbolos}

% \includeonly{chapters/chapter_1}
% \includeonly{chapters/chapter_2}
% \includeonly{chapters/conclusion}

% \includeonly{postexto/anexo_a}
% \includeonly{postexto/apendice_a}



% %----------------------------------------------------------------------------------------
% %   DOCUMENT CONTENTS
% %----------------------------------------------------------------------------------------

\begin{document}

    % Comment this after finishing the thesis, so you can start fixing the \flushbottom vs \raggedbottom
    % https://tex.stackexchange.com/questions/65355/flushbottom-vs-raggedbottom
    \raggedbottom

    % Retira espaço extra obsoleto entre as frases `Double space between sentences`
    % https://tex.stackexchange.com/questions/4705/double-space-between-sentences
    \frenchspacing

    

% How to fix destination with the same identifier (name{page.A}) has been already used, duplicate ignored?
% https://tex.stackexchange.com/questions/386446/how-to-fix-destination-with-the-same-identifier-namepage-a-has-been-already
\hypersetup{pageanchor=false}


% ELEMENTOS PRÉ-TEXTUAIS
% \includepdf{pictures/FrenteCapaUFSC.pdf}

% To automatically put a [Go To Top] on each section
\addGoToSummary

\pretextual

% Capa
\imprimircapa

% Folha de rosto (o * indica que haverá a ficha bibliográfica)
\imprimirfolhaderosto*

% Inserir a ficha bibliografica
%
% Isto é um exemplo de Ficha Catalográfica, ou ``Dados internacionais de
% catalogação-na-publicação''. Você pode utilizar este modelo como referência.
% Porém, provavelmente a biblioteca da sua universidade lhe fornecerá um PDF
% com a ficha catalográfica definitiva após a defesa do trabalho. Quando estiver
% com o documento, salve-o como PDF no diretório do seu projeto e substitua todo
% o conteúdo de implementação deste arquivo pelo comando abaixo:
%
% \begin{fichacatalografica}
%    \includepdf{pretexto/ficha_catalografica.pdf}
% \end{fichacatalografica}


\ifenglish

Legal Notes

There is no warranty for any part of the documented software. The authors have taken care in the
preparation of this thesis, but make no expressed or implied warranty of any kind and assume no
responsibility for errors or omissions. No liability is assumed for incidental or consequential
damages in connection with or arising out of the use of the information or programs contained here.
\cite{koma-scrguien}

\else

Notas legais

Não há garantia para qualquer parte do software documentado. Os autores tomaram cuidado na
preparação desta tese, mas não fazem nenhuma garantia expressa ou implícita de qualquer tipo e não
assumem qualquer responsabilidade por erros ou omissões. Não se assume qualquer responsabilidade por
danos incidentais ou consequentes em conexão ou decorrentes do uso das informações ou programas aqui
contidos. \cite{koma-scrguien}

\fi


% http://portalbu.ufsc.br/ficha
% http://portal.bu.ufsc.br/servicos/ficha-de-identificacao-da-obra/
\begin{fichacatalografica}
    \vspace*{\fill}

    \begin{center}

        \chooselang
        {Cataloging at source by the University Library of the Federal University of Santa Catarina.}
        {Catalogação na fonte pela Biblioteca Universitária da Universidade Federal de Santa Catarina.}

        \chooselang
        {File compiled at \currenttime h of the day \today.}
        {Arquivo compilado às \currenttime h do dia \today.}

        \framebox[\textwidth]
        {
            \begin{minipage}{0.98\textwidth}

                \ttfamily
                \imprimirautor

                \hspace{0.5cm} \imprimirtitulo~:~\imprimirsubtitulo~/~\imprimirautor;
                \imprimirorientadorRotulo,~\imprimirorientador;~\imprimircoorientadorRotulo,~\imprimircoorientador
                ~--~\imprimirlocal,~\currenttime,~\imprimirdata.

                % Prints how much pages there are on the document and links to the last page
                \hspace{0.5cm} \pageref{LastPage} p.
                \bigskip

                \hspace{0.5cm} \imprimirtipotrabalho~--~\imprimirinstituicao,
                \imprimircentro,~\imprimirprograma.
                \bigskip

                \hspace{0.5cm} \chooselang{Includes references}{Inclui referências}
                \bigskip

                \hspace{0.5cm}
                \begin{inparaenum}
                    \chooselang{\palavraschaveingles}{\palavraschaveportugues}
                \end{inparaenum}
                I. \imprimirorientador~
                II. \imprimircoorientador~
                III. \imprimirprograma~
                IV. \imprimirtitulo~
                \bigskip

                \hspace{7.75cm} CDU 02:141:005.7

            \end{minipage}
        }

    \end{center}

\end{fichacatalografica}



% Inserir errata

% Inserir folha de aprovação

% Isto é um exemplo de Folha de aprovação, elemento obrigatório da NBR
% 14724/2011 (seção 4.2.1.3). Você pode utilizar este modelo até a aprovação
% do trabalho. Após isso, substitua todo o conteúdo deste arquivo por uma
% imagem da página assinada pela banca com o comando abaixo:
% \includepdf{folhadeaprovacao_final.pdf}


\addtotextpreliminarycontent{\lang{Approval Sheet}{Folha de Aprovação}}

\begin{folhadeaprovacao}

    \begin{center}
        {\imprimirautor}

        \begin{center}
            \ABNTEXchapterfont\bfseries\MakeUppercase{\imprimirtitulo}\ifnotempty{\imprimirsubtitulo}{: \imprimirsubtitulo}
        \end{center}

        \begin{minipage}{\textwidth}
            \lang
            {
                This \imprimirtipotrabalho~ was considered appropriate to get the \imprimirformacao,
                \ifnotempty{\imprimirarea}{in the area of \imprimirarea,}
                and it was approved by the \imprimirprograma~ of \imprimircentro~ of \imprimirinstituicao.
            }
            {
                Este(a) \imprimirtipotrabalho~ foi julgado adequado(a) para obtenção do Título de \imprimirformacao,
                \ifnotempty{\imprimirarea}{na área de concentração \imprimirarea,}
                e foi aprovado em sua forma final pelo \imprimirprograma~
                do \imprimircentro~ da \imprimirinstituicao.
            }
         \end{minipage}%
    \end{center}

    \begin{center}
        \imprimirlocal, \imprimirdata.
    \end{center}

    \assinatura{%
        \textbf{\imprimircoordenador} \\
        \imprimircoordenadorRotulo~\lang{of}{do} \imprimirprograma
    }

    % \newpage
    \begin{flushleft}
        \textbf{\lang{Examination Board}{Banca Examinadora}:}
    \end{flushleft}

    \assinatura{%
        \textbf{\imprimirorientador} \\ \imprimirorientadorRotulo\\
        \imprimirinstituicao~--~\imprimirinstituicaosigla
    }

    \ifnotempty{\imprimircoorientador}{%
        \assinatura{%
            \textbf{\imprimircoorientador} \\ \imprimircoorientadorRotulo \\
            \imprimirinstituicao~--~\imprimirinstituicaosigla
        }
    }

    \assinatura{%
        \textbf{Prof. Convidado 1, \lang{PhD.}{Dr.}} \\
        Instituição 1 -- Sigla 1
    }

    \assinatura{%
        \textbf{Prof. Convidado 2, \lang{PhD.}{Dr.}} \\
        Instituição 2 -- Sigla 2
    }

\end{folhadeaprovacao}



% Dedicatória
\begin{dedicatoria}
    \vspace*{\fill}
    \centering
    \noindent
    \textit{ Este trabalho é dedicado às crianças adultas que,\\
        quando pequenas, sonharam em se tornar cientistas.} \vspace*{\fill}
\end{dedicatoria}

% Agradecimentos
\include{pretexto/agradecimentos}

% Epígrafe


\addtotextpreliminarycontent{\lang{Epigraph}{Epigrafe}}

\begin{epigrafe}

\vspace*{\fill}\lang
{
    \begin{flushright}
        \textit{``Learn from yesterday, live for today, hope for tomorrow. The important thing is not to stop questioning.''} \\ Albert Einstein
    \end{flushright}
    \begin{flushright}
        \textit{``The true sign of intelligence is not knowledge but imagination.''} \\  Albert Einstein
    \end{flushright}
    \begin{flushright}
        \textit{``Peace cannot be kept by force; it can only be achieved by understanding.''} \\ Albert Einstein
    \end{flushright}
    \begin{flushright}
        \textit{``Whoever is careless with the truth in small matters cannot be trusted with important matters.''} \\ Albert Einstein
    \end{flushright}
    \begin{flushright}
        \textit{``Extraordinary claims require extraordinary evidence''} \\ Carl Sagan
    \end{flushright}
    \begin{flushright}
        \textit{``Catholic, which I was until I reached the age of reason.''} \\ George Carlin
    \end{flushright}
    \begin{flushright}
        \textit{``We made too many wrong mistakes.''} \\ Yogi Berra
    \end{flushright}
}
{
    \begin{flushright}
        \textit{``Assim como aquele pecado da juventude, este documento te perseguirá pelo resto da vida. \showfont''} \\ Enio Valmor Kassick
    \end{flushright}
    \begin{flushright}
        \textit{``Estupidez trará mais autoconfiança do que o conhecimento e a bravura juntas.''} \\ Adriano Ruseler
    \end{flushright}
}

\end{epigrafe}




% RESUMOS
%
% Ajusta o espaçamento dos parágrafos do resumo
\setlength{\absparsep}{18pt}


\newcommand{\imprimirbrazilabstract}{%
    \cleardoublepage\phantomsection
    \addtotextpreliminarycontent{Resumo em Português}
    \begin{otherlanguage*}{brazil}
    \begin{resumo}[Resumo]

        Faz~=se um estudo sobre o que é e
        para que servem os Formatadores de Código,
        assim como as abordagens utilizadas nas mais variadas Ferramentas de Formatação de Código.
        Os softwares Formatadores de Código~=Fonte atuais,
        também conhecidos como \textit{Source Code Beautifiers},
        são limitados a um conjunto similar,
        ou mesmo à uma única linguagem de programação,
        além de muitos serem limitados no que eles podem fazer ao formatar o código~=fonte.
        Nesse contexto,
        propõe~=se uma ferramenta que permita por meio de gramáticas,
        a especificação de quais linguagens de programação deseja~=se realizar a formatação.
        Assim,
        centralizando em um único programa a formatação de código das mais diversas linguagens de programação pela especificação de suas gramáticas.

        \imprimirpalavraschave{Palavras~=chaves}{\begin{inparaitem}[]\palavraschaveportugues\end{inparaitem}}

    \end{resumo}
    \end{otherlanguage*}
}


\newcommand{\imprimirenglishabstract}{%
    % https://tex.stackexchange.com/questions/20987/changing-babel-package-inside-a-single-chapter
    % https://tex.stackexchange.com/questions/36526/multiple-language-document-babel-selectlanguage-vs-begin-endotherlanguage
    \cleardoublepage\phantomsection
    \addtotextpreliminarycontent{English's Abstract}
    \begin{otherlanguage*}{english}
    \begin{resumo}[Abstract]

        A study about nowadays used Source Code Formatters,
        and as well,
        the programming algorithms used in most Source Code Formatting Tools.
        Cutting edge Source Code Formatting Softwares,
        also known as Source Code Beautifiers,
        are limited to a common set,
        or even to a single programming language,
        and many formatters are limited in what they can do.
        In this context,
        it is also proposed a new Source Code Formatting Tool allowing users to input their favourite languages grammars.
        Therefore,
        formatting all programming languages they would like a single formatting software,
        by the input of their languages grammars specification.

        \imprimirpalavraschave{Keywords}{\begin{inparaitem}[]\palavraschaveingles\end{inparaitem}}

    \end{resumo}
    \end{otherlanguage*}
}


% \newcommand{\imprimirfrenchabstract}{%
%     \addtotextpreliminarycontent{Français Résumé}
%     \begin{resumo}[Résumé]
%       \begin{otherlanguage*}{french}
%           Il s'agit d'un résumé en français.

%           \imprimirpalavraschave{Mots-clés}{latex. abntex. publication de textes.}
%       \end{otherlanguage*}
%     \end{resumo}
% }


% \newcommand{\imprimirspanishabstract}{%
%     \addtotextpreliminarycontent{Español Resumen}
%     \begin{resumo}[Resumen]
%       \begin{otherlanguage*}{spanish}
%           Este es el resumen en español.

%           \imprimirpalavraschave{Palabras clave}{latex. abntex. publicación de textos.}
%       \end{otherlanguage*}
%     \end{resumo}
% }


\makeatletter
\ifenglish
    \@ifundefined{imprimirbrazilabstract}{}{\imprimirbrazilabstract}

    % https://tex.stackexchange.com/questions/331108/times-new-roman-in-latex-just-some-text
    % https://tex.stackexchange.com/questions/11707/how-to-force-output-to-a-left-or-right-page
    % https://tex.stackexchange.com/questions/132966/do-not-display-chapter-title-in-memoir-class
    \cleardoublepage\phantomsection
    \pretextualchapter{Resumo Expandido}
    \addtotextpreliminarycontent{Resumo Expandido}

    \begin{otherlanguage*}{brazil}
        \setlength{\parskip}{0.2cm}
        \setlength{\parindent}{0.0cm}
        \fontfamily{ptm}\selectfont

        \section*{Introdução}
        O resumo expandido é previsto na Resolução Normativa nº 95/CUn/2017, Art. 55, § 2, de 4 de
        abril de 2017, e exigido para teses e dissertações escritas em idiomas estrangeiros (com
        exceção dos cursos pertinentes ao estudo de idiomas estrangeiros – Programa de Pós-Graduação
        em Estudos da Tradução e Programa de Pós-Graduação em Inglês: Estudos Linguísticos e
        Literários).

        O resumo expandido é considerado um elemento pré-textual e deverá ser incluído no trabalho
        após o resumo e antes do abstract. Deverá iniciar em página impar (no anverso de uma folha)
        continuando no verso da folha. O texto deverá seguir o formato A5, com margens espelhadas:
        superior 2,0 cm, inferior 1,5 cm, interna 2,5 cm e externa 1,5. Deve ser empregada a fonte
        Time New Roman.  Todo o texto deve ser digitado em tamanho 10,5. O espaçamento entre as
        linhas deverá ser simples. A expressão “resumo expandido” deve seguir a mesma tipografia das
        demais sessões primárias do trabalho.

        O texto do resumo expandido deve ser redigido em português e conter as seguintes seções (ver
        modelo): Introdução, Objetivos, Metodologia, Resultados e Discussão e Considerações Finais.
        Deve apresentar no mínimo duas (02) e, no máximo, cinco (05) páginas contendo a mesma
        formatação em A5 do resumo e do abstract, bem como palavras-chave. \englishword{\showfont}

        \section*{Objetivos}
        Lorem ipsum dolor sit amet, consectetur adipiscing elit. Phasellus vitae dolor lacus. Ut
        accumsan vitae felis nec porttitor. Integer interdum fringilla feugiat. Nullam pulvinar sit
        amet tellus eget maximus. Donec sit amet magna eget justo semper fermentum vel eget velit.
        In iaculis imperdiet mauris, ac ornare libero placerat non. Nulla libero lectus, ullamcorper
        ac ornare eget, pulvinar ac nulla. Curabitur vestibulum non nisl eget sagittis. Proin
        gravida lacus id eros bibendum interdum. Mauris ullamcorper elementum tortor sed consequat.
        Integer tempus, est a lobortis vehicula, nisi mi fringilla augue, non semper leo metus in
        quam. Etiam in leo maximus, pulvinar mi eget, vehicula risus. Donec sed dui semper, dictum
        eros at, suscipit felis.

        Nam sagittis vel orci at tempus. Nulla non pellentesque eros.
        Quisque cursus leo massa, eu ultricies nisl lacinia a. Nulla sit amet elementum ligula.
        Proin sodales venenatis dictum. Ut et est cursus, vulputate velit et, viverra odio. Interdum
        et malesuada fames ac ante ipsum primis in faucibus. Maecenas purus diam, tempor a semper
        et, finibus a ex. Cras sagittis felis urna, et consequat arcu lacinia ut. Praesent blandit
        venenatis ante nec porta. Duis rutrum, tellus vitae ullamcorper auctor, lectus ex laoreet
        est, ac tristique ipsum arcu vitae nibh. Nam efficitur felis ut mi consectetur, nec auctor
        odio ornare. In tempor vulputate urna, vitae cursus enim egestas eu. Proin diam augue,
        dignissim vitae ligula eget, lobortis ornare odio. Duis quis elit augue. Fusce quis rhoncus
        tortor. Donec hendrerit at massa a mattis. Sed ipsum neque, aliquam ut sem sed, ultrices
        varius ligula. Suspendisse blandit, dolor ac rhoncus lacinia, dolor purus cursus purus, et
        accumsan orci neque a leo.

        \section*{Metodologia}
        Quisque efficitur dolor in lectus dapibus elementum. Nam ultrices blandit consectetur.
        Nullam ultricies sit amet odio quis placerat. Aenean eget est elit. Maecenas et nulla dolor.
        Orci varius natoque penatibus et magnis dis parturient montes, nascetur ridiculus mus. In
        pulvinar velit sed mi sagittis ornare. Aenean rutrum suscipit egestas. Phasellus pharetra
        eget ex in volutpat. Quisque eu arcu nunc. Vivamus arcu ligula, pharetra at rhoncus sit
        amet, pulvinar sed eros. Sed porta ipsum ipsum, et fermentum magna volutpat sed. Vivamus
        pharetra facilisis orci, sit amet luctus nisl pretium id. Sed consequat, arcu et congue
        pulvinar, risus enim aliquet purus, eget venenatis libero leo sit amet metus. Maecenas vitae
        elit sapien. Fusce mollis libero et gravida placerat. Proin ut quam quis justo aliquam
        dictum. Donec volutpat convallis suscipit. Vivamus metus nisl, placerat ac enim vitae,
        tempus ultricies odio.

        Aliquam ac vehicula arcu, non bibendum nulla. Morbi libero sem,
        imperdiet vel quam et, posuere tempus nunc. Maecenas dictum magna sit amet ligula facilisis
        commodo. Aliquam tellus diam, ornare vel elementum in, dignissim id purus. Ut at tortor non
        sem molestie euismod non at turpis. Phasellus vitae bibendum tellus. Suspendisse odio enim,
        faucibus eget congue quis, semper sit amet tortor. Sed ac lectus est. Pellentesque nec
        mattis mi, et varius dolor. Aliquam quis massa ac tellus malesuada sollicitudin. Maecenas
        ultrices risus massa, nec auctor risus sagittis id. Praesent a sapien nulla. Donec
        tincidunt, metus quis hendrerit facilisis, enim augue convallis elit, sed consequat lacus
        odio vitae magna.

        \section*{Resultados e Discussão}
        Nullam sed cursus leo. Donec commodo volutpat hendrerit. Fusce et tempus lectus, feugiat
        consequat est. Class aptent taciti sociosqu ad litora torquent per conubia nostra, per
        inceptos himenaeos. Nam quis cursus mauris, non tempus orci. Phasellus lobortis et mauris at
        vulputate. Sed nec nisl elementum lorem commodo gravida non a enim. Phasellus neque erat,
        aliquet ac ligula ac, maximus vestibulum sem. Vestibulum vel tincidunt turpis. Donec lacinia
        rutrum dolor dapibus bibendum. Mauris pharetra nibh nec tincidunt iaculis. Vivamus pharetra
        bibendum nisl eget blandit. In lobortis diam non justo eleifend, id lobortis ante fringilla.
        Donec libero tortor, suscipit vestibulum vestibulum id, rutrum accumsan turpis. Phasellus
        sollicitudin luctus tincidunt. Suspendisse potenti. Nam semper metus et mi pharetra, in
        pretium ligula fermentum. Integer consectetur, orci non placerat feugiat, dui ex gravida
        augue, vel placerat ligula augue vel velit. Aliquam sollicitudin pellentesque congue. Donec
        vitae turpis in ante posuere posuere. Pellentesque eu justo leo. Donec quis elit vitae leo
        varius luctus quis eget justo.

        Vestibulum elementum ex neque, quis commodo tortor porttitor
        mattis. Mauris vel sagittis turpis. Aenean ligula turpis, eleifend at felis sed, cursus
        condimentum orci. Fusce accumsan est odio, eu venenatis massa sodales in. Curabitur a tempor
        nisl. Quisque consequat sed arcu a congue. In viverra, ex ut hendrerit condimentum, urna sem
        euismod eros, nec suscipit turpis dolor eget augue. Aenean posuere tellus et consectetur
        condimentum. Mauris et massa et nulla fringilla interdum. Duis quis posuere elit. Donec at
        ex non arcu faucibus rutrum et vel lectus. Vivamus pellentesque vestibulum rutrum. Sed
        pretium, purus sed efficitur feugiat, nisi justo eleifend nibh, id suscipit nunc massa nec
        lectus. In euismod enim eu sapien dictum sodales. Fusce sit amet vulputate orci. Nulla
        rutrum mauris at purus aliquet, ac sollicitudin leo laoreet. Etiam elementum posuere
        feugiat. Maecenas sed libero non augue fermentum ultricies eget at mi. Aenean auctor
        bibendum lacus, dignissim aliquet est tempus eget. Maecenas tempus, nulla id rhoncus
        suscipit, augue leo auctor mi, eget tincidunt magna mi quis dui. Maecenas ut elit in turpis
        tincidunt ultrices. Nulla id nulla aliquet, porttitor eros quis, egestas justo. Nunc nisi
        quam, egestas a accumsan fermentum, ultricies ac elit.

        Nulla porta auctor vestibulum. Sed
        consectetur lacus molestie iaculis ullamcorper. Proin porta posuere massa a lacinia. Nunc a
        lacinia orci, non vehicula ante. Vestibulum ipsum velit, congue et neque aliquam, imperdiet
        ornare augue. Donec et congue sapien. Pellentesque consequat consectetur neque ut varius. In
        aliquam ex quis ante venenatis dapibus. Vivamus et imperdiet urna. Vestibulum quis nibh
        magna. In a congue lectus, eu sodales nunc. Suspendisse id.

        \section*{Considerações Finais}
        Lorem ipsum dolor sit amet, consectetur adipiscing elit. Phasellus vitae dolor lacus. Ut
        accumsan vitae felis nec porttitor. Integer interdum fringilla feugiat. Nullam pulvinar sit
        amet tellus eget maximus. Donec sit amet magna eget justo semper fermentum vel eget velit.
        In iaculis imperdiet mauris, ac ornare libero placerat non. Nulla libero lectus, ullamcorper
        ac ornare eget, pulvinar ac nulla. Curabitur vestibulum non nisl eget sagittis. Proin
        gravida lacus id eros bibendum interdum. Mauris ullamcorper elementum tortor sed consequat.
        Integer tempus, est a lobortis vehicula, nisi mi fringilla augue, non semper leo metus in
        quam. Etiam in leo maximus, pulvinar mi eget, vehicula risus. Donec sed dui semper, dictum
        eros at, suscipit felis.

        Nam sagittis vel orci at tempus. Nulla non pellentesque eros.
        Quisque cursus leo massa, eu ultricies nisl lacinia a. Nulla sit amet elementum ligula.
        Proin sodales venenatis dictum. Ut et est cursus, vulputate velit et, viverra odio. Interdum
        et malesuada fames ac ante ipsum primis in faucibus. Maecenas purus diam, tempor a semper
        et, finibus a ex. Cras sagittis felis urna, et consequat arcu lacinia ut. Praesent blandit
        venenatis ante nec porta. Duis rutrum, tellus vitae ullamcorper auctor, lectus ex laoreet
        est, ac tristique ipsum arcu vitae nibh. Nam efficitur felis ut mi consectetur, nec auctor
        odio ornare. In tempor vulputate urna, vitae cursus enim egestas eu. Proin diam augue,
        dignissim vitae ligula eget, lobortis ornare odio. Duis quis elit augue. Fusce quis rhoncus
        tortor. Donec hendrerit at massa a mattis. Sed ipsum neque, aliquam ut sem sed, ultrices
        varius ligula. Suspendisse blandit, dolor ac rhoncus lacinia, dolor purus cursus purus, et
        accumsan orci neque a leo.


        \imprimirpalavraschave{Palavras-chaves}{\begin{inparaitem}[]\palavraschaveportugues\end{inparaitem}}

    \end{otherlanguage*}

    \@ifundefined{imprimirenglishabstract}{}{\imprimirenglishabstract}

\else
    \@ifundefined{imprimirbrazilabstract}{}{\imprimirbrazilabstract}
    \@ifundefined{imprimirenglishabstract}{}{\imprimirenglishabstract}
\fi

\@ifundefined{imprimirfrenchabstract}{}{\imprimirfrenchabstract}
\@ifundefined{imprimirspanishabstract}{}{\imprimirspanishabstract}
\makeatother



% inserir lista de ilustrações
\pdfbookmark[0]{\listfigurename}{lof}
\listoffigures*
\cleardoublepage

% inserir lista de tabelas
\pdfbookmark[0]{\listtablename}{lot}
\listoftables*
\cleardoublepage

% inserir códigos fonte
% ---
\pdfbookmark[0]{\lstlistingname}{lol}
\lstlistoflistings*
\cleardoublepage

% inserir lista de abreviaturas e siglas


\addtotextpreliminarycontent{\chooselang{List of Acronyms}{Lista de Siglas}}

\begin{siglas}
    \item[ABNT] Associação Brasileira de Normas Técnicas \chooselang{}{, Brazilian Association of Technical Standards}
    \item[abnTeX] ABsurdas Normas para TeX \chooselang{}{, Absurd Standards for TeX}
\end{siglas}



% Inserir lista de símbolos


\addtotextpreliminarycontent{\chooselang{List of Symbols}{Lista de Símbolos}}

% Devam aparecer na mesma ordem de ocorrência no texto.
\begin{simbolos}
    \item[$ \Gamma $] \chooselang{Greek letter Gama}{Letra grega Gama}
    \item[$ \Lambda $] \chooselang{Lambda}{Lambda}
    \item[$ \zeta $] \chooselang{Minimal Greek letter zeta}{Letra grega minúscula zeta}
    \item[$ \in $] \chooselang{Belongs}{Pertence}
\end{simbolos}


% Disable the [Go To Top] on the table of contents section
\removeGoToSummary

% Inserir o sumario
\pdfbookmark[0]{\contentsname}{toc}
\tableofcontents*
\cleardoublepage

% To automatically put a [Go To Top] on each section
\addGoToSummary


% How to fix destination with the same identifier (name{page.A}) has been already used, duplicate ignored?
% https://tex.stackexchange.com/questions/386446/how-to-fix-destination-with-the-same-identifier-namepage-a-has-been-already
\hypersetup{pageanchor=true}






    % ELEMENTOS TEXTUAIS
    %
    % Configura estilo das páginas.
    \textual

    % To automatically put a [Go To Top/Back] ←← | ← on each section
    \addGoToSummary

    \setlength\beforechapskip{50pt}
    \setlength\midchapskip{20pt}
    \setlength\afterchapskip{20pt}

    % Introdução (exemplo de capítulo sem numeração, mas presente no Sumário)
    

% The \phantomsection command is needed to create a link to a place in the document that is not a
% figure, equation, table, section, subsection, chapter, etc.
%
% When do I need to invoke \phantomsection?
% https://tex.stackexchange.com/questions/44088/when-do-i-need-to-invoke-phantomsection
\cleardoublepage
\phantomsection


% Is it possible to keep my translation together with original text?
% https://tex.stackexchange.com/questions/5076/is-it-possible-to-keep-my-translation-together-with-original-text
\chapter{\chooselang{Introduction}{Introdução}}


\chooselang
{
    Questions like ``What are good programming practices?'' Or ``Why are these practices are good?''
    Are not easy to answer. But each programmer learns to write their codes in a certain way, with
    certain features like using 4 or 8 spaces to indent lines, always leave a blank line before each
    control structure as if or for statements, and alike rules. % TODO, put reference for this

    But entering the universe of good practices, there are many things for discoursing. So in this
    work implementation tool called ``Object Beautifier'' specifically dedicates on how to perform
    the best layout/display of programming code on the computer screen, so that maximize and
    facilitate the understanding of same. Therefore, allowing the programmer to disperse more tempe
    thinking about its coding algorithms problem, other than trying to decipher the information that
    is presented to it on the screen through infinit different code layouts. % TODO, put reference for this

    Within this work\textquotesingle s area, we need to also think long and hard about how to share
    the programming code of the programmers among you. Now, the problem of human diversity, like all
    big scientific questions -- how do you explain something like that -- It can be broken down into
    sub-questions. It happens many times, which is a good practice for a `Programmer A', is not the
    same to another `Programmer B'. For example, imagine some code where a programmer decided to put
    before each `if' statement, a blank line. It is therefore expected that whenever we see a blank
    line we can potentially find a matching `if', which can be considered a quite useful pattern
    matching as empty line may call better your attention. % TODO, put reference for this

    But again this is something heavily dependent of what each one learning through their life time.
    Imagine another programmer do not liked this rule, and when he was writing your code involving
    an `if', he did not put such blank line another programmer is expecting. So when the first
    programmer start reading its the code and look for `if', he will be expecting for blank lines
    before its if\textquotesingle s. But will lose some time searching until realize another
    programmer does not put them, or perhaps he forgot to insert them. % TODO, put reference for this

    These differences are due to the diversity of ways we learn programming, i.e., to the ways we
    are used to doing coding, as much as the abilities and objectives of every programmer developed.
    Hence, nowadays it becomes a big problem because we increasingly need more and more programmers
    working together developing several and diverse computing systems. Where the latter is due to
    the fact of the complexity of computer systems growing increasingly, therefore over requiring
    programmers working and sharing their codes and ideas. % TODO, put reference for this

    Moreover besides only worrying about how the code is displayed on their computer screen, we need
    to worry about on how it will be saved in the file system on its `plain-text' mode. Since for
    code sharing, it is vital for you to use a versioning control system which enable project
    manager\textquotesingle s and programmers themselves, take control of their code changes. It does allow to
    easily perform the tracking of code changes and allow you to better understand what each
    programmer is doing every time he formalizes a change in the code through a `commit', as in
    `git' systems for example. % TODO, put reference for this

    That is because while working with a versioning system like `git', we need to keep the code
    among a single style or which we may call a `good practice' set as standard for everybody, due
    the fact of letting each programmer to write as he pleases, there will be plenty of noise on the
    code review and we are figuring out what actually each programmer did. Hence, if every
    programmer re-writes the history making changes like inserting new lines before each if, we end
    up with too much noise and focus of a versioning system is to look at only those changes that
    are significant to the code, such as the creation of new functions and not the addition of new
    blank lines (ref find\_some). % TODO, put reference for this (ref coding\_horror)

    Talking about the last thing pointed out, we could also think about an approach to creating a
    new version control system which focuses only on significant changes to the code, while
    reviewing code changes. However, this approach could not be ideal, as for example, it would
    allow programmers to start tedious wars of unproductive code adjustments. For example, imagine
    how it would be for your every day and have to go through your code re-adding new lines before
    each one of your beloved if\textquotesingle s, just because some night shift programmer had just
    removed them? % TODO, put reference for this

    \section{Goals}

    Establish relationships between good programming practices and efficiency in programming, in
    addition to a new tool to support programmers in order to automate the long and diverse
    programming process in teams of developers with different programming `best practices'.

    \subsection{Specific Goals}

    \begin{enumerate}
        \item A study on universal programming tools, which from a single software, to work well
        behaved accross all programming languages. Moreover, explain the differences for other
        softwares and the beneficits of a unique tool, instead of several heavly different ones.

        \item Define, determine and classify which one are good programming practices and perform an
        in-depth study on the good practices on visual layout area, also known as code `Beautifying'.

        \item A study on the variety of existing tools for the support of good programming
        practices, beyond a comparative analysis between them, determining their weaknesses and
        strengths.

        \item The definition of a flow pattern of development allowing teams of developers with
        different programming best practices, to work without intervene with each other up to start
        wars of `best good practices'.

        \item Propose a unique tool that allowing several and distinct programming `best practices'
        being implemented in several programming languages, which can be configured and set
        accordindgly to their whishes.
    \end{enumerate}

    \section{Search Method}

    The work will be based on research in articles, books, theses, dissertations, trusted authors
    websites, and through new demonstrated evidences based on arguments in the monograph evolution
    road. Also, present results after building a new tool which proposes a solution for the problems
    presented and detailed.

    In this proposal last chapter which lies in the topic \autoref{sec:implementation}, there is a
    series of weblinks and references preselected and may be used in the release build of this work.
    Noticing the texts of the last section probably will end up gradually moved to the first section
    of the text where there is the theoretical research, while correlated research are incorporated
    in the main written work.

    Moreover, at the end of the first part of this work, the completion of the subject entitled of
    Course Conclusion Work 1, leaving only the information for the implementation of the proposed
    tool to be implemented in the second part of this named thesis on \nameref{sec:implementation}.
}
{ % Portuguese




    Perguntas como ``O que são boas práticas de programação?'' ou ainda ``O por quê estas práticas
    são boas?'', não são fáceis de responder. Mas cada programador aprende a escrever seus códigos
    em uma determinada maneira, com determinadas características como utilizar 4 ou 8 espaços para
    indentação de linhas, sempre deixar uma linha em branco antes de cada estrutura de controle como
    if\textquotesingle s, for\textquotesingle s, e afins.

    Mas entrando o universo de boas práticas, há muitos coisas sobre discorrer. Assim neste trabalho
    especificamente trabalhá-se sobre como realizar a melhor disposição/exibição do código de
    programação na tela do computador, de modo que maximize e facilite o entendimento do mesmo.
    Portanto permitindo que o programador dispersa mais tempe pensando sobre o problema, do que
    tentar decifrar a informação que é apresentado para ele na tela.

    Dentro desta área de trabalho, precisa-se também pensar muito bem sobre como compartilhar os
    códigos de programação dos programadores entre si. Isso por que entra agora o problema da
    diversidade de boas práticas de programação. Ela acontece por que muitas vezes, aquilo que é uma
    boa prática para um `programador A', não é para o outro `programador B'. Por exemplo, imagine um
    código onde um programador decidiu colocar antes de cada `if', uma linha em branco. Portanto é
    de se esperar que sempre que vemos uma linha em branco nos podemos potencialmente encontrar um
    `if'. Entretanto imagine que outro programador não gostou dessa regra e quando ele foi escrever
    seu código que envolvia um `if', ele não colocou a essa tal linha em branco que o outro
    programador vinha colocando. Então quando o primeiro programador for ler o código e procurar por
    `if'es, ele vai estar esperando por linhas em branco. Mas vai perder algum tempo procurando até
    perceber que o outro programador não as colocou.

    Essas diferenças dão-se devido a diversidade de meios de se aprender programação, tanto quanto
    aos gostos, aptidões e objetivos de cada programador. Assim hoje em dia isso torna-se um grande
    problema por que cada vez mais precisamos de mais e mais programadores trabalhem juntos entre
    si, desenvolvendo os mais diversos sistemas computações. Onde este último deve-se ao fato de que
    a complexidade dos sistemas computacionais cresce cada vez mais, portanto requer-se que mais e
    mais programadores trabalhem e compartilhem códigos.

    Então além de nos preocupar-mos somente como o código é exibido na tela do computador, nós
    precisamos nos preocupar sobre como ele será salvo no sistema de arquivos. Já que ao
    compartilhar o código, é vital o uso de um sistema de versionamento para permitir a gerências de
    projetos e os programadores em si, terem o controle de mudanças do código. O que permiti e
    facilmente possa realizar o rastreamento de mudanças e permitir que se possa entender melhor o
    que cada programador está fazendo a cada vez que ele formaliza um mudança no código através de
    uma `commit', como no sistemas `git` por exemplo.

    Isso por que quando trabalhos em um sistema de versionamento como `git' precisamos manter o
    código dentre um único estilo ou boa prática definida como padrão, devido ao fato de que se
    deixar-mos cada programador escrever como ele quiser, teremos muito ruído durante a revisão do
    código e estamos determinando o que o programador fez/escreveu, se cada programador re-escreve o
    histórico fazendo alterações como colocar linhas novas antes de cada if. Assim teremos ruído por
    que o foco de um sistema de versionamento é olhar somente as mudanças que são significativas
    para o código, como a criação de novas funções e não a adição de novas linhas em branco.

    Sobre o último ponto, podemos pensar também sobre uma abordagem da criação de um novo sistema de
    versão que foque somente nas mudanças significativas para o código, durante o momento da
    revisão. Entretanto essa abordagem não é ideal por que, por exemplo, ela dá margem para que
    programadores entrem em guerras tediantes e não produtivas de ajustes de código. Por exemplo,
    imagine o quão seria todo dia que você acorda e começa a trabalhar, você tem que passar pelo
    código colocando linhas novas antes de cada um dos if\textquotesingle s por que o programador do
    turno da noite tinha acabado de remover eles?


\section{Objetivos}

    Estabelecer relações entre boas práticas de programação e eficiência em programar, além de uma
    nova ferramenta ao apoio do programador com o intuito de automatizar o longo e diverso processo
    de programação em equipes de desenvolvedores com distintas boas práticas de programação.


\subsection{Objetivos específicos}

    \begin{enumerate}

        \item

        Um estudo sobre ferramentas universais de programação, que permitam que a partir de um único
        software, seja programado em todas as linguagens de programação. Assim explicar as
        diferenças para os outros softwares e os porquês de querer-se uma ferramenta única, ao invés
        de diversas.

        \item

        Definir, estudar, determinar e classificar o que são boas práticas de programação e realizar
        um estudo aprofundado sobre a as boas práticas da área de disposição visual, conhecidas
        também como `Beautifying'.

        \item

        Um estudo sobre as mais diversas ferramentas existentes para o apoio de boas práticas de
        programação, além de uma análise comparativa entre elas, determinando suas fraquezas e
        pontos fortes.

        \item

        A definição de um padrão de floxo de desenvolvimento que permita equipes de programadores
        com distintas boas práticas de programação, trabalhem em si sem intervir e iniciar guerras
        de boas práticas.

        \item

        Propor uma ferramenta única que permita diversas e distintas boas práticas de programação serem
        implementadas nas mais diversas linguagens de programação e que elas possam ser configuradas
        e definidas ao gosto dos programadores que a usa.

    \end{enumerate}


\section{Método de pesquisa}

    O trabalho será baseado em pesquisas em artigos, livros, teses, dissertações, sites de autores
    confiáveis, e por meio de novas provas demonstradas e baseadas através de argumentos no decorrer
    da evolução da monografia. Também sera apresentado os resultados decorridos da construção de uma
    nova ferramenta que proprõe a solução de um dos problemas apresentados e explicados.

    No último capítulo desta proposta encontra-se no tópico \autoref{sec:implementation} encontra-se
    uma série de links e referências que forma pré-selecionadas e poderão ser utilizadas na
    construção final deste trabalho. Notes que em si, as partes da última seção serão gradativamente
    movida para primeira parte do texto onde encontra-se pesquisa teórica, no decorrer que suas
    informações correlacionadas são incorporadas no trabalho escrito.

    Assim no final da primeira parte desta obra que dará-se no final da conclusão da disciplina
    intitulada de Trabalho de Conclusão de Curso 1, restarão somente as informações destinadas a
    implementação da ferramenta proposta, que serão implementadas na segunda parte da monografia
    denominada \nameref{sec:implementation}, que será desenvolvida no final da conclusão da
    disciplina de Trabalho de Conclusão de Curso 2.
}




    % PARTE
    \part{Preparação da pesquisa}

    % Capitulo com exemplos de comandos inseridos de arquivo externo
    \include{chapters/chapter_2}

    % PARTE
    % \part{Referenciais teóricos}

    % Capitulo de revisão de literatura
    % \include{chapters/chapter_3}

    % PARTE
    % \part{Resultados}

    % Primeiro capitulo de Resultados
    % \include{chapters/chapter_4}

    % Segundo capitulo de Resultados
    % \include{chapters/chapter_5}

    % Terceiro capitulo de Resultados
    % \include{chapters/chapter_6}

    % Finaliza a parte no bookmark do PDF
    % para que se inicie o bookmark na raiz
    % e adiciona espaço de parte no Sumário
    \phantompart

    % Conclusão (outro exemplo de capítulo sem numeração e presente no sumário)
    

\chapter[]{\lang{Conclusion}{Conclusão}}

    The difference from this proposal to remaining formatting tools,
    is the tradeoff between end\hyp{}users and developers responsibilities.
    Most tools rarely expose to end\hyp{}users their language syntax specification,
    in contrast,
    this proposal completely exposes the language to the end\hyp{}user as simple plain\hyp{}text,
    not requiring the tool to know any language syntax neither semantics.
    Moreover,
    with no syntax knowledge required,
    the tool be can used with any languages their user wishes to.

\begin{enumerate}[leftmargin=*]
    \item
        There are many different tools, sometimes paid, and difficult to
        complete. \cite{universalCodeFormatter}
    \item
        Many programming languages exist, so always having Beautifier
        software for each of them is very laborious
        \cite{universalCodeFormatter}. But the approach to a Universal
        Beautifier proposed in this work, would allow easily new languages to be
        added, being completely different from previous ones, or alike. And in
        case of similarities between them, it is enough to reuse their
        configuration structures already implemented.
    \item
        Looking for a Beautifier for each one of them because programmers
        currently work daily with several of these languages, and they are not
        similar. So you need to configure several beautifiers to do the
        formatting. This is a problem because only a few beautifiers are more
        complete, and every time you need to make a change in the formatting
        style, you must manually propagate the same change over several
        different program configuration files, which is bad because it takes the
        user a lot of time to learn how to handle many different types of
        settings. \cite{universalIndentGUI}
    \item
        In the case of ideal Beautifier, a change in your styling is
        propagated to all languages. And if you want to leave some language out
        of it, you just need to remove it from the list on which the
        configuration block applies to, and `a)' leave it out so no change is
        applied to. Or `b)' create a new block including only the block within
        the desired settings.
\end{enumerate}





    % ELEMENTOS PÓS-TEXTUAIS
    %
    \postextual
    \setlength\beforechapskip{0pt}
    \setlength\midchapskip{15pt}
    \setlength\afterchapskip{15pt}

    % Referências bibliográficas
    \bibliography{refs}

    % Glossário, consulte o manual da classe abntex2 para orientações sobre o glossário.
    % \glossary

    % Apêndices, inicia os apêndices
    \begin{apendicesenv}

        % Imprime uma página indicando o início dos apêndices
        \partapendices

        \setlength\beforechapskip{50pt}
        \setlength\midchapskip{20pt}
        \setlength\afterchapskip{20pt}

        


%
% How to fix the Underfull \vbox badness has occurred while \output is active on my memoir chapter style?
% https://tex.stackexchange.com/questions/387881/how-to-fix-the-underfull-vbox-badness-has-occurred-while-output-is-active-on-m
%

% ---

\chooselang
{\chapter[Appendix A]{Since this page is not being completely filled, it is generating the bottom bottom of the page}}
{\chapter[Apêndice A]{Como esta página não está sendo completamente preenchida, ele está gerando a caixa inferior inferior da página}}
% ---


% Multiple-language document - babel - selectlanguage vs begin/end{otherlanguage}
% https://tex.stackexchange.com/questions/36526/multiple-language-document-babel-selectlanguage-vs-begin-endotherlanguage
\begin{otherlanguage*}{english}

\showfont

1. How to display the font size in use in the final output,
2. How to display the font size in use in the final output,
3. How to display the font size in use in the final output,
4. How to display the font size in use in the final output,
5. How to display the font size in use in the final output,
6. How to display the font size in use in the final output,
7. How to display the font size in use in the final output,
8. How to display the font size in use in the final output,
9. How to display the font size in use in the final output,


% As this page is not being completely filled, it is generating the page bottom bad box.
% Fix Underfull \vbox (badness 10000) has occurred while \output is active
%
% \flushbottom vs \raggedbottom
% https://tex.stackexchange.com/questions/65355/flushbottom-vs-raggedbottom
\newpage



\section[Some encoding tests]{\showfont}

1. How to display the font size in use in the final output,
2. How to display the font size in use in the final output,
3. How to display the font size in use in the final output,
4. How to display the font size in use in the final output,
5. How to display the font size in use in the final output,
6. How to display the font size in use in the final output,

7. How to display the font size in use in the final output,
8. How to display the font size in use in the final output,
9. How to display the font size in use in the final output,
10. How to display the font size in use in the final output,
11. How to display the font size in use in the final output,
12. How to display the font size in use in the final output,

\subsection{\showfont}

1. How to display the font size in use in the final output,
2. How to display the font size in use in the final output,
3. How to display the font size in use in the final output,
4. How to display the font size in use in the final output,
5. How to display the font size in use in the final output,
6. How to display the font size in use in the final output,

7. How to display the font size in use in the final output,
8. How to display the font size in use in the final output,
9. How to display the font size in use in the final output,
10. How to display the font size in use in the final output,
11. How to display the font size in use in the final output,
12. How to display the font size in use in the final output,

\subsubsection{\showfont}

1. How to display the font size in use in the final output,
2. How to display the font size in use in the final output,
3. How to display the font size in use in the final output,
4. How to display the font size in use in the final output,
5. How to display the font size in use in the final output,
6. How to display the font size in use in the final output,

7. How to display the font size in use in the final output,
8. How to display the font size in use in the final output,
9. How to display the font size in use in the final output,
10. How to display the font size in use in the final output,
11. How to display the font size in use in the final output,
12. How to display the font size in use in the final output,

\subsubsubsection{\showfont}

1. How to display the font size in use in the final output,
2. How to display the font size in use in the final output,
3. How to display the font size in use in the final output,
4. How to display the font size in use in the final output,
5. How to display the font size in use in the final output,
6. How to display the font size in use in the final output,
7. How to display the font size in use in the final output,

8. How to display the font size in use in the final output,
9. How to display the font size in use in the final output,
10. How to display the font size in use in the final output,
11. How to display the font size in use in the final output,
12. How to display the font size in use in the final output,


Lipsum me [31-35]

\end{otherlanguage*}




    \end{apendicesenv}

    % Anexos, inicia os anexos
    \begin{anexosenv}

        % Imprime uma página indicando o início dos anexos
        \partanexos

        \setlength\beforechapskip{50pt}
        \setlength\midchapskip{20pt}
        \setlength\afterchapskip{20pt}

        


%
% How to fix the Underfull \vbox badness has occurred while \output is active on my memoir chapter style?
% https://tex.stackexchange.com/questions/387881/how-to-fix-the-underfull-vbox-badness-has-occurred-while-output-is-active-on-m
%

% ----------------------------------------------------------
\lang
{\chapter[Sample example]{How to display the font size in use in the final output}}
{\chapter[Anexo exemplo]{Como exibir o tamanho da fonte em uso na saída final}}
% ----------------------------------------------------------


% Multiple-language document - babel - selectlanguage vs begin/end{otherlanguage}
% https://tex.stackexchange.com/questions/36526/multiple-language-document-babel-selectlanguage-vs-begin-endotherlanguage
\begin{otherlanguage*}{english}

\showfont

1. How to display the font size in use in the final output,
2. How to display the font size in use in the final output,
3. How to display the font size in use in the final output,


\section[Some encoding tests]{\showfont}

1. How to display the font size in use in the final output,
2. How to display the font size in use in the final output,
3. How to display the font size in use in the final output,
4. How to display the font size in use in the final output,
5. How to display the font size in use in the final output,
6. How to display the font size in use in the final output,

7. How to display the font size in use in the final output,
8. How to display the font size in use in the final output,
9. How to display the font size in use in the final output,
10. How to display the font size in use in the final output,
11. How to display the font size in use in the final output,
12. How to display the font size in use in the final output,

\subsection{\showfont}

1. How to display the font size in use in the final output,
2. How to display the font size in use in the final output,
3. How to display the font size in use in the final output,
4. How to display the font size in use in the final output,
5. How to display the font size in use in the final output,
6. How to display the font size in use in the final output,

7. How to display the font size in use in the final output,
8. How to display the font size in use in the final output,
9. How to display the font size in use in the final output,
10. How to display the font size in use in the final output,
11. How to display the font size in use in the final output,
12. How to display the font size in use in the final output,

\subsubsection{\showfont}

1. How to display the font size in use in the final output,
2. How to display the font size in use in the final output,
3. How to display the font size in use in the final output,
4. How to display the font size in use in the final output,
5. How to display the font size in use in the final output,
6. How to display the font size in use in the final output,

7. How to display the font size in use in the final output,
8. How to display the font size in use in the final output,
9. How to display the font size in use in the final output,
10. How to display the font size in use in the final output,
11. How to display the font size in use in the final output,
12. How to display the font size in use in the final output,

\subsubsubsection{\showfont}

1. How to display the font size in use in the final output,
2. How to display the font size in use in the final output,
3. How to display the font size in use in the final output,
4. How to display the font size in use in the final output,
5. How to display the font size in use in the final output,
6. How to display the font size in use in the final output,
7. How to display the font size in use in the final output,

8. How to display the font size in use in the final output,
9. How to display the font size in use in the final output,
10. How to display the font size in use in the final output,
11. How to display the font size in use in the final output,
12. How to display the font size in use in the final output,


Lipsum me [55-65]

\end{otherlanguage*}




    \end{anexosenv}

    % INDICE REMISSIVO
    \phantompart
    \printindex

\end{document}
