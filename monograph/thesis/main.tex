% Simple Sectioned Essay Template - LaTeX Template
%
% This template has been downloaded from:
% http://www.latextemplates.com
%
% `proposal.tex`
% Based on
%
% 1. https://github.com/royertiago/tcc
% 2. http://portal.bu.ufsc.br/normalizacao/
% 3. https://github.com/evandrocoan/ufscthesisx
% 4. http://www.latextemplates.com/template/simple-sectioned-essay
%
% Initially translated from Portuguese with help of https://github.com/omegat-org/omegat
% <Computer Assisted Translation of LaTeX document>
% https://tex.stackexchange.com/questions/313732/computer-assisted-translation-of-latex-document
%
% In case a translation back to Portuguese is required, keep both languages toguether.
% <Is it possible to keep my translation together with original text?>
% https://tex.stackexchange.com/questions/5076/is-it-possible-to-keep-my-translation-together-with-original-text
%
% You can build this using the command:
% latexmk -pdf -jobname=output -output-directory=cache -aux-directory=cache -pdflatex="pdflatex -interaction=nonstopmode" -use-make main.tex

% When the bibliography includes a cyclic reference to another bibliography,
% you need to run `pdfTeX` 5 times on the following order:
% 1. `pdfTeX`,
% 2. `biber`,
% 3. `pdfTeX`
% 4. `pdfTeX`
% 5. `pdfTeX`
% 6. `biber`
% 7. `pdfTeX`

% Allows you to write your thesis both in English and Portuguese
% https://tex.stackexchange.com/questions/5076/is-it-possible-to-keep-my-translation-together-with-original-text
\newif\ifenglish\englishfalse
\newif\ifadvisor\advisorfalse

% Uncomment the line `\englishtrue` to set the document default language to English
% \englishtrue
\advisortrue

% https://tex.stackexchange.com/questions/131002/how-to-expand-ifthenelse-so-that-it-can-be-used-in-parshape
\newcommand{\lang}[2]{\ifenglish#1\else#2\fi}
\newcommand{\advisor}[2]{\ifadvisor#1\else#2\fi}

% https://tex.stackexchange.com/questions/385895/how-to-make-passoptionstopackage-add-the-option-as-the-last
\ifenglish
    \newcommand{\swapcontents}[2]{#1 #2}
    \PassOptionsToPackage{brazil,main=english,spanish,french}{babel}
\else
    \newcommand{\swapcontents}[2]{#2 #1}
    \PassOptionsToPackage{main=brazil,english,spanish,french}{babel}
\fi

% Simple alias for English and Portuguese words
\newcommand{\brazilword}[1]{\foreignlanguage{brazil}{#1}}
\newcommand{\englishword}[1]{\foreignlanguage{english}{#1}}

% Allow you to write `Evandro's house` in latex as `Evandro\s house` instead of `Evandro\textquotesingle{}s house`
% https://tex.stackexchange.com/questions/31091/space-after-latex-commands
\newcommand{\s}[0]{\textquotesingle{}s\xspace}
\newcommand{\q}[0]{\textquotesingle{}\xspace}

% Uncomment the following line if you want to use other biblatex settings
% \PassOptionsToPackage{style=numeric,repeatfields=true,backend=biber,backref=true,citecounter=true}{biblatex}

% Disable the empty pages automatically put by memoir class, except the ones by \cleardoublepage
% \PassOptionsToClass{openany}{memoir}

% Fixes several `abntex2` class problems
\input{setup/setup.tex}

% The UFSC font size is 10.5, but memoir embedded by `abntex2` only accepts 10 and 11pt.
% However, problem will be fixed the `ufscthesisx` package.
% http://tug.ctan.org/tex-archive/macros/latex/contrib/cleveref/cleveref.pdf
\documentclass[
\lang{english}{brazilian,brazil},
10pt,          % Padrão UFSC para versão final
a5paper,       % Padrão UFSC para versão final
% 12pt,        % Pode usar tamanho 12pt para defesa
% a4paper,     % Pode usar a4 para defesa
twoside,       % Impressão nos dois lados da folha
chapter=TITLE, % Título de capítulos em caixa alta
section=TITLE, % Título de seções em caixa alta
]{abntex2}

% Load the UFSC thesis package
\usepackage{setup/ufscthesisx}

% Load extra commands for tables, lists, summaries, etc.
\input{setup/utilities.tex}

% % Utilize o arquivo aftertext/references.bib para incluir sua bibliografia.
\addbibresource{aftertext/references.bib}

% FIXME: Preencha com seus dados
\autor{\brazilword{Evandro Coan}}
\titulo{\lang{A Grammar Programmable Source Code Formatting Tool}{Uma Ferramenta de Formatação Programável Por Gramáticas}}

% FIXME: Se houver subtítulo, descomente a linha abaixo
% \subtitulo{\lang{Subtitle}{Subtítulo}}

% FIXME: Siglas para grau de formação Dr./Dra., Me./Ma, Bel. Bela. (inglês: PhD., MSc., Bs.)
\orientador[\lang{Supervisor}{Orientador}]{\brazilword{Rafael de Santiago}}

% FIXME: Se houver coorientador, descomente a linha abaixo
% \coorientador[\lang{Co-supervisor}{Coorientador(a)}]{\brazilword{Nome do coorientador(a)}}

% FIXME: Preencher com o nome do Coordenador de TCCs/Teses do seu curso
\coordenador[Coordenador]{\brazilword{Renato Cislaghi}}

% FIXME: Local da sua defesa
\local{\brazilword{Florianópolis, Santa Catarina} -- \lang{Brazil}{Brasil}}

% FIXME: Ano da sua defesa
\ano{2019}
\biblioteca{\lang{University Library}{Biblioteca Universitária}}

% FIXME: Sigla da sua instituição
\instituicaosigla{UFSC}
\instituicao{\brazilword{Universidade Federal de Santa Catarina}}

% FIXME: Preencha com Tese, Dissertação, Monografia ou Trabalho de Conclusão de Curso, Bachelor's Thesis, etc
\tipotrabalho{\lang{Bachelor\s Thesis}{Trabalho de Conclusão de Curso}}

% FIXME: Se houver Área de Concentração, descomente a linha abaixo
\area{\lang{Formal Languages}{Linguagens Formais}}

% FIXME: Preencha com Doutor, Bacharel ou Mestrando
\formacao{\lang
	{Bachelor of Science degree in Computer Science}
	{Bacharel em Ciências da Computação}%
}
\programa{\lang
	{Undergraduate Program in Computer Science}
	{Programa de Graduação em Ciências da Computação}%
}

% FIXME: Preencha com Departamento de XXXXXX, Centro de XXXXXX
\centro{\lang
	{INE -- Department of Informatics and Statistics, CTC -- Technological Center}
    {INE -- Departamento de Informática e Estatística, CTC -- Centro Tecnológico}%
}

% FIXME: Data da sua defesa
\data{\lang{30 of march of}{30 de março de} 2019}

% O preambulo deve conter tipo do trabalho, objetivo, nome da instituição e a área de concentração.
\preambulo{\lang%
    {%
        \imprimirtipotrabalho~submitted to the \imprimirprograma~of
        \imprimirinstituicao~for degree acquirement in \imprimirformacao.%
    }{%
        \imprimirtipotrabalho~submetido ao \imprimirprograma~da
        \imprimirinstituicao~para a obtenção do Grau de \imprimirformacao.%
    }%
}

% Allows you to use ~= instead of `\hyp{}`
% https://tex.stackexchange.com/questions/488008/how-to-create-an-alternative-to-shortcut-or-hyp
\useshorthands{~}\defineshorthand{~=}{\hyp{}}

\palavraschaveufsc{palavraschaveingles}   {Text Formatter}
\palavraschaveufsc{palavraschaveportugues}{Formatador do texto}

\palavraschaveufsc{palavraschaveingles}   {Source Code Beautifier}
\palavraschaveufsc{palavraschaveportugues}{Embelezador de código~=fonte}

\palavraschaveufsc{palavraschaveingles}   {Pretty~=printing}
\palavraschaveufsc{palavraschaveportugues}{Impressão~=bonita}

\palavraschaveufsc{palavraschaveingles}   {Context~=free Grammars}
\palavraschaveufsc{palavraschaveportugues}{Gramáticas Livre de Contexto}

\palavraschaveufsc{palavraschaveingles}   {Programming Languages Syntax}
\palavraschaveufsc{palavraschaveportugues}{Sintaxe de Linguagens de Programação}

% Altere o arquivo 'settings.tex' para incluir customizações de aparência da sua tese
%----------------------------------------------------------------------------------------
%   Thesis Tweaks and Utilities
%----------------------------------------------------------------------------------------
\makeatletter


% Uncomment this if you are debugging pages' badness Underfull & Overflow
% https://tex.stackexchange.com/questions/115908/geometry-showframe-landscape
% https://tex.stackexchange.com/questions/387077/what-is-the-difference-between-usepackageshowframe-and-usepackageshowframe
% https://tex.stackexchange.com/questions/387257/how-to-do-the-memoir-headings-fix-but-not-have-my-text-going-over-the-page-botto
% https://tex.stackexchange.com/questions/14508/print-page-margins-of-a-document
% \usepackage[showframe,pass]{geometry}

% To use the font Times New Roman, instead of the default LaTeX font
% more up-to-date than '\usepackage{mathptmx}'
% \usepackage{newtxtext}
% \usepackage{newtxmath}

% https://tex.stackexchange.com/questions/182569/how-to-manually-set-where-a-word-is-split
\hyphenation{Ge-la-im}
\hyphenation{Cis-la-ghi}

% Add missing translations for Portuguese
% https://tex.stackexchange.com/questions/8564/what-is-the-right-way-to-redefine-macros-defined-by-babel
\@ifpackageloaded{babel}{\@ifpackagewith{babel}{brazil}{\addto\captionsbrazil{%
  \renewcommand{\mytextpreliminarylistname}{Breve Sumário}
}}{}}{}
\@ifundefined{advisor}{\newcommand{\advisor}[2]{#1}}{}

% Selects a sans serif font family
% \renewcommand{\sfdefault}{cmss}

% Selects a monospaced (“typewriter”) font family
% \renewcommand{\ttdefault}{cmtt}

% Spacing between lines and paragraphs
% https://tex.stackexchange.com/questions/70212/ifpackageloaded-question
\@ifclassloaded{memoir}
{
  % New custom chapter style VZ14, see other chapters styles in:
  % http://repositorios.cpai.unb.br/ctan/info/latex-samples/MemoirChapStyles/MemoirChapStyles.pdf
  \newcommand\thickhrulefill{\leavevmode \leaders \hrule height 1ex \hfill \kern \z@}
  \makechapterstyle{VZ14} { %
    % \thispagestyle{empty}
    \setlength\beforechapskip{50pt}
    \setlength\midchapskip{20pt}
    \setlength\afterchapskip{20pt}
    \renewcommand\chapternamenum{}
    \renewcommand\printchaptername{}
    \renewcommand\chapnamefont{\Huge\scshape}
    \renewcommand\printchapternum {%
      \chapnamefont\null\thickhrulefill\quad
      \@chapapp\space\thechapter\quad\thickhrulefill
    }
    \renewcommand\printchapternonum {%
      \par\thickhrulefill\par\vskip\midchapskip
      \hrule\vskip\midchapskip
    }
    \renewcommand\chaptitlefont{\huge\scshape\centering}
    \renewcommand\afterchapternum {%
      \par\nobreak\vskip\midchapskip\hrule\vskip\midchapskip
    }
    \renewcommand\afterchaptertitle {%
      \par\vskip\midchapskip\hrule\nobreak\vskip\afterchapskip
    }
  }

  % Apply the style `VZ14` just created
  % \chapterstyle{VZ14}

  % http://mirrors.ibiblio.org/CTAN/macros/latex/contrib/memoir/memman.pdf
  \setlength\beforechapskip{0pt}
  \setlength\midchapskip{15pt}
  \setlength\afterchapskip{15pt}

  % Memoir: Warnings “The material used in the headers is too large” w/ accented titles
  % https://tex.stackexchange.com/questions/387293/how-to-change-the-page-layout-with-memoir
  \setheadfoot{30.0pt}{\footskip}
  \checkandfixthelayout
}{}

% Controlling the spacing between one paragraph and another, try also \onelineskip
% Default value for UFSC 0.0cm
\setlength{\parskip}{\advisor{0.0cm}{0.2cm}}

% Paragraph size is given by
% Default value for UFSC 1.5cm
% \setlength{\parindent}{1.3cm}

% https://tex.stackexchange.com/questions/148647/how-to-remove-space-before-enumerate
% https://tex.stackexchange.com/questions/433543/behaviour-of-enumitem-setlist
\advisor{}{
    \setlist*[enumerate]{label=\arabic*,}
    \setlist*[enumerateoptional]{label=\arabic*,}

    % Patch the `abntex2` citacao environment removing the extra space from its top
    % https://tex.stackexchange.com/questions/300340/topsep-itemsep-partopsep-and-parsep-what-does-each-of-them-mean-and-wha
    \xpatchcmd{\citacao}
    {\list{}}
    {\list{}{\topsep=0pt}}
    {}
    {\FAILEDPATCHINGCITACAO}
}


% Color settings across the document
\@ifpackageloaded{xcolor}
{
  % RGB colors in absolute values from 0 to 255 by using `RGB` tag
  \definecolor{darkblue}{RGB}{26,13,178}

  % Colors names definitions as RGB colors in percentage notation by using `rgb` tag
  \definecolor{mygreen}{rgb}{0,0.6,0}
  \definecolor{mygray}{rgb}{0.5,0.5,0.5}
  \definecolor{mymauve}{rgb}{0.58,0,0.82}
  \definecolor{figcolor}{rgb}{1,0.4,0}
  \definecolor{tabcolor}{rgb}{1,0.4,0}
  \definecolor{eqncolor}{rgb}{1,0.4,0}
  \definecolor{linkcolor}{rgb}{1,0.4,0}
  \definecolor{citecolor}{rgb}{1,0.4,0}
  \definecolor{seccolor}{rgb}{0,0,1}
  \definecolor{abscolor}{rgb}{0,0,1}
  \definecolor{titlecolor}{rgb}{0,0,1}
  \definecolor{biocolor}{rgb}{0,0,1}
  \definecolor{blue}{RGB}{41,5,195}

  % PDF Hyperlinks settings
  \@ifpackageloaded{hyperref}
  {
    \hypersetup
    {
      colorlinks=true,     % false: boxed links; true: colored links
      linkcolor=darkblue,  % color of internal links
      citecolor=darkblue, % color of links to bibliography
      filecolor=black,     % color of file links
      urlcolor=\advisor{black}{darkgreen},
      bookmarksdepth=4,
      pdfencoding=auto,%
      psdextra,
    }
  }
}{}

% ---
% Filtering and Mapping Bibliographies
% \DeclareFieldFormat{url}{Disponível~em:\addspace\url{#1}}

% ---
\DeclareSourcemap{
  \maps[datatype=bibtex]{
    % remove fields that are always useless
    \map{
      \step[fieldset=abstract, null]
      \step[fieldset=pagetotal, null]
    }
    % % remove URLs for types that are primarily printed
    % \map{
    %   \pernottype{software}
    %   \pernottype{online}
    %   \pernottype{report}
    %   \pernottype{techreport}
    %   \pernottype{standard}
    %   \pernottype{manual}
    %   \pernottype{misc}
    %   \step[fieldset=url, null]
    %   \step[fieldset=urldate, null]
    % }
    \map{
      \pertype{inproceedings}
      % remove mostly redundant conference information
      \step[fieldset=venue, null]
      \step[fieldset=eventdate, null]
      \step[fieldset=eventtitle, null]
      % do not show ISBN for proceedings
      \step[fieldset=isbn, null]
      % Citavi bug
      \step[fieldset=volume, null]
    }
  }
}

% Backref package settings, pages with citations in bibliography
\newcommand{\biblatexcitedntimes}{\autocap{c}ited \arabic{citecounter} times}
\newcommand{\biblatexcitedonetime}{\autocap{c}ited one time}
\newcommand{\biblatexcitednotimes}{\autocap{n}o citation in the text}

\@ifpackageloaded{babel}{\@ifpackagewith{babel}{brazil}{\addto\captionsbrazil{%
  \renewcommand{\biblatexcitedntimes}{\autocap{c}itado \arabic{citecounter} vezes}
  \renewcommand{\biblatexcitedonetime}{\autocap{c}itado uma vez}
  \renewcommand{\biblatexcitednotimes}{\autocap{n}enhuma citação no texto}
}}{}}{}
\@ifpackageloaded{biblatex}
{%
  % https://tex.stackexchange.com/questions/483707/how-to-detect-whether-the-option-citecounter-was-enabled-on-biblatex
  \ifx\blx@citecounter\relax
    \message{Is citecounter defined? NO!^^J}
  \else
    \message{Is citecounter defined? YES!^^J}
    \ifbacktracker
      \message{Is backtracker defined? YES!^^J}
      \renewbibmacro*{pageref}
      {
        \iflistundef{pageref}
        {\printtext{\biblatexcitednotimes}}
        {%
          \printtext
          {%
            \ifnumgreater{\value{citecounter}}{1}
              {\biblatexcitedntimes}
              {\biblatexcitedonetime}
          }%
          \setunit{\addspace}%
          \ifnumgreater{\value{pageref}}{1}
            {\bibstring{backrefpages}\ppspace}
            {\bibstring{backrefpage}\ppspace}%
          \printlist[pageref][-\value{listtotal}]{pageref}%
        }%
      }

      \DefineBibliographyStrings{brazil}
      {
        backrefpage  = {na página},
        backrefpages = {nas páginas},
      }

      \DefineBibliographyStrings{english}
      {
        backrefpage  = {on page},
        backrefpages = {on pages},
      }
    \else
      \message{Is backtracker defined? NO!^^J}
    \fi
  \fi
}{}

% https://tex.stackexchange.com/questions/391695/is-possible-to-remove-the-link-color-of-the-comma-on-the-citation-link
% \DeclareFieldFormat{citehyperref}{#1}

% https://tex.stackexchange.com/questions/19105/how-can-i-put-more-space-between-bibliography-entries-biblatex
\advisor{}{\setlength\bibitemsep{2.1\itemsep}}

% % https://tex.stackexchange.com/questions/203764/reduce-font-size-of-bibliography-overfull-bibliography
% \newcommand{\bibliographyfontsize}{\fontsize{10.0pt}{10.5pt}\selectfont}
% \renewcommand*{\bibfont}{\bibliographyfontsize}

% Uncomment this to insert the abstract into your bibliography entries when the abstract is available
% https://tex.stackexchange.com/questions/398666/how-to-correctly-insert-and-justify-abstract
\ifadvisor
\else
  \DeclareFieldFormat{abstract}%
  {%
    \vspace*{-0.5mm}\par\justifying
    \begin{adjustwidth}{1cm}{}
      \textbf{\bibsentence\bibstring{abstract}:} #1
    \end{adjustwidth}
  }
  \renewbibmacro*{finentry}%
  {%
    \iffieldundef{abstract}
    {\finentry}
    {\finentrypunct
      \printfield{abstract}%
      \renewcommand*{\finentrypunct}{}%
      \finentry
    }
  }
\fi


% https://tex.stackexchange.com/questions/14314/changing-the-font-of-the-numbers-in-the-toc-in-the-memoir-class
\renewcommand{\cftpartfont}{\ABNTEXpartfont\color{black}}
\renewcommand{\cftpartpagefont}{\ABNTEXpartfont\color{black}}

\renewcommand{\cftchapterfont}{\ABNTEXchapterfont\color{black}}
\renewcommand{\cftchapterpagefont}{\ABNTEXchapterfont\color{black}}

\renewcommand{\cftsectionfont}{\ABNTEXsectionfont\color{black}}
\renewcommand{\cftsectionpagefont}{\ABNTEXsectionfont\color{black}}

\renewcommand{\cftsubsectionfont}{\ABNTEXsubsectionfont\color{black}}
\renewcommand{\cftsubsectionpagefont}{\ABNTEXsubsectionfont\color{black}}

\renewcommand{\cftsubsubsectionfont}{\ABNTEXsubsubsectionfont\color{black}}
\renewcommand{\cftsubsubsectionpagefont}{\ABNTEXsubsubsectionfont\color{black}}

\renewcommand{\cftparagraphfont}{\ABNTEXsubsubsubsectionfont\color{black}}
\renewcommand{\cftparagraphpagefont}{\ABNTEXsubsubsubsectionfont\color{black}}

% Memoir has another mechanism for the job: \cftsetindents{‹kind›}{indent}{numwidth}. Here kind is
% chapter, section, or whatever; the indent specifies the ‘margin’ before the entry starts; and the
% width is of the box into which the number is typeset (so needs to be wide enough for the largest
% number, with the necessary spacing to separate it from what comes after it in the line.
% http://www.tex.ac.uk/FAQ-tocloftwrong.html
% https://tex.stackexchange.com/questions/264668/memoir-indentation-of-unnumbered-sections-in-table-of-contents
% https://tex.stackexchange.com/questions/394227/memoir-toc-indent-the-second-line-by-numberspace
%
% `\cftlastnumwidth` and these `\cftsetindents` are defined by the abntex2 class,
% obeying the `ABNTEXsumario-abnt-6027-2012`. \newlength{\cftlastnumwidth}
% \setlength{\cftlastnumwidth}{\cftsubsubsectionnumwidth}
% \addtolength{\cftlastnumwidth}{-1em}

% http://www.tex.ac.uk/FAQ-tocloftwrong.html
% Use \setlength\cftsectionnumwidth{4em} to override all these values at once
\ifadvisor
\else
  \makechapterstyle{fixedabntex2indentation}
  {%
    \renewcommand{\chapterheadstart}{}
    \setlength{\beforechapskip}{20pt}
    \setlength{\midchapskip}{20pt}
    \setlength{\afterchapskip}{15pt}

    \ifx \chapternamenumlength \undefined
      \newlength{\chapternamenumlength}
    \fi

    % tamanhos de fontes de chapter e part
    \ifthenelse{\equal{\ABNTEXisarticle}{true}}{%
      \setlength\beforechapskip{\baselineskip}%
      \renewcommand{\chaptitlefont}{\ABNTEXsectionfont\ABNTEXsectionfontsize}%
    }{%else
       \setlength{\beforechapskip}{0pt}%
       \renewcommand{\chaptitlefont}{\ABNTEXchapterfont\ABNTEXchapterfontsize}%
    }

    \renewcommand{\chapnumfont}{\chaptitlefont}
    \renewcommand{\parttitlefont}{\ABNTEXpartfont\ABNTEXpartfontsize}
    \renewcommand{\partnumfont}{\ABNTEXpartfont\ABNTEXpartfontsize}
    \renewcommand{\partnamefont}{\ABNTEXpartfont\ABNTEXpartfontsize}

    % tamanhos de fontes de section, subsection, subsubsection e subsubsubsection
    \setsecheadstyle{\ABNTEXsectionfont\ABNTEXsectionfontsize\ABNTEXsectionupperifneeded}
    \setsubsecheadstyle{\ABNTEXsubsectionfont\ABNTEXsubsectionfontsize\ABNTEXsubsectionupperifneeded}
    \setsubsubsecheadstyle{\ABNTEXsubsubsectionfont\ABNTEXsubsubsectionfontsize\ABNTEXsubsubsectionupperifneeded}
    \setsubsubsubsecheadstyle{\ABNTEXsubsubsubsectionfont\ABNTEXsubsubsubsectionfontsize\ABNTEXsubsubsubsectionupperifneeded}

    % Impressão do número do capítulo
    \renewcommand{\chapternamenum}{}

    % Impressão do nome do capítulo
    \renewcommand{\printchaptername}{%
       \chaptitlefont%
       \ifthenelse{\boolean{abntex@apendiceousecao}}{\appendixname}{}%
    }

    % Impressão do título do capítulo
    \def\printchaptertitle##1{%
      \chaptitlefont%
      \ifthenelse{\boolean{abntex@innonumchapter}}{\centering\ABNTEXchapterupperifneeded{##1}}{%
      \ifthenelse{\boolean{abntex@apendiceousecao}}{%
        \centering%
        \settowidth{\chapternamenumlength}{\printchaptername\printchapternum\afterchapternum}%
        \ABNTEXchapterupperifneeded{##1}%
      }{%
        \settowidth{\chapternamenumlength}{\printchaptername\printchapternum\afterchapternum}%
        \parbox[t]{\columnwidth-\chapternamenumlength}{\ABNTEXchapterupperifneeded{##1}}}%
      }%
    }

    % https://tex.stackexchange.com/questions/264668/memoir-indentation-of-unnumbered-sections-in-table-of-contents
    \renewcommand{\tocinnonumchapter}{%
      \addtocontents{toc}{\cftsetindents{chapter}{2.5em}{2em}}%
      \cftinserthook{toc}{A}}

    % Impressão do número do capítulo (no capítulo e não toc)
    \renewcommand{\printchapternum}{%
      \setboolean{abntex@innonumchapter}{false}%
      \chapnumfont%
      ~~\thechapter~%
      \ifthenelse{\boolean{abntex@apendiceousecao}}{%
        \tocinnonumchapter%
        ~\ABNTEXcaptiondelim~~%
      }{}%
    }

    \renewcommand{\ABNTEXcaptiondelim}{~\textendash~}
    \renewcommand{\afterchapternum}{}

    % Impressão do capítulo não numerado
    \renewcommand\printchapternonum{%
      \setboolean{abntex@innonumchapter}{true}%
    }
  }
  \chapterstyle{fixedabntex2indentation}

  \cftsetindents{part}          {0em} {3em}
  \cftsetindents{chapter}       {0em} {3em}
  \cftsetindents{section}       {0em} {4.3em}
  \cftsetindents{subsection}    {0em} {5.2em}
  \cftsetindents{subsubsection} {0em} {5.1em}
  \cftsetindents{paragraph}     {0em} {6.0em}
  \cftsetindents{subparagraph}  {0em} {7.0em}
\fi


\makeatother



% When writing a large document, it is sometimes useful to work on selected sections of the document
% to speed up compilation time: https://en.wikibooks.org/wiki/TeX/includeonly
\newif\ifforcedinclude\forcedincludefalse

% \addtoincludeonly{beforetext/agradecimentos}
% \addtoincludeonly{beforetext/epigrafe}
% \addtoincludeonly{beforetext/fichacatalografica}
% \addtoincludeonly{beforetext/folhadeaprovacao}
% \addtoincludeonly{beforetext/resumos}
% \addtoincludeonly{beforetext/siglas}
% \addtoincludeonly{beforetext/simbolos}

% Part 1
\addtoincludeonly{chapters/introduction}
\addtoincludeonly{chapters/motivation}
% \addtoincludeonly{chapters/beautifiers}

% Part 2
% \addtoincludeonly{chapters/object_beautifier}
% \addtoincludeonly{chapters/conclusion}

% \addtoincludeonly{aftertext/apendice_a}
% \addtoincludeonly{aftertext/abstracts}

% Control whether the full document will be generated
% Note: It will also generate severals errors like the following, which can be ignored
%       Latexmk: Missing input file: 'chapters/test.aux'
%
% You can make latex stop generate these errors, if you generate a full version
% of the document, before uncommenting these lines.
%
% Uncomment these two lines, to only partially generate the document
% \doincludeonly
% \forcedincludetrue

\ifenglish
    \hypersetup
    {
        pdfauthor={Author},
        pdfsubject={Thesis' Abstract},
        pdfcreator={LaTeX with abnTeX2 for UFSC},
        pdfkeywords={abnt}{latex}{UFSC}{abntex2}{thesis},
    }
\else
    \hypersetup
    {
        pdfauthor={Autores},
        pdfsubject={Resumo da tese},
        pdfcreator={LaTeX com abnTeX2 para UFSC},
        pdfkeywords={abnt}{latex}{UFSC}{abntex2}{tese},
    }
\fi

% https://tex.stackexchange.com/questions/171999/overfull-hbox-in-biblatex
% https://tex.stackexchange.com/questions/499457/why-my-document-is-not-hyphenation-on-words-starting-with-upper-case-letter-i
\emergencystretch=5em

% https://www.overleaf.com/learn/latex/Inserting_Images
\graphicspath{{pictures/}}

% https://tex.stackexchange.com/questions/23313/how-can-i-reduce-padding-after-figure
% https://tex.stackexchange.com/questions/499580/how-to-keep-my-default-floating-environment-spacing-before-them-while-reducing
% \xpretocmd{\figure}{\setlength{\belowcaptionskip}{-10pt}}{}{}

% https://tex.stackexchange.com/questions/85113/xrightarrow-text
\makeatletter
\newcommand{\xRightarrow}[2][]{\ext@arrow 0359\Rightarrowfill@{#1}{#2}}
\newcommand{\xLeftarrow}[2][]{\ext@arrow 0359\Leftarrowfill@{#1}{#2}}
\makeatother

% https://tex.stackexchange.com/questions/32208/footnote-runs-onto-second-page
\interfootnotelinepenalty=10000

\begin{document}
    % FIXME: Comment this after finishing the thesis, so you can start fixing the \flushbottom vs \raggedbottom
    % https://tex.stackexchange.com/questions/65355/flushbottom-vs-raggedbottom
    \raggedbottom

    % https://tex.stackexchange.com/questions/4705/double-space-between-sentences
    \frenchspacing

    % Uncomment this to put a ←← | ← (Go To Top/Go Back) on each section header
    \advisor{}{\addGoToSummary}

    % ELEMENTOS PRÉ-TEXTUAIS
    

% ELEMENTOS PRÉ-TEXTUAIS
\ifforcedinclude\else
    % Fix the \textpreliminarycontents not showing up when @twoside is disabled
    \newboolean{ufscThesisXisMemoirTwoSidesEnabled}
    \if@twoside
        \setboolean{ufscThesisXisMemoirTwoSidesEnabled}{true}
    \else
        \setboolean{ufscThesisXisMemoirTwoSidesEnabled}{false}
    \fi
    \setboolean{@twoside}{true}

    % pretextual settings
    % https://tex.stackexchange.com/questions/386446/how-to-fix-destination-with-the-same-identifier-namepage-a-has-been-already
    % https://tex.stackexchange.com/questions/67989/pdftex-warning-has-been-referenced-but-does-not-exist-replaced-by-a-fixed-one
    \hypersetup{pageanchor=false}
    \PRIVATEbookmarkthis{Capa}
    \addtotextpreliminarycontent{Capa}
    \pretextual

    % Capa
    % \includepdf{pictures/FrenteCapaUFSC.pdf}
    % https://tex.stackexchange.com/questions/227711/blank-page-after-titlingpage
    \advisor{}{\AtBeginShipoutNext{\AtBeginShipoutNext{\AtBeginShipoutDiscard}}}
    \imprimircapa

    % https://tex.stackexchange.com/questions/386446/how-to-fix-destination-with-the-same-identifier-namepage-a-has-been-already
    % https://tex.stackexchange.com/questions/67989/pdftex-warning-has-been-referenced-but-does-not-exist-replaced-by-a-fixed-one
    \hypersetup{pageanchor=true}

    % Custom list throw LaTeX Error: Command \mycustomfiction already defined?
    % https://tex.stackexchange.com/questions/388489/custom-list-throw-latex-error-command-mycustomfiction-already-defined/
    \advisor{}{%
        % Manually add the `\textpreliminarycontents` to the Table of Contents here
        % to keep the hyper link pointing to the beginning of the page, instead of
        % the beginning of `\textpreliminarycontents`
        % https://tex.stackexchange.com/questions/44088/when-do-i-need-to-invoke-phantomsection
        \phantomsection\addcontentsline{toc}{chapter}{\mytextpreliminarylistname}

        % https://tex.stackexchange.com/questions/234398/list-of-figures-and-tables-when-there-are-no-figures-or-tables
        \whenlistisnotempty{\mytextpreliminarylistname}{%
            \begin{KeepFromToc}
                \textpreliminarycontents
            \end{KeepFromToc}
        }

        \clearpage
    }

    % Fix the \textpreliminarycontents not showing up when @twoside is disabled
    \ifthenelse{\boolean{ufscThesisXisMemoirTwoSidesEnabled}}
    {\setboolean{@twoside}{true}}
    {\setboolean{@twoside}{false}}

    % Folha de rosto (o * indica que haverá a ficha bibliográfica)
    % https://tex.stackexchange.com/questions/74439/table-of-contents-incorrect-page-numbering
    \addtotextpreliminarycontent{\folhaderostoname}
    \imprimirfolhaderosto*{}

    % Inserir a ficha bibliografica
    %
    % Isto é um exemplo de Ficha Catalográfica, ou ``Dados internacionais de
    % catalogação-na-publicação''. Você pode utilizar este modelo como referência.
    % Porém, provavelmente a biblioteca da sua universidade lhe fornecerá um PDF
    % com a ficha catalográfica definitiva após a defesa do trabalho. Quando estiver
    % com o documento, salve-o como PDF no diretório do seu projeto e substitua todo
    % o conteúdo de implementação deste arquivo pelo comando abaixo:
    \PRIVATEbookmarkthis{Ficha Catalográfica}
    \addtotextpreliminarycontent{Ficha Catalográfica}

    

\ifenglish

Legal Notes

There is no warranty for any part of the documented software. The authors have taken care in the
preparation of this thesis, but make no expressed or implied warranty of any kind and assume no
responsibility for errors or omissions. No liability is assumed for incidental or consequential
damages in connection with or arising out of the use of the information or programs contained here.
\cite{koma-scrguien}

\else

Notas legais

Não há garantia para qualquer parte do software documentado. Os autores tomaram cuidado na
preparação desta tese, mas não fazem nenhuma garantia expressa ou implícita de qualquer tipo e não
assumem qualquer responsabilidade por erros ou omissões. Não se assume qualquer responsabilidade por
danos incidentais ou consequentes em conexão ou decorrentes do uso das informações ou programas aqui
contidos. \cite{koma-scrguien}

\fi


% http://portalbu.ufsc.br/ficha
% http://portal.bu.ufsc.br/servicos/ficha-de-identificacao-da-obra/
\begin{fichacatalografica}
    \vspace*{\fill}

    \begin{center}

        Catalogação na fonte pela Biblioteca Universitária da Universidade Federal de Santa Catarina.

        Arquivo compilado às \currenttime h do dia \today.

        \framebox[\textwidth]
        {
            \begin{minipage}{0.98\textwidth}

                \ttfamily
                \imprimirautor

                \hspace{0.5cm} \imprimirtitulo~:~\imprimirsubtitulo~/~\imprimirautor;
                orientador(a),~\imprimirorientador;~co\hyp{}orientador(a),~\imprimircoorientador
                ~--~\imprimirlocal,~\currenttime,~\imprimirdata.

                % Prints how much pages there are on the document and links to the last page
                \hspace{0.5cm} \pageref{LastPage} p.
                \bigskip

                \hspace{0.5cm} \imprimirtipotrabalho~--~\imprimirinstituicao,
                \imprimircentro,~\imprimirprograma.
                \bigskip

                \hspace{0.5cm} Inclui referências
                \bigskip

                \hspace{0.5cm}
                    1. Uma Palavra\hyp{}chave ~
                    2. Outra Palavra\hyp{}chave ~
                    3. Mais Palavras\hyp{}chave ~
                    I. \imprimirorientador ~
                    II. \imprimircoorientador ~
                    III. \imprimirprograma ~
                    IV. \imprimirtitulo ~
                \bigskip

                \hspace{7.75cm} CDU 02:141:005.7

            \end{minipage}
        }

    \end{center}

\end{fichacatalografica}


    % https://tex.stackexchange.com/questions/91440/how-to-include-multiple-pages-in-latex
    % \includepdf{pictures/Ficha_Catalografica.pdf}
    \ifforcedinclude\else\cleardoublepage\fi
\fi


% Inserir errata

% Inserir folha de aprovação. Isto é um exemplo de Folha de aprovação, elemento obrigatório da
% NBR 14724/2011 (seção 4.2.1.3). Você pode utilizar este modelo até a aprovação do trabalho.
% Após isso, substitua todo o conteúdo deste arquivo por uma imagem da página assinada pela
% banca com o comando abaixo:
\ifforcedinclude\else\cleardoublepage\fi


\addtotextpreliminarycontent{\chooselang{Approval Sheet}{Folha de Aprovação}}

\begin{folhadeaprovacao}

    \begin{center}
        {\ABNTEXchapterfont\large\imprimirautor}

        \begin{center}
            \ABNTEXchapterfont\bfseries\Large\imprimirtitulo
        \end{center}

        \begin{minipage}{\textwidth}
            \chooselang
            {
                This thesis was considered appropriate to obtain the Doctor 's Degree in Electrical Engineering, in the area of concentration in Power Electronics and Electrical Drive, and approved in its final form by the Post-Graduate Program in Electrical Engineering of the Federal University of Santa Catarina.
            }
            {
                Esta Tese foi julgada adequada para obtenção do Título de Doutor em Engenharia Elétrica, na área de concentração em Eletrônica de Potência e Acionamento Elétrico, e aprovada em sua forma final pelo Programa de Pós--Graduação em Engenharia Elétrica da Universidade Federal de Santa Catarina.
            }
        \end{minipage}%

    \end{center}
    \begin{center}
        Florianópolis, \imprimirdata.
    \end{center}

    \assinatura
    {
        \textbf{Prof. Marcelo Lobo Heldwein, Dr.} \\
        \chooselang{Coordinator of the}{Coordenador do} \imprimirprograma
    }

    \assinatura
    {
        \textbf{\imprimirorientador} \\ \imprimirorientadorRotulo \\
        \imprimirinstituicao~--~\imprimirinstituicaosigla
    }

    \assinatura
    {
        \textbf{\imprimircoorientador} \\ \imprimircoorientadorRotulo \\
        \imprimirinstituicao~--~\imprimirinstituicaosigla
    }

    \newpage
    \begin{flushleft}
        \textbf{\chooselang{Examination Board:}{Banca Examinadora:}}
    \end{flushleft}

    \assinatura{\textbf{Prof. Arnaldo José Perin, \chooselang{PhD.}{Dr.}} \\
    \imprimirinstituicao~--~\imprimirinstituicaosigla}

    \assinatura{\textbf{Prof. Denizar Cruz Martins, \chooselang{PhD.}{Dr.}} \\
    \imprimirinstituicao~--~\imprimirinstituicaosigla}

    \assinatura{\textbf{Prof. Roberto Francisco Coelho, \chooselang{PhD.}{Dr.}} \\
    \imprimirinstituicao~--~\imprimirinstituicaosigla}

    \assinatura{\textbf{Prof. Samir Ahmad Mussa, \chooselang{PhD.}{Dr.}} \\
    \imprimirinstituicao~--~\imprimirinstituicaosigla}

    \assinatura{\textbf{Prof. Telles Brunelli Lazzarin, \chooselang{PhD.}{Dr.}} \\
    \imprimirinstituicao~--~\imprimirinstituicaosigla}

\end{folhadeaprovacao}


% \includepdf{pictures/folhadeaprovacao_final.pdf}


% % Dedicatória
% \ifforcedinclude\else\cleardoublepage\fi
% \ifforcedinclude\else

\addtotextpreliminarycontent{\lang{Dedication}{Dedicatória}}

\begin{dedicatoria}

    \vspace*{\fill}
    \centering
    \noindent
    \textit{\lang
    {
        This work is dedicated to adult children who, \\
        When small, dreamed of becoming scientists.
    }
    {
        Este trabalho é dedicado às crianças adultas que,\\
        quando pequenas, sonharam em se tornar cientistas.
    }}
    \vspace*{\fill}

\end{dedicatoria}


\fi

% % Agradecimentos
% \ifforcedinclude\else\cleardoublepage\fi
% 

\addtotextpreliminarycontent{\chooselang{Acknowledgement}{Agradecimentos}}

\begin{agradecimentos}

\chooselang
{
    Greetings.
}
{
    Os agradecimentos principais são direcionados à Gerald Weber, Miguel Frasson,
    Leslie H. Watter, Bruno Parente Lima, Flávio de Vasconcellos Corrêa, Otavio Real
    Salvador, Renato Machnievscz\footnote{Os nomes dos integrantes do primeiro
    projeto abn\TeX\ foram extraídos de
    \url{http://codigolivre.org.br/projects/abntex/}} e todos aqueles que
    contribuíram para que a produção de trabalhos acadêmicos conforme
    as normas ABNT com \LaTeX{} fosse possível.

    Agradecimentos especiais são direcionados ao Centro de Pesquisa em Arquitetura
    da Informação\footnote{\url{http://www.cpai.unb.br/}} da Universidade de
    Brasília (CPAI), ao grupo de usuários
    \emph{latex-br}\footnote{\url{http://groups.google.com/group/latex-br}} e aos
    novos voluntários do grupo
    \emph{\abnTeX{}}\footnote{\url{http://groups.google.com/group/abntex2} e
    \url{http://abntex2.googlecode.com/}}~que contribuíram e que ainda
    contribuirão para a evolução do \abnTeX{}.
}

\end{agradecimentos}


%Mesmo padrão da seção primária, porém sem indicativo numérico. Assim como: Dedicatória, Resumo, Abstract, Sumário, Listas, Referências, Apêndices e Anexos.
%
%
%Corpo do texto, fonte 10,5, justificado, recuo especial da primeira linha de 1 cm, espaçamento simples.
%


% % Epígrafe
% \ifforcedinclude\else\cleardoublepage\fi
% 

\addtotextpreliminarycontent{\lang{Epigraph}{Epigrafe}}

\begin{epigrafe}

\vspace*{\fill}\lang
{
    \begin{flushright}
        \textit{``Learn from yesterday, live for today, hope for tomorrow. The important thing is not to stop questioning.''} \\ Albert Einstein
    \end{flushright}
    \begin{flushright}
        \textit{``The true sign of intelligence is not knowledge but imagination.''} \\  Albert Einstein
    \end{flushright}
    \begin{flushright}
        \textit{``Peace cannot be kept by force; it can only be achieved by understanding.''} \\ Albert Einstein
    \end{flushright}
    \begin{flushright}
        \textit{``Whoever is careless with the truth in small matters cannot be trusted with important matters.''} \\ Albert Einstein
    \end{flushright}
    \begin{flushright}
        \textit{``Extraordinary claims require extraordinary evidence''} \\ Carl Sagan
    \end{flushright}
    \begin{flushright}
        \textit{``Catholic, which I was until I reached the age of reason.''} \\ George Carlin
    \end{flushright}
    \begin{flushright}
        \textit{``We made too many wrong mistakes.''} \\ Yogi Berra
    \end{flushright}
}
{
    \begin{flushright}
        \textit{``Assim como aquele pecado da juventude, este documento te perseguirá pelo resto da vida. \showfont''} \\ Enio Valmor Kassick
    \end{flushright}
    \begin{flushright}
        \textit{``Estupidez trará mais autoconfiança do que o conhecimento e a bravura juntas.''} \\ Adriano Ruseler
    \end{flushright}
}

\end{epigrafe}





% Ajusta o espaçamento dos parágrafos do resumo
\setlength{\absparsep}{18pt}

% RESUMOS
\ifforcedinclude\else\cleardoublepage\fi


% Is it possible to keep my translation together with original text?
% https://tex.stackexchange.com/questions/5076/is-it-possible-to-keep-my-translation-together-with-original-text
\begin{comment}

    Segundo a \citeonline[3.1-3.2]{NBR6028:2003}, o resumo deve ressaltar o
    objetivo, o método, os resultados e as conclusões do documento. A ordem e a extensão
    destes itens dependem do tipo de resumo (informativo ou indicativo) e do
    tratamento que cada item recebe no documento original. O resumo deve ser
    precedido da referência do documento, com exceção do resumo inserido no
    próprio documento. (\ldots) As palavras-chave devem figurar logo abaixo do
    resumo, antecedidas da expressão Palavras-chave:, separadas entre si por
    ponto e finalizadas também por ponto.

    ufscthesis: O texto do resumo deve ser digitado, em um único bloco, sem espaço de parágrafo. O resumo deve
    ser significativo, composto de uma sequência de frases concisas, afirmativas e não de uma
    enumeração de tópicos. Não deve conter citações. Deve usar o verbo na voz passiva. Abaixo do
    resumo, deve-se informar as palavras-chave (palavras ou expressões significativas retiradas do
    texto) ou, termos retirados de thesaurus da área. \showfont

\end{comment}



\swapcontents
{
    % Changing babel package inside a single chapter
    % https://tex.stackexchange.com/questions/20987/changing-babel-package-inside-a-single-chapter
    %
    % Multiple-language document - babel - selectlanguage vs begin/end{otherlanguage}
    % https://tex.stackexchange.com/questions/36526/multiple-language-document-babel-selectlanguage-vs-begin-endotherlanguage
    \addtotextpreliminarycontent{English's Abstract}
    \begin{otherlanguage*}{english}
    \begin{resumo}[Abstract]

        A study is made of what Beautifiers are for, as well as approaches to what are good
        programming practices and why we should follow them for good efficiency when writing code in
        the most diverse programming languages. Current source code formatting software, also known
        as Beautifiers, is limited to a similar set, or even a single language, and, in addition to
        many, be limited to what they can do for you when processing / formatting the code.

        Therefore, it is expected at the end of the work, to know what tools exist and which of them
        are the best that can be used to help the programmer while writing codes of the most diverse
        programming languages. Besides the proposal of a new tool with the intuition of centralizing
        in a single program the approach of the most diverse programming languages.

        \imprimirpalavraschave{Keywords}{\wordslistunlabled{\palavraschaveingles}}

    \end{resumo}
    \end{otherlanguage*}
}
{
    \addtotextpreliminarycontent{Resumo em Português}
    \begin{otherlanguage*}{brazil}
    \begin{resumo}[Resumo]

        Faz-se um estudo sobre o que é, para que servem os Beautifiers, assim como abordagens sobre
        o que são boas práticas de programação e por que devemos segui-las para um boa eficiência ao
        escrever códigos nas mais diversas linguagens de programação. Os softwares formatadores de
        código fonte atuais, também conhecidos como Beautifiers, são limitados a um conjunto
        similar, ou mesmo à uma única linguagem, e além de muitos, serem limitados ao que eles podem
        fazer por você ao processar/formatar o código.

        Portanto espera-se o final do trabalho, conhecer-se quais são as ferramentas que existem e
        quais delas são as melhores que podem ser utilizadas para o auxilio do programador durante a
        escrita de códigos das mais diversas linguagens de programação. Além de proposta de uma nova
        ferramenta com o intuído de centralizar em uma único programa o abordagem das mais diversas
        linguagens de programação.

        \imprimirpalavraschave{Palavras-chaves}{\wordslistunlabled{\palavraschaveportugues}}

    \end{resumo}
    \end{otherlanguage*}
}



% % resumo em francês
% \begin{resumo}[Résumé]
%   \begin{otherlanguage*}{french}
%       Il s'agit d'un résumé en français.

%       \imprimirpalavraschave{Mots-clés}{latex. abntex. publication de textes.}
%   \end{otherlanguage*}
% \end{resumo}


% % resumo em espanhol
% \begin{resumo}[Resumen]
%   \begin{otherlanguage*}{spanish}
%       Este es el resumen en español.

%       \imprimirpalavraschave{Palabras clave}{latex. abntex. publicación de textos.}
%   \end{otherlanguage*}
% \end{resumo}





% Some tables of contents
\ifforcedinclude\else
{
    % https://tex.stackexchange.com/questions/179506/disable-colorlinks-locally-or-just-for-the-toc
    \hypersetup{hidelinks}

    % inserir lista de figuras
    \ifforcedinclude\else\cleardoublepage\fi
    % https://tex.stackexchange.com/questions/234398/list-of-figures-and-tables-when-there-are-no-figures-or-tables
    \whenlistisnotempty{\listfigurename}{%
        \addtotextpreliminarycontent{\listfigurename}
        % https://tex.stackexchange.com/questions/121879/remove-spacing-between-per-chapter-figures-in-lof
        {\renewcommand{\addvspace}[1]{}
        \listoffigures*}
    }{\pdfbookmark[0]{\listfigurename}{lof}}

    % inserir lista de quadros
    \ifforcedinclude\else\cleardoublepage\fi
    % https://tex.stackexchange.com/questions/234398/list-of-figures-and-tables-when-there-are-no-figures-or-tables
    \whenlistisnotempty{\listofquadrosname}{%
        \addtotextpreliminarycontent{\listofquadrosname}
        % https://tex.stackexchange.com/questions/121879/remove-spacing-between-per-chapter-figures-in-lof
        {\renewcommand{\addvspace}[1]{}
        \listofquadros*}
    }{\pdfbookmark[0]{\listofquadrosname}{loq}}

    % inserir lista de tabelas
    \ifforcedinclude\else\cleardoublepage\fi
    % https://tex.stackexchange.com/questions/234398/list-of-figures-and-tables-when-there-are-no-figures-or-tables
    \whenlistisnotempty{\listtablename}{%
        \addtotextpreliminarycontent{\listtablename}
        % https://tex.stackexchange.com/questions/121879/remove-spacing-between-per-chapter-figures-in-lof
        {\renewcommand{\addvspace}[1]{}
        \listoftables*}
    }{\pdfbookmark[0]{\listtablename}{lot}}

    % inserir códigos fonte (List of Listings `lol`)
    % https://tex.stackexchange.com/questions/511519/latex-keeps-showing-minted-environment-as-figures-instead-of-listening/511579#511579
    \ifforcedinclude\else\cleardoublepage\fi
    % https://tex.stackexchange.com/questions/234398/list-of-figures-and-tables-when-there-are-no-figures-or-tables
    \whenlistisnotempty{\lstlistlistingname}{%
        \addtotextpreliminarycontent{\lstlistlistingname}
        % https://tex.stackexchange.com/questions/121879/remove-spacing-between-per-chapter-figures-in-lof
        {\renewcommand{\addvspace}[1]{}
        \lstlistoflistings*}
    }{\pdfbookmark[0]{\lstlistlistingname}{lol}}
}
\fi


% % inserir lista de abreviaturas e siglas
% \ifforcedinclude\else\cleardoublepage\fi
% 

\addtotextpreliminarycontent{\lang{List of Acronyms}{Lista de Siglas}}

\begin{siglas}
    \item[ABNT] \lang{Brazilian Association of Technical Standards}{Associação Brasileira de Normas Técnicas}
    \item[abnTeX] \lang{Absurd Standards for TeX}{ABsurdas Normas para TeX}
\end{siglas}



% % Inserir lista de símbolos
% \ifforcedinclude\else\cleardoublepage\fi
% 
% Devam aparecer na mesma ordem de ocorrência no texto.

\begin{simbolos}
    \item[$ \Gamma $] Letra grega Gama
    \item[$ \Lambda $] Lambda
    \item[$ \zeta $] Letra grega minúscula zeta
    \item[$ \in $] Pertence
\end{simbolos}


% How to remove the self-reference of the ToC from the ToC?
% https://tex.stackexchange.com/questions/10943/how-to-remove-the-self-reference-of-the-toc-from-the-toc
\ifforcedinclude\else\cleardoublepage\fi

\begin{KeepFromToc}
    % https://tex.stackexchange.com/questions/35/what-does-overfull-hbox-mean
    % https://tex.stackexchange.com/questions/59122/how-to-avoid-using-sloppy-document-wide-to-fix-overfull-hbox-problems
    % https://tex.stackexchange.com/questions/257007/adding-color-to-table-of-contents-and-section-headings
    {
        % https://tex.stackexchange.com/questions/179506/disable-colorlinks-locally-or-just-for-the-toc
        \hypersetup{hidelinks}

        % https://tex.stackexchange.com/questions/65711/underfull-vbox-badness-10000-with-memoir
        \raggedbottom

        % https://tex.stackexchange.com/questions/49887/overfull-hbox-warning-for-toc-entries-when-using-memoir-documentclass
        % \makeatletter
            % \renewcommand{\@pnumwidth}{2em}
            % \renewcommand{\@tocrmarg}{3em}
        % \makeatother

        % https://tex.stackexchange.com/questions/57544/memoir-mysterious-overfull-hbox-in-toc-when-mathptmx-is-used
        % \setlength{\cftchapternumwidth}{2.25em}

        % Add the table of contents to the brief table of contents
        % https://tex.stackexchange.com/questions/234398/list-of-figures-and-tables-when-there-are-no-figures-or-tables
        \whenlistisnotempty{\contentsname}{%
            \addtotextpreliminarycontent{\contentsname}
            \tableofcontents
        }{\pdfbookmark[0]{\contentsname}{toc}}
    }

\end{KeepFromToc}



    % ELEMENTOS TEXTUAIS
    \textual
    \setlength\beforechapskip{50pt}
    \setlength\midchapskip{20pt}
    \setlength\afterchapskip{20pt}

    % PARTE
    \advisor{}{%
        \ifforcedinclude\else\part{\lang{Research}{Pesquisa}}\fi
        \label{primeira_parte}
    }

    % Introdução (exemplo de capítulo sem numeração, mas presente no Sumário)
    

% The \phantomsection command is needed to create a link to a place in the document that is not a
% figure, equation, table, section, subsection, chapter, etc.
%
% When do I need to invoke \phantomsection?
% https://tex.stackexchange.com/questions/44088/when-do-i-need-to-invoke-phantomsection
\phantomsection


% Is it possible to keep my translation together with original text?
% https://tex.stackexchange.com/questions/5076/is-it-possible-to-keep-my-translation-together-with-original-text
\chapter{\lang{Introduction}{Introdução}}


\begin{englishtext}

    % TODO, put reference for this
    Questions like ``What are good programming practices?'' Or ``Why are these
    practices are good?''Are not easy to answer. But each programmer learns to
    write their codes in a certain way, with certain features like using 4 or 8
    spaces to indent lines, always leave a blank line before each control
    structure as if or for statements, and alike rules.

    % TODO, put reference for this
    But entering the universe of good practices, there are many things for
    discoursing. So in this work implementation tool called `Object Beautifier'
    specifically dedicates on how to perform the best layout/display of
    programming code on the computer screen, so that maximize and facilitate the
    understanding of same. Therefore, allowing the programmer to disperse more
    tempe thinking about its coding algorithms problem, other than trying to
    decipher the information that is presented to it on the screen through
    infinit different code layouts.

    % TODO, put reference for this
    Within this work\s area, we need to also think long and hard about how to
    share the programming code of the programmers among you. Now, the problem of
    human diversity, like all big scientific questions -- how do you explain
    something like that -- It can be broken down into sub-questions. It happens
    many times, which is a good practice for a `Programmer A', is not the same
    to another `Programmer B'. For example, imagine some code where a programmer
    decided to put before each `if' statement, a blank line. It is therefore
    expected that whenever we see a blank line we can potentially find a
    matching `if', which can be considered a quite useful pattern matching as
    empty line may call better your attention.

    % TODO, put reference for this
    But again this is something heavily dependent of what each one learning
    through their life time. Imagine another programmer do not liked this rule,
    and when he was writing your code involving an `if', he did not put such
    blank line another programmer is expecting. So when the first programmer
    start reading its the code and look for `if', he will be expecting for blank
    lines before its if\s. But will lose some time searching until realize
    another programmer does not put them, or perhaps he forgot to insert them.

    % TODO, put reference for this
    These differences are due to the diversity of ways we learn programming,
    i.e., to the ways we are used to doing coding, as much as the abilities and
    objectives of every programmer developed. Hence, nowadays it becomes a big
    problem because we increasingly need more and more programmers working
    together developing several and diverse computing systems. Where the latter
    is due to the fact of the complexity of computer systems growing
    increasingly, therefore over requiring programmers working and sharing their
    codes and ideas.

    % TODO, put reference for this
    Moreover besides only worrying about how the code is displayed on their
    computer screen, we need to worry about on how it will be saved in the file
    system on its `plain-text' mode. Since for code sharing, it is vital for you
    to use a versioning control system which enable project manager\s and
    programmers themselves, take control of their code changes. It does allow to
    easily perform the tracking of code changes and allow you to better
    understand what each programmer is doing every time he formalizes a change
    in the code through a `commit', as in `git' systems for example.

    % TODO, put reference for this (ref coding\_horror)
    That is because while working with a versioning system like `git', we need
    to keep the code among a single style or which we may call a `good practice'
    set as standard for everybody, due the fact of letting each programmer to
    write as he pleases, there will be plenty of noise on the code review and we
    are figuring out what actually each programmer did. Hence, if every
    programmer re-writes the history making changes like inserting new lines
    before each if, we end up with too much noise and focus of a versioning
    system is to look at only those changes that are significant to the code,
    such as the creation of new functions and not the addition of new blank
    lines (ref find\_some).

    % TODO, put reference for this
    Talking about the last thing pointed out, we could also think about an
    approach to creating a new version control system which focuses only on
    significant changes to the code, while reviewing code changes. However, this
    approach could not be ideal, as for example, it would allow programmers to
    start tedious wars of unproductive code adjustments. For example, imagine
    how it would be for your every day and have to go through your code
    re-adding new lines before each one of your beloved if\s, just because some
    night shift programmer had just removed them?


    \cite{softwarePortfolio}
    \cite{legacyAssets}
    \cite{massMaintenance}
    \cite{prettyPrinting}
    \cite{architectureFormatting}
    \cite{independentFramework}
    \cite{programIndentation}
    \cite{industrialApplication}

    \section{Goals}

    Establish relationships between good programming practices and efficiency in
    programming, in addition to a new tool to support programmers in order to
    automate the long and diverse programming process in teams of developers
    with different programming `best practices'. \cite{pushdownAutomata}


    \subsection{Specific Goals}

    \begin{enumerate}
        \item A study on universal programming tools, which from a single
        software, to work well behaved across all programming languages.
        Moreover, explain the differences for other softwares and the benefits
        of a unique tool, instead of several heavily different ones.

        \item Define, determine and classify which one are good programming
        practices and perform an in-depth study on the good practices on visual
        layout area, also known as code `Beautifying'.

        \item A study on the variety of existing tools for the support of good
        programming practices, beyond a comparative analysis between them,
        determining their weaknesses and strengths.

        \item The definition of a flow pattern of development allowing teams of
        developers with different programming best practices, to work without
        intervene with each other up to start wars of `best good practices'.

        \item Propose a unique tool that allowing several and distinct
        programming `best practices' being implemented in several programming
        languages, which can be configured and set accordingly to their wishes.
    \end{enumerate}



    \section{Search Method}

    The work will be based on research in articles, books, theses,
    dissertations, trusted authors websites, and through new demonstrated
    evidences based on arguments in the monograph evolution road. Also, present
    results after building a new tool which proposes a solution for the problems
    presented and detailed. \cite{aspectOriented}

    In this proposal last chapter which lies in the topic
    \autoref{sec:implementation}, there is a series of weblinks and references
    preselected and may be used in the release build of this work. Noticing the
    texts of the last section probably will end up gradually moved to the first
    section of the text where there is the theoretical research, while
    correlated research are incorporated in the main written work.
    \cite{aspectOrientationReview}

    Moreover, at the end of the first part of this work, the completion of the
    subject entitled of Course Conclusion Work 1, leaving only the information
    for the implementation of the proposed tool to be implemented in the second
    part of this named thesis on \nameref{sec:implementation}.



    \section{What does coding is?}

    Coding is like writing and reading a book for the large people, you like it
    to look beautifully. Or at least do you expect such when you buy a book, for
    example, to learn programming for you first time. You expect: % Reference to
    % book writing style/formatting articles

    \begin{enumerate}
        \item Things to be well organized, so you do not get lost.

        \item The colors to be properly placed, so you do not get distracted
           from the main content.

        \item The spacing between paragraphs, words, chapters, sections
           subsections, etc, to be well adjusted. Not everything cluttered in
           only one file, line, function, class, or whatsoever so.
    \end{enumerate}



    \section{Spaces and Tabs}

    The problem is that I will certainly not notice when I paste something
    indented with spaces instead of tabs. This is problem because for some file
    types as `.sublime-settings' files (or a Makefile), which has the setting
    `translate\_tabs\_to\_spaces' set to false, so I would expect to all
    `.sublime-settings' files to be indented with tabs, not spaces.
    \cite{tabsAndSpacesConversion}

    The setting `translate\_tabs\_to\_spaces' set to false works fine until I paste
    something on a setting's files which is indented with spaces, instead of
    tabs. This is a problem because as I am over git versioning, I can easily
    create files with mixed tabs and spaces on the history, and some day later
    Sublime Text will fix the indentation to tabs, which will cause noise on the
    git history due the tab/space conversion war. % Cite reference to the war

    I think this can have a performance problem as when I am pasting something
    big on Sublime Text. Then to perform the conversion on the would not be
    easily possible and it should be performed afterwards by the user. Now
    Sublime Text should warn the user when he is pasting something indented with
    spaces instead of tabs in a file which is expected to be indented with tabs?

    The detection of whether the contents of the clipboard should not be
    expensive as we should just check some lines (which would not cover the
    cases where there is already mixed indentation on the clipboard contents).
    But there would be a performance problem when pasting something with very
    big first lines. On this case a threshold should stop Sublime Text from
    looking forward and ceases the detection as inconclusive for this paste and
    just paste like it is.



    \subsection{Computer Assisted Programming}

    Your computer should help you with with these unforeseen tasks. Why should I
    spend my precious time checking whether I am actually copying something
    space indented, when I am actually coping something tab indented?

    Therefore, how to do such a thing on this 21\q{}st century? Perhaps we
    should sit and cry while waiting for some greater force to come and rescue
    us. Or may be you should stop crying and actually do something about other
    than keep waiting for you mommy to come and save you from the darkness
    growing behind you back leading you to endless unsleepy nights fixing your
    code just because everything just went wrong.



    \section{The Upper Stream}

    TODO.


    \subsection{How to keep up with the upstream}

    TODO.



    \section{Common Tasks}

    So you are developing a software which is under version control, however to
    deploy your tests, you need to copy some big folders into the deployment or
    testing system. Then how do you do it?

    Copying and pasting them probably the most straight forward idea, which is
    nice if you are going to it only a few times in a life time like two or
    three. However if you are going to do it move than these limits,
    please don\q t do that. It is bad for the planet and is worsening your
    health for nothing other than more headaches.

    As a promptly good computer user, at this point you already have some tool,
    either graphical or by command line which can help you easily and fastly
    setup the folder\q s. Easily like:

    \begin{enumerate}
        \item You open the tool
        \item Click on the new button
        \item Name your sync task as `My cuttie'
        \item Copy and paste there source and destine addresses
        \item Hit the `sync now' button
    \end{enumerate}



    \section{The rsync side}

    Doing everything out of the box by a graphical interface seems not
    practical. Command Line Interfaces (CLI) are simpler to be built and allows
    their programmers to saver their efforts in actually writing the tool
    instead of designing a reasonable Graphical User Interface (GUI)
    \cite{quantificationOfInterface}.

    GUI interfaces are awesome but for their proper usage, which is mostly
    defined by their aim public. Non-computer programmers, perhaps even novice
    programmers, cannot easily deal with command lines, but experienced
    programmers should be able to get great advantage from it usage.
    \cite{commandLineInterface}.

    Following we may see an example about the simpleness of a shell script,
    which runs several commands to accomplish a clean build of the testing
    environment:

    \begin{lstlisting}[caption={rebuild\_workspace.sh}]
    #!/bin/sh
    printf "$(date)\nRemoving folders...\n"

    rm -rf "Installed Packages"
    rm -rf "Lib"
    rm -rf "Local"
    rm -rf "Packages"

    printf "Unzipping files...\n"
    unzip -q "Packages.zip"

    mkdir -p "./Deployment/Code A"
    mkdir -p "./Deployment/Code B"

    printf "Syncing folders...\n"
    rsync -r \
         "/cygdrive/d/Development/Environment/Code A/" \
         "/cygdrive/c/Test/Deployment/Code A/"

    rsync -r \
         "/cygdrive/d/Development/Environment/Code B/" \
         "/cygdrive/c/Test/Deployment/Code B/"
    \end{lstlisting}
    \vspace*{-4mm}

    On preceding example, the `rebuild\_workspace.sh' script is located on the
    testing folder `/cygdrive/c/Test', then when calling it we get some folders
    removed, a file unpacked on the current folder, and our code synced from the
    versioning system directly to testing environment. You can read more about
    `rsync' utility on \citeonline{synchronizingFolders}.


\end{englishtext}


% Portuguese
\lang{}{

    Perguntas como ``O que são boas práticas de programação?'' ou ainda ``O por
    quê estas práticas são boas?'', não são fáceis de responder. Mas cada
    programador aprende a escrever seus códigos em uma determinada maneira, com
    determinadas características como utilizar 4 ou 8 espaços para indentação de
    linhas, sempre deixar uma linha em branco antes de cada estrutura de
    controle como if\s, for\s, e afins.

    Mas entrando o universo de boas práticas, há muitos coisas sobre discorrer.
    Assim neste trabalho especificamente trabalhá-se sobre como realizar a
    melhor disposição/exibição do código de programação na tela do computador,
    de modo que maximize e facilite o entendimento do mesmo. Portanto permitindo
    que o programador dispersa mais tempe pensando sobre o problema, do que
    tentar decifrar a informação que é apresentado para ele na tela.

    Dentro desta área de trabalho, precisa-se também pensar muito bem sobre como
    compartilhar os códigos de programação dos programadores entre si. Isso por
    que entra agora o problema da diversidade de boas práticas de programação.
    Ela acontece por que muitas vezes, aquilo que é uma boa prática para um
    `programador A', não é para o outro `programador B'. Por exemplo, imagine um
    código onde um programador decidiu colocar antes de cada `if', uma linha em
    branco. Portanto é de se esperar que sempre que vemos uma linha em branco
    nos podemos potencialmente encontrar um `if'. Entretanto imagine que outro
    programador não gostou dessa regra e quando ele foi escrever seu código que
    envolvia um `if', ele não colocou a essa tal linha em branco que o outro
    programador vinha colocando. Então quando o primeiro programador for ler o
    código e procurar por `if'es, ele vai estar esperando por linhas em branco.
    Mas vai perder algum tempo procurando até perceber que o outro programador
    não as colocou.

    Essas diferenças dão-se devido a diversidade de meios de se aprender
    programação, tanto quanto aos gostos, aptidões e objetivos de cada
    programador. Assim hoje em dia isso torna-se um grande problema por que cada
    vez mais precisamos de mais e mais programadores trabalhem juntos entre si,
    desenvolvendo os mais diversos sistemas computações. Onde este último
    deve-se ao fato de que a complexidade dos sistemas computacionais cresce
    cada vez mais, portanto requer-se que mais e mais programadores trabalhem e
    compartilhem códigos.

    Então além de nos preocupar-mos somente como o código é exibido na tela do
    computador, nós precisamos nos preocupar sobre como ele será salvo no
    sistema de arquivos. Já que ao compartilhar o código, é vital o uso de um
    sistema de versionamento para permitir a gerências de projetos e os
    programadores em si, terem o controle de mudanças do código. O que permiti e
    facilmente possa realizar o rastreamento de mudanças e permitir que se possa
    entender melhor o que cada programador está fazendo a cada vez que ele
    formaliza um mudança no código através de uma `commit', como no sistemas
    `git` por exemplo.

    Isso por que quando trabalhos em um sistema de versionamento como `git'
    precisamos manter o código dentre um único estilo ou boa prática definida
    como padrão, devido ao fato de que se deixar-mos cada programador escrever
    como ele quiser, teremos muito ruído durante a revisão do código e estamos
    determinando o que o programador fez/escreveu, se cada programador
    re-escreve o histórico fazendo alterações como colocar linhas novas antes de
    cada if. Assim teremos ruído por que o foco de um sistema de versionamento é
    olhar somente as mudanças que são significativas para o código, como a
    criação de novas funções e não a adição de novas linhas em branco.

    Sobre o último ponto, podemos pensar também sobre uma abordagem da criação
    de um novo sistema de versão que foque somente nas mudanças significativas
    para o código, durante o momento da revisão. Entretanto essa abordagem não é
    ideal por que, por exemplo, ela dá margem para que programadores entrem em
    guerras tediantes e não produtivas de ajustes de código. Por exemplo,
    imagine o quão seria todo dia que você acorda e começa a trabalhar, você tem
    que passar pelo código colocando linhas novas antes de cada um dos if\s por
    que o programador do turno da noite tinha acabado de remover eles?


\section{Objetivos}

    Estabelecer relações entre boas práticas de programação e eficiência em
    programar, além de uma nova ferramenta ao apoio do programador com o intuito
    de automatizar o longo e diverso processo de programação em equipes de
    desenvolvedores com distintas boas práticas de programação.


\subsection{Objetivos específicos}

    \begin{enumerate}

        \item

        Um estudo sobre ferramentas universais de programação, que permitam que
        a partir de um único software, seja programado em todas as linguagens de
        programação. Assim explicar as diferenças para os outros softwares e os
        porquês de querer-se uma ferramenta única, ao invés de diversas.

        \item

        Definir, estudar, determinar e classificar o que são boas práticas de
        programação e realizar um estudo aprofundado sobre a as boas práticas da
        área de disposição visual, conhecidas também como `Beautifying'.

        \item

        Um estudo sobre as mais diversas ferramentas existentes para o apoio de
        boas práticas de programação, além de uma análise comparativa entre
        elas, determinando suas fraquezas e pontos fortes.

        \item

        A definição de um padrão de floxo de desenvolvimento que permita equipes
        de programadores com distintas boas práticas de programação, trabalhem
        em si sem intervir e iniciar guerras de boas práticas.

        \item

        Propor uma ferramenta única que permita diversas e distintas boas
        práticas de programação serem implementadas nas mais diversas linguagens
        de programação e que elas possam ser configuradas e definidas ao gosto
        dos programadores que a usa.

    \end{enumerate}


\section{Método de pesquisa}

    O trabalho será baseado em pesquisas em artigos, livros, teses,
    dissertações, sites de autores confiáveis, e por meio de novas provas
    demonstradas e baseadas através de argumentos no decorrer da evolução da
    monografia. Também sera apresentado os resultados decorridos da construção
    de uma nova ferramenta que proprõe a solução de um dos problemas
    apresentados e explicados.

    No último capítulo desta proposta encontra-se no tópico
    \autoref{sec:implementation} encontra-se uma série de links e referências
    que forma pré-selecionadas e poderão ser utilizadas na construção final
    deste trabalho. Notes que em si, as partes da última seção serão
    gradativamente movida para primeira parte do texto onde encontra-se pesquisa
    teórica, no decorrer que suas informações correlacionadas são incorporadas
    no trabalho escrito.

    Assim no final da primeira parte desta obra que dará-se no final da
    conclusão da disciplina intitulada de Trabalho de Conclusão de Curso 1,
    restarão somente as informações destinadas a implementação da ferramenta
    proposta, que serão implementadas na segunda parte da monografia denominada
    \nameref{sec:implementation}, que será desenvolvida no final da conclusão da
    disciplina de Trabalho de Conclusão de Curso 2.


}



    

% Is it possible to keep my translation together with original text?
% https://tex.stackexchange.com/questions/5076/is-it-possible-to-keep-my-translation-together-with-original-text
\chapter{\lang{Theory base}{Fundamentação Teórica}}

Fazer depois que a fundamentação teórica estiver concluída.


\section{Compiladores e Tradutores}
\label{compiladoresEtradutores}

\lang{%
    This work aims to propose a translator \cite{generatingInterpretiveTranslators},
    where the input and output languages are the same language.
    Such translation objective is to change the language representational structure,
    but without affecting the language lex,
    syntactic or semantics, i.e.,
    the language meaning.

    This program class is commonly know as text formatters.
    The differential from this work from others is the goal of a single expandable tool,
    capable of manipulating all existent and future programming languages,
    based on the use of deterministic \cite{introductionToContextFreeGrammars}
    and controlled nondeterministic
    % \cite{TODO:section explaning what does controlled means}
    context free grammars.
}{%
    Em linguagens formais,
    tradutores são ferramentas que operam realizando a transformação de um programa de entrada,
    em um programa de saída \cite{generatingInterpretiveTranslators}.
    Diferente de um compilador,
    a linguagem de destino da ``tradução'' é do \textbf{mesmo nível} que a linguagem de origem.
    Por exemplo,
    dado um programa de entrada em C++ e
    um programa de saída em Java,
    tem~=se um processo de tradução (Figura \ref{fig:pictures/ProcessoTraducao.png}).
    \advisor{A tradução é diferente de um processo de compilação,
    que é dotado de mais etapas \cite{translatorGenerationCompilier}.
    }{Pelo outro lado,
    dado um programa de entrada em C++ e
    um programa de saída em \textit{Assembly},
    tem~=se um processo de compilação \cite{translatorGenerationCompilier}.
    }
    \begin{figure}[h]
    \centering
    \includegraphics[width=1.0\textwidth]{pictures/ProcessoTraducao.png}
    \caption[Processo de Tradução]{Processo de Tradução -- Fonte Própria,
    \citeonline{ahoCompilerDragonBook}}
    \label{fig:pictures/ProcessoTraducao.png}
    \end{figure}

    No processo de compilação ou
    tradução,
    um Analisador Léxico cria múltiplos \textit{tokens}.
    Um token é composto por diversos atributos como a posição e
    o \textit{lexema}, i.e.,
    a sequencia de caracteres que este token representa no programa de entrada.
    Uma vez que o programa é ``\textit{tokenizado}'' pelo Analisador Léxico,
    o Analisador Sintático constrói a Árvore Sintática do programa.

    Utilizando a Árvore Sintática do programa de entrada,
    o tradutor constrói uma nova Árvore Sintática correspondente a Árvore Sintática da linguagem do programa destino,
    utilizada para construir o código~=fonte do programa destino.
    Em um processo de compilação,
    não seria criado uma nova Árvore Sintática como no processo de tradução,
    mas sim a geração de código objeto ou
    binário \cite{ahoCompilerDragonBook}.

    Analisadores Sintáticos podem ser Ascendentes\footnote{
    Do inglês, \textit{Bottom~=Up}
    }
    ou Descendentes\footnote{
    Do inglês, \textit{Top~=Down}
    }.
    Devido a essa característica ambos possuem as suas vantagens e
    desvantagens.
    Um Analisador Ascendente realiza a construção da Árvore Sintática das folhas até a raíz,
    o contrário de um Analisador Descendente que realiza a construção da Árvore Sintática a partir da raíz até as folhas\footnote{
    Como pode ser observados em seus nomes,
    ambos os Analisadores tanto da familia LL (Descendentes,
    \textit{Left-to-right, Leftmost derivation}) ou LR (Ascendentes, \textit{Left~=to~=right,
    Rightmost derivation}) fazem a leitura do programa de entrada da esquerda para a direita.
    }.

    Uma vantagem de um Analisador Ascendente é o suporte de uma maior classe de Gramáticas Determinísticas.
    Uma vantagem de um Analisador Descendente é a facilidade da recuperação de erros em relação aos Analisadores Ascendentes\footnote{
    Conceito abordado na
    \fullref{analisadoresSintaticos}
    }
    \cite{sippu1982,lr1ErrorRecovery,larkJosefGrosch}.


\section{Gramáticas}

    Gramáticas são conjuntos de regras que definem uma linguagem.
    Em linguagens formais,
    sendo $\alpha$ um Não~=Terminal,
    $\beta$ um Terminal,
    $V_n$ um conjunto de Não~=Terminais,
    $V_t$ um conjunto de Terminais e
    $V = V_n \cup V_t$,
    uma Gramática é definida por quatro componentes:
    \begin{enumerate}%[nosep,nolistsep]
        \item \advisor{O}{Um} conjunto $V_t$ de símbolos Terminais\advisor{ (também chamados
        de tokens ou símbolos do alfabeto da linguagem).
        Cada Terminal corresponde a um símbolo presente na linguagem.
        }{,
        chamados algumas vezes de ``\textit{tokens}'' devido a sua forte conexão.
        Cada Terminal corresponde a um símbolo presente no alfabeto da linguagem.
        }%
        Durante a Análise Léxica,
        os símbolos Terminais serão utilizados definir os lexemas que são a base principal dos tokens.
        Na composição da Árvore Sintática,
        os ``\textit{tokens}'' ou Terminais,
        serão sempre as folhas da Árvore Sintática.

        \item \advisor{O}{Um} conjunto $V_n$ de símbolos
        Não~=Terminais\advisor{ (algumas vezes chamados de ``variáveis sintáticas'').}{,
        algumas vezes chamados de ``variáveis sintáticas''.
        }
        Não~=Terminais servem para agrupar vários Não~=Terminais e\slash{}ou Terminais.
        Na composição da Árvore Sintática,
        os símbolos Não~=Terminais sempre serão os nós da Árvore Sintática\footnote{
        Desde que a Gramática da linguagem não contenha símbolos inúteis,
        i.e.,
        todos os símbolos da Gramática são férteis e
        permitem a geração de palavras além do conjunto vazio $\varnothing$ \cite{hopcroftBook}
        }.
        Por convenção,
        a intersecção entre o conjunto de símbolos Terminais e
        Não~=Terminais é sempre vazia,
        i.e.,
        $V_n \cap V_t = \varnothing$,
        uma vez que não seja respeitando essa convenção,
        não se tem como deterministicamente analisar o programa de entrada,
        podendo~=se indefinidamente realizar a substituição da mesma regra de produção.

        \item \label{definicaoDeGramatica}Um conjunto de produções $P$.
        Uma produção consiste em uma dupla elementos.
        O primeiro elemento é a cabeça ou
        lado esquerdo e
        representa a substituição ou
        consumo que será feito no programa de entrada.
        Ele é obrigatoriamente constituído de no mínimo um Não~=Terminal e
        um ou mais Não~=Terminais ou
        Terminais.
        O segundo elemento é a cauda ou
        lado direito da produção,
        composto de Terminais e\slash{}ou Não~=Terminais.
        Formalmente defini~=se uma produção pela seguinte regra,
        onde ``*'' representa o operador de fechamento do conjunto \cite{hopcroftBook}:
        $$P = \{\; \alpha ::= \beta \;|\; \alpha \in V^* V_n V^* \land \beta \in V^* \;\}$$

        \item Um símbolo inicial selecionado a partir do conjunto de símbolos Não~=Terminais.
        O símbolo inicial é utilizado para definir qual será a raíz da Árvore Sintática,
        e.g.,
        o última regra de produção utilizada para terminar o reconhecimento do programa de entrada em um Analisador Ascendente (ver seção \ref{reducoesEderivacoes}),
        e a primeira regra utilizada em um Analisador Descendente ou
        gerar~=se palavras desta linguagem\footnote{
        Processo natural quanto um Analisador Sintático realiza o reconhecimento de um programa de entrada.
        Para mais informações,
        veja a \fullref{reducoesEderivacoes}.
        }.
    \end{enumerate}


\subsection{Hierarquia de Chomsky}
\label{hierarquiaDeChomsky}

    Todas as Gramáticas que existem são no mínimo\footnote{
    Caso contrário não serão Gramáticas,
    mas qualquer outra definição no qual a Teoria de Linguagens Formais e
    Compiladores pode não se aplicar.
    }
    Gramáticas Tipo 0,
    também conhecidas como Gramáticas Irrestritas por que não possuem nenhuma restrição de complexidade de tempo,
    como os outros tipos de Gramáticas a serem definidos nas próximas seções{}.
    A partir da adição restrições sobre a definição formal de Gramática recém apresentada,
    também pode~=se \advisor{compreender}{realizar diversas classificações como} a hierarquia de \citeonline{chomskyGrammars1956},
    onde uma linguagem pode ser classificada como Regular,
    Livre de Contexto,
    Sensível ao Contexto e
    Irrestrita (Figura \ref{fig:pictures/HierarquiaDeChomsky.png}).
    \begin{figure}[h]
    \centering
    \includegraphics[width=1.0\textwidth]{pictures/HierarquiaDeChomsky.png}
    \begin{minipage}{\textwidth} \footnotesize
    *Para Gramáticas Regulares Determinísticas,
    complexidade linear ao tamanho da palavra de entrada para determinar se uma dada palavra pertence ou
    não a linguagem.
    Para Gramáticas Regulares Não~=Determinísticas,
    complexidade polinomial para construir as Árvores de Derivações e
    determinar se dada palavra pertence ou
    não a Linguagem com algoritmos como CYK \cite{hopcroftBook,cykParsingAlgorithm}.
    Por fim,
    para Automatos Finitos Não~=Determinísticos ou
    Analisadores com Backtracking,
    tempo exponencial.

    **Para Gramáticas Livres de Contexto Determinísticas,
    também conhecidas como LR(K),
    complexidade linear ao tamanho da palavra de entrada (veja a seção \ref{gramaticasVersusLinguagens}).
    Para Gramáticas Livre de Contexto Não~=Determinísticas,
    vale o mesmo que para Linguagens Regulares Não~=Determinísticas logo acima,
    mas no lugar de Automatos Finitos,
    utilizam~=se Máquinas de Pilha.

    ***Para verificar se uma dada sentença pertence ou não a linguagem.
    \end{minipage}
    \caption[Hierarquia de Chomsky]{Hierarquia de Chomsky -- Fonte Própria\protect\footnotemark,
    \citeonline{sipserBook,ahoTheoryOfParsing,efficientNonDeterministicParsers,johnCocke}}
    \label{fig:pictures/HierarquiaDeChomsky.png}
    \end{figure}
    \footnotetext{
    Veja \citeonline{computationalComplexityAuroraBarak,complexityClasses} para aprender mais sobre Classes de Complexidade.
    }

    Toda Gramática Regular ou
    Livre de Contexto,
    é também uma Gramática Irrestrita ou
    Sensível ao Contexto,
    uma vez que Gramáticas Livres de Contexto ou
    Regulares são um subconjunto das Gramáticas Irrestritas ou
    Sensíveis ao Contexto como apresentado na Figura \ref{fig:pictures/HierarquiaDeChomsky.png}.
    Por isso,
    também pode~=se chamar uma dada Gramática Regular de Irrestrita ou
    Livre de Contexto.

    Quando diz~=se que existe Gramática Livre de Contexto para uma dada Linguagem,
    pode~=se ter o equívoco de pensar que este é o melhor tipo,
    i.e.,
    o tipo mais eficiente em tempo computational de Gramática no qual dada Linguagem pode ser representada.
    Entretanto,
    precisa~=se tomar cuidado quando fala~=se sobre Gramáticas e
    Linguagens.

    Para uma dada Linguagem,
    dizer que existe uma Gramática Livre de Contexto para ela não significa que esta Linguagem é Livre de Contexto.
    Sempre pode~=se escrever uma Gramática menos eficiente do que o tipo mínimo de Gramática que uma Linguagem pode ser escrita.
    Utilizando o Lema do Bombeamento\footnote{
    Do inglês \textit{Pumping Lemma}
    }
    \cite{hopcroftBook,sipserBook} pode~=se determinar se dada Gramática é o tipo mínimo de Gramática para dada Linguagem.

    Assim,
    não pode~=se dizer uma dada Linguagem é Livre de Contexto simplesmente por que existe uma Gramática Livre de Contexto para dada Linguagem,
    pois também é preciso que esta Gramática seja o tipo mínimo no qual esta Linguagem pode ser escrita e
    para saber se este tipo de Gramática é o mínimo,
    utiliza~=se o Lema do Bombeamento.


\subsection{Gramáticas Regulares}

    Gramáticas Regulares (também conhecidas como Tipo 3) são todas aquelas reconhecidas por Automatos Finitos Determinísticos e\slash{}ou Não~=Determinísticos.
    Gramáticas de Linguagens Regulares pela definição formal,
    são todas aquelas nos quais todas as Produções $P$ da Gramática possuem a seguinte forma:
    $$ P = \{\; \alpha ::= a \beta \;|\; \alpha \in V_n \land a \in V_t
                \land \beta \in \{\; V_n \cup \varepsilon\; \} \;\} $$

\subsection{Gramáticas Livres de Contexto}

    Gramáticas Livres de Contexto (também conhecidas como Tipo 2) \cite{hopcroftBook} são todas aquelas reconhecidas por Automatos de Pilha Não~=Determinísticos.
    Gramáticas de Linguagens Livre de Contexto pela definição formal,
    são todas aquelas nos quais todas as Produções $P$ da Gramática possuem a seguinte forma:
    $$ P = \{\; \alpha ::= \beta \;|\; \alpha \in V_n \land \beta \in V^* \;\} $$


\subsection{Gramáticas Sensíveis ao Contexto}

    Gramáticas Sensíveis ao Contexto (também conhecidas como Tipo 1) são todas aquelas reconhecidas por Automatos Linearmente Limitados,
    que tratam~=se somente de Máquinas de Turing \cite{sipserBook} com Fita (ou memória) Finita.
    Gramáticas de Linguagens Sensíveis ao Contexto pela definição formal,
    são todas aquelas nos quais todas as Produções $P$ da Gramática possuem a seguinte forma:
    $$ P = \{\; \alpha ::= \beta \;|\; \alpha \in V^* V_n V^* \land \beta \in V^*
                \land \vert\alpha\vert \leq \vert\beta\vert \;\} $$


\subsection{Gramáticas Irrestritas}

    Por fim,
    as Gramáticas Irrestritas ou (também conhecidas como Tipo 0) possuem a mesma definição do
    que a definição válida de uma Gramática como apresentado anteriormente (no
    \fullref{definicaoDeGramatica}).
    Gramáticas Irrestritas são reconhecidas somente por Máquinas de Turing\footnote{
    Máquinas de Turing possuem por definição fita (ou memória) ilimitada,
    mas não infinita,
    pois em um dado momento,
    somente uma quantidade finita de símbolos podem estar na fita,
    que continuamente pode crescer ilimitadamente.
    },
    e diferente das Gramáticas Sensíveis ao Contexto,
    a Máquina de Turing não possui parada garantida.

    Linguagens do Tipo 0 (ou Irrestritas) representam problemas incomputáveis e
    que podem ser representados de procedimentos \cite{sipserBook}.
    Já Linguagens do Tipo 1 (ou Sensíveis ao Contexto),
    representam todos os problemas computáveis e
    sua implementação pode ser representada por algoritmos,
    pois possuem parada garantida,
    apesar de terem em pior caso,
    tempo exponential ao contrário de tempo infinito como nas Linguagens Irrestritas.


\section{Analisadores Sintáticos}
\label{analisadoresSintaticos}

    Analisadores são equivalentes a Mecanismos Reconhecedores como Automatos Finitos,
    Automatos de Pilha ou
    Máquinas de Turing.
    No caso de outros Mecanismos como Automatos Finitos,
    o reconhecimento é feito a partir da especificação ou
    construção do Automato que reconhece palavras de dada linguagem.
    Ambos Gramáticas e
    Automatos são equivalentes e
    existem algoritmos de conversão entre um e
    outro \cite{hopcroftBook}.

    Analisador Sintático\footnote{
    Além de Analisadores Sintáticos (Gramáticas Livre de Contexto),
    existem muitos outros como Analisadores Semânticos (Gramáticas Sensíveis ao Contexto) \cite{contextSensitiveParsing}.
    }
    é um nome dado para Analisadores que recebem como entrada uma Gramática que representa os aspectos estruturais de uma Linguagem,
    i.e.,
    sua sintaxe \cite{ahoCompilerDragonBook}.
    Analisadores Sintáticos possuem muito mais utilidade do que somente checar se a sintaxe do programa de entrada está correta.
    Uma vez que eles também geram a Árvore Sintática do programa\footnote{
    Como visto no começo desde capítulo na \fullref{compiladoresEtradutores}.
    }
    que é utilizada para realizar a análise semântica e
    geração de código.


\subsection{Gramáticas $versus$ Linguagens}
\label{gramaticasVersusLinguagens}

    É importante fazer a distinção entre Gramáticas Livre de Contexto e
    as Linguagens Livre de Contexto.
    \citeonline{parikh1966} provou que existem linguagens nas quais não existem Gramáticas Não~=Ambíguas que representem estas linguagens.
    Tais linguagens são conhecidas como Linguagens Inerentemente Ambíguas\footnote{
    Do inglês,
    \textit{Inherently Ambiguous Languages}.
    }
    onde não existem Gramáticas Livre de Contexto Determinísticas capazes de representa~=las e
    tais Linguagens somente podem ser reconhecidas por Analisadores com Backtracking \cite{ahoCompilerDragonBook} ou
    Automatos de Pilha Não~=Determinísticos.

    A maior classe de Gramáticas Determinísticas suportadas por Analisadores Sintáticos são as Gramáticas LR(K)\footnote{
    Do inglês, \textit{Left~=to~=right,
    Rightmost derivation} em reverso com K símbolos de \textit{lookahead}.
    \textit{Rightmost} significa que ao realizar as derivações,
    escolhe~=se sempre o Não~=Terminal mais a direita.
    }.
    Analisadores LR(K) \cite{ahoCompilerDragonBook} são Ascendentes e
    reconhecem um subconjunto das Linguagens Livre de Contexto (Figura \ref{fig:pictures/LinguagensDeterministicas.png}).
    Já os Analisadores LL(K)\footnote{
    Do inglês, \textit{Left-to-right,
    Leftmost derivation} com K símbolos de \textit{lookahead}.
    \textit{Leftmost} significa que ao realizar as derivações,
    escolhe~=se sempre o Não~=Terminal mais a esquerda.
    }
    são Descendentes \cite{antlrBookTerrentParr,llStarAntlr} e
    reconhecem somente um subconjunto das Linguagens LR(K)\footnote{
    Diz~=se que uma linguagem é LR(K) ou
    LL(K) quando ela é reconhecida por este Analisador
    }.

    A Figura \ref{fig:pictures/LinguagensDeterministicas.png} não é inteiramente um Diagrama de Venn \cite{generalizedVennDiagrams},
    inicialmente,
    nas camadas mais externas,
    ele é uma relação abstrata entre Linguagens Ambíguas e
    Gramáticas Determinísticas.
    O Conjunto das Gramáticas Livre de Contexto Determinísticas está contido dentro das Linguagens Livre de Contexto\footnote{
    Também existem Gramáticas Sensíveis ao Contexto Determinísticas \cite{contextSensitiveParsing},
    entretanto,
    algoritmos de análise possuem em pior caso,
    complexidade exponencial.
    }.
    O primeiro nível significa que todas as Linguagens Inerentemente Ambíguas\footnote{
    É comum confundir~=se e
    chamar Gramáticas de Inerentemente Ambíguas,
    mas esse termo não existe para gramáticas.
    Ou elas são Ambíguas ou
    Não.
    Somente uma Linguagem pode ser Inerentemente Ambígua.
    }
    são representáveis somente por Gramáticas Ambíguas.

    O segundo nível significa que Linguagens Não~=Inerentemente Ambíguas\footnote{
    Somente utilizado para enfatizar o conjunto de Linguagens no qual existem Gramáticas Ambíguas e
    Determinísticas (ou Não~=Ambíguas).
    }
    podem ser representadas por Gramáticas Ambíguas e\slash{}ou Determinísticas.
    No terceiro nível encontra~=se as Gramáticas que são mais importantes,
    as Gramáticas Determinísticas\footnote{
    As Gramáticas Determinísticas representam o conjunto de Linguagens que podem ser Analisadas Deterministicamente e
    tais Linguagens também podem ser conhecidas como LR(K),
    LR(K).
    Reveja os parágrafos após a Figura \ref{fig:pictures/HierarquiaDeChomsky.png}
    },
    que podem ser classificadas como LR(K),
    LL(K)\advisor{etc, }{ entre outros, }i.e.,
    de acordo com o tipo de Analisador que pode ser construído.
    \begin{figure}[h]
    \centering
    \includegraphics[width=1.0\textwidth]{pictures/LinguagensDeterministicas.png}
    \caption[Gramáticas Determinísticas \textit{versus} suas Linguagens]{Gramáticas Determinísticas \textit{versus} suas Linguagens -- Fonte Própria,
    \citeonline{llVersusLrContainment,llContainmentInLalr,beatty1982,ahoCompilerDragonBook}}
    \label{fig:pictures/LinguagensDeterministicas.png}
    \end{figure}

    Diz~= que um Analisador Determinístico para dada Gramática pode ser construído quando a criação de sua Tabela de Análise\footnote{
    Do inglês, \textit{Parsing Table}
    }
    \cite{ahoCompilerDragonBook} acontece sem conflitos.
    É importante notar que usualmente o processo de análise por um Analisador,
    seja ele LR(K) ou
    LL(K),
    acontece em duas etapas.
    Com a exceção dos Analisadores LL(K) que também podem ser facilmente construídos programaticamente,
    i.e.,
    com o programador construindo manualmente como deve acontecer cada transição de estado do analisador \cite{ahoCompilerDragonBook}.

    Na primeira etapa do Analisador utilizam~=se algoritmos de construção da Tabela de Análise.
    Quando a tabela está construída sem conflitos (este Analisador portanto,
    é Determinístico),
    entra em cena o algoritmo de análise na segunda etapa,
    que utilizando a Tabela de Análise,
    realiza o reconhecimento do programa de entrada.

    Como única diferença entre os Analisadores LR(K),
    LALR(K) e
    SLR(K) é exatamente construção da Tabela de Análise,
    ambos possuem a mesma complexidade de análise linear \cite{knuthLrParser1965} ao tamanho do programa\footnote{
    Quando refere~=se a programa,
    fala~=se da \textit{string} ou
    texto que será analisado e
    decidir se tal programa é um programa da linguagem que se está analisando.
    Um ponto curioso,
    caso o programa não seja aceito pelo analisador,
    ele não é um programa com erros,
    mas um programa inválido,
    i.e.,
    de uma outra linguagem,
    que não é a linguagem que está sendo analisada.
    Comumente ou
    informalmente,
    chamamos estes programas como programas com erros (de sintaxe).
    }
    de entrada que será analisado \cite{linearLL1AndLR1Grammars,generalContextFreeParsingAlgorithm}.

    No caso de conflitos na Tabela de Análise,
    a Gramática não pode ser analisada deterministicamente e
    algoritmos de análise com backtracking (ou CYK,
    reveja a seção \ref{hierarquiaDeChomsky}) precisam ser utilizados para construção da Árvore Sintática.
    Como mostrado para Máquinas de Turing Não~=Determinísticas na seção \ref{mecanismosReconhecedores},
    Analisadores com Backtracking também funcionam em pior caso,
    com tempo exponencial e
    podem escolher uma estratégia como Busca em Profundidade\footnote{
    Veja a \fullref{buscaEmLarguraEProfundidade} para saber mais.
    } para executar os Ramos de Computação Não~=Determinístico.


\subsection{Reduções e Derivações}
\label{reducoesEderivacoes}

    Diferente Máquinas Reconhecedoras específicas como Automatos Finitos,
    os Analisadores recebem diretamente como entrada uma Gramática de uma dada Linguagem.
    Mas diferente de Gramáticas e
    Analisadores LL(K),
    Analisadores LR(K) especificamente funcionam de modo contrário.
    Eles operam por meio de Reduções ao invés de Derivações como no caso das Gramáticas e
    Analisadores LL(K) \cite{sipserBook}.

    Uma Derivação acontece quando uma regra de produção como ``$S \Rightarrow a a $'' de uma Gramática expande e
    tem~=se como resultado ``$a a$'' a partir do símbolo de origem ``$S$''.
    Já uma Redução acontece quanto a dada regra de produção como ``$S \Rightarrow a a $'' de uma Gramática reduz e
    tem~=se como resultado ``$S$'' a partir do símbolo de origem ``$a a$''.

    Tanto Derivações quanto Reduções podem ser descritas em termos que quantos passos são necessários para que se possa sair de um ponto até outro \cite{ahoCompilerDragonBook}:
    \begin{enumerate}%[nosep,nolistsep]
        \item Quanto um derivação é denotada como ``$S \Rightarrow a a $'',
        isso significa que somente um passo é necessário para sair do símbolo inicial ``$S$'' e
        chegar no símbolo final ``$a a$''.
        \item Quanto um derivação é denotada como ``$S \xRightarrow{*} a a $'',
        isso significa que são necessários,
        desde zero (nenhum) até infinitos passos para sair do símbolo inicial ``$S$'' e
        chegar no símbolo final ``$a a$''.
        \item Quanto um derivação é denotada como ``$S \xRightarrow{+} a a $'',
        isso significa que são necessários,
        desde um passo até infinitos passos para sair do símbolo inicial ``$S$'' e
        chegar no símbolo final ``$a a$''.
    \end{enumerate}

    Para reduções,
    estas mesmas condições se aplicam,
    mas em ordem reversa,
    i.e., $\Leftarrow$, $\xLeftarrow{*}$ e $\xLeftarrow{+}$,
    ao invés de $\Rightarrow$,
    $\xRightarrow{*}$ e
    $\xRightarrow{+}$.
    Enquanto Gramáticas são geradores de palavras que partem do Símbolo Inicial da Gramática até gerarem uma palavra da linguagem,
    Analisadores Ascendentes como LR(K) são reconhecedores de palavras.

    Diferente de Gramáticas,
    Analisadores Ascendentes partem de uma palavra da Linguagem até chegarem no Símbolo Inicial da Gramática,
    consumindo toda a palavra de entrada e
    chegando em um Estado de Aceitação.
    Já Analisadores Descendentes como LL(K),
    partem do Símbolo Inicial da Gramática até consumirem toda palavra de entrada,
    também chegando um em Estado de Aceitação.

    Ambos os Analisadores Ascendentes ou
    Descendentes,
    terminam no final do processo,
    gerando toda a Árvore de Derivação.
    Entretanto,
    caso no final do processo,
    não chegue~=se em um Estado de Aceitação,
    i.e.,
    no Símbolo Inicial da Gramática.
    Tem~=se somente a construção de uma Árvore de Derivação partial \cite{allStarAntlr}.

    No caso dos Analisadores Ascendentes,
    será uma floresta de árvores,
    por que somente no final da análise,
    com a chegada ao símbolo inicial da Gramática,
    completa~=se custura de todas a árvores que foram parcialmente construídas durante o processo de análise (\textit{Bottom~=Up}).

    Já no caso dos Analisadores Descendentes,
    não existe uma floresta de árvores.
    Como parte~=se diretamente do Símbolo Inicial da Gramática,
    a Árvore de Derivação desde o começo é construído como sendo uma única árvore (\textit{Top~=Down}).


\subsection{Analisadores LR(K)}

    Como pode ser observado na Figura \ref{fig:pictures/LinguagensDeterministicas.png},
    existem Gramáticas SLR(K) que não são Gramáticas LL(K) por que para uma Gramáticas ser LL(K),
    ela precisa respeitar 3 propriedades,
    \begin{inparaenum}
        \item Não possuir Recursão a Esquerda,
        \item Estar fatorada e
        \item $\forall\; A\, \in\, V_n\; |\; A\,
                \xRightarrow{*}\, \varepsilon\,
                \land\, First(A)\, \cap\, Follow(A) = \varnothing$
    \end{inparaenum}
    \cite{ahoCompilerDragonBook}.

    Entretanto,
    Gramáticas LR(K), LALR(K) e
    SLR(K) não precisam de nenhuma dessas restrições.
    No caso da Recursão a Esquerda,
    o algoritmo de criação da Tabela de Análise Sintática da Gramática LR(K),
    LALR(K) ou SLR(K),
    não possui o problema de entrar em um loop infinito assim como acontecem com as Gramáticas LL(K),
    portanto aceitando~=se Gramáticas com Recursão a Esquerda.

    Uma vez que Analisadores LR(K) requerem uma quantidade de memória exponencial \cite{complexityOfLRKTesting} ao tamanho da Gramáticas de entrada para operar,
    \citeonline{lalrDeRemer1982} criaram os Analisadores LALR(K)\footnote{
    Do inglês, \textit{Look~=Ahead} LA(K) LR(0),
    onde LR(0) é um Analisador LR(K) com $K=0$
    }
    e SLR(K)\footnote{
    Do inglês, \textit{Simple LR(K) parser}
    }
    com o objeto de viabilizar a implementação de Analisadores Ascendentes Determinísticos.

    Gramáticas de Linguagens Determinísticas são chamadas de LR,
    por que todas as Linguagens Determinísticas são reconhecidas por Analisadores LR(K),
    uma vez que \citeonline{knuthLrParser1965} provou que todas as Gramáticas Determinísticas são aceitas por um Analisador LR(K).

    Assim,
    além da hierarquia de Chomsky,
    também classifica~=se as Gramáticas de acordo com o tipo de Analisador que reconhece as linguagens representadas por elas.
    Como mostrado na Figura \ref{fig:pictures/LinguagensDeterministicas.png},
    nem todas as Gramáticas Livre de Contexto são de Determinísticas e
    uma Gramática é Determinística somente se ela pode ser reconhecida por um Analisador LR(K).

    Portanto uma maneira fácil de decidir se uma dada Gramática é Determinística ou
    não,
    é tentar construir a sua tabela de um Analisador LR(K).
    Caso consiga~=se construir com sucesso (sem conflitos) a Tabela de Análise Sintática \cite{ahoCompilerDragonBook},
    a Gramática é LR(K),
    caso contrário a Gramática não é Determinística.

    A mesma técnica pode ser aplicada no caso de Analisadores menos poderosos como LALR(K),
    entretanto,
    uma vez que não se consiga construir a Tabela de Análise Sintática,
    não se pode ter certeza se dada Gramática é ou não Determinística.


\subsection{Análise Semântica}

    Usualmente,
    somente depois que a Árvore Sintática é construída,
    realiza~=se o processo de Análise Semântica \cite{ahoCompilerDragonBook},
    i.e.,
    a verificação da corretude do programa escrito em relação os aspectos Não~=estruturais,
    por exemplo,
    é sintaticamente correto escrever a declaração de uma mesma variável duas vezes ou
    mais.

    Entretanto,
    para algumas linguagens é semanticamente errado redeclarar uma variável duas vezes ou
    mais.
    O Analisador Sintático representado por uma Gramática Livre de Contexto não tem poder suficiente para realizar tais verificações devido as limitações desse tipo de Gramática,
    que restringem~=se a estrutura do programa e
    não a seu significado (semântica).

    Nem todas as linguagens podem ser Analisadas completamente em diferentes etapas,
    como Análise Léxica, Sintática e Semântica. Muitas vezes,
    estas três etapas acontecem em paralelo como realizado na implementação do compilador da Linguagem C \cite{jourdan2017,whyCcannotBeParsedWithALR1Parser}.

    A Gramática da linguagem C não é Livre de Contexto devido as ambiguidades (conhecido também como Não~=Determinismo) existentes como a expressão ``\textit{x * y ;}''.
    Tal sentença pode ser ou
    a declaração de um ponteiro chamado \textit{y} do tipo \textit{x},
    ou a multiplicação de dois números armazenados nas variáveis \textit{x} e
    \textit{y},
    portanto ela não pode ser aceita por um Analisador LR(K) tradicional.

    Uma otimização que o compilador C faz para poder fazer o Analisador ``Determinístico'' da linguagem C,
    e assim saber se a expressão ``\textit{x * y ;}'' trata~=se de de uma mera multiplicação ou
    a declaração de uma variável,
    é exatamente a realização simultânea da Análise Léxica,
    Sintática e
    Semântica.
    Uma vez que um novo \textit{token} é reconhecido,
    ele é alimentado para o Analisador Sintático,
    que também o alimenta para o Analisador Semântico.

    Assim,
    o Analisador Sintático é capaz de consultar a Tabela de Símbolos \cite{ahoCompilerDragonBook} e
    descobrir se dado token ou
    tratar~=se de um tipo ou
    uma variável numérica.
    Entretanto,
    requer~=se cuidado sobre como estas alterações são feitas,
    pois pode~=se pensar que as Gramáticas de todas as Linguagens de Programação são ``Livres de Contexto'' e
    Determinísticas.
    E uma vez que a Gramática não é mais Livre de Contexto ou
    Determinística,
    pode~=se mover Aspectos Sensíveis ao Contexto para o Analisador Semântico,
    assim,
    deixando a Gramática somente com aspectos determinísticos.


\subsection{Alterações nos Analisadores Sintáticos}

    Dependendo de como o Analisador Sintático de Gramáticas Livre de Contexto é alterado,
    o conjunto de Gramáticas aceitos por tal Analisador pode deixar de serem Livres de Contexto.
    As Gramáticas somente continuarão Livre de Contexto caso estas alterações sejam somente mover checagens da etapa de Análise Sintática para a etapa de Análise Semântica sem realizar alterações no Analisador Sintático.

    Quanto se adiciona suporte a Aspectos Sensíveis ao Contexto \cite{contextSensitiveParsing} a Gramáticas Livre de Contexto por meio de alterações do Analisador Sintático,
    como feito no Analisador da Linguagem C,
    o Analisador da Gramática deixa de ser Livre de Contexto,
    suportando assim,
    algumas Gramáticas Sensíveis ao Contexto e\slash{}ou também algumas Gramáticas Não~=Determinísticas.

    Note que,
    a pesar disso não impede~=se que a Gramática da Linguagem C,
    como mostrado na seção anterior,
    seja analisada com eficiência.
    Mas isso deixa a brechas para que ela possa não ser analisada com eficiência.
    A diferença para um Analisador onde a Gramática é inteiramente Livre de Contexto,
    é que elas tem performance \textit{garantida} pela sua Classe de Complexidade (Veja seção \ref{classesDeComplexidade}).

    Sintaxe e
    Semântica de Linguagens são completamente ortogonais.
    Gramáticas de Linguagens Irrestritas\footnote{
    Não a linguagem no qual elas representam,
    mas a própria Gramática em si \cite{finiteAutomataTuringComplete}
    }
    podem ser Turing Completas\footnote{
    A Turing Completude acontece quando uma dada linguagem pode simular o funcionamento completo de uma Máquina de Turing
    }
    devido a sua equivalência com Máquinas de Turing e
    são capazes de realizar qualquer operação computacional.
    Mas,
    isso não pode ser confundido com as \textit{Strings} ou
    Programas gerados por essas Gramáticas \cite{areThereDomainSpecificLanguages}.

    Tais programas podem ou
    não ser Turing Completos.
    Do lado oposto,
    até Linguagens Regulares podem gerar programas que são Turing Completos,
    mesmo que seu dispositivo reconhecedor equivalente,
    os Automatos Finitos,
    não tenham Turing Completude\footnote{
    Caso isso esteja confuso,
    reveja a Figura \ref{fig:pictures/HierarquiaDeChomsky.png} e
    note que de todas as Linguagens,
    quem tem Turing Completude são as Linguagens Irrestritas,
    enquanto Automatos Finitos são um subconjunto das Máquinas de Turing \cite{finiteAutomataTuringComplete}
    }
    \cite{turingCompleteRegularLanguages,finiteAutomataTuringComplete}.

    Na seção \ref{sec:software_implementation},
    será mostrado a implementação de uma Gramática ``Livre de Contexto'' em um Analisador LALR(1),
    onde Aspectos Sensíveis ao Contexto serão analisados pelo Analisador Semântico,
    tal como feito na implementação do Compilador da Linguagem C apresentado.
    Mas com a diferença de que utiliza~= um Analisador LALR(1) genérico,
    ao contrário de um Analisador feito exclusivamente para a linguagem alvo.

    Este Analisador LALR(1),
    possui suporte a pequenos ``\textit{\englishword{hacks}}'' ou
    otimizações que permitem adicionar alguns aspectos Sensíveis ao Contexto ao Analisador LALR(1).
    Assim,
    as Gramáticas aceitas por esse Analisador incluem somente algumas Gramáticas Não~=Determinísticas,
    não incluindo todas as Gramáticas Livre de Contexto Ambíguas ou
    Sensíveis ao Contexto devido a limitações das alterações do algoritmo de Análise LALR(1) \cite{larkContextualLexer}.


\section{\advisor{Compiladores e }{}Classes de Complexidade}
\label{classesDeComplexidade}

    Como um todo,
    o conjunto de Linguagens Regulares pode ser considerado com complexidade linear\footnote{
    Complexidade linear é um caso particular de complexidade polinomial onde o grau do Polinômio é 1,
    i.e.,
    $\Theta(n)$.
    Aprenda mais sobre complexidade linear com \citeonline{cormenIntroductionToAlgorithms,computationalComplexityAuroraBarak}.
    }
    em tempo computacional para determinar de dada palavra pertence ou
    não a linguagem,
    por que toda Gramática Regular Não~=Determinística pode ser convertida em uma Gramática Regular Determinística \cite{sipserBook}.

    Infelizmente isso não é verdade para Gramáticas Livres de Contexto,
    por que Gramáticas Livre de Contexto Determinísticas $\Theta(n)$ e
    Não~=Determinísticas não são equivalentes.
    Gramáticas Não~=Determinísticas possuem complexidade exponential,
    quando Analisadas por um Analisador com Backtracking.
    Em contra~=partida,
    Gramáticas Livre de Contexto Não~=Determinísticas também podem ser Analisadas em tempo polinomial $\Theta(n^3)$ utilizando algoritmos de parsing tal como CYK (seção \ref{hierarquiaDeChomsky} e
    \citeonline{larkContextualLexer}).


\advisor{}{%
\subsection{Complexidade Computacional com Computadores Quânticos}

    Com a exceção de alguns problemas específicos \cite{theGoodAndBadQuantumComputing},
    a execução probabilística de Computadores Quânticos \cite{nonlinearQuantumComputers},
    baseados nas leis da Física Quântica \cite{dicke1963QuantumPhysicsIntroduction},
    podem cortar\footnote{
    Devido as probabilidades envolvidas,
    somente um ou
    alguns dos ramos de computação serão seguidos durante a execução do Algoritmo Quântico,
    pelo Computador Quântico.
    }
    caminho ``pulando'' ramos de Computação Não~=Determinísticos (da computação Clássica) com a superposição quântica.
    Assim,
    conseguindo resolver alguns problemas que são exponenciais,
    em tempo polinomial ao tamanho da entrada,
    utilizando algoritmos específicos para computadores quânticos \cite{quantumComputerSurvey,quantumSimulatorChagas}.

    Esta é a gama de problemas nos quais Computadores Quânticos são úteis \cite{quantumComputingForNonPhysicists},
    não sendo assim,
    substitutos completos da Computação Tradicional (ou Clássica) \cite{efficientQuantumComputation},
    somente otimizadores na resolução de alguns problemas que podem ser otimizados devido as propriedades específicas\slash{}probabilísticas das leis Física Quântica \cite{churchTuringQuantumComputer}.

    Pode~=se confundir Computadores Quânticos como equivalentes a Analisadores Não~=Determinísticos devido as nomenclaturas utilizadas.
    Enquanto Analisadores são Não~=Determinísticos devido à ambiguidades nas Gramáticas de Entrada,
    Computadores Quânticos são Não Determinísticos devido à serem baseado em modelos Probabilísticos,
    i.e.,
    Computadores Quânticos não são equivalentes a Analisadores Não~=Determinísticos devido a sua execução ser probabilística \cite{polynomialQuantumComputers,probabilisticQuantumComputation,quantumSimulatorChagas}.

    Diferente dos Computadores Tradicionais,
    Computadores Quânticos são construídos com base nas leis da Física Quântica,
    que são radicalmente diferentes das Leis da Física Tradicional ou
    Clássica, i.e.,
    as Leis de Newton.
    As Leis da Física Clássica regem os elementos muitos grandes na escala galáxias,
    planetas, células e
    virus \cite{halliday2013fundamentals}.
    Já as Leis da Física Quântica regem as elementos muito pequenos na escala de átomos,
    elétrons, prótons, fótons e
    \textit{quarks} \cite{dicke1963QuantumPhysicsIntroduction}.
}

\subsection{Complexidade Teórica $versus$ Real}

    Na Figura \ref{fig:pictures/ParserNonDeterministic.png},
    encontra~=se uma Árvore de Computação de um Analisador Não~=Determinístico.
    Diz~=se que que o tempo de execução de um Analisador Não~=Determinístico é Não~=Determinístico  Polinomial $NP$\footnote{
    Do inglês, \textit{Non~=Deterministic Polynomial Time},
    comumente conhecida pela pergunta,
    $P \stackrel{?}{=} NP$, i.e.,
    a classe de problemas com complexidade de tempo polinomial,
    está estritamente contida na classe de problemas $NP$ (Não~=Determinísticos Polinomiais) \cite{computationalComplexityAuroraBarak}?
    }
    ao tamanho da entrada,
    por que um Analisador Não~=Determinístico executa simultaneamente todos os ramos de Computação Não~=Determinísticos \cite{hopcroftBook}.

    Como mostrado na Figura \ref{fig:pictures/ParserNonDeterministic.png},
    após a cada um dos passo de computação 1,
    2, 3 e 4,
    todos os 15 ramos de computação foram concluídos.
    Cada um desses passos é corresponde a um item a ser analisado na entrada do programa.
    E esta computação,
    acontece em tempo Não~=Determinístico Polinomial,
    com expoente de $n$ igual a $1$,
    i.e.,
    $n^1$.
    \begin{figure}[h]
    \centering
    \includegraphics[width=1.0\textwidth]{pictures/ParserNonDeterministic.png}
    \caption[Árvore de Computação com 4 Passos de um Problema da Classe $NP$]{Árvore de Computação com 4 Passos de um Problema da Classe $NP$ -- Fonte Própria}
    \label{fig:pictures/ParserNonDeterministic.png}
    \end{figure}

    O que torna a computação Não~=Determinística\footnote{
    E pertencente a classe dos problemas Não~=Determinístico Polinomial.
    } é o fato de cada um dos itens 1,
    2, 3 e 4 da entrada,
    permitirem simultaneamente a escolha de mais de um caminho na escolha do próximo estado do Analisador,
    i.e.,
    mais de um Ramo de Computação,
    devido a ambiguidades da Gramática de entrada \cite{antlrBookTerrentParr}.

    Nesse contexto,
    $P$ representa o conjunto de problemas resolvidos tempo Determinístico Polinomial (por Máquinas de Turing Determinísticas),
    enquanto $NP$ o conjunto de Problemas resolvido em tempo Não~=Determinístico Polinomial (por Máquinas de Turing Não~=Determinísticas).
    Assim,
    um problema Não~=Determinístico Polinomial somente pode ser resolvido por uma Máquinas de Turing Determinística em tempo exponencial.

    O tempo de execução será linear ao tamanho da entrada caso o Analisador Não~=Determinístico seja de uma Linguagem Regular e
    implementado através de um Automato Finito.
    O tempo de execução será polinomial ao tamanho da entrada caso o Analisador Não~=Determinístico seja de uma Linguagem Livre de Contexto Ambígua (Gramática Não~=Determinística) e
    implementado através de algum algoritmo com tempo polinomial como CYK \cite{allStarAntlr}.

    Como já explicado nas observações da Figura \ref{fig:pictures/HierarquiaDeChomsky.png},
    existem duas classes distintas de Complexidade para Gramáticas Livre de Contexto Não~=Determinísticas.
    Quando faz~=se uma Análise de uma Gramáticas Livre de Contexto Não~=Determinística,
    tem~=se como resultado várias possíveis Árvores de Derivação\footnote{
    Como a Gramática é Não~=Determinística,
    existem muitas possíveis Árvores de Derivação,
    (devido à ambiguidade da Gramática).
    }.

    Algoritmos de construção das Árvores de Derivação com Backtracking para Gramáticas Livre de Contexto Não~=Determinística,
    possuem tempo exponencial de execução.
    Já algoritmos como CYK,
    que simplesmente dizem se dada palavra pertence ou
    não a Linguagem,
    possuem complexidade polinomial ao tamanho da palavra de entrada \cite{hopcroftBook}.


\subsection{Mecanismos Reconhecedores}
\label{mecanismosReconhecedores}

    Uma vez que o conjunto de Linguagens Determinísticas LR(K) (com tempo linear) está contida no conjunto das Linguagens Livres de Contexto,
    não considera~=se tempos Análise Lineares ou
    Polinomiais de execução para Linguagens Sensíveis ao Contexto ou
    Irrestritas,
    por que tudo o que é eficiente é Livre de Contexto e
    Determinístico.

    Algoritmos de Análise para essas outras classes de Linguagens serão exponenciais em pior caso \cite{contextSensitiveParsing} e
    quando analisados por dispositivos equivalentes a Máquinas de Turing Não~=Determinísticas terão no mínimo tempo polinomial\footnote{
    Quando fala~=se de complexidade de tempo polinomial para Máquinas Não~=Determinísticas,
    resulta~=se em uma complexidade de tempo exponencial ao simular o funcionamento dessa Máquina Não~=Determinística em um computador,
    ou seja,
    a Real Complexidade do problema termina sendo exponencial,
    enquanto teoricamente a complexidade é polinomial.
    Veja as figuras \ref{fig:pictures/ParserNonDeterministic.png} e
    \ref{fig:pictures/ParserDeterministic.png} e
    as compare.
    }
    de execução.

    Máquinas de Turing Não~=Determinísticas que resolvem os problemas da Classe $NP$ em tempo polinomial não existem fisicamente,
    portanto sua complexidade de tempo reduzida não pode ser alcançada e
    seu tempo de execução é exponential,
    pois para simular o funcionamento de uma Máquina de Turing Não~=Determinística,
    utiliza~=se uma Máquina de Turing Determinística\footnote{
    Máquinas de Turing Determinísticas são equivalentes aos computadores de propósito geral
    }
    \cite{sipserBook,turingMachinesRoyer}.

    Assim,
    máquinas de Turing Determinísticas e
    Não~=Determinísticas são equivalentes,
    pois sempre é possível simular o funcionamento de uma Máquina de Turing Não~=Determinística,
    utilizando uma Máquina de Turing Determinística \cite{hopcroftBook}.

    Na Figura \ref{fig:pictures/ParserDeterministic.png},
    encontra~=se a mesma Árvore de Computação apresentada na Figura \ref{fig:pictures/ParserNonDeterministic.png},
    mas com a diferença de que desta vez utiliza~=se uma máquina de Turing Determinística ao contrário de uma Máquina de Turing Não~=Determinística.
    Com isso,
    ao invés de um tempo polinomial ao tamanho da entrada,
    tem~=se um tempo exponential ao tamanho da entrada.
    \begin{figure}[h]
    \centering
    \includegraphics[width=1.0\textwidth]{pictures/ParserDeterministic.png}
    \caption[Árvore de Computação com 15 Passos]{Árvore de Computação com 15 Passos -- Fonte Própria}
    \label{fig:pictures/ParserDeterministic.png}
    \end{figure}

    Para que uma Máquina de Turing Determinística possa processar uma Gramática Não~=Determinística,
    é necessário execute cada um dos ramos de computação.
    Já Máquinas de Turing Não~=Determinísticas\footnote{
    Máquinas de Turing da classe $NP$,
    somente existem teoricamente
    }
    executam simultaneamente todos os ramos de computação Não~=determinísticos,
    conseguindo assim, desempenho linear ou
    polinomial ao tamanho da entrada compondo os problemas da classe $NP$ (com tempo Não~=Determinístico Polinomial) \cite{hopcroftBook}.


\subsection{Busca em Largura e Profundidade}
\label{buscaEmLarguraEProfundidade}

    Quando uma Máquina de Turing Determinística é utilizado para simular o funcionamento de uma Máquina de Turing Não~=Determinística,
    ela precisa decidir como escolher executar os Ramos de Computação Não~=Determinísticos \cite{sipserBook}.
    Duas principais abordagens distintas e
    conhecidas\footnote{
    Além dessas duas abordagens,
    existem muitas outras técnicas que podem ser cridas como misturas dessas duas estratégias extremas,
    como heurísticas e inteligências artificias
    }
    são a Busca em Largura\footnote{
    Do inglês, \textit{Breadth-First Search (BFS)}
    }
    e Busca em Profundidade\footnote{
    Do inglês, \textit{Depth-First Search (DFS)}
    }.
    Os algoritmos funcionamento desses tipos de busca são detalhadas em \citeonline{cormenIntroductionToAlgorithms}.

    Cada uma delas apresenta suas vantagens e
    desvantagens.
    Uma vantagem da Busca em Profundidade é possibilidade de ``sorte'',
    caso o primeiro ramo Não~=Determinístico que escolhe~=se seja uma solução para o problema,
    i.e.,
    leve o Analisador a um estado de aceitação,
    mas ao mesmo tempo de pode~=se ter ``sorte'',
    pode~=se ter o ``azar'' de que o primeiro ramo Não~=Determinístico seja um ramo infinito de computações que nunca levarão o Analisador à um estado de aceitação.

    A Figura \ref{fig:pictures/ParserDeterministic.png} mostrou um exemplo de uso do algoritmo de Busca em Profundidade.
    Já na Figura \ref{fig:pictures/ParserDeterministicBreadth.png},
    encontra~=se a variação de execução de um Analisador Determinístico que utilizou o algoritmo de Busca em Largura.
    \begin{figure}[h]
    \centering
    \includegraphics[width=1.0\textwidth]{pictures/ParserDeterministicBreadth.png}
    \caption[Árvore de Computação com 15 Passos utilizado Busca em Largura]{Árvore de Computação com 15 Passos utilizado Busca em Largura -- Fonte Própria}
    \label{fig:pictures/ParserDeterministicBreadth.png}
    \end{figure}

    Tanto o algoritmo de Busca em Largura quanto Busca em Profundidade,
    não precisam exatamente seguir resolvendo o problema pela esquerda ou
    direita.
    O que importa é a sua característica de avançar até o fim de algum dado ramo de computação,
    ou seguir executando todos os ramos que fazem parte de um mesmo nível de computação \cite{cormenIntroductionToAlgorithms,efficientBreadthFirstSearch}.
}


    

\chapter{Estado da Arte}
\label{source_code_beautifiers}

% \section{\lang{What does coding is}{Programação de computadores}}

\lang{
    Coding is like writing and
    reading a book for the large people,
    you like it to look beautifully.
    Or at least do you expect such when you buy a book,
    for example,
    to learn programming for you first time \cite{howNovicesRead}.
    You expect:
    \begin{enumerate}
        \item
            Things to be well organized,
            so you do not get lost;
        \item
            The colors to be properly placed,
            so you do not get distracted from the main content;
        \item
            The spacing between paragraphs,
            words, chapters, sections subsections, etc,
            to be well adjusted.
            Not everything cluttered in only one file,
            line, function, class,
            or whatsoever so;
    \end{enumerate}
}{%
    Programar pode ser considerado pela maioria das pessoas como escrever um livro,
    e para tal,
    a facilidade de leitura é desejável\advisor{.}{
    e
    gostar~=se mais caso este ``livro'' tenha uma boa aparência.%
    }
    Existem livros \advisor{difíceis de ler}{muito horríveis de se ler}:
    fontes inadequadas ou
    pequenas,
    com folhas que brilham contra a luz.
    No mínimo espera~=se algumas características chave ao comprar~=se um livro \cite{visualizationsInAFunctionalProgramming}.
    Por exemplo,
    quando compra~=se um livro,
    \advisor{}{para aprender a programar pela primeira vez,
    }espera~=se que \cite{howNovicesRead}:
    \begin{enumerate}
        \item
            Seu conteúdo esteja bem organizado,
            para que o leitor não se perca durante sua leitura;
        \item
            Que suas cores \advisor{sejam}{estejam} propriamente escolhidas e
            utilizadas,
            para que elas não distraiam o leitor ou tirem o foco do principal,
            o conteúdo do livro;
        \item
            Que o espaçamento entre os parágrafos,
            palavras, capítulos, seções, subseções, etc, estejam propriamente ajustados,
            e não\advisor{}{ todo} aglomerado ou desordenado em um único parágrafo,
            frase, capítulo\advisor{.}{, etc.}
    \end{enumerate}
}


\advisor{}{%
    \section{\lang{Computer Assisted Programming}{Programação auxiliada}}

    \lang{%
        Your computer should help you with these unforeseen tasks.
        Why should I spend my precious time checking whether I am actually copying something space indented,
        when I am actually coping something tab indented?
        Perhaps we should sit and
        cry while waiting for some greater force to come and
        rescue us.
        Or may be you should stop crying and
        actually do something about other than keep waiting for you mommy to come and
        save you from the darkness growing behind you back leading you to endless unsleepy nights fixing your code just because everything just went wrong.
    }{%
        Seu computador deveria ajudar você com aquelas imprevistas tarefas.
        Por que eu deveria gastar meu precioso tempo verificando se algo que estou copiando,
        está indentado por ``TAB''\s ou
        espaços?
        Talvez devêssemos sentar e
        chorar enquanto aguardamos que alguma força maior do Universo venha e
        nos salve.
        Ou talvez você deva para de chorar e
        realmente fazer alguma coisa a respeito além de continuar esperando que sua mamãe venha e
        resgate você da escuridão crescendo pela suas costas levando você a infinitas noites acordado corrigindo seu código simplesmente por que tudo deu errado.
    }
}


\section{Formatadores de código}
\label{formatadores_de_codigo}

\lang{%
    A robust Code Beautifier can get a lot more complicated just with the basic definitions of formatting (\typeref{source_code_beautifiers}) applied over each language own characteristics as for example:
    \begin{enumerate}[nosep,nolistsep]
        \item
            Add spaces before if\s name as in `if(var)' versus `if (var)';
        \item
            Add spaces inside if\s as in `if(var)' versus `if( var )';
        \item
            Add spaces before for\s name as in `for(var)' versus `for (var)';
        \item
            Add spaces inside for\s as in `for(var)' versus `for( var )'.
    \end{enumerate}
}{%
    Conhecido como ``\textit{pretty~=printing}'' ou
    embelezadores\footnote{
    Do inglês,
    \textit{Source Code Beautifiers}.
    }
    \cite{prettyPrintingForSoftware},
    uma ferramenta de formatação pode ser complicada de se utilizar,
    somente com um conjunto básico de definições.
    Por exemplo,
    para permitir um melhor controle do usuário,
    a ferramenta de formatação pode permitir que exista uma configuração específica para cada aspecto da linguagem.

    Uma possível implementação para tal ajuste fino,
    pode ser uma configuração específica como por exemplo a entrada booleana ``\textit{use\_spaces\_after\_if}'',
    em um arquivo de configuração para definir caso deva~=se ou
    não ser adicionados espaços após cada ``if'' ao fazer a formatação da linguagem,
    i.e.,
    ``if (var)'' ao contrário de ``if(var)''.
}

\lang{
    As may be noticed,
    the list may became quite big,
    and if fact such big list of rules has been implemented.
    Looking over the Beautifier called `Uncrustify' \cite{uncrustifySourceCode},
    we can find about 500 settings with specifications like these above.
}{%
    Como pode ser percebido,
    uma lista contendo todas as configurações de formatação para cada aspecto da linguagem,
    ficará muito grande quando todos os aspectos das linguagens mais complexas como C ou
    C++ forem implementados.
    Em softwares como Uncrustify \cite{uncrustifySourceCode},
    encontra~=se mais de 500 configurações tais como o \typeref{uncrustifySettingSample},
    veja mais sobre configurações na \fullref{trabalhosRelacionados}.
}
\begin{lstlisting}[caption={Trecho do Arquivo de Configuração de Uncrustify},label={uncrustifySettingSample}]
# If the body of the namespace is longer than this
# number, it won't be indented. Requires
# indent_namespace=true. Default=0 (no limit)
indent_namespace_limit = 0  # number

# Whether the 'extern "C"' body is indented
indent_extern = false       # false/true

# Whether the 'class' body is indented
indent_class = true         # false/true

# Whether to indent the stuff after a
# leading base class colon
indent_class_colon = false  # false/true

# Whether to indent the stuff after a
# leading class initializer colon
indent_constr_colon = false # false/true
\end{lstlisting}

\lang{%
    The problem about is,
    even if you go through all these settings,
    which will take you quite some time,
    you still only configuring a few languages closely related.
    On this case,
    C, C++, Java, Pawn,
    etc.
    For all other languages you still need to find out another source code formatter tool,
    which will be certainly limited \cite{eclisePeriodSettings} and
    still need to configure all over again.
}{%
    O problema é,
    mesmo que se passe por todas essas configurações,
    realizando os ajustes de preferência do usuário,
    isso levará um bom tempo.
    Infelizmente,
    você ainda está apenas configurando uma e
    talvez algumas linguagens fortemente relacionadas,
    dependendo do suporte que a ferramenta oferece.

    No caso da ferramenta Uncrustify,
    serão configurados linguagens como \textit{C},
    \textit{C++}, \textit{C\#} e \textit{ObjectiveC},
    veja mais linguagens na \fullref{trabalhosRelacionados}.
    Para todos os outras linguagens,
    você ainda precisará encontrar outra ferramenta de formatação,
    que será certamente mais limitada no que ela pode customizar \cite{eclisePeriodSettings,uncrustifyNotMeetingCodingStandards},
    uma vez que Uncrustify já é uma das mais completas.

    Ao realiza~=se a migração de uma ferramenta de formatação como Uncrustify para outra,
    precisa~=se configurar novamente todas as novas opções já definidas para Uncrustify.
    Outra forte desvantagem do arquivo de configurações do Uncrustify segue na dificuldade de entender e
    visualizar o que está sendo configurado para que possa~=se realizar a migração.

    Algumas opções são claras e
    fáceis de se entender,
    já outras,
    não consegue~=se ter a mínima ideia do que elas estão fazendo e
    qual será o seu resultado final.
    Por exemplo,
    você consegue claramente entender o que a última regra do \typeref{uncrustifySettingSample} significa?
}

\advisor{}{%
    \lang{%
        Below there are some basic formatting rules for illustration:

        % \begin{bluebox}
        \begin{enumerate}[nosep,nolistsep]
            \item Add new lines after `\{' and before `\}';
            \item Add new lines before `\{';
            \item Remove empty lines;
            \item Add comment lines before function;
            \item Add new lines after `;';
            \item Add new lines after `\}';
            \item Remove new lines;
            \item Reduce whitespace;
            \item Fix bad indentation.
        \end{enumerate}
        % \end{bluebox}
        \vspace{-4mm}\begin{flushright}\textcite{prettyPrinter}\end{flushright}

        Mostly,
        code formatting is resumed into this set of changes.
    }{%
    }
}


\section{\lang{For what is their use?}{Qual a utilidade de Formatadores?}}

\advisor{%
    Pode~=se compreender uma frase sem pontuação,
    mas é mais fácil de fazer essa operação utilizando~=as.
    Por exemplo:

    \noindent
    umaboapontuacaocomcertezatornamascoisasmaisfaceisdeseler

    Um leitor hábil pode compreender facilmente a oração acima,
    mas a pontuação tornaria a tarefa mais fácil \cite{theTidyverseStyleGuide}.
}{%
    Um bom estilo de formatação é como uma boa pontuação,
    você consegue viver sem \cite{theTidyverseStyleGuide}:

    \noindent
    mascomcertezatornamascoisasmaisfaceisdeseler
}

\lang{%
    For now could not find any strong evidence or
    correlation about code comprehension and
    source code beautifying \cite{improvingCodeReadability},
    except perhaps for team annoyance:
}{%
    Não pôde~=se encontrar nenhuma forte evidência\footnote{
    Com forte evidência refere~=se a afirmações absolutas como:
    ``--Ao utilizar formatadores de código,
    é garantido que seu projeto de desenvolvimento de código~=fonte irá sofre ganhos em tempo de desenvolvimento e
    qualidade de software''.
    }
    ou relação sobre a compreensão de códigos~=fonte e
    formatadores de código.
    Alguns estudos como \citeonline{improvingCodeReadability},
    implementam modelos de Inteligências Artificiais utilizados para classificar códigos~=fonte como ``bem legíveis'' ou
    ``mal legíveis''.
    Inicialmente estes estudos começam coletando dados de pessoas classificando códigos~=fonte como legíveis ou
    não,
    para então treinar Inteligências Artificiais classificando códigos~=fonte.

    Além desse estudo científico para classificar códigos~=fonte,
    encontra~=se em outros lugares como \citeonline{pep8operatorsBlankLine},
    dicas ou
    notas explicando quais características de formatação de código~=fonte ajudam ou
    prejudicam a legibilidade de código.
}%
\begin{citacao}
    % \setlength{\itemindent}{5pt}
    Um dos melhores métodos de destruição de equipes de desenvolvedores de software,
    é engajar~=se em guerras de formatação passiva~=agressiva.
    Elas destroem os relacionamentos entre colegas,
    e dependendo do tipo de formatação feita,
    também podem prejudicar a capacidade de comparar efetivamente as revisões no sistema de controle de versão,
    o que é realmente assustador.
    Não pode~=se nem imaginar o quão ruim seria caso a liderança das equipes fosse a culpada por esse comportamento.
    Isso que é liderar por exemplo.
    Por mau exemplo.
    \cite[tradução nossa\protect\footnotemark]{deathToTheSpaceInfidels}\footnotetext{
    \englishword{%
    One of absolute worst,
    worst methods of teamicide for software developers is to engage in these kinds of passive~=aggressive formatting wars.
    I know because I've been there.
    They destroy peer relationships,
    and depending on the type of formatting,
    can also damage your ability to effectively compare revisions in source control,
    which is really scary.
    I can't even imagine how bad it would get if the lead was guilty of this behavior.
    That's leading by example,
    all right.
    Bad example.
    }}.
\end{citacao}

\lang{%
    On \citeonline{programIndentation},
    was analyzed the level of program comprehension which can be gained by the indentation and
    was comprehended that the indentation levels of 2 and
    4 spaces proved to have the best comprehension levels again other levels.
}{%
    Em \citeonline{programIndentation},
    foi analisado o nível de compreensão do programa que pode ser obtido pela indentação e
    foi constatado que os níveis de indentação de 2 e
    4 espaços provaram ter os melhores níveis de compreensão do que outros níveis.
}%
\begin{citacao}
    % \setlength{\itemindent}{5pt}
    Por mais absurdo que possa parecer,
    lutar sobre espaços em branco e
    outras questões aparentemente triviais de layout de código~=fonte é realmente justificado.
    Desde que feito com moderação,
    abertamente,
    de uma forma justa e
    com construção de consenso,
    sem esfaquear companheiros de equipe ao longo do caminho.
    \cite[tradução nossa\protect\footnotemark]{deathToTheSpaceInfidels}\footnotetext{
    \englishword{%
    So yes,
    absurd as it may sound,
    fighting over whitespace characters and
    other seemingly trivial issues of code layout is actually justified.
    Within reason of course -- when done openly,
    in a fair and
    concensus building way,
    and without stabbing your teammates in the face along the way.
    }}.
\end{citacao}

Como não pode~=se dizer certamente que ao utilizar ferramentas de formatação irá~=se ter um melhor desempenho,
cabe a cada desenvolvedor ou
time de desenvolvimento decidir por si e
sua experiência se ele sente a necessidade de algo mais durante o período de desenvolvimento.


\section{\lang{Related Works}{Trabalhos Relacionados}}
\label{trabalhosRelacionados}

Nesta seção,
são relatados os trabalhos relacionados obtidos através de pesquisa na literatura utilizando metodologia especificada no \typeref{firstChapterIntroduction}.
Segundo \citeonline{uncrustifySourceCode},
Uncrustify é capaz de formatar as linguagens:
\begin{inparaenum}[1)]
\item C;
\item C++;
\item C\#;
\item ObjectiveC;
\item D;
\item Java;
\item Pawn e;
\item VALA.
\end{inparaenum}%
De acordo com o site \citetitle{uncrustifyWebSite},
a versão ``0.69.0'' possui 671 configurações para personalizar a formatação do código~=fonte como mostrado no \typeref{uncrustifySettingSample}.

Existem muitas ferramentas e\slash{}ou editores de texto,
e cada uma dessas ferramentas possuí suas vantagens e
desvantagens \cite{prettyPrintingOfVisualSentences,anAbstractPrettyPrinter,improvingRefactoringSpeed}.
Por isso,
alguns desenvolvedores de software\advisor{}{
que são mais inquietos ou
entusiastas por novas tecnologias
} podem utilizar simultaneamente duas ou
mais ferramentas dependendo de suas especialidades e
da tarefa que o desenvolvedor está realizando no momento.

Trabalhos como \citeonline{editorConfig},
não são diretamente um formatador de código~=fonte\advisor{.}{,
tema foco desta monografia.
} Mas sua relação com este trabalho é seu proposito de servir como uma ponte de compartilhamento de configurações entre todos os editores de texto.
Com isso,
facilitando a troca de um editor pelo outro durante o desenvolvimento.
Do mesmo jeito,
este trabalho foca em propor uma ferramenta de formatação extensível várias linguagens de programação (Veja o \fullref{software_implementation}).

\citeonline{universalIndentGUI},
é outra ferramenta que não é um formatador de código~=fonte,
mas sim uma interface comum a várias e
diferentes ferramentas de formatação de código~=fonte.
Na \typeref{UniversalIndentGUIScreenshot4},
pode~=se ver como é a interface com o usuário que esta ferramenta proporciona.

\begin{figure}[!htb]
\caption{Exemplo da Tela de Configuração de UniversalIndentGUI}
\label{UniversalIndentGUIScreenshot4}
\centering
\includegraphics[width=1.0\textwidth]{UniversalIndentGUIScreenshot4.png}
\fonte{\citeonline{UniversalIndentGUIScreenshot4}}
\end{figure}

A interface da \typeref{UniversalIndentGUIScreenshot4},
permite que se escolha qual será o formatador de código que será utilizado,
e permite que carregue~=se um arquivo de código~=fonte para ser formatado,
e na medida que seleciona~=se as opções pela interface gráfica,
pode~=se visualizar as mudanças no código~=fonte de exemplo carregado.
Entretanto,
\citeonline{universalIndentGUI},
enfrenta problemas como as limitações de sua interface gráfica.
Como já mencionado na \typeref{trabalhosRelacionados},
Uncrustify na sua versão ``0.69.0'' possuí 671 configurações de formatação de código~=fonte,
mas a interface gráfica de \citeonline{universalIndentGUI},
não possui \advisor{tal número de}{nem próximo de tantas} opções de configuração.

Todas as opções que não aparecem na interface gráfica de \citeonline{universalIndentGUI},
não podem ser utilizadas ao realizar a formatação do código~=fonte porque os arquivos de configuração gerados e
lidos pela ferramenta não conterão estas opções.
Qualquer outra opção adicional inserida manualmente pelo usuário será apagada,
assim não podendo ser utilizadas mais opções além das quais a\advisor{}{
limitada} interface gráfica apresenta.

Outra desvantagem da visualização das mudanças,
como mostrado na \typeref{UniversalIndentGUIScreenshot4},
acontece quando o código~=fonte de amostra carregado não contém partes que são afetados\slash{}formatados pela configuração utilizada.
Nesse caso,
não será apresentado nenhuma mudança durante a visualização.

Com o que foi apresentado sobre as ferramentas de formatação de código~=fonte,
já pode~=se ter uma visão sobre como funcionam e
são todas as ferramentas de formatação.
Ou elas serão configuradas diretamente pelo usuário,
editando arquivos de configuração \cite{uncrustifySourceCode},
ou elas terão uma interface gráfica que permita com que o usuário visualize as suas alterações diretamente em uma amostra de código~=fonte \cite{universalIndentGUI}.

Quanto ao seu funcionamento interno,
formatadores de código~=fonte seguem o processo de tradução como apresentado na \fullref{compiladoresEtradutores}.
Diferentemente do usual para um tradutor,
um formatador realiza um processo de tradução onde o nível da linguagem e
a própria linguagem de origem e
destino são o mesmo.
Por exemplo,
ao realizar a tradução de um programa em \textit{C++} para \textit{C++}.


\section{\lang{Obfuscators}{Ofuscadores}}

Ofuscadores\footnote{Do inglês,
\textit{Obfuscator},
\citetitle{obfuscationWikipedia}.
}
são formatadores de código~=fonte que funcionam com objetivo contrário ao dos \textit{Beautifiers},
veja o \typeref{exemploDeOfuscador}.
Em vez de melhorar a leitura do código~=fonte,
sua função é impossibilitar que o código~=fonte seja compreendido.
Tais técnicas e
utilidades para estes tipos de software podem ser encontradas com mais detalhes em \citeonline{codeObfuscationTechniques}.
\begin{quadro}[!htb]
\caption[Exemplo de Ofuscador de Código]{Exemplo de Ofuscador de Código \cite{familyOfSourceCodeObfuscators}}
\label{exemploDeOfuscador}
\begin{bluebox}
\begin{code}
\caption{Antes do ofuscamento}
\begin{minted}[xleftmargin=2em]{javascript}
for (i=0; i < M.length; i++)
{
   // Adjust position of clock hands
   var ML = (ns) ? document.layers['nsMinutes' + i]:ieMinutes[i].style;

   ML.top = y[i] + HandY + ( i*HandHeight ) * Math.sin(min) + scroll;
   ML.left = x[i] + HandX + ( i*HandWidth ) * Math.cos(min);
}
\end{minted}
\end{code}

\begin{code}
\caption{Depois do ofuscamento}
\begin{minted}[xleftmargin=2em]{javascript}
for(O79=0;O79<l6x.length;O79++){var O63=(l70)?
document.layers["nsM\151\156u\164\145s"+O79]:
ieMinutes[O79].style;O63.top=l61[O79]+O76+(O79*O75)
*Math.sin(O51)+l73;O63.left=l75[O79]+l77+(O79*l76)
*Math.cos(O51);}
\end{minted}
\end{code}
\end{bluebox}
\end{quadro}

Dentre as utilidades dos Ofuscadores está minimizar a quantidade de dados transmitidos entre um Servidor e
um Cliente Web ao reduzir o tamanho dos arquivos de linguagens como \textit{JavaScript},
quando eles são baixados pelos Navegadores de Internet ao acessar um site que utiliza essas linguagens de script.
Como exemplo,
pode~=se imaginar um arquivo com grande volume de comentários\advisor{.}{
de documentação.} Nesse arquivo,
a transmissão dos comentários pode ser considerada inútil e
consequentemente podem removidos.
Portanto,
fazendo seu ofuscamento pode~=se reduzir seu tamanho em até pela metade ou
mais.

Eventualmente,
existem linguagens onde o uso de ofuscadores possui limitações.
Por exemplo,
diferente da linguagem \textit{JavaScript} utilizada no \typeref{exemploDeOfuscador},
a sintaxe da linguagem YAML\footnote{
Do inglês,
\textit{YAML Ain't Markup Language},
i.e.,
a sigla de YAML é uma definição recursiva.
}
\cite{yamlSpecificModelChecking} obrigatoriamente deve utilizar espaços para indentação e
linhas novas para separar blocos de informação.
Tais características enfraquecem o poder do ofuscamento uma vez que ele possui um maior efeito quando pode~=se introduzir maior caos no documento,
removendo~=se as indentações e
linhas novas. Entretanto,
mesmo sem esses elementos chave,
ainda pode~=se introduzir caos removendo todos os comentários e
renomeando variáveis para nomes \advisor{sem significado de uso}{obtusos\footnote{Sem significado de uso.}} como ``A88''.


\advisor{}{%
\subsection{Lista de curiosidades}

\lang{%
    On a distributed version control system,
    continuous integration would be continuously running the system tests as long new code is integrated into main system from the distributed clients \cite{continuousIntegration}.
}{}

\begin{enumerate}
    % https://tex.stackexchange.com/questions/123104/inline-citations-with-only-author-title-and-year
    \item \citeauthortitleyear{trackingChanges};
    \item \citeauthortitleyear{aspectOriented};
    \item \citeauthortitleyear{annotationAssistant};
    \item \citeauthortitleyear{codePlagiarismDetection};
    \item \citeauthortitleyear{softwarePortfolio};
    \item \citeauthortitleyear{legacyAssets};
    \item \citeauthortitleyear{massMaintenance};
    \item \citeauthortitleyear{prettyPrinting};
    \item \citeauthortitleyear{architectureFormatting};
    \item \citeauthortitleyear{independentFramework};
    \item \citeauthortitleyear{industrialApplication};
    \item \citeauthortitleyear{toolsForProjectManagement};
    \item \citeauthortitleyear{codeClassification};
    \item \citeauthortitleyear{codeScanningPatterns};
    \item \citeauthortitleyear{debuggingIntoExamples};
    \item \citeauthortitleyear{programUnderstanding};
    \item \citeauthortitleyear{documentingAndSharingKnowledge};
    \item \citeauthortitleyear{autofoldingForSourceCode};
    \item \citeauthortitleyear{learningSupportSystem};
    \item \citeauthortitleyear{syntaxHighlightingInfluencing};
    \item \citeauthortitleyear{improvingCodeReadability};
    \item \citeauthortitleyear{howNovicesRead};
    \item \citeauthortitleyear{theRoleOfMethodChains};
    \item \citeauthortitleyear{codeComprehensionComparedToOO};
    \item \citeauthortitleyear{enhancingLegacySoftwareSystemAnalysis};
    \item \citeauthortitleyear{moldableCodeEditor};
    \item \citeauthortitleyear{blindAndSightedProgrammers};
    \item \citeauthortitleyear{pushdownAutomata};
    \item \url{https://github.com/r-lib/styler};
    \item \url{https://github.com/github/linguist};
    \item \url{https://github.com/wbond/package_control_channel/issues/4310} Write a formatter utility;
    \item \url{https://github.com/PyCQA/pydocstyle};
    \item \url{https://github.com/PyCQA/pycodestyle};
    \item \url{https://blog.codinghorror.com/check-in-early-check-in-often/};
    \item \url{https://www.youtube.com/watch?v=2GqpdfIAhz8};
    \item \url{http://langserver.org/};
    \item \url{https://github.com/Microsoft/language-server-protocol};
    \item \url{https://github.com/SublimeCodeIntel/SublimeCodeIntel};
    \item \url{https://code.visualstudio.com/blogs/2016/06/27/common-language-protocol};
    \item \url{https://www.eclipse.org/community/eclipse_newsletter/2017/may/article1.php};
    \item \url{https://github.com/Microsoft/language-server-protocol/wiki/Protocol-Implementations}.
\end{enumerate}
}


    % PARTE
    \advisor{}{%
        \ifforcedinclude\else\part{\lang{Implementation}{Implementação}}\fi
        \label{segunda_parte}
    }

    % Capitulo com exemplos de comandos inseridos de arquivo externo
    

% Is it possible to keep my translation together with original text?
% https://tex.stackexchange.com/questions/5076/is-it-possible-to-keep-my-translation-together-with-original-text
\chapter{Uma Ferramenta de Formatação}
\label{software_implementation}

Neste capítulo será explicado o funcionamento e
implementação de uma nova ferramenta de formatação.
A proposta desta nova ferramenta é permitir que usuários possam entrar com a gramática de qualquer linguagem,
por meio de uma metagramática\footnote{
Em Ciências da Computação,
quando algo é prefixado com ``meta'',
isso significa que ele refere~=se sobre o seu tipo ou
categoria \cite{theUseOfMetaRules}.
Por exemplo,
``metadata'' são dados sobre os dados.
} para então formatar o código~=fonte da linguagem descrita pela gramática.


\section{Uma Gramática de Gramáticas}

Na \fullref{introducaoGramaticas},
foi explicado o que são gramáticas.
Mas,
como gramáticas podem ser expressadas?
Isso depende de como seu analisador foi implementado\advisor{.}{,
sendo assim,
um detalhe de implementação.%
} \advisor{Analisadores}{Usualmente,
analisadores} seguem uma notação comum como EBNF\footnote{
Do inglês,
\textit{Extended Backus–Naur Form} uma extensão do padrão BNF (\textit{Backus–Naur Form}).
}\cite{teachingEbnf,antlrBookTerrentParr},
\advisor{que diverge de acordo com detalhes de implementação.
}{%
que não difere muito de um analisador para outro,
exceto por detalhes de implementação específicos de cada analisador.
}

Para realizar a implementação da nova ferramenta de formatação de código~=fonte,
foi realizado a construção de uma nova gramática de gramáticas de uma nova linguagem chamada de ``ObjectBeauty'',
uma metalinguagem \cite{compilersCompilerMetaLanguage}.
Na \typeref{MyWorflowForLarkTraduzido},
é apresentado o fluxo de uso comum para um analisador.
Neste processo,
o desenvolvedor da linguagem escreve a gramática de especificação\advisor{}{
desta linguagem} que é entregue a algum analisador e
gera~=se um compilador para tal linguagem.
\begin{figure}[h]
\centering
\includegraphics[width=1.0\textwidth]{MyWorflowForLarkTraduzido.png}
\caption[Fluxo de uso comum de um analisador]{Fluxo de uso comum de um analisador -- Fonte Própria \cite{larkErrorRecovery}}
\label{MyWorflowForLarkTraduzido}
\end{figure}

\advisor{Este trabalho faz um uso diferente}{Já este trabalho faz um uso fora do comum}.
Como mostrado na \typeref{MyWorflowForLarkTraduzido2},
primeiro especifica~=se uma metalinguagem que será utilizada pelos usuários da nova ferramenta de Formatação de Código.
Para escrever esta nova metalinguagem,
utilizou~=se o Analisador Lark \cite{larkContextualLexer}.
Usualmente,
o Analisador Lark é utilizado somente como um gerador de compiladores (\typeref{MyWorflowForLarkTraduzido}),
entretanto,
neste contexto Lark é utilizado como um compilador de compiladores (\typeref{MyWorflowForLarkTraduzido2}).
Para este trabalho,
foi realizado um \textit{fork} \cite{overviewOfGitHubForks,mayTheForkBeWithYou,collaborationAmongGitHubUsers} do Analisador Lark,
renomeado o Analisador Lark para ``pushdown''\footnote{%
O código~=fonte do \textit{fork} pode ser encontrado em \url{https://github.com/evandrocoan/pushdownparser}.
}.
\begin{figure}[h]
\centering
\includegraphics[width=1.0\textwidth]{MyWorflowForLarkTraduzido2.png}
\caption[Uso feito pela nova ferramenta de Formatação de Código]{Uso feito pela nova ferramenta de Formatação de Código -- Fonte Própria \cite{larkErrorRecovery}}
\label{MyWorflowForLarkTraduzido2}
\end{figure}

Foi realizado um \textit{fork} do Analisador Lark para poder~=se realizar pequenas alterações que facilitam o entendimento do funcionamento interno da ferramenta como adição de logs e
alterações nos algoritmos de iteração nas árvores geradas pela ferramenta.
Por isso,
em alguns lugares do código~=fonte é encontrado o nome ``pushdown'' ao invés de ``lark''.
Já em outros,
continua~=se sendo chamado Lark de Lark para simplificar a retrocompatibilidade com a biblioteca original e
facilitar a realização da integração de novos updates vindos do repositório original do Analisador Lark para o \textit{fork} realizado.

Em vez de permitir com que o usuário final da aplicação opere diretamente com o analisador da \typeref{MyWorflowForLarkTraduzido},
foi criado uma nova metagramática (uma gramática de gramáticas) como mostrado nas \typeref{MyWorflowForLarkTraduzido2}.
Esta nova metagramática simplifica o processo de escrita de gramáticas ao criar uma nova especificação de gramáticas,
somente com os recursos necessários para se possa trabalhar com formatação de código~=fonte.
\advisor{Não}{A final, não}
é objetivo deste trabalho fazer a análise completa de programas,
pela sua sintaxe, semântica,
e gerar código~=binário executável.

Na \typeref{ParsersPublicAudienceTraduzido},
pode~=se encontrar uma relação entre o funcionamento das diversas partes da ferramenta de Formatação de Código e
a audiência alvo.
Basicamente existem três grupos distintos de usuários ou
audiência:
\begin{inparaenum}[1)]
\item quem escreve ou
desenvolve a ferramenta de Formatação de Código proposta por este trabalho e
define as regras da metalinguagem (especificada pela sua metagramática,
i.e.,
a gramática de gramáticas),
\item quem escreve ou
desenvolve gramáticas de linguagens para serem formatadas de acordo com as regras da metalinguagem e
\item quem escreve ou
desenvolve programas de computador e
deseja realizar a formatação de seus códigos~=fonte.
\end{inparaenum}%
\begin{figure}[h]
\centering
\includegraphics[width=1.0\textwidth]{ParsersPublicAudienceTraduzido.png}
\caption[Relacionamentos entre os Diferentes Públicos deste Projeto]{Relacionamentos entre os Diferentes Públicos deste Projeto -- Fonte Própria \cite{larkErrorRecovery}}
\label{ParsersPublicAudienceTraduzido}
\end{figure}

Até este ponto,
já falou~=se de metagramática e
metalinguagem com a exceção dos metaprogramas \cite{tradeoffsInMetaprogramming}.
Nas \typeref{MyWorflowForLarkTraduzido2,ParsersPublicAudienceTraduzido},
por simplificação foram omitidos o relacionamento dos metaprogramas com a metagramática e
metalinguagem.
Metaprogramas fazem parte da entrada do metacompilador (\typeref{MetacompilerMetagrammarMetaprogram}) junto com a metagramática para gerar um novo compilador (ou Formatador de Código).
Neste trabalho,
os metaprogramas serão as gramáticas que serão utilizadas pelos formatadores de código~=fonte.

Os metaprogramas (ou gramáticas) são entradas diretas do metacompilador,
o Analisador Lark na \typeref{MyWorflowForLarkTraduzido2},
um Analisador LALR(1).
Na \typeref{ParsersPublicAudienceTraduzido},
não pode~=se ver diretamente que as gramáticas das linguagens serão os metaprogramas,
mas o quadro em azul mais a esquerda ligado por linhas pontilhadas explica que os erros léxicos e
sintáticos nas gramáticas de entrada serão mostrados pelo Analisador Lark.
Isso acontece por que as gramáticas (ou metaprogramas) são entradas diretamente no Analisador Lark.

Na \typeref{MetacompilerMetagrammarMetaprogram},
encontra~=se uma extensão da \typeref{MyWorflowForLarkTraduzido2},
e pode~=se ver claramente as relações entre Metagramáticas,
Metacompiladores e Metaprogramas. Por simplificação,
mostra~=se o nodo ``Arvore de Sintaxe'' sem explicitamente falar sobre sua Análise Semântica e
propriamente a construção do Compilador (ou do Formatador de Código).
Vale lembrar que trata~=se de um Compilador de Compiladores,
e não um Compilador de Analisadores.
Por isso vemos que os Metaprogramas (ou gramáticas) são entradas diretas dos Metacompilador,
e não do Formatador de Código.
\begin{figure}[h]
\centering
\includegraphics[width=1.0\textwidth]{MetacompilerMetagrammarMetaprogram.png}
\caption[Relação entre Metagramáticas, Metacompiladores e Metaprogramas]{Relação entre Metagramáticas, Metacompiladores e Metaprogramas -- Fonte Própria \cite{larkErrorRecovery}}
\label{MetacompilerMetagrammarMetaprogram}
\end{figure}

Esta não é a primeira vez que uma metagramática com simplificações foi escrita.
Em trabalhos como \citeonline{rustSublimeTextSyntaxSyntec,sublimeTextSyntax,vsCodeSyntaxHighlighthing} foram realizados as mesmas simplificações aqui apresentadas.
Existem algumas diferenças técnicas da metagramática deste trabalho com as dos recém~=apresentados.
Como por exemplo,
a implementação da metagramática realizada ainda não suporta a classificação do mesmo trecho de código~=fonte por múltiplos tipos de escopo \cite{vsCodeSyntaxHighlighthing}.

Foi escolhida a criação de uma nova metagramática por que as implementações de metagramáticas já existentes como \citeonline{rustSublimeTextSyntaxSyntec,vsCodeSyntaxHighlighthing}:
\begin{enumerate}[1)]
\item Não utilizam explicitamente nenhum analisador,
realizando a programação das produções da gramática diretamente no código~=fonte (\fullref{gramaticasVersusLinguagens}).
\item Não são capazes de reconhecer todas as características de todas as linguagens de programação (devido a optimizações para maior performance).
\item Não possuem sintaxe própria,
i.e.,
utilizam~=se de outras linguagens como YAML,
XML e
JSON para fazer a especificação da metagramática.
\end{enumerate}
Fazendo a especificação de uma nova metagramática,
é possível adaptar~=se a especificação da sintaxe das gramáticas de acordo as necessidades específicas sem ter que depender de características de outras linguagens como YAML,
XML ou
JSON.


\subsection{Escopos}

Na \typeref{TexMateScopes},
é mostrado na primeira linha o trecho de código~=fonte ``function f1 () \{'' e
nas demais linhas são apresentados as diversas classificações de escopos aplicados a cada um dos trechos do código~=fonte de amostra.
Por exemplo,
a palavra ``function'' possui simultaneamente os escopos
\begin{inparaenum}[1)]
\item ``source.js''
\item ``meta.function.js'' e
\item ``storage.type.function.js''.
\end{inparaenum}
\begin{figure}[h]
\centering
\includegraphics[width=1.0\textwidth]{TexMateScopes.png}
\caption[Exemplo de Classificação de Código~=Fonte com Múltiplos Escopos]{Exemplo de Classificação de Código~=Fonte com Múltiplos Escopos -- Fonte \citeonline{vsCodeSyntaxHighlighthing}}
\label{TexMateScopes}
\end{figure}

Os nomes utilizados na \typeref{TexMateScopes} podem ser qualquer texto que usuário especificador daquela gramática atribuiu.
Entretanto,
pode~=se perceber que o nome dos escopos recém apresentados parecem seguir um padrão.
Por conversão,
desenvolvedores de gramáticas para os editores de texto como \citeonline{sublimeTextSyntax,vsCodeSyntaxHighlighthing},
seguem uma conversão de nomes para que as utilizações dos escopos gerados pelas gramáticas sejam compatíveis entre si.

Fazendo o uso de uma conversão para nomes de escopos,
as gramáticas ficam compatíveis com um maior número de arquivos de temas (ou configurações de cores),
onde são especificados os nomes dos escopos serão utilizados para especificar as cores a serem utilizadas pelo editor de texto.
Para mais informações sobre a utilização de arquivos de temas em editores de texto veja \citeonline{sublimeTextScopeNaming,vsCodeSyntaxHighlighthing}.


\section{Metalinguagem}
\label{metalinguagemGrammar}

Como já explicado na seção anterior,
uma metagramática é gramática de gramáticas e
foi utilizado o Analisador Lark \cite{larkContextualLexer} como um metacompilador ou
compilador de compiladores.
Nesta seção será discutido como a metalinguagem (especificada pela metagramática) utilizada foi construída,
começando com o seu símbolo inicial.
No \typeref{simboloInicialDaMetagramatica},
defini~=se que o programa é constituído de três grandes áreas,
que devem acontecer uma em sequencia da outra\footnote{
Para mais informações sobre a sintaxe de entrada de gramáticas do Analisador Lark consulte a documentação \cite{larkGrammarReference,larkStyleCheat}.
}:
\begin{enumerate}
\item A produção ``preamble\_statements'' define características globais da gramática como um nome,
e um escopo que será atribuído a toda gramática.
\item A produção ``language\_construct\_rules'' define qual será o símbolo inicial da gramática.
Em comparação com linguagens de programação como ``C'',
ele pode ser considerado similar ao método ``main''.
\item A produção ``miscellaneous\_language\_rules'' permite a definição de diversos contextos\footnote{
Contexto refere~=se a um bloco de operadores ou
conjunto de instruções como ``include'' e
``match''.
} com grupos de produções da gramática (\fullref{definicaoDeGramatica}),
que podem ser incluídos a partir do símbolo inicial da gramática definido no item ``language\_construct\_rules''.
\end{enumerate}%
\begin{lstlisting}[caption={Simbolo Inicial da Metagramática ``ObjectBeauty''},label={simboloInicialDaMetagramatica},style=yaml_style]
language_syntax: _NEWLINE? preamble_statements _NEWLINE?
                    language_construct_rules _NEWLINE?
                    ( miscellaneous_language_rules _NEWLINE? )*
                    _NEWLINE?

preamble_statements: ( (
                        target_language_name_statement
                        | master_scope_name_statement
                        | constant_definition
                    ) _NEWLINE )+

language_construct_rules: "contexts" ": " indentation_block
miscellaneous_language_rules: /[^:\n]+/ ": " indentation_block

target_language_name_statement: "name" ": " free_input_string
master_scope_name_statement: "scope" ": " free_input_string
\end{lstlisting}

Entre os \typeref{exemploDeGramaticaPawn1,exemploDeGramaticaPawn2,exemploDeGramaticaPawn3,exemploDeGramaticaPawn4},
encontra~=se pequenos exemplos de gramáticas escritas na metalinguagem ``ObjectBeauty'' brevemente apresentada.
No \typeref{exemploDeGramaticaPawn1},
encontra~=se a definição do símbolo inicial da gramática da linguagem sendo descrita (pela metagramática) e
pode~=se ver a metalinguagem sendo utilizada para definir uma linguagem chamada de ``Abstract Machine Language''.
Por padrão,
toda gramática ``ObjectBeauty'' precisa ter um contexto inicial ou
símbolo inicial chamado de ``contexts''.

O \typeref{exemploDeGramaticaPawn1} faz uso dos operadores ``include'' e
``match''.
O operador ``include'' serve incluir partes de outras gramáticas ou
mesmo gramáticas inteiras no contexto da gramática atual.
Entretanto,
a implementação de ``include'' realizada neste trabalho somente consegue realizar includes de contextos definidos no mesmo arquivo.

No exemplo do \typeref{exemploDeGramaticaPawn1},
o operador ``include'' está incluindo contextos da gramática atual que serão definidas mais tarde neste mesmo arquivo.
Já o operador ``match'' utilizado no final serve para realizar propriamente o reconhecimento do programa de entrada e
atribuir a ele o escopo ``constant.boolean.language.pawn''.

Mais tarde,
as informações de escopo atribuídas por operadores como ``match'' e
``captures'' serão utilizadas pelo formatador de código~=fonte.
Com estas informações,
o Formatador de Código será capaz de realizar as operações de formatação somente sobre os trechos de código~=fonte que o usuário definir.
Realizando assim,
a formatação seletiva de código~=fonte,
contrário da formatação total de código~=fonte como acontece nos demais trabalhos (\fullref{performanceDoFormator}).
\begin{lstlisting}[caption={Exemplo de Gramática -- Símbolo Inicial},label={exemploDeGramaticaPawn1},style=yaml_style]
name: Abstract Machine Language
scope: source.sma

contexts: {
    include: parens
    include: numbers
    include: check_brackets

    match: (true|false) {
        scope: constant.boolean.language.pawn
    }
}
\end{lstlisting}

No \typeref{exemploDeGramaticaPawn2},
é introduzido o uso dos operadores ``push'',
``meta\_scope'' e
``pop''.
O operadores ``push'' e
``pop'' são responsáveis por manter uma pilha de contextos que permite aplicar um mesmo escopo por várias linhas utilizado o operador ``meta\_scope''.
A diferença entre o operador ``scope'' e
``meta\_scope'' é que o operador ``scope'' atribuí o escopo diretamente ao texto reconhecido pelo um operador ``match''.
Já o operador ``meta\_scope'' permite aplicar o escopo a todo o texto desde o primeiro até o último ``match'',
que desempilha com o operador ``pop'',
o contexto empilhado inicialmente com um ``push''.
\begin{lstlisting}[caption={Exemplo de Gramática -- Contextos},label={exemploDeGramaticaPawn2},style=yaml_style]
parens: {
    match: \( {
        scope: parens.begin.pawn
        push: {
            meta_scope: meta.group.pawn
            match: \) {
                scope: parens.end.pawn
                pop: true
            }
            include: numbers
        }
    }
}
\end{lstlisting}

No \typeref{exemploDeGramaticaPawn3},
é introduzido o uso do operador ``captures''.
O operador ``captures'' atribuí simultaneamente diversos escopos com uma única expressão regular.
Cada um dos números listados equivalem a um dos grupos de captura da expressão regular utilizada no operador ``captures''.
O operador ``scope'' pode ser considerado um caso especial do operador ``captures'' quando utiliza~=se o Grupo de Captura 0.

Motores de expressões regulares geralmente suportam um recurso chamado de Grupos de Captura \cite{expressionGrammarsWithRegexLikeCaptures}.
Por exemplo,
a expressão regular ``foo(bar)zoo(car)'' possuí 3 grupos de captura quando analisado o texto de entrada ``foobarzoocar'':
\begin{inparaenum}[1)] \setcounter{enumi}{-1}
\item foobarzoocar
\item bar
\item car,
\end{inparaenum}%
onde o grupo de captura 0 refere~=se a toda a expressão regular encontrada.
Portanto,
ao invés de utilizar~=se o operador ``scope:
constant.numeric.pawn'',
poderia~=se utilizar equivalentemente o operador ``captures:
0.
constant.numeric.pawn''.
\begin{lstlisting}[caption={Exemplo de Gramática -- Grupos de Captura},label={exemploDeGramaticaPawn3},style=yaml_style]
numbers: {
    match: '(\d+)(\.\{2\})(\d+)' {
        captures: {
            0: constant.numeric.pawn
            1: constant.numeric.int.pawn
            2: keyword.operator.switch-range.pawn
            3: constant.numeric.int.pawn
        }
    include: numeric
}
\end{lstlisting}

No \typeref{exemploDeGramaticaPawn4},
é mostrado mais alguns exemplos de uso do operador ``match'' classificando diversos tipos de numéricos (da linguagem sendo descrita pela gramática).
É importante notar que a ordem no qual os operadores como ``match'' aparecem é importante.
Ao realizar o reconhecido o programa de entrada utilizando esta gramática,
a Árvore de Sintaxe Abstrata\footnote{%
Do inglês (AST), Abstract Syntax Tree.
} \cite{ahoCompilerDragonBook} será interpretada diversas vezes,
partindo no símbolo inicial até chegar ao último símbolo da gramática.

O processo de interpretação irá reiniciar indefinidamente até que nenhum texto seja mais consumido por nenhum dos operadores da gramática.
Assim,
uma vez que um trecho de código~=fonte já foi classificado,
ele será ignorado quando os próximos operadores forem aplicados,
evitando assim que o programa execute infinitamente.
\begin{lstlisting}[caption={Exemplo de Gramática -- Tipos numéricos},label={exemploDeGramaticaPawn4},style=yaml_style]
numeric: {
    match: ([-]?0x[\da-f]+) {
        scope: constant.numeric.hex.pawn
    }
    match: \b(\d+\.\d+)\b {
        scope: constant.numeric.float.pawn
    }
    match: \b(\d+)\b {
        scope: constant.numeric.int.pawn
    }
}
\end{lstlisting}

Por fim,
no \typeref{exemploDeGramaticaPawn5} é apresentado um exemplo não relacionado com formatação de código~=fonte.
A construção utilizada é comum para gramáticas que serão utilizadas para realizar a aplicação de cores em editores de texto \cite{vsCodeSyntaxHighlighthing}.
Com ela é possível colorir o código~=fonte,
destacando~=o como inválido no editor de texto,
uma vez que uma inconsistência sintática foi encontrada na linguagem sendo analisada.

Construções como a do \typeref{exemploDeGramaticaPawn5} funcionam usualmente quando elas são a última regra da gramática.
Uma vez que todas as regras que consomem o programa de entrada e
o classifica em escopos terminam seu trabalho,
não deveria existir mais nenhum texto ser reconhecido.
Caso exista,
ou a gramática não estava preparada para reconhecer todo o programa de entrada,
ou estes trechos de código~=fonte são frutos de algum erro no programa de entrada.
\begin{lstlisting}[caption={Exemplo de Gramática -- Reconhecimento de Erros},label={exemploDeGramaticaPawn5},style=yaml_style]
check_brackets: {
    match: \) {
        scope: invalid.illegal.stray-bracket-end
    }
}
\end{lstlisting}


\section{Analisador Semântico}

Depois que um metaprograma da metalinguagem apresentada na seção anterior é reconhecido pelo Analisador Lark,
o Analisador Lark entrega a Árvore de Sintaxe da gramática da linguagem sendo descrita pelo metaprograma (\typeref{MetacompilerMetagrammarMetaprogram}).
Todas as verificações de corretude da sintaxe da gramática são verificadas pelo Analisador Lark,
com base na metagramática da metalinguagem.
Portanto,
somente resta ser implementado o Analisador Semântico para verificar se a linguagem descrita respeita as regras semânticas da metalinguagem explicadas na seção anterior,
\nameref{metalinguagemGrammar}.

Neste trabalho (\typeref{CodeFormatterClassDiagram}),
o Analisador Semântico que deriva de ``Transformer'' recebe como entrada a Árvore de Sintaxe do programa de entrada,
e uma vez que o Analisador Semântico termina seu trabalho,
ele devolve Árvore de Sintaxe Abstrata completa.
Então,
utilizando a Árvore de Sintaxe Abstrata,
``Backend'' que deriva de ``Interpreter'',
realiza a formatação de código~=fonte recebendo um programa de entrada e
as configurações do formatador (\typeref{ParsersPublicAudienceTraduzido,MetacompilerMetagrammarMetaprogram}).
\begin{figure}[h]
\centering
\includegraphics[width=1.0\textwidth]{CodeFormatterClassDiagram.png}
\caption[Diagrama das Principais Classes]{Diagrama das Principais Classes -- Fonte Própria}
\label{CodeFormatterClassDiagram}
\end{figure}

No \typeref{semanticAnalizerConstructor},
pode~=se ver o construtor do Analisador Semântico.
Pode~=se estranhar seu construtor não receber como parâmetro a Árvore de Sintaxe.
Entretanto,
a ela não é passada pelo construtor mas sim por um método chamado ``transform(tree)''.
Esta é uma característica do Analisador Lark utilizado.
A função ``transform(tree)'' do Analisador Lark simplesmente inicia a análise do programa visitando todos os nós da Árvore de Sintaxe,
partindo das folhas até chegar na raíz (\fullref{compiladoresEtradutores}).
\begin{lstlisting}[caption={Construtor do Analisador Semântico},label={semanticAnalizerConstructor},style=python_style]
class TreeTransformer(pushdown.Transformer):
    """
        Transforms the Derivation Tree nodes into meaningful string representations,
        allowing simple recursive parsing and conversion to Abstract Syntax Tree.
    """

    def __init__(self):
        ## Saves all the semantic errors detected so far
        self.errors = []

        ## Saves all warnings noted so far
        self.warnings = []

        ## Whether the mandatory/obligatory global scope name statement was declared
        self.is_master_scope_name_set = False

        ## Whether the mandatory/obligatory global language name statement was declared
        self.is_target_language_name_set = False

        ## Can only be one scope called `contexts`
        self.has_called_language_construct_rules = False

        ## Pending constants declarations
        self.constant_usages = {}

        ## Pending constants usages
        self.constant_definitions = {}

        ## A list of miscellaneous_language_rules include contexts defined for duplication checking
        self.defined_includes = {}

        ## A list of required includes to check for missing includes
        self.required_includes = {}

        ## A list of regular expressions used on match statements,
        ## for validation when the constants definitions are completely know
        self.pending_match_statements = []

        ## Responsible for calculating all open and close commands scoping
        self.open_blocks = {}
        self.indentation_level = 0
        self.indentation_blocks = []
\end{lstlisting}

A visita dos nodos da Árvore de Sintaxe pela função ``transform(tree)'' acontece simplesmente chamando os métodos que a classe ``TreeTransformer'' (Definido no \typeref{semanticAnalizerConstructor}) define e
que possuem os mesmos nomes que os símbolos não~=terminais definidos pela metagramática.
Então,
cada nodo ou
função deve retornar qual será o novo nodo que irá o substituir na Árvore de Sintaxe.
Assim,
no final do processo,
um a um,
cada nodo da Árvore de Sintaxe será convertido para um nodo da Árvore de Sintaxe Abstrata.

Um jeito fácil de excluir um nodo da Árvore de Sintaxe é simplesmente definir uma função com o nome de seu não~=terminal que returna ``null''.
Por exemplo,
o trecho da metagramática apresentado no \typeref{simboloInicialDaMetagramatica} possui alguns símbolos não~=terminais como ``preamble\_statements'' e
``language\_construct\_rules''. Então,
para estes dados símbolos,
serão chamados os métodos da classe ``TreeTransformer'' que possuem os nomes ``preamble\_statements'' e
``language\_construct\_rules''.

Caso não existam os métodos ``preamble\_statements'' e
``language\_construct\_rules'' na classe ``TreeTransformer'',
os nós ``preamble\_statements'' e
``language\_construct\_rules'' da Árvore de Sintaxe não serão visitados e
``poderão'' ser excluídos da Análise Semântica (mas não da Árvore de Sintaxe Abstrata).
Nodos não analisados pela classe ``TreeTransformer'' não serão excluídos da Árvore de Sintaxe Abstrata,
eles serão mantidos intactos,
a não ser que algum outro nodo os altere diretamente na Árvore de Sintaxe.

Mesmo que não existam os métodos ``preamble\_statements'' e
``language\_construct\_rules'' definidos na classe ``TreeTransformer'',
eles também podem ser visitados diretamente a partir de algum nodo pai ou
até algum de seus filhos.
Inclusive,
esta foi uma das alterações realizadas no fork ``pushdown'' do Analisador Lark.
Por padrão,
ao iterar pela árvore,
o Analisador Lark somente passa como parâmetro da função o nodo correspondente ao método atualmente sendo chamado e
uma lista de nodos filhos.
Entretanto,
``pushdown'' também passa um terceiro parâmetro que é uma referência para o nó raíz da árvore e
mantém a variável ``parent'',
acessível como um atributo da classe ``TreeTransformer''.


\section{Formatador de Código}

O Formatador de Código não é composto somente uma parte única e
altamente acoplada.
Mas \advisor{}{pelo contrário,} um conjunto de partes altamente coesas e
completamente independentes onde cada uma dessas partes recebe a Árvore de Sintaxe Abstrata do Analisador Semântico,
para então realizar a formatação do programa de entrada junto com as configurações que esta parte aceita.


\subsection{Performance}
\label{performanceDoFormator}

Continuamente chamar diversos algoritmos independentes possui uma perda de performance em comparação com os formatadores de código~=fonte apresentados no \fullref{source_code_beautifiers}.
Estes formatadores tem como principal características realizar a formatação de código~=fonte em uma única passada,
i.e.,
a Árvore de Sintaxe de programa de entrada é completamente reconstruída,
para então ser serializada novamente em texto de acordo com as configurações de formatador.

Ao leitor mais desatento,
pode parecer que então não existe muita diferença entre a ferramenta de formatação proposta neste trabalho para as já existentes.
Entretanto,
este trabalho permite que o usuário entre com a gramática da linguagem a ser formatada,
diferentes dos outros trabalhos onde esta gramática já é incluída ao formatador.
Formadores de código~=fonte como apresentados na \fullref{source_code_beautifiers} são construídos programando~=se as produções da gramática diretamente no código~=fonte do formatador (Descendentes Recursivos,
reveja a \fullref{gramaticasVersusLinguagens}).

Ao não permitir~=se que o usuário possa entrar com a gramática do programa,
restringe~=se o formatador a somente funcionar com as gramáticas que foram programadas dentro do seu código~=fonte.
Uma vez que o usuário da ferramenta precisa programar o formatador de código para ter suporte a sua linguagem,
isso dificulta a adição do suporte de novas linguagens ao formatador,
pois precisa~=se programar as suas gramáticas diretamente dentro do código~=fonte do formatador.


\subsection{Formatação}

No \typeref{construtorDoFormatador},
pode ser encontrado o construtor do formatador de código~=fonte.
Comparando~=o com o construtor do Analisador Semântico \typeref{semanticAnalizerConstructor},
pode~=se notar algumas diferenças.
Em ambos os casos,
o processo todo se completará ao percorrer toda a árvore.
No caso do Analisador Semântico,
a Árvore de Sintaxe,
e no caso do Formatador de Código,
a Árvore de Sintaxe Abstrata.

Diferentemente do Analisador Semântico,
o Formatador de Código faz herança do tipo ``Interpreter'' em vez de ``Transformer'' (\typeref{CodeFormatterClassDiagram}).
A diferença é simples,
``Interpreter'' visita a árvore partindo das folhas até chegar na raíz visitando todos os nodos filhos (\fullref{compiladoresEtradutores}),
já ``Transformer'' visita a árvore partindo da raíz até chegar nos nodos pais,
i.e.,
ele não visita os nodos filhos automaticamente como ``Transformer'' faz.

``Interpreter'' também recebe diretamente no construtor qual será a árvore que ele irá iterar sobre.
Nos demais aspectos,
``Interpreter'' funciona igual ao ``Transformer'',
exceto no ponto onde ``Transformer'' cria uma nova árvore no final do processo,
enquanto ``Interpreter'' não cria árvore alguma.
O parâmetro chamado ``program'',
que o construtor de ``Transformer'' recebe,
é o programa a ser formatado pelo formatador.
No final processo,
``Interpreter'' terá em sua variável ``self.program'' o novo programa completamente formatado.
\begin{lstlisting}[caption={Construtor do Formatador},label={construtorDoFormatador},style=python_style]
class Backend(pushdown.Interpreter):

    def __init__(self, formatter, tree, program, settings):
        super().__init__()
        self.tree = tree
        self.program = formatter( program, settings )

        ## A list of lists, where each list saves all the matches performed by
        ## the last match_statement on scope_name_statement
        self.last_match_stack = []

        ## This is set to False every push statement, and set to True, after
        ## every match statement. This way we can know whether there is a match
        ## statement after a push statement.
        self.is_there_push_after_match = False
        self.is_there_scope_after_match = False

        self.cached_includes = {}
        self.cache_includes( tree )

        self.visit( tree )
        log( 4, "Tree: \n%s", tree.pretty( debug=0 ) )
\end{lstlisting}

Enquanto ``Interpreter'' é responsável por somente ``passear'' pela Árvore de Sintaxe Abstrata,
a classe ``AbstractFormatter'' \typeref{construtorDeParsedProgram} é responsável por realmente fazer a formatação de código~=fonte de acordo com a instruções vindas da Árvore de Sintaxe Abstrata.
No final do processo,
``AbstractFormatter'' terá na variável ``new\_program'' todos os pedaços do programa formatado.
Uma vez que ``Interpreter'' termina de construir todos os pedaços de código~=fonte formatado,
a função ``get\_new\_program'' irá unir dos os pedaços e
salvá~=los na variável ``cached\_new\_program'',
para evitar ter que recalcular o novo programa toda vez que pedir~=se a sua nova versão.
\begin{lstlisting}[caption={Construtor de ``AbstractFormatter''},label={construtorDeParsedProgram},style=python_style]
class AbstractFormatter(object):
    """
        Represents a program as chunks of data as (text_chunk_start_position,
        text_chunk).
    """

    def __init__(self, program, settings):
        super().__init__()
        self.initial_size = len( program )
        self.program = program
        self.settings = OrderedDict( sorted( settings.items(), key=lambda item: len( str( item ) ) ) )

        self.new_program = []
        self.cached_new_program = []
        log( 4, "program %s: `%s`", len( str( self.program ) ), self.program )
\end{lstlisting}

A linha ``sorted'' realiza a ordenação do configurações sem nenhuma necessidade prática neste caso.
Ela garante que a ordem no qual as configurações irão ser processadas seja sempre a mesma.
No caso da Adição de Cores,
``sorted'' é uma função obrigatória para garantir que os nomes das cores da configurações sejam atribuídas de acordo com a ordem de funcionamento do tema \cite{vsCodeSyntaxHighlighthing,sublimeTextScopeNaming}.

A linha ``len( program )'' não possui nenhuma influência nos resultados do programa,
seja para formatação ou
adição de cores.
Entretanto,
ela é constantemente verificado durantes as operações que realizam o consumo do programa de entrada pelas regras da metagramática para garantir que o tamanho programa original não seja perdido.
É necessário uma sincronia entre os índices do programa original com o programa formatado para que ambos possam ser sincronizados no final do processo.
Assim,
uma que os algoritmos de adição de cores ou
formatação estivem propriamente testados,
não seria mais necessário manter o uso da variável ``len( program )''.

Durante o processo de consumo,
o programa de entrada é modificado e
os caracteres que foram consumidos são substituídos pelo caractere ``§''.
A existência do caractere de marcação de consumo ``§'' é uma fraqueza do algoritmo implementado,
pois pode levar programas que contenham este caractere ao erro.
Ele também trás uma ineficiência no consumo de novos caracteres.
Os caracteres que já foram substituídos por ``§'',
serão reanalisados continuamente até o final do análise do programa de entrada.
A utilização do caractere ``§'' foi realizada para simplificar a implementação do algoritmo de consumo.

Com o caractere ``§'' colocado no lugar dos caracteres que já foram consumidos,
é possível reconstruir o programa de entrada no final da análise simplesmente ordenando todos pedaços que de programa formatado que estão armazenada na variável ``new\_program''.
Como o tamanho do programa original não fui modificado,
também é possível facilmente integrar no programa formatado,
todas as partes do programa original que não foram formatados,
no caso do formatador,
ou coloridas do caso da adição de cores.


\subsection{Exemplo}

A formatação de código~=fonte recebe como entrada o programa em texto simples,
e como resultado,
retorna uma página HTML \cite{parallelParserForHTML}.
Com isso,
sendo possível observar com mais facilidade o código~=fonte original e
formatado,
e as metainformações atribuídas pela gramática da linguagem como atributos das tags HTML.
No \typeref{exemploDeFormatacaoDeCodigo},
encontra~=se um HTML gerado com as metainformações agregadas ao programa não~=formatado ``if(something) bar'',
onde ``if( \ something \ ) bar'' é o resultado da formatação de código~=fonte.
\begin{quadro}[h]
\caption{Exemplo de Formatação de Código}
\label{exemploDeFormatacaoDeCodigo}
\begin{bluebox}
\lstset{xleftmargin=2em,aboveskip=0pt}

    \begin{lstlisting}[caption={Exemplo de HTML Gerado pelo Formatador de Código},label={exemploDeHTMLGerado},style=HTML5]
    <body style="white-space: pre; font-family: monospace;">
        <span setting="unformatted" grammar_scope="if.statement.definition" setting_scope="" original_program="if(">if(</span>
        <span setting="1" grammar_scope="if.statement.body" setting_scope="if.statement.body" original_program="something">  something  </span>
        <span setting="unformatted" grammar_scope="if.statement.definition" setting_scope="" original_program=")">)</span>
        <span grammar_scope="none" setting_scope="none"> bar</span>
    </body>
    \end{lstlisting}
    \begin{lstlisting}[caption={Exemplo de Gramática Utilizada pelo Formatador de Código},label={exemploDeGramaticaUtilizada},style=yaml_style]
    scope: source.sma
    name: Abstract Machine Language
    contexts: {
        match: if\( {
            scope: if.statement.definition
            push: {
                meta_scope: if.statement.body
                match: \) {
                    scope: if.statement.definition
                    pop: true
                }
            }
        }
    }
    \end{lstlisting}
    \begin{lstlisting}[caption={Exemplo de Configuração Utilizada pelo Formatador de Código},label={exemploDeConfiguracaoUtilizada},style=python_style]
    {
        "if.statement.body" : 2,
    }
    \end{lstlisting}
    \begin{lstlisting}[caption={Exemplo de Formatador de Código},label={exemploDeFormatadorDeCodigo},style=python_style]
    class SingleSpaceFormatter(AbstractFormatter):

        def format_text(self, matched_text, matched_setting):
            matched_text = matched_text.strip( " " )

            if matched_setting:
                return " " * matched_setting + matched_text + " " * matched_setting

            else:
                return matched_text
    \end{lstlisting}
\end{bluebox}
\end{quadro}

Como pode ser percebido,
a gramática de entrada no \typeref{exemploDeFormatacaoDeCodigo} não consome a palavra ``bar'' do programa de entrada ``if(something) bar''.
Esta é uma característica importante deste formatador de código.
Todo texto que não é consumido,
ou pela gramática de entrada,
ou pelo formatador de código ou
adição de cores,
será mantido intacto no final do processo de formatação ou
adição de cores.
Assim,
pode~=se ter o formatador de código já em funcionamento com gramática mais simples possível,
ou que já atenda as características mínimas que deseja~=se formatar ou
adicionar de cores.

    

\chapter{Draft for implementation}

This chapter will not be presented on the monograph definitive version.
It only is a collection of articles,
tools and websites which should be considered useful for this work, i.e., ideas and references.

1.2      Rascunho da Monografia para TCC1

1.2.1    * Introdução       Uma breve descrição dos objetivos e da
           justificativa para realização do projeto. Uma breve explicação de
           qual a necessidade e importância deste projeto no escopo da
           Ciência da Computação. Os objetivo gerais e específicos buscados
           após a conclusão e análise deste projeto.

1.2.2    * Bibliográfia     Fazer uma análise do estado da arte em relação
         ao que existe hoje em dia de cunho científico, comentando sobre os
         diversos trabalhos na área de beautifying.

1.2.3    * Classificações   Escrever sobre quais são os tipos possíveis de
         beautifying. Quais são as técnicas mais eficientes, para quais
         linguagens ele se aplicam.

Cronograma:

Id      Atividade                               Data início    Data fim
1.2.2.a Escrever sobre os Formatadores Atuais   31/07/2017     30/08/2017
1.2.2.b Escrever sobre as Pesquisas Atuais      01/09/2017     15/09/2017
1.2.3.a Escrever as Classes de Beautifying      16/09/2017     30/09/2017
1.2.3.b Escrever os Tipos de Beautifying        01/10/2017     30/10/2017
1.2.4.a Fazer a Revisão do Texto Escrito        31/10/2017     20/11/2017

Coloca ChannelManager no tópico e boas praticas, e comenta sobre o modelo de
fork e canais.

Inclui a IA para reconhecer o formatação nos módulos de beautifying. Ela eh
uma heurística, que cada bloco implementa e faz ele gerar um arquivo de
configuração que representa a atual formatação do código (aqui esta o
verdadeiro desafio do trabalho, pesquise trabalhos correlatos).

Inclui sobre a implementação  do semantic linefeed implementado na seção do
linefeed.

Somente incluí somente o que é mais importante para entender o trabalho, não
queira mostrar tudo o que você fez. Uma trabalho extensivo não é necessário,
basta somente apontar como uma referencia que inclua o que você fez.

Mas não esqueça de incluir como foi implementado, i.e., os diagramas UML, se
o sistema é extensível, as bibliotecas utilizadas, como os testes foram
feitos e os resultados deles.

Coloca no capítulo de motivação a seção de trabalhos relacionados. Trabalhos
relacionados com beautifying e com as boas práticas de programação (code clean,
GOF, DEITEL (forminhas das boas práticas)). E deixa claro qual é o problema
que se está resolvendo.

Cria uma capitulo de comparação com os trabalhos relacionados, tanto a parte
teórica (boas práticas), quanto a parte prática (beautifier). Complexidade do
algoritmo do beautifier e o que esse trabalho tem de diferente dos outros.

Coloque evidencias de que funciona o formatador, de boas práticas, mas
práticas e críticas.

Fazer um texto mais didático com exemplos, para os leitores leigos.


recebi alguns comentários dos professores da banca sobre o seu TCC que
gostaria de repassar para voce.

Os dois professores se manifestaram a respeito da formatação que voce
usou. Como eu já havia comentado com voce, eles pedem que voce adote o
formato oficial da BU quando fizer o texto do TCC II no semestre que
vem, pois isso é obrigatório.

Reconhecem o mérito de voce fazer o texto em ingles mas comentam
também sobre a necessidade de fazer uma boa revisão no texto para
corrigir erros gramaticais. Talvez fosse o caso de voce buscar algum
profissional para isso, na versão final do texto

Quanto a estrutura, pedem que voce seja mais claro e conciso nos
objetivos específicos do tcc (há duas seções de objetivos) e o texto
está mais com estilo de um tutorial do que de um TCC.

No resumo, o objetivo também fica dúbio. Diz que vai estudar e depois
também propor uma linguagem. Acho que se a ideia é propor algo, focar
nisso e o resto é para apoio e suporte para fins de comparação

A revisão de literatura está um tanto confusa. Na seção da proposta
tem algumas partes que deveriam estar na fundamentação.

De minha parte, eu concordo com as observações e vamos agora trabalhar
em melhorar a organização do texto. Mas é importante que voce
transcreva o texto para o template da BU.


\medskip
\begin{bluebox}
\begin{enumerate}[leftmargin=*]

    \item Implement tabstops with white space align. The solution - move
    tabstops to fit the text between them and align them with matching tabstops
    on adjacent lines. \url{http://nickgravgaard.com/elastic-tabstops/}
    \url{https://forum.sublimetext.com/t/elastic-tabs/128}

\end{enumerate}
\end{bluebox}

Some existing libraries,
and to be potentially used as `syntect` for assistance in building the software product:

\begin{bluebox}
\begin{enumerate}[leftmargin=*,parsep=0pt]

    \item \url{https://github.com/jbeder/yaml-cpp}
    \item \url{https://github.com/trishume/syntect}
    \item \url{https://github.com/onqtam/doctest}
    \item \url{https://github.com/c42f/tinyformat}
    \item \url{https://github.com/limetext/lime}
    \item \url{https://forum.sublimetext.com/t/disassembling-sublime-text/24824}

\end{enumerate}
\end{bluebox}

The following is a basic list of formatters/beautifiers accessed at
\lword{\url{http://www.softpanorama.org/Utilities/beautifiers.shtml}} on march/2017:

\medskip
\begin{sloppypar}
\begin{bluebox}\RaggedRight
\begin{enumerate}[leftmargin=*,parsep=0pt]

    \item CB210.ZIP - C Beautifier 2.10 - polish C source code (19,406 bytes, 06/22/92)
    \item CL121.ZIP - Codelister 1.21 - print C code with stats (51,110 bytes, 01/10/94)

    \item CPC200.ZIP - CodePrint for C/C++ 2.00 is a full-featured command line driven source
    code reformatter and pretty printer for C++ and C; over 20 customization features including
    auto-indent, adjustable tab spacing, indent styles, flow lines, comment alignment, and line
    editing for consistent white space (140,605 bytes, 01/26/96)

    \item CSCOP120.ZIP - c-scope 1.20 analyzes C source code and produces various reports
    (48,505 bytes, 06/30/95)

    \item HTML : \url{http://www.digital-mines.com/htb/}
    \item HTML : \url{http://www.datacomm.ch/mwoog/software/perl/beautifier.html}
    \item HTML : \url{http://www.watson-net.com/free/perl/s_fhtml.asp}
    \item SQL : \url{http://www.netbula.com/products/sqlb}
    \item Oracle PLSQL : \url{http://www.revealnet.com}
    \item GPL \url{http://www.geocities.com/~starkville/vancbj.html}
    \item GPL \url{http://kevinkelley.mystarband.net/java/dent.html}
    \item Free \url{http://www.tiobe.com/jacobe.htm}
    \item Free \url{http://www.mmsindia.com/JPretty.html}
    \item Free \url{http://members.magnet.at/johann.langhofer/products/jxbeauty/overview.html} (has JBuilder support)
    \item Free \url{http://www.semdesigns.com/Products/Formatters/JavaFormatter.html}
    \item Commercial \$24.99 \url{http://smartbeautify.com}
    \item Commercial \$129 \url{http://www.jindent.com}
    \item Google \url{http://directory.google.com/Top/Computers/Programming/Languages/Java/Development_Tools/Code_Beautifiers/?tc=1}
    \item Java, SQL, HTML, C++ : \url{http://www.semdesigns.com/Products/DMS/DMSToolkit.html}
    \item Java JIndent \url{http://home.wtal.de/software-solutions/jindent}
    \item Java Pat \url{http://javaregex.com/cgi-bin/pat/jbeaut.asp}
    \item Java JStyle \url{http://www.redrival.com/greenrd/java/jstyle}
    \item Java JPrettyPrinter \url{http://www.epoch.com.tw/download/ms/java/java.htm}
    \item Java JxBeauty \url{http://members.nextra.at/johann.langhofer/download/jxbeauty} and the JxBeauty Home
    \item Java beautify percolator
    \item Java list \url{http://www.java.about.com/compute/java/library/weekly/aa102499.htm}
    \item Java html present VasJava2HTML
    \item Java code colorifier and beautifier \url{http://www.mycgiserver.com/~lisali/jccb}
    \item Perl : \url{http://www.consultix-inc.com/www.consultix-inc.com/talk.htm}
    \item Perl : \url{http://www.consultix-inc.com/www.consultix-inc.com/perl_beautifier.html}
    \item Fortran beautifier : \url{http://www.aeem.iastate.edu/Fortran/tools.html}

    \item C++ : BCPP site is at \url{http://dickey.his.com/bcpp/bcpp.html} or at \url{http://www.clark.net/pub/dickey}.
    BCPP ftp site is at \url{ftp://dickey.his.com/bcpp/bcpp.tar.gz}

    \item C++ : \url{http://www.consultix-inc.com/c++b.html}
    \item C : \url{http://www.chips.navy.mil/oasys/c/} and mirror at Oasys
    \item C++, C, Java, Oracle Pro-C Beautifier \url{http://www.geocities.com/~starkville/main.html}

    \item C++, C beautifier \url{http://users.erols.com/astronaut/vim/ccb-1.07.tar.gz} and site at
    \url{http://users.erols.com/astronaut/vim/#vimlinks_src}

    \item GC! GreatCode! is a powerful C/C++ source code beautifier Windows 95/98/NT/2000
    \url{http://perso.club-internet.fr/cbeaudet}

    \item C++ beautifier `SourceStyler' \url{https://web.archive.org/web/20061205061102/http://ochresoftware.com/}
    \item JavaScript : \url{http://jsbeautifier.org/}

\end{enumerate}
\end{bluebox}
\end{sloppypar}


\section{Related Programs for Beautifying}

After the search of the scientific publications on the subject,
it was found some works in the specific area similar to the works done by code formatters (Beautifiers).
However,
source code beautifying is ambiguous definition which can be easily confused with `Prettyprinting',
which is about coloring the text and displaying it to the user.
`Prettyprint' is not is sought in this work implementation,
but rather make changes in the text on how it is structured,
presented to the user and saved on the file system.

Following we may find some references and publications for source code beautifying,
which should be used futurely by this monograph study:

% How to add `parsep` to `itemsep` and set `parsep` to 0pt, when declaring my list?
% https://tex.stackexchange.com/questions/366904/how-to-add-parsep-to-itemsep-and-set-parsep-to-0pt-when-declaring-my-list
\begin{sloppypar}
\begin{bluebox}\RaggedRight
\begin{enumerate}[leftmargin=*,parsep=0pt]

    \item CodeBeautify is an online code beautifier which allows you to beautify
    your source code: \url{http://codebeautify.org/}.

    \item A universal code formatter, written in Dart:
    \url{https://pub.dartlang.org/packages/unifmt}.

    \item Google-java-format is a program that reformats Java source code to
    comply with Google Java Style:
    \url{https://github.com/google/google-java-format}.

    \item CodeFormatter is a Sublime Text 2/3 plugin that supports format
    (beautify) source code.
    \url{https://github.com/akalongman/sublimetext-codeformatter} and
    \url{https://github.com/aukaost/SublimePrettyYAML}

    \item UniversalIndentGUI offers a live preview for setting the parameters of
    nearly any indenter. You change the value of a parameter and directly see
    how your reformatted code will look like. Save your beauty looking code or
    create an anywhere usable batch/shell script to reformat whole directories
    or just one file even out of the editor of your choice that supports
    external tool calls: \url{http://universalindent.sourceforge.net/} and
    \url{https://github.com/danblakemore/universal-indent-gui}.

    \item Language-agnostic pretty-printing through machine learning (uh, like,
    is this possible? YES, apparently). By Terence Parr (primary developer),
    Fangzhou (Morgan) Zhang (help with initial development), Jurgen Vinju
    (co-author of academic paper, help with empirical results and algorithm
    discussions). \url{https://github.com/antlr/codebuff}

    \item To every developer in this world, the closest thing to their heart is
    the text editor of their choice. Over the last few years many new text
    editors has come into the market in both free and paid model, but
    unfortunately not all of them were able to make a real dent on the developer
    community. I remember in my college days we uses to use Notepad++ as our
    beloved text editor, as at that point of time it was one of the popular and
    free text editor with a lot of features for coding. But as time goes on, the
    entire development community started to lean towards sublime text since it’s
    launch.
    \url{https://www.isaumya.com/sublime-text-vs-atom-which-one-i-prefer-most-and-why/}

    \item As a developer, your code editor is one of the most important parts of
    your setup. It can save your wrists and fingers from repetitive strain
    injuries. It can save your eyes from going blind after a coding marathon.
    \url{https://hackernoon.com/virtualstudio-code-the-editor-i-didnt-think-i-needed-16970c8356d5}

    \item VS Code is an Editor while VS is an IDE.
    \url{https://stackoverflow.com/questions/30527522/what-are-the-differences-between-visual-studio-code-and-visual-studio}

    \item What is the difference between VS Code and VS Community?
    Visual Studio Code is a streamlined code editor with support for development operations like
    debugging, task running and version control. It aims to provide just the tools a developer needs
    for a quick code-build-debug cycle and leaves more complex workflows to fuller featured IDEs.
    For more details about the goals of VS Code, see Why VS Code.
    \url{https://code.visualstudio.com/docs/supporting/faq#_licensing}

    \item Reg Replace is a plugin for Sublime Text 2 that allows the creating of commands consisting of
    sequences of find and replace instructions.
    \url{https://forum.sublimetext.com/t/regreplace-plugin/3810}

    \item The main reason I moved was that I find that it’s much slower, the simple things like opening a
    new window for a project should be instantaneous and sadly it’s far from it. As I've said before
    it's all about personal preference, I've gone back to Sublime but Adam for example is sticking
    with it...
    \url{http://engageinteractive.co.uk/blog/atom-review}

    \item \citeonline{aPrettyGoodFormatting}

    \item \url{https://www.researchgate.net/publication/228540036_An_industrial_application_of_context-sensitive_formatting}

    \item \url{http://www.suodenjoki.dk/us/archive/2010/cpp-checkstyle.htm}

    \item \url{http://www.basicinputoutput.com/2014/08/uncrustify-your-bios.html}

    \item \url{http://prettyprinter.de/}

    \item \url{https://github.com/ryanmaxwell/UncrustifyX}

    \item \url{http://www.softpanorama.org/Utilities/beautifiers.shtml}

    \item Understanding the Syntax Parsing
    \url{https://forum.sublimetext.com/t/understanding-the-syntax-parsing/28569}

    "So, part of what I've been working on is a code beautifier that, more or less, aligns and
    indents the code properly based on scanning through the source document."
    ...
    "It hasn't escaped my notice that this is to some degree exactly what the syntax file is doing."

    \item

    {\bfseries Towards a universal code formatter through machine learning:}
    In this paper, we solve the formatter construction problem using a novel approach, one that
    automatically derives formatters for any given language without intervention from a language
    expert. We introduce a code formatter called CODEBUFF that uses machine learning to abstract
    formatting rules from a representative corpus, using a carefully designed feature set. Our
    experiments on Java, SQL, and ANTLR grammars show that CODEBUFF is efficient, has excellent
    accuracy, and is grammar invariant for a given language. It also generalizes to a 4th language
    tested during manuscript preparation.
    \begin{enumerate}[nolistsep,topsep=0pt,label=$\star$]
        \item \url{http://dl.acm.org/citation.cfm?id=2997383}
        \item \url{http://homepages.cwi.nl/~jurgenv/papers/SLE16.pdf}
    \end{enumerate}

    \item \url{https://www.google.com/search?q=universal+source+code+formatter}
    \begin{enumerate}[nolistsep,topsep=0pt,label=$\star$]
        \item \url{https://www.google.com/search?q=universal+source+code+beautifier}
    \end{enumerate}

    \item \url{http://en.wikipedia.org/wiki/Indent_style}
    \begin{enumerate}[nolistsep,topsep=0pt,label=$\star$]
        \item \url{https://en.wikipedia.org/wiki/Programming_style}
        \item \url{https://en.wikipedia.org/wiki/Scope_(computer_science)}
    \end{enumerate}

    \item \url{http://wiki.c2.com/?CodingStyle}
    \begin{enumerate}[nolistsep,topsep=0pt,label=$\star$]
        \item \url{https://github.com/google/code-prettify}
        \item \url{https://github.com/uncrustify/uncrustify}
    \end{enumerate}

    \item \url{https://en.wikipedia.org/wiki/Prettyprint}
    \begin{enumerate}[nolistsep,topsep=0pt,label=$\star$]
        \item \url{https://www.researchgate.net/search.Search.html?query=formatting%20source%20code&type=publication}
        \item \url{https://www.researchgate.net/search.Search.html?query=pretty%20print%20source%20code&type=publication}
    \end{enumerate}

    \item \url{https://github.com/gchpaco/gopprint}
    \begin{enumerate}[nolistsep,topsep=0pt,label=$\star$]
        \item \url{http://dl.acm.org.sci-hub.io/citation.cfm?id=357115}
        \item \url{https://www.cs.indiana.edu/~sabry/papers/yield-pp.pdf}
    \end{enumerate}

    \item \url{http://www.worldcat.org/title/beautiful-code-a-customizable-code-beautifier-for-java/oclc/56564674}
    \begin{enumerate}[nolistsep,topsep=0pt,label=$\star$]
        \item \url{https://www.researchgate.net/publication/34736049_Beautiful_code_a_customizable_code_beautifier_for_Java}
        \item \url{https://vufind.carli.illinois.edu/vf-ncc/Record/ncc_118189/Holdings}
    \end{enumerate}

    \item \url{https://www.researchgate.net/publication/4283921_Smart_Formatter_Learning_Coding_Style_from_Existing_Source_Code}
    \begin{enumerate}[nolistsep,topsep=0pt,label=$\star$]
        \item \url{http://www.ing.unisannio.it/mdipenta/index.html}
        \item \url{https://github.com/iain/rspec-smart-formatter}
    \end{enumerate}

    \item \url{https://www.researchgate.net/publication/2543984_Source_Code_Files_as_Structured_Documents}
    \begin{enumerate}[nolistsep,topsep=0pt,label=$\star$]
        \item \url{https://en.wikipedia.org/wiki/SrcML}
    \end{enumerate}

    \item \url{https://www.researchgate.net/publication/228540036_An_industrial_application_of_context-sensitive_formatting}
    \begin{enumerate}[nolistsep,topsep=0pt,label=$\star$]
        \item \url{https://www.researchgate.net/publication/234809222_Program_indentation_and_comprehensibility}
    \end{enumerate}

\end{enumerate}
\end{bluebox}
\end{sloppypar}





    % Finaliza a parte no bookmark do PDF
    % para que se inicie o bookmark na raiz
    % e adiciona espaço de parte no Sumário
    \phantompart

    % Conclusão (outro exemplo de capítulo sem numeração e presente no sumário)
    

\chapter{\lang{Conclusion}{Conclusão}}
\label{chapter:conclusion}

\lang{%
    The difference from this proposal to remaining formatting tools,
    is the tradeoff between end\hyp{}users and developers responsibilities.
    Most tools rarely expose to end\hyp{}users their language syntax specification,
    in contrast,
    this proposal completely exposes the language to the end\hyp{}user as simple plain\hyp{}text,
    not requiring the tool to know any language syntax neither semantics.
    Moreover,
    with no syntax knowledge required,
    the tool be can used with any languages their user wishes to.

\begin{enumerate}[leftmargin=*]
    \item
        There are many different tools, sometimes paid, and difficult to
        complete. \cite{universalCodeFormatter};
    \item
        Many programming languages exist, so always having Beautifier
        software for each of them is very laborious
        \cite{universalCodeFormatter}. But the approach to a Universal
        Beautifier proposed in this work, would allow easily new languages to be
        added, being completely different from previous ones, or alike. And in
        case of similarities between them, it is enough to reuse their
        configuration structures already implemented;
    \item
        Looking for a Beautifier for each one of them because programmers
        currently work daily with several of these languages, and they are not
        similar. So you need to configure several beautifiers to do the
        formatting. This is a problem because only a few beautifiers are more
        complete, and every time you need to make a change in the formatting
        style, you must manually propagate the same change over several
        different program configuration files, which is bad because it takes the
        user a lot of time to learn how to handle many different types of
        settings \cite{universalIndentGUI};
    \item
        In the case of ideal Beautifier, a change in your styling is
        propagated to all languages. And if you want to leave some language out
        of it, you just need to remove it from the list on which the
        configuration block applies to, and `a)' leave it out so no change is
        applied to. Or `b)' create a new block including only the block within
        the desired settings.
\end{enumerate}
}{%
    Como esta ferramenta difere das demais já existentes?
    A diferença desta nova proposta de ferramenta de formatação de código~=fonte para as demais é a troca de responsabilidades entre usuários finais da ferramenta e
    os desenvolvedores desta ferramenta e
    das gramáticas para os usuários finais.
    Formatadores de código~=fonte usualmente expõem as mudanças que podem ocorrer ao formatar o código~=fonte sem quebrar a suas regras sintáticas ou
    semânticas.

    A maior parte das ferramentas raramente permite que usuários finais tenham o controle total das mudanças no código~=fonte.
    Enquanto utilizando a ferramenta desenvolvida neste trabalho,
    é possível escrever regras de formatação que quebrem a sintaxe e
    semântica da linguagem sendo formatada.
    Por exemplo,
    na linguagem ``Go'' \cite{programmingLanguageGolang},
    diferentemente de todas as demais linguagens,
    é um erro de sintaxe adicionar a chave ``\{'' de abertura de bloco em uma linha nova.

    Utilizando~=se as ferramentas usuais de formatação,
    fica impedido que configurações do usuário quebre o código~=fonte da linguagem sendo formatada,
    a não ser em casos de \textit{bugs} na ferramenta.
    As ferramentas em geral tentam reconstruir a árvore de sintaxe da linguagem a ser formatada.
    Neste trabalho,
    o usuário final pode escolher entre simplesmente especificar a gramática da linguagem com o mínimo necessário para atingir somente as suas necessidades.
    Mas ao mesmo tempo,
    ele também pode realizar a especificação completa de toda a sintaxe da sua linguagem.

    Mesmo com a especificação completa da sintaxe da linguagem,
    ainda não será o suficiente para impedir quebras nos códigos~=fonte sendo formatados,
    pois a sintaxe da linguagem não cobre os seus aspectos semânticos.
    Neste caso,
    o usuário final precisará conhecer quais são as regras semânticas da linguagem na qual ele está realizando a formatação,
    e configurar o formatador para que ele não quebre nenhuma das regras semânticas da linguagem.
}


\section{\lang{Future Works}{Trabalhos Futuros}}

O autor desse trabalho sugere alguns trabalhos futuros.
Antes de novos formatadores de código~=fonte sejam implementados,
estes pontos precisam ser revistos.
Ao realizar estas mudanças,
qualquer implementação já realizada para formatação será perdida devido ao número de mudanças realizadas,
como em um efeito borboleta \cite{standardButterflyEffect}.
Mais especificamente,
sobre a metalinguagem e
gerador de formatadores criados,
sugerem~=se as seguintes alterações:
\begin{enumerate}
\item Adicionar quaisquer regras semânticas da metalinguagem que não foram adicionadas no analisador semântico;
\item Reduzir o uso de memória e
otimizar a performance em tempo de execução,
uma vez que os algoritmos e
estratégias adotas não levaram estes pontos em consideração;
\item Corrigir erros de interpretação da metalinguagem ou
da especificação da metagramática quando alguns operadores como ``scope'' que tem um uso opcional,
são omitidos;
\item Implementar operadores como ``captures'' e
``set'' para tornar o uso da metalinguagem mais fácil ou
melhorar a sua performance em casos de uso específicos;
\item Adicionar suporte à especificação de múltiplos escopos a um mesmo trecho de código \cite{vsCodeSyntaxHighlighthing},
definindo alguma estrutura de dados adequada,
capaz de permitir consultas e
aritméticas de escopos \cite{textMateScopeExclusion} com performance constante $\Theta(1)$;
\item Melhorar a legibilidade e
facilitar a escrita das gramáticas,
removendo a necessidade de chaves de abertura ``\{'' e
fechamento ``\}'' de blocos,
fazendo a separação de blocos ser feita de acordo com a indentação como em linguagens como Python e
YAML.
\end{enumerate}%

A atual estrutura de composição dos formatadores de código~=fonte não suporta que diversos formatadores realizem a formatação simultaneamente.
Somente um formatador,
que estende da classe ``AbstractFormatter'' pode estar em funcionamento ao mesmo tempo.
Entretanto,
este formatador recebe como parâmetro de sua função ``format\_text'',
o trecho de código~=fonte a ser formatado e
o escopo dele.
Com essas informações ele poderia em tese,
chamar diferentes formatadores mais especializados de acordo com o seu parâmetro escopo.

Em relação aos formatadores gerados pelo gerador de formatadores,
eles precisaram ser completamente reescritos,
uma vez que as melhorias feitas no gerador de formatadores e
metalinguagem linguagem forem concluídas.
A implementação atual dos formatadores pode ser considerada nula,
uma vez que toda sua lógica de funcionamento é construída como um simples iteração descendente pela Árvore de Sintaxe Abstrata gerada pelo analisador semântico.

Em \citeonline{objectBeautifierFutureWorks},
pode ser encontrada a concepção inicial de implementação da ferramenta de formatação de código~=fonte deste trabalho.
Uma vez que este trabalho for publicado,
seu código~=fonte estará disponível.
Então,
seguindo as referências de \citeonline{objectBeautifierFutureWorks} será possível encontrar a especificação completa sobre última versão implementação de formatadores de código~=fonte com os requerimentos deste trabalho.



    % ELEMENTOS PÓS-TEXTUAIS
    \postextual
    \setlength\beforechapskip{0pt}
    \setlength\midchapskip{15pt}
    \setlength\afterchapskip{15pt}

    % Referências bibliográficas
    \begingroup
        % https://tex.stackexchange.com/questions/163559/how-to-modify-line-spacing-per-entry-of-bibliography
        % \linespread{1.18}\selectfont

        % https://tex.stackexchange.com/questions/17128/using-bibtex-to-make-a-list-of-references-without
        % \nocite{*}
        \printbibliography[title=REFERÊNCIAS]
    \endgroup

    % Glossário, consulte o manual da classe abntex2 para orientações sobre o glossário.
    % \ifforcedinclude\else\glossary\fi

    % Inicia os apêndices
    \begin{apendicesenv}
        % Imprime uma página indicando o início dos apêndices
        % \ifforcedinclude\else\partapendices\fi
        \setlength\beforechapskip{50pt}
        \setlength\midchapskip{20pt}
        \setlength\afterchapskip{20pt}

        


%
% How to fix the Underfull \vbox badness has occurred while \output is active on my memoir chapter style?
% https://tex.stackexchange.com/questions/387881/how-to-fix-the-underfull-vbox-badness-has-occurred-while-output-is-active-on-m
%

% ---

\lang
{\chapter[Appendix A]{Since this page is not being completely filled, it is generating the bottom bottom of the page}}
{\chapter[Apêndice A]{Como esta página não está sendo completamente preenchida, ele está gerando a caixa inferior inferior da página}}
% ---


% Multiple-language document - babel - selectlanguage vs begin/end{otherlanguage}
% https://tex.stackexchange.com/questions/36526/multiple-language-document-babel-selectlanguage-vs-begin-endotherlanguage
\begin{otherlanguage*}{english}

\showfont

1. How to display the font size in use in the final output,
2. How to display the font size in use in the final output,
3. How to display the font size in use in the final output,
4. How to display the font size in use in the final output,
5. How to display the font size in use in the final output,
6. How to display the font size in use in the final output,
7. How to display the font size in use in the final output,
8. How to display the font size in use in the final output,
9. How to display the font size in use in the final output,


% As this page is not being completely filled, it is generating the page bottom bad box.
% Fix Underfull \vbox (badness 10000) has occurred while \output is active
%
% \flushbottom vs \raggedbottom
% https://tex.stackexchange.com/questions/65355/flushbottom-vs-raggedbottom
\newpage



\section[Some encoding tests]{\showfont}

1. How to display the font size in use in the final output,
2. How to display the font size in use in the final output,
3. How to display the font size in use in the final output,
4. How to display the font size in use in the final output,
5. How to display the font size in use in the final output,
6. How to display the font size in use in the final output,

7. How to display the font size in use in the final output,
8. How to display the font size in use in the final output,
9. How to display the font size in use in the final output,
10. How to display the font size in use in the final output,
11. How to display the font size in use in the final output,
12. How to display the font size in use in the final output,

\subsection{\showfont}

1. How to display the font size in use in the final output,
2. How to display the font size in use in the final output,
3. How to display the font size in use in the final output,
4. How to display the font size in use in the final output,
5. How to display the font size in use in the final output,
6. How to display the font size in use in the final output,

7. How to display the font size in use in the final output,
8. How to display the font size in use in the final output,
9. How to display the font size in use in the final output,
10. How to display the font size in use in the final output,
11. How to display the font size in use in the final output,
12. How to display the font size in use in the final output,

\subsubsection{\showfont}

1. How to display the font size in use in the final output,
2. How to display the font size in use in the final output,
3. How to display the font size in use in the final output,
4. How to display the font size in use in the final output,
5. How to display the font size in use in the final output,
6. How to display the font size in use in the final output,

7. How to display the font size in use in the final output,
8. How to display the font size in use in the final output,
9. How to display the font size in use in the final output,
10. How to display the font size in use in the final output,
11. How to display the font size in use in the final output,
12. How to display the font size in use in the final output,


Lipsum me [31-35]

\end{otherlanguage*}



    \end{apendicesenv}

    % Inicia os anexos
    \begin{anexosenv}
        % Imprime uma página indicando o início dos anexos
        % \ifforcedinclude\else\partanexos\fi
        \setlength\beforechapskip{50pt}
        \setlength\midchapskip{20pt}
        \setlength\afterchapskip{20pt}

        % 


%
% How to fix the Underfull \vbox badness has occurred while \output is active on my memoir chapter style?
% https://tex.stackexchange.com/questions/387881/how-to-fix-the-underfull-vbox-badness-has-occurred-while-output-is-active-on-m
%

% ---

\lang
{\chapter[Appendix A]{Since this page is not being completely filled, it is generating the bottom bottom of the page}}
{\chapter[Apêndice A]{Como esta página não está sendo completamente preenchida, ele está gerando a caixa inferior inferior da página}}
% ---


% Multiple-language document - babel - selectlanguage vs begin/end{otherlanguage}
% https://tex.stackexchange.com/questions/36526/multiple-language-document-babel-selectlanguage-vs-begin-endotherlanguage
\begin{otherlanguage*}{english}

\showfont

1. How to display the font size in use in the final output,
2. How to display the font size in use in the final output,
3. How to display the font size in use in the final output,
4. How to display the font size in use in the final output,
5. How to display the font size in use in the final output,
6. How to display the font size in use in the final output,
7. How to display the font size in use in the final output,
8. How to display the font size in use in the final output,
9. How to display the font size in use in the final output,


% As this page is not being completely filled, it is generating the page bottom bad box.
% Fix Underfull \vbox (badness 10000) has occurred while \output is active
%
% \flushbottom vs \raggedbottom
% https://tex.stackexchange.com/questions/65355/flushbottom-vs-raggedbottom
\newpage



\section[Some encoding tests]{\showfont}

1. How to display the font size in use in the final output,
2. How to display the font size in use in the final output,
3. How to display the font size in use in the final output,
4. How to display the font size in use in the final output,
5. How to display the font size in use in the final output,
6. How to display the font size in use in the final output,

7. How to display the font size in use in the final output,
8. How to display the font size in use in the final output,
9. How to display the font size in use in the final output,
10. How to display the font size in use in the final output,
11. How to display the font size in use in the final output,
12. How to display the font size in use in the final output,

\subsection{\showfont}

1. How to display the font size in use in the final output,
2. How to display the font size in use in the final output,
3. How to display the font size in use in the final output,
4. How to display the font size in use in the final output,
5. How to display the font size in use in the final output,
6. How to display the font size in use in the final output,

7. How to display the font size in use in the final output,
8. How to display the font size in use in the final output,
9. How to display the font size in use in the final output,
10. How to display the font size in use in the final output,
11. How to display the font size in use in the final output,
12. How to display the font size in use in the final output,

\subsubsection{\showfont}

1. How to display the font size in use in the final output,
2. How to display the font size in use in the final output,
3. How to display the font size in use in the final output,
4. How to display the font size in use in the final output,
5. How to display the font size in use in the final output,
6. How to display the font size in use in the final output,

7. How to display the font size in use in the final output,
8. How to display the font size in use in the final output,
9. How to display the font size in use in the final output,
10. How to display the font size in use in the final output,
11. How to display the font size in use in the final output,
12. How to display the font size in use in the final output,


Lipsum me [31-35]

\end{otherlanguage*}



    \end{anexosenv}

    % INDICE REMISSIVO
    \ifforcedinclude\else
        \phantompart
        \printindex
    \fi

\end{document}

