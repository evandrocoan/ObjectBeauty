
%
% Simple Sectioned Essay Template - LaTeX Template
%
% This template has been downloaded from:
% http://www.latextemplates.com
%
% `proposal.tex`
% Based on
%
% 1. https://github.com/royertiago/tcc
% 2. http://portal.bu.ufsc.br/normalizacao/
% 3. https://github.com/evandrocoan/ufscthesisx
% 4. http://www.latextemplates.com/template/simple-sectioned-essay
%
% Initially translated from Portuguese with help of https://github.com/omegat-org/omegat
% <Computer Assisted Translation of LaTeX document>
% https://tex.stackexchange.com/questions/313732/computer-assisted-translation-of-latex-document
%
% In case a translation back to Portuguese is required, keep both languages toguether.
% <Is it possible to keep my translation together with original text?>
% https://tex.stackexchange.com/questions/5076/is-it-possible-to-keep-my-translation-together-with-original-text
%
% You can build this using the command:
% latexmk -pdf -jobname=output -output-directory=cache -aux-directory=cache -pdflatex="pdflatex -interaction=nonstopmode" -use-make main.tex



%----------------------------------------------------------------------------------------
%   PACKAGES AND OTHER DOCUMENT CONFIGURATIONS
%----------------------------------------------------------------------------------------

% Uncomment the line `\englishtrue` to set the document default language to english.
%
% Is it possible to keep my translation together with original text?
% https://tex.stackexchange.com/questions/5076/is-it-possible-to-keep-my-translation-together-with-original-text
\newif\ifenglish
\englishfalse
\englishtrue

% How to expand \ifthenelse so that it can be used in \parshape?
% https://tex.stackexchange.com/questions/131002/how-to-expand-ifthenelse-so-that-it-can-be-used-in-parshape
\newcommand{\chooselang}[2]{\ifenglish#1\else#2\fi}

\ifenglish
    % How to make \PassOptionsToPackage add the option as the last option?
    % https://tex.stackexchange.com/questions/385895/how-to-make-passoptionstopackage-add-the-option-as-the-last
    \PassOptionsToPackage{brazil,main=english,spanish,french}{babel}
    \newcommand{\swapcontents}[2]{#1 #2}
\else
    \PassOptionsToPackage{main=brazil,english,spanish,french}{babel}
    \newcommand{\swapcontents}[2]{#2 #1}
\fi

% You need to run `pdfTeX` 5 times on the following order: 1. `pdfTeX`, 2. `bibtex`, 3. `pdfTeX` 4.
% `pdfTeX` 5. `pdfTeX` 6. `pdfTeX`, because the bibliography includes a cyclic reference to another
% bibliography, so we need a last pass to fix the bibliography undefined references.
%
% Fix recurring LaTeX Warning
% https://github.com/abntex/abntex2/pull/189
%
% To fix the warning `LaTeX Warning: Label(s) may have changed. Rerun to get cross-references right`,
% open the file `D:\User\Documents\latex\texmfs\install\tex\latex\abntex2\abntex2cite.sty` and
% comment out these two lines:
% 547: % \renewcommand{\bibcite}[2]{%
% 548: %   \@newl@bel{b}{#1}{\hyper@@link[cite]{}{cite.#1}{#2}}}%
\input{ufscthesisx/setup/setup}

% Load the UFSC thesis settings
\usepackage{ufscthesisx/setup/ufscthesisx}

% Load all required basic packages




% Incompatible color definition when using tikz with color package
% https://tex.stackexchange.com/questions/150369/incompatible-color-definition-when-using-tikz-with-color-package
\usepackage{xcolor}

\definecolor{dkgreen}{rgb}{0,0.6,0}
\definecolor{gray}{rgb}{0.5,0.5,0.5}
\definecolor{mauve}{rgb}{0.58,0,0.82}

\definecolor{link_color}{RGB}{26,13,178}


% For web links and paths with \path{..} and \url{https://www.python.org/downloads/}
%
% https://tex.stackexchange.com/questions/3033/forcing-linebreaks-in-url
% ftp://tug.ctan.org/pub/tex-archive/macros/latex/contrib/hyperref/doc/options.pdf
\PassOptionsToPackage{hyphens}{url}
\usepackage[backref,colorlinks,linkcolor=link_color,citecolor=dkgreen]{hyperref}

% How to fix URL overfull & underfull on emumeration?
% % https://tex.stackexchange.com/questions/366803/how-to-fix-url-overfull-underfull-on-emumeration
%
% Forcing linebreaks in \url
% https://tex.stackexchange.com/questions/3033/forcing-linebreaks-in-url/10401
\usepackage{url}
\makeatletter
\g@addto@macro{\UrlBreaks}{\UrlOrds}
\makeatother

% \lettrine{O}{nce} upon a time...
% \lettrine[findent=2pt]{\fbox{\textbf{T}}}{ }his thesis deals with...
%
% https://tex.stackexchange.com/questions/164298/starting-a-paragraph-with-a-big-letter
\usepackage{lettrine}

% Required for including pictures, resizebox
\usepackage{graphicx}

% Allows putting an [H] in \begin{figure} to specify the exact location of the figure
\usepackage{float}

% Allows in-line images such as the example fish picture
\usepackage{wrapfig}

% How to automatically force latex to not justify the text when it is not wise?
% https://tex.stackexchange.com/questions/365801/how-to-automatically-force-latex-to-not-justify-the-text-when-it-is-not-wise
\usepackage{array,ragged2e}

% Use its macro adjustwidth* to extend tables out of outer text border.
% https://tex.stackexchange.com/questions/366155/how-to-write-a-table-a-little-larger-than-the-paragraphs-with-centered-columns
\usepackage[strict]{changepage}

% No spacing between enumerated items with \usepackage{enumerate}
% https://tex.stackexchange.com/questions/119919/no-spacing-between-enumerated-items-with-usepackageenumerate
\usepackage[shortlabels]{enumitem}

\usepackage{tabularx}
\usepackage{multirow}








%
% New Macros
%

% How could the `\everypar` justification statement be used?
% https://tex.stackexchange.com/questions/365818/how-could-the-everypar-justification-statement-be-used
\newbox\linebox \newbox\snapbox
\def\eatlines{
  \setbox\linebox\lastbox % check the last line
  \ifvoid\linebox
  \else % if it’s not empty
    \unskip\unpenalty % take whatever is
    {\eatlines} % above it;
    \setbox\snapbox\hbox{\unhcopy\linebox}
    \ifdim\wd\snapbox<.98\wd\linebox
       \box\snapbox % take the one or the other,
    \else \box\linebox \fi
  \fi
}


% How could the `\everypar` justification statement be used?
% https://tex.stackexchange.com/questions/365818/how-could-the-everypar-justification-statement-be-used
\everypar={\setbox0=\lastbox \par
   \vbox\bgroup \everypar={}\def\par{\endgraf\eatlines\egroup}}


% Creates a new environment which can be used as:
%
% \begin{foo}
%   Text...
%
%   Text ...
% \end{foo}
%
% https://tex.stackexchange.com/questions/62333/push-long-words-in-a-new-line
\newenvironment{foo}
{\par
\hyphenpenalty=10000
\exhyphenpenalty=10000
}
{\par}


% How to break long URLs using common hyphenation but adding a line feed indicator?
%
% some text \brkurl{http://www.example.com/this/directory/here}
%
% https://tex.stackexchange.com/questions/69824/how-to-break-long-urls-using-common-hyphenation-but-adding-a-line-feed-indicator
\def\addurlspace#1{%
\ifx\relax#1%
\else
\ifx/#1\space\fi
\ifx.#1\space\fi
#1%
\ifx/#1\space\fi
\ifx.#1\space\fi
\expandafter\addurlspace
\fi}

\makeatletter

\@namedef{OT1-zwidthchar}{255}
\@namedef{T1-zwidthchar}{"17}

\def\brkurl#1{%
\edef\savedhchar{\the\hyphenchar\font}%
\global\setbox1\hbox{}%
\setbox0=\vbox{\hsize=2pt\rightskip=0pt plus 1fill
\hfuzz\maxdimen
\tracinglostchars0
\overfullrule0pt
\hyphenchar\font=\csname \f@encoding-zwidthchar\endcsname
\noindent \hskip0pt \addurlspace #1\relax
\par
\loop
\setbox4 \lastbox
\ifvoid4 \else
\global\setbox1\hbox{\unhbox4\unskip\unskip\discretionary{\hbox{\rlap{$\leftarrow$}}}{}{}\unhbox1}%
\unskip
\unskip
\unpenalty
\unskip
\repeat
}%
\unhbox1
\hyphenchar\font\savedhchar
\relax}

\makeatother


% Change background color for text block
% https://tex.stackexchange.com/questions/238294/change-background-color-for-text-block
\usepackage{framed}
\usepackage[most]{tcolorbox}
\definecolor{shadecolor}{RGB}{219, 229, 241}
\newtcolorbox{myquote}{
colback=shadecolor,
grow to right by=-2mm,
grow to left by=-2mm,
boxrule=0pt,
boxsep=0pt,
breakable,
}

% Make first row of table all bold
%
% Usage:
% 1. Add `B` on the borders and `^` before each column definition.
% 2. `\rowstyle{\bfseries}` before the row you want to bold.
%
% Example:
% \begin{tabularx}{\linewidth}
% {|
%     *1{                 >{\RaggedRight\arraybackslash\hsize=1.1\hsize }BX       |} % Riscos
%     *3{@{\hspace{3.0pt}}>{\Centering\arraybackslash                   }^p{0.9cm}|} % Probabilidade, Impacto, Prioridade
%     *2{                 >{\RaggedRight\arraybackslash\hsize=0.95\hsize}^X       |} % Resposta, Prevenção
% }
%
% \hline
%
% \rowstyle{\bfseries}
% Riscos  & 1 & 2 & 3 & Estratégia de resposta & Ações de prevenção \\ \hline
%
%
% https://tex.stackexchange.com/questions/4811/make-first-row-of-table-all-bold
\usepackage{array}
\newcolumntype{B}{>{\global\let\currentrowstyle\relax}}
\newcolumntype{^}{>{\currentrowstyle}}
\newcommand{\rowstyle}[1]{\gdef\currentrowstyle{#1}%
  #1\ignorespaces
}



%
% New commands
%

% Allow to push long words on new lines when they do not fit entirely on the current line.
% https://tex.stackexchange.com/questions/62333/push-long-words-in-a-new-line
\newcommand\lword[1]{\leavevmode\nobreak\hskip0pt plus\linewidth\penalty50\hskip0pt plus-\linewidth\nobreak{#1}}
\newcommand\lurl[1]{\leavevmode\nobreak\hskip0pt plus\linewidth\penalty50\hskip0pt plus-\linewidth\nobreak{\url{#1}}}


% For the new command \latex
\usepackage{xspace}

% Write the word LaTeX nicely.
\newcommand{\latex}{\LaTeX\xspace}


% Create a bold title all in upper case.
\newcommand{\Title}[1]{\textbf{\MakeUppercase{#1}}}







% Bad boxes settings and programming environments
\input{ufscthesisx/utilities/badboxes}


% Writing code in latex document. Usage: \begin & \end {lstlisting}
% http://stackoverflow.com/questions/3175105/writing-code-in-latex-document
\usepackage{listings}

% How to insert code with accents with listings?
% https://tex.stackexchange.com/questions/30512/how-to-insert-code-with-accents-with-listings
\usepackage{listingsutf8}

% Incompatible color definition when using tikz with color package
% https://tex.stackexchange.com/questions/150369/incompatible-color-definition-when-using-tikz-with-color-package
\usepackage{xcolor}

\definecolor{dkgreen}{rgb}{0,0.6,0}
\definecolor{gray}{rgb}{0.5,0.5,0.5}
\definecolor{mauve}{rgb}{0.58,0,0.82}

\lstset{frame=,
  language=Java,
  aboveskip=3mm,
  belowskip=3mm,
  showstringspaces=false,
  columns=flexible,
  basicstyle={\small\ttfamily},
  numbers=left,
  numberstyle=\color{gray},
  keywordstyle=\color{blue},
  commentstyle=\color{dkgreen},
  stringstyle=\color{mauve},
  breaklines=true,
  breakatwhitespace=true,
  tabsize=3
}

% Defining `lstset` parameters for multiple languages & How can I highlight YAML code in a pretty way with listings?
%
% Usage \begin{lstlisting}[style=yaml_style] ... \end{lstlisting}
%
% https://tex.stackexchange.com/questions/45711/defining-lstset-parameters-for-multiple-languages
% https://tex.stackexchange.com/questions/152829/how-can-i-highlight-yaml-code-in-a-pretty-way-with-listings
\newcommand\YAMLcolonstyle{\color{red}}
\newcommand\YAMLkeystyle{\color{black}}
\newcommand\YAMLvaluestyle{\color{blue}}
\newcommand\ProcessThreeDashes{\llap{\color{cyan}\mdseries-{-}-}}

\lstdefinestyle{yaml_style}{
  frame=,
  aboveskip=3mm,
  belowskip=3mm,
  showstringspaces=false,
  columns=flexible,
  numbers=left,
  numberstyle=\color{gray},
  breaklines=true,
  breakatwhitespace=true,
  tabsize=2,
  keywords={true,false,null,y,n},
  keywordstyle=\color{darkgray},
  basicstyle=\YAMLkeystyle,                                 % assuming a key comes first
  sensitive=false,
  comment=[l]{\#},
  morecomment=[s]{/*}{*/},
  commentstyle=\color{purple}\ttfamily,
  stringstyle=\YAMLvaluestyle\ttfamily,
  moredelim=[l][\color{orange}]{\&},
  moredelim=[l][\color{magenta}]{*},
  moredelim=**[il][\YAMLcolonstyle{:}\YAMLvaluestyle]{:},   % switch to value style at :
  morestring=[b]',
  morestring=[b]",
  literate = {---}{{\ProcessThreeDashes}}3
             {>}{{\textcolor{red}\textgreater}}1
             {|}{{\textcolor{red}\textbar}}1
             {\ -\ }{{\mdseries\ -\ }}3,
  inputencoding=utf8, % Listings in Latex with UTF-8 (or at least german umlauts)
  extendedchars=true, % http://stackoverflow.com/questions/1116266/listings-in-latex-with-utf-8-or-at-least-german-umlauts
  literate=%
  {é}{{\'{e}}}1
  {è}{{\`{e}}}1
  {ê}{{\^{e}}}1
  {ë}{{\¨{e}}}1
  {É}{{\'{E}}}1
  {Ê}{{\^{E}}}1
  {û}{{\^{u}}}1
  {ù}{{\`{u}}}1
  {ú}{{\'{u}}}1
  {â}{{\^{a}}}1
  {à}{{\`{a}}}1
  {á}{{\'{a}}}1
  {ã}{{\~{a}}}1
  {Á}{{\'{A}}}1
  {Â}{{\^{A}}}1
  {Ã}{{\~{A}}}1
  {ç}{{\c{c}}}1
  {Ç}{{\c{C}}}1
  {õ}{{\~{o}}}1
  {ó}{{\'{o}}}1
  {ô}{{\^{o}}}1
  {Õ}{{\~{O}}}1
  {Ó}{{\'{O}}}1
  {Ô}{{\^{O}}}1
  {î}{{\^{i}}}1
  {Î}{{\^{I}}}1
  {í}{{\'{i}}}1
  {Í}{{\~{Í}}}1
}

% Change background color for text block
% https://tex.stackexchange.com/questions/238294/change-background-color-for-text-block
\usepackage{framed}
\usepackage[most]{tcolorbox}
\definecolor{shadecolor}{RGB}{219, 229, 241}
\newtcolorbox{myquote}{
colback=shadecolor,
grow to right by=-2mm,
grow to left by=-2mm,
boxrule=0pt,
boxsep=0pt,
breakable
}




% Input a empty list of commands when on debug mode
\input{ufscthesisx/utilities/commands_list}



%----------------------------------------------------------------------------------------
%   File settings
%----------------------------------------------------------------------------------------

% Print page margins of a document
% https://tex.stackexchange.com/questions/14508/print-page-margins-of-a-document
\usepackage[showframe,pass]{geometry}

% To use the font Times New Roman, instead of the default LaTeX font
% more up-to-date than '\usepackage{mathptmx}'
\usepackage{newtxtext}
\usepackage{newtxmath}

% Always use it as should improve full justification
% https://tex.stackexchange.com/questions/10377/texttt-overfull-hbox-problem
% https://tex.stackexchange.com/questions/66052/should-i-load-microtype-with-pdflatex
\usepackage{microtype}

% Thesis settings
%----------------------------------------------------------------------------------------
%   Thesis Tweaks and Utilities
%----------------------------------------------------------------------------------------
\makeatletter


% Uncomment this if you are debugging pages' badness Underfull & Overflow
% https://tex.stackexchange.com/questions/115908/geometry-showframe-landscape
% https://tex.stackexchange.com/questions/387077/what-is-the-difference-between-usepackageshowframe-and-usepackageshowframe
% https://tex.stackexchange.com/questions/387257/how-to-do-the-memoir-headings-fix-but-not-have-my-text-going-over-the-page-botto
% https://tex.stackexchange.com/questions/14508/print-page-margins-of-a-document
% \usepackage[showframe,pass]{geometry}

% To use the font Times New Roman, instead of the default LaTeX font
% more up-to-date than '\usepackage{mathptmx}'
% \usepackage{newtxtext}
% \usepackage{newtxmath}

% https://tex.stackexchange.com/questions/182569/how-to-manually-set-where-a-word-is-split
\hyphenation{Ge-la-im}
\hyphenation{Cis-la-ghi}

% Add missing translations for Portuguese
% https://tex.stackexchange.com/questions/8564/what-is-the-right-way-to-redefine-macros-defined-by-babel
\@ifpackageloaded{babel}{\@ifpackagewith{babel}{brazil}{\addto\captionsbrazil{%
  \renewcommand{\mytextpreliminarylistname}{Breve Sumário}
}}{}}{}
\@ifundefined{advisor}{\newcommand{\advisor}[2]{#1}}{}

% Selects a sans serif font family
% \renewcommand{\sfdefault}{cmss}

% Selects a monospaced (“typewriter”) font family
% \renewcommand{\ttdefault}{cmtt}

% Spacing between lines and paragraphs
% https://tex.stackexchange.com/questions/70212/ifpackageloaded-question
\@ifclassloaded{memoir}
{
  % New custom chapter style VZ14, see other chapters styles in:
  % http://repositorios.cpai.unb.br/ctan/info/latex-samples/MemoirChapStyles/MemoirChapStyles.pdf
  \newcommand\thickhrulefill{\leavevmode \leaders \hrule height 1ex \hfill \kern \z@}
  \makechapterstyle{VZ14} { %
    % \thispagestyle{empty}
    \setlength\beforechapskip{50pt}
    \setlength\midchapskip{20pt}
    \setlength\afterchapskip{20pt}
    \renewcommand\chapternamenum{}
    \renewcommand\printchaptername{}
    \renewcommand\chapnamefont{\Huge\scshape}
    \renewcommand\printchapternum {%
      \chapnamefont\null\thickhrulefill\quad
      \@chapapp\space\thechapter\quad\thickhrulefill
    }
    \renewcommand\printchapternonum {%
      \par\thickhrulefill\par\vskip\midchapskip
      \hrule\vskip\midchapskip
    }
    \renewcommand\chaptitlefont{\huge\scshape\centering}
    \renewcommand\afterchapternum {%
      \par\nobreak\vskip\midchapskip\hrule\vskip\midchapskip
    }
    \renewcommand\afterchaptertitle {%
      \par\vskip\midchapskip\hrule\nobreak\vskip\afterchapskip
    }
  }

  % Apply the style `VZ14` just created
  % \chapterstyle{VZ14}

  % http://mirrors.ibiblio.org/CTAN/macros/latex/contrib/memoir/memman.pdf
  \setlength\beforechapskip{0pt}
  \setlength\midchapskip{15pt}
  \setlength\afterchapskip{15pt}

  % Memoir: Warnings “The material used in the headers is too large” w/ accented titles
  % https://tex.stackexchange.com/questions/387293/how-to-change-the-page-layout-with-memoir
  \setheadfoot{30.0pt}{\footskip}
  \checkandfixthelayout
}{}

% Controlling the spacing between one paragraph and another, try also \onelineskip
% Default value for UFSC 0.0cm
\setlength{\parskip}{\advisor{0.0cm}{0.2cm}}

% Paragraph size is given by
% Default value for UFSC 1.5cm
% \setlength{\parindent}{1.3cm}

% https://tex.stackexchange.com/questions/148647/how-to-remove-space-before-enumerate
% https://tex.stackexchange.com/questions/433543/behaviour-of-enumitem-setlist
\advisor{}{
    \setlist*[enumerate]{label=\arabic*,}
    \setlist*[enumerateoptional]{label=\arabic*,}

    % Patch the `abntex2` citacao environment removing the extra space from its top
    % https://tex.stackexchange.com/questions/300340/topsep-itemsep-partopsep-and-parsep-what-does-each-of-them-mean-and-wha
    \xpatchcmd{\citacao}
    {\list{}}
    {\list{}{\topsep=0pt}}
    {}
    {\FAILEDPATCHINGCITACAO}
}


% Color settings across the document
\@ifpackageloaded{xcolor}
{
  % RGB colors in absolute values from 0 to 255 by using `RGB` tag
  \definecolor{darkblue}{RGB}{26,13,178}

  % Colors names definitions as RGB colors in percentage notation by using `rgb` tag
  \definecolor{mygreen}{rgb}{0,0.6,0}
  \definecolor{mygray}{rgb}{0.5,0.5,0.5}
  \definecolor{mymauve}{rgb}{0.58,0,0.82}
  \definecolor{figcolor}{rgb}{1,0.4,0}
  \definecolor{tabcolor}{rgb}{1,0.4,0}
  \definecolor{eqncolor}{rgb}{1,0.4,0}
  \definecolor{linkcolor}{rgb}{1,0.4,0}
  \definecolor{citecolor}{rgb}{1,0.4,0}
  \definecolor{seccolor}{rgb}{0,0,1}
  \definecolor{abscolor}{rgb}{0,0,1}
  \definecolor{titlecolor}{rgb}{0,0,1}
  \definecolor{biocolor}{rgb}{0,0,1}
  \definecolor{blue}{RGB}{41,5,195}

  % PDF Hyperlinks settings
  \@ifpackageloaded{hyperref}
  {
    \hypersetup
    {
      colorlinks=true,     % false: boxed links; true: colored links
      linkcolor=darkblue,  % color of internal links
      citecolor=darkblue, % color of links to bibliography
      filecolor=black,     % color of file links
      urlcolor=\advisor{black}{darkgreen},
      bookmarksdepth=4,
      pdfencoding=auto,%
      psdextra,
    }
  }
}{}

% ---
% Filtering and Mapping Bibliographies
% \DeclareFieldFormat{url}{Disponível~em:\addspace\url{#1}}

% ---
\DeclareSourcemap{
  \maps[datatype=bibtex]{
    % remove fields that are always useless
    \map{
      \step[fieldset=abstract, null]
      \step[fieldset=pagetotal, null]
    }
    % % remove URLs for types that are primarily printed
    % \map{
    %   \pernottype{software}
    %   \pernottype{online}
    %   \pernottype{report}
    %   \pernottype{techreport}
    %   \pernottype{standard}
    %   \pernottype{manual}
    %   \pernottype{misc}
    %   \step[fieldset=url, null]
    %   \step[fieldset=urldate, null]
    % }
    \map{
      \pertype{inproceedings}
      % remove mostly redundant conference information
      \step[fieldset=venue, null]
      \step[fieldset=eventdate, null]
      \step[fieldset=eventtitle, null]
      % do not show ISBN for proceedings
      \step[fieldset=isbn, null]
      % Citavi bug
      \step[fieldset=volume, null]
    }
  }
}

% Backref package settings, pages with citations in bibliography
\newcommand{\biblatexcitedntimes}{\autocap{c}ited \arabic{citecounter} times}
\newcommand{\biblatexcitedonetime}{\autocap{c}ited one time}
\newcommand{\biblatexcitednotimes}{\autocap{n}o citation in the text}

\@ifpackageloaded{babel}{\@ifpackagewith{babel}{brazil}{\addto\captionsbrazil{%
  \renewcommand{\biblatexcitedntimes}{\autocap{c}itado \arabic{citecounter} vezes}
  \renewcommand{\biblatexcitedonetime}{\autocap{c}itado uma vez}
  \renewcommand{\biblatexcitednotimes}{\autocap{n}enhuma citação no texto}
}}{}}{}
\@ifpackageloaded{biblatex}
{%
  % https://tex.stackexchange.com/questions/483707/how-to-detect-whether-the-option-citecounter-was-enabled-on-biblatex
  \ifx\blx@citecounter\relax
    \message{Is citecounter defined? NO!^^J}
  \else
    \message{Is citecounter defined? YES!^^J}
    \ifbacktracker
      \message{Is backtracker defined? YES!^^J}
      \renewbibmacro*{pageref}
      {
        \iflistundef{pageref}
        {\printtext{\biblatexcitednotimes}}
        {%
          \printtext
          {%
            \ifnumgreater{\value{citecounter}}{1}
              {\biblatexcitedntimes}
              {\biblatexcitedonetime}
          }%
          \setunit{\addspace}%
          \ifnumgreater{\value{pageref}}{1}
            {\bibstring{backrefpages}\ppspace}
            {\bibstring{backrefpage}\ppspace}%
          \printlist[pageref][-\value{listtotal}]{pageref}%
        }%
      }

      \DefineBibliographyStrings{brazil}
      {
        backrefpage  = {na página},
        backrefpages = {nas páginas},
      }

      \DefineBibliographyStrings{english}
      {
        backrefpage  = {on page},
        backrefpages = {on pages},
      }
    \else
      \message{Is backtracker defined? NO!^^J}
    \fi
  \fi
}{}

% https://tex.stackexchange.com/questions/391695/is-possible-to-remove-the-link-color-of-the-comma-on-the-citation-link
% \DeclareFieldFormat{citehyperref}{#1}

% https://tex.stackexchange.com/questions/19105/how-can-i-put-more-space-between-bibliography-entries-biblatex
\advisor{}{\setlength\bibitemsep{2.1\itemsep}}

% % https://tex.stackexchange.com/questions/203764/reduce-font-size-of-bibliography-overfull-bibliography
% \newcommand{\bibliographyfontsize}{\fontsize{10.0pt}{10.5pt}\selectfont}
% \renewcommand*{\bibfont}{\bibliographyfontsize}

% Uncomment this to insert the abstract into your bibliography entries when the abstract is available
% https://tex.stackexchange.com/questions/398666/how-to-correctly-insert-and-justify-abstract
\ifadvisor
\else
  \DeclareFieldFormat{abstract}%
  {%
    \vspace*{-0.5mm}\par\justifying
    \begin{adjustwidth}{1cm}{}
      \textbf{\bibsentence\bibstring{abstract}:} #1
    \end{adjustwidth}
  }
  \renewbibmacro*{finentry}%
  {%
    \iffieldundef{abstract}
    {\finentry}
    {\finentrypunct
      \printfield{abstract}%
      \renewcommand*{\finentrypunct}{}%
      \finentry
    }
  }
\fi


% https://tex.stackexchange.com/questions/14314/changing-the-font-of-the-numbers-in-the-toc-in-the-memoir-class
\renewcommand{\cftpartfont}{\ABNTEXpartfont\color{black}}
\renewcommand{\cftpartpagefont}{\ABNTEXpartfont\color{black}}

\renewcommand{\cftchapterfont}{\ABNTEXchapterfont\color{black}}
\renewcommand{\cftchapterpagefont}{\ABNTEXchapterfont\color{black}}

\renewcommand{\cftsectionfont}{\ABNTEXsectionfont\color{black}}
\renewcommand{\cftsectionpagefont}{\ABNTEXsectionfont\color{black}}

\renewcommand{\cftsubsectionfont}{\ABNTEXsubsectionfont\color{black}}
\renewcommand{\cftsubsectionpagefont}{\ABNTEXsubsectionfont\color{black}}

\renewcommand{\cftsubsubsectionfont}{\ABNTEXsubsubsectionfont\color{black}}
\renewcommand{\cftsubsubsectionpagefont}{\ABNTEXsubsubsectionfont\color{black}}

\renewcommand{\cftparagraphfont}{\ABNTEXsubsubsubsectionfont\color{black}}
\renewcommand{\cftparagraphpagefont}{\ABNTEXsubsubsubsectionfont\color{black}}

% Memoir has another mechanism for the job: \cftsetindents{‹kind›}{indent}{numwidth}. Here kind is
% chapter, section, or whatever; the indent specifies the ‘margin’ before the entry starts; and the
% width is of the box into which the number is typeset (so needs to be wide enough for the largest
% number, with the necessary spacing to separate it from what comes after it in the line.
% http://www.tex.ac.uk/FAQ-tocloftwrong.html
% https://tex.stackexchange.com/questions/264668/memoir-indentation-of-unnumbered-sections-in-table-of-contents
% https://tex.stackexchange.com/questions/394227/memoir-toc-indent-the-second-line-by-numberspace
%
% `\cftlastnumwidth` and these `\cftsetindents` are defined by the abntex2 class,
% obeying the `ABNTEXsumario-abnt-6027-2012`. \newlength{\cftlastnumwidth}
% \setlength{\cftlastnumwidth}{\cftsubsubsectionnumwidth}
% \addtolength{\cftlastnumwidth}{-1em}

% http://www.tex.ac.uk/FAQ-tocloftwrong.html
% Use \setlength\cftsectionnumwidth{4em} to override all these values at once
\ifadvisor
\else
  \makechapterstyle{fixedabntex2indentation}
  {%
    \renewcommand{\chapterheadstart}{}
    \setlength{\beforechapskip}{20pt}
    \setlength{\midchapskip}{20pt}
    \setlength{\afterchapskip}{15pt}

    \ifx \chapternamenumlength \undefined
      \newlength{\chapternamenumlength}
    \fi

    % tamanhos de fontes de chapter e part
    \ifthenelse{\equal{\ABNTEXisarticle}{true}}{%
      \setlength\beforechapskip{\baselineskip}%
      \renewcommand{\chaptitlefont}{\ABNTEXsectionfont\ABNTEXsectionfontsize}%
    }{%else
       \setlength{\beforechapskip}{0pt}%
       \renewcommand{\chaptitlefont}{\ABNTEXchapterfont\ABNTEXchapterfontsize}%
    }

    \renewcommand{\chapnumfont}{\chaptitlefont}
    \renewcommand{\parttitlefont}{\ABNTEXpartfont\ABNTEXpartfontsize}
    \renewcommand{\partnumfont}{\ABNTEXpartfont\ABNTEXpartfontsize}
    \renewcommand{\partnamefont}{\ABNTEXpartfont\ABNTEXpartfontsize}

    % tamanhos de fontes de section, subsection, subsubsection e subsubsubsection
    \setsecheadstyle{\ABNTEXsectionfont\ABNTEXsectionfontsize\ABNTEXsectionupperifneeded}
    \setsubsecheadstyle{\ABNTEXsubsectionfont\ABNTEXsubsectionfontsize\ABNTEXsubsectionupperifneeded}
    \setsubsubsecheadstyle{\ABNTEXsubsubsectionfont\ABNTEXsubsubsectionfontsize\ABNTEXsubsubsectionupperifneeded}
    \setsubsubsubsecheadstyle{\ABNTEXsubsubsubsectionfont\ABNTEXsubsubsubsectionfontsize\ABNTEXsubsubsubsectionupperifneeded}

    % Impressão do número do capítulo
    \renewcommand{\chapternamenum}{}

    % Impressão do nome do capítulo
    \renewcommand{\printchaptername}{%
       \chaptitlefont%
       \ifthenelse{\boolean{abntex@apendiceousecao}}{\appendixname}{}%
    }

    % Impressão do título do capítulo
    \def\printchaptertitle##1{%
      \chaptitlefont%
      \ifthenelse{\boolean{abntex@innonumchapter}}{\centering\ABNTEXchapterupperifneeded{##1}}{%
      \ifthenelse{\boolean{abntex@apendiceousecao}}{%
        \centering%
        \settowidth{\chapternamenumlength}{\printchaptername\printchapternum\afterchapternum}%
        \ABNTEXchapterupperifneeded{##1}%
      }{%
        \settowidth{\chapternamenumlength}{\printchaptername\printchapternum\afterchapternum}%
        \parbox[t]{\columnwidth-\chapternamenumlength}{\ABNTEXchapterupperifneeded{##1}}}%
      }%
    }

    % https://tex.stackexchange.com/questions/264668/memoir-indentation-of-unnumbered-sections-in-table-of-contents
    \renewcommand{\tocinnonumchapter}{%
      \addtocontents{toc}{\cftsetindents{chapter}{2.5em}{2em}}%
      \cftinserthook{toc}{A}}

    % Impressão do número do capítulo (no capítulo e não toc)
    \renewcommand{\printchapternum}{%
      \setboolean{abntex@innonumchapter}{false}%
      \chapnumfont%
      ~~\thechapter~%
      \ifthenelse{\boolean{abntex@apendiceousecao}}{%
        \tocinnonumchapter%
        ~\ABNTEXcaptiondelim~~%
      }{}%
    }

    \renewcommand{\ABNTEXcaptiondelim}{~\textendash~}
    \renewcommand{\afterchapternum}{}

    % Impressão do capítulo não numerado
    \renewcommand\printchapternonum{%
      \setboolean{abntex@innonumchapter}{true}%
    }
  }
  \chapterstyle{fixedabntex2indentation}

  \cftsetindents{part}          {0em} {3em}
  \cftsetindents{chapter}       {0em} {3em}
  \cftsetindents{section}       {0em} {4.3em}
  \cftsetindents{subsection}    {0em} {5.2em}
  \cftsetindents{subsubsection} {0em} {5.1em}
  \cftsetindents{paragraph}     {0em} {6.0em}
  \cftsetindents{subparagraph}  {0em} {7.0em}
\fi


\makeatother



% When writing a large document, it is sometimes useful to work on selected sections of the document
% to speed up compilation time: https://en.wikibooks.org/wiki/TeX/includeonly
%
% \includeonly{pretexto/agradecimentos}
% \includeonly{pretexto/epigrafe}
% \includeonly{pretexto/fichacatalografica}
% \includeonly{pretexto/folhadeaprovacao}
% \includeonly{pretexto/resumos}
% \includeonly{pretexto/siglas}
% \includeonly{pretexto/simbolos}

% \includeonly{chapters/chapter_1}
% \includeonly{chapters/chapter_2}
% \includeonly{chapters/conclusion}

% \includeonly{postexto/anexo_a}
% \includeonly{postexto/apendice_a}



% %----------------------------------------------------------------------------------------
% %   DOCUMENT CONTENTS
% %----------------------------------------------------------------------------------------

\begin{document}

    % Retira espaço extra obsoleto entre as frases `Double space between sentences`
    % https://tex.stackexchange.com/questions/4705/double-space-between-sentences
    \frenchspacing

    

% How to fix destination with the same identifier (name{page.A}) has been already used, duplicate ignored?
% https://tex.stackexchange.com/questions/386446/how-to-fix-destination-with-the-same-identifier-namepage-a-has-been-already
\hypersetup{pageanchor=false}



% ELEMENTOS PRÉ-TEXTUAIS
% \includepdf{pictures/FrenteCapaUFSC.pdf}

\pretextual

% Capa
\imprimircapa

% Folha de rosto (o * indica que haverá a ficha bibliográfica)
\imprimirfolhaderosto*

% Inserir a ficha bibliografica
%
% Isto é um exemplo de Ficha Catalográfica, ou ``Dados internacionais de
% catalogação-na-publicação''. Você pode utilizar este modelo como referência.
% Porém, provavelmente a biblioteca da sua universidade lhe fornecerá um PDF
% com a ficha catalográfica definitiva após a defesa do trabalho. Quando estiver
% com o documento, salve-o como PDF no diretório do seu projeto e substitua todo
% o conteúdo de implementação deste arquivo pelo comando abaixo:
%
% \begin{fichacatalografica}
%    \includepdf{pretexto/ficha_catalografica.pdf}
% \end{fichacatalografica}


\ifenglish

Legal Notes

There is no warranty for any part of the documented software. The authors have taken care in the
preparation of this thesis, but make no expressed or implied warranty of any kind and assume no
responsibility for errors or omissions. No liability is assumed for incidental or consequential
damages in connection with or arising out of the use of the information or programs contained here.
\cite{koma-scrguien}

\else

Notas legais

Não há garantia para qualquer parte do software documentado. Os autores tomaram cuidado na
preparação desta tese, mas não fazem nenhuma garantia expressa ou implícita de qualquer tipo e não
assumem qualquer responsabilidade por erros ou omissões. Não se assume qualquer responsabilidade por
danos incidentais ou consequentes em conexão ou decorrentes do uso das informações ou programas aqui
contidos. \cite{koma-scrguien}

\fi


% http://portalbu.ufsc.br/ficha
% http://portal.bu.ufsc.br/servicos/ficha-de-identificacao-da-obra/
\begin{fichacatalografica}
    \vspace*{\fill}

    \begin{center}

        Catalogação na fonte pela Biblioteca Universitária da Universidade Federal de Santa Catarina.

        Arquivo compilado às \currenttime h do dia \today.

        \framebox[\textwidth]
        {
            \begin{minipage}{0.98\textwidth}

                \ttfamily
                \imprimirautor

                \hspace{0.5cm} \imprimirtitulo~:~\imprimirsubtitulo~/~\imprimirautor;
                orientador(a),~\imprimirorientador;~co\hyp{}orientador(a),~\imprimircoorientador
                ~--~\imprimirlocal,~\currenttime,~\imprimirdata.

                % Prints how much pages there are on the document and links to the last page
                \hspace{0.5cm} \pageref{LastPage} p.
                \bigskip

                \hspace{0.5cm} \imprimirtipotrabalho~--~\imprimirinstituicao,
                \imprimircentro,~\imprimirprograma.
                \bigskip

                \hspace{0.5cm} Inclui referências
                \bigskip

                \hspace{0.5cm}
                    1. Uma Palavra\hyp{}chave ~
                    2. Outra Palavra\hyp{}chave ~
                    3. Mais Palavras\hyp{}chave ~
                    I. \imprimirorientador ~
                    II. \imprimircoorientador ~
                    III. \imprimirprograma ~
                    IV. \imprimirtitulo ~
                \bigskip

                \hspace{7.75cm} CDU 02:141:005.7

            \end{minipage}
        }

    \end{center}

\end{fichacatalografica}



% Custom list throw LaTeX Error: Command \mycustomfiction already defined?
% https://tex.stackexchange.com/questions/388489/custom-list-throw-latex-error-command-mycustomfiction-already-defined/
\textpreliminarycontents
\cleardoublepage

% Inserir errata

% Inserir folha de aprovação. Isto é um exemplo de Folha de aprovação, elemento obrigatório da NBR
% 14724/2011 (seção 4.2.1.3). Você pode utilizar este modelo até a aprovação do trabalho. Após isso,
% substitua todo o conteúdo deste arquivo por uma imagem da página assinada pela banca com o comando
% abaixo:
% \includepdf{folhadeaprovacao_final.pdf}


\addtotextpreliminarycontent{\chooselang{Approval Sheet}{Folha de Aprovação}}

\begin{folhadeaprovacao}

    \begin{center}
        {\ABNTEXchapterfont\large\imprimirautor}

        \begin{center}
            \ABNTEXchapterfont\bfseries\Large\imprimirtitulo
        \end{center}

        \begin{minipage}{\textwidth}
            \chooselang
            {
                This thesis was considered appropriate to obtain the Doctor 's Degree in Electrical Engineering, in the area of concentration in Power Electronics and Electrical Drive, and approved in its final form by the Post-Graduate Program in Electrical Engineering of the Federal University of Santa Catarina.
            }
            {
                Esta Tese foi julgada adequada para obtenção do Título de Doutor em Engenharia Elétrica, na área de concentração em Eletrônica de Potência e Acionamento Elétrico, e aprovada em sua forma final pelo Programa de Pós--Graduação em Engenharia Elétrica da Universidade Federal de Santa Catarina.
            }
        \end{minipage}%

    \end{center}
    \begin{center}
        Florianópolis, \imprimirdata.
    \end{center}

    \assinatura
    {
        \textbf{Prof. Marcelo Lobo Heldwein, Dr.} \\
        \chooselang{Coordinator of the}{Coordenador do} \imprimirprograma
    }

    \assinatura
    {
        \textbf{\imprimirorientador} \\ \imprimirorientadorRotulo \\
        \imprimirinstituicao~--~\imprimirinstituicaosigla
    }

    \assinatura
    {
        \textbf{\imprimircoorientador} \\ \imprimircoorientadorRotulo \\
        \imprimirinstituicao~--~\imprimirinstituicaosigla
    }

    \newpage
    \begin{flushleft}
        \textbf{\chooselang{Examination Board:}{Banca Examinadora:}}
    \end{flushleft}

    \assinatura{\textbf{Prof. Arnaldo José Perin, \chooselang{PhD.}{Dr.}} \\
    \imprimirinstituicao~--~\imprimirinstituicaosigla}

    \assinatura{\textbf{Prof. Denizar Cruz Martins, \chooselang{PhD.}{Dr.}} \\
    \imprimirinstituicao~--~\imprimirinstituicaosigla}

    \assinatura{\textbf{Prof. Roberto Francisco Coelho, \chooselang{PhD.}{Dr.}} \\
    \imprimirinstituicao~--~\imprimirinstituicaosigla}

    \assinatura{\textbf{Prof. Samir Ahmad Mussa, \chooselang{PhD.}{Dr.}} \\
    \imprimirinstituicao~--~\imprimirinstituicaosigla}

    \assinatura{\textbf{Prof. Telles Brunelli Lazzarin, \chooselang{PhD.}{Dr.}} \\
    \imprimirinstituicao~--~\imprimirinstituicaosigla}

\end{folhadeaprovacao}



% Dedicatória
\addtotextpreliminarycontent{\chooselang{Dedication}{Dedicatória}}
\begin{dedicatoria}
    \vspace*{\fill}
    \centering
    \noindent
    \textit{\chooselang
    {
        This work is dedicated to adult children who, \\
        When small, dreamed of becoming scientists.
    }
    {
        Este trabalho é dedicado às crianças adultas que,\\
        quando pequenas, sonharam em se tornar cientistas.
    }}
    \vspace*{\fill}
\end{dedicatoria}

% Agradecimentos


\addtotextpreliminarycontent{\chooselang{Acknowledgement}{Agradecimentos}}

\begin{agradecimentos}

\chooselang
{
    Greetings.
}
{
    Os agradecimentos principais são direcionados à Gerald Weber, Miguel Frasson,
    Leslie H. Watter, Bruno Parente Lima, Flávio de Vasconcellos Corrêa, Otavio Real
    Salvador, Renato Machnievscz\footnote{Os nomes dos integrantes do primeiro
    projeto abn\TeX\ foram extraídos de
    \url{http://codigolivre.org.br/projects/abntex/}} e todos aqueles que
    contribuíram para que a produção de trabalhos acadêmicos conforme
    as normas ABNT com \LaTeX{} fosse possível.

    Agradecimentos especiais são direcionados ao Centro de Pesquisa em Arquitetura
    da Informação\footnote{\url{http://www.cpai.unb.br/}} da Universidade de
    Brasília (CPAI), ao grupo de usuários
    \emph{latex-br}\footnote{\url{http://groups.google.com/group/latex-br}} e aos
    novos voluntários do grupo
    \emph{\abnTeX{}}\footnote{\url{http://groups.google.com/group/abntex2} e
    \url{http://abntex2.googlecode.com/}}~que contribuíram e que ainda
    contribuirão para a evolução do \abnTeX{}.
}

\end{agradecimentos}


%Mesmo padrão da seção primária, porém sem indicativo numérico. Assim como: Dedicatória, Resumo, Abstract, Sumário, Listas, Referências, Apêndices e Anexos.
%
%
%Corpo do texto, fonte 10,5, justificado, recuo especial da primeira linha de 1 cm, espaçamento simples.
%


% Epígrafe


\addtotextpreliminarycontent{\lang{Epigraph}{Epigrafe}}

\begin{epigrafe}

\vspace*{\fill}\lang
{
    \begin{flushright}
        \textit{``Learn from yesterday, live for today, hope for tomorrow. The important thing is not to stop questioning.''} \\ Albert Einstein
    \end{flushright}
    \begin{flushright}
        \textit{``The true sign of intelligence is not knowledge but imagination.''} \\  Albert Einstein
    \end{flushright}
    \begin{flushright}
        \textit{``Peace cannot be kept by force; it can only be achieved by understanding.''} \\ Albert Einstein
    \end{flushright}
    \begin{flushright}
        \textit{``Whoever is careless with the truth in small matters cannot be trusted with important matters.''} \\ Albert Einstein
    \end{flushright}
    \begin{flushright}
        \textit{``Extraordinary claims require extraordinary evidence''} \\ Carl Sagan
    \end{flushright}
    \begin{flushright}
        \textit{``Catholic, which I was until I reached the age of reason.''} \\ George Carlin
    \end{flushright}
    \begin{flushright}
        \textit{``We made too many wrong mistakes.''} \\ Yogi Berra
    \end{flushright}
}
{
    \begin{flushright}
        \textit{``Assim como aquele pecado da juventude, este documento te perseguirá pelo resto da vida. \showfont''} \\ Enio Valmor Kassick
    \end{flushright}
    \begin{flushright}
        \textit{``Estupidez trará mais autoconfiança do que o conhecimento e a bravura juntas.''} \\ Adriano Ruseler
    \end{flushright}
}

\end{epigrafe}





% RESUMOS
% Ajusta o espaçamento dos parágrafos do resumo
\setlength{\absparsep}{18pt}


% Is it possible to keep my translation together with original text?
% https://tex.stackexchange.com/questions/5076/is-it-possible-to-keep-my-translation-together-with-original-text
\begin{comment}

    Segundo a \citeonline[3.1-3.2]{NBR6028:2003}, o resumo deve ressaltar o
    objetivo, o método, os resultados e as conclusões do documento. A ordem e a extensão
    destes itens dependem do tipo de resumo (informativo ou indicativo) e do
    tratamento que cada item recebe no documento original. O resumo deve ser
    precedido da referência do documento, com exceção do resumo inserido no
    próprio documento. (\ldots) As palavras-chave devem figurar logo abaixo do
    resumo, antecedidas da expressão Palavras-chave:, separadas entre si por
    ponto e finalizadas também por ponto.

    ufscthesis: O texto do resumo deve ser digitado, em um único bloco, sem espaço de parágrafo. O resumo deve
    ser significativo, composto de uma sequência de frases concisas, afirmativas e não de uma
    enumeração de tópicos. Não deve conter citações. Deve usar o verbo na voz passiva. Abaixo do
    resumo, deve-se informar as palavras-chave (palavras ou expressões significativas retiradas do
    texto) ou, termos retirados de thesaurus da área. \showfont

\end{comment}



\swapcontents
{
    % Changing babel package inside a single chapter
    % https://tex.stackexchange.com/questions/20987/changing-babel-package-inside-a-single-chapter
    %
    % Multiple-language document - babel - selectlanguage vs begin/end{otherlanguage}
    % https://tex.stackexchange.com/questions/36526/multiple-language-document-babel-selectlanguage-vs-begin-endotherlanguage
    \addtotextpreliminarycontent{English's Abstract}
    \begin{otherlanguage*}{english}
    \begin{resumo}[Abstract]

        A study is made of what Beautifiers are for, as well as approaches to what are good
        programming practices and why we should follow them for good efficiency when writing code in
        the most diverse programming languages. Current source code formatting software, also known
        as Beautifiers, is limited to a similar set, or even a single language, and, in addition to
        many, be limited to what they can do for you when processing / formatting the code.

        Therefore, it is expected at the end of the work, to know what tools exist and which of them
        are the best that can be used to help the programmer while writing codes of the most diverse
        programming languages. Besides the proposal of a new tool with the intuition of centralizing
        in a single program the approach of the most diverse programming languages.

        \imprimirpalavraschave{Keywords}{\wordslistunlabled{\palavraschaveingles}}

    \end{resumo}
    \end{otherlanguage*}
}
{
    \addtotextpreliminarycontent{Resumo em Português}
    \begin{otherlanguage*}{brazil}
    \begin{resumo}[Resumo]

        Faz-se um estudo sobre o que é, para que servem os Beautifiers, assim como abordagens sobre
        o que são boas práticas de programação e por que devemos segui-las para um boa eficiência ao
        escrever códigos nas mais diversas linguagens de programação. Os softwares formatadores de
        código fonte atuais, também conhecidos como Beautifiers, são limitados a um conjunto
        similar, ou mesmo à uma única linguagem, e além de muitos, serem limitados ao que eles podem
        fazer por você ao processar/formatar o código.

        Portanto espera-se o final do trabalho, conhecer-se quais são as ferramentas que existem e
        quais delas são as melhores que podem ser utilizadas para o auxilio do programador durante a
        escrita de códigos das mais diversas linguagens de programação. Além de proposta de uma nova
        ferramenta com o intuído de centralizar em uma único programa o abordagem das mais diversas
        linguagens de programação.

        \imprimirpalavraschave{Palavras-chaves}{\wordslistunlabled{\palavraschaveportugues}}

    \end{resumo}
    \end{otherlanguage*}
}



% % resumo em francês
% \begin{resumo}[Résumé]
%   \begin{otherlanguage*}{french}
%       Il s'agit d'un résumé en français.

%       \imprimirpalavraschave{Mots-clés}{latex. abntex. publication de textes.}
%   \end{otherlanguage*}
% \end{resumo}


% % resumo em espanhol
% \begin{resumo}[Resumen]
%   \begin{otherlanguage*}{spanish}
%       Este es el resumen en español.

%       \imprimirpalavraschave{Palabras clave}{latex. abntex. publicación de textos.}
%   \end{otherlanguage*}
% \end{resumo}





% inserir lista de ilustrações
\addtotextpreliminarycontent{\chooselang{List of Figures}{Lista de Figuras}}
\pdfbookmark[0]{\listfigurename}{lof}
\listoffigures*
\cleardoublepage

% inserir lista de tabelas
\addtotextpreliminarycontent{\chooselang{List of Tables}{Lista de Tabelas}}
\pdfbookmark[0]{\listtablename}{lot}
\listoftables*
\cleardoublepage

% inserir códigos fonte
\addtotextpreliminarycontent{\chooselang{List of Source Codes}{Lista de Códigos Fonte}}
\pdfbookmark[0]{\lstlistingname}{lol}
\lstlistoflistings*
\cleardoublepage


% inserir lista de abreviaturas e siglas


\addtotextpreliminarycontent{\lang{List of Acronyms}{Lista de Siglas}}

\begin{siglas}
    \item[ABNT] \lang{Brazilian Association of Technical Standards}{Associação Brasileira de Normas Técnicas}
    \item[abnTeX] \lang{Absurd Standards for TeX}{ABsurdas Normas para TeX}
\end{siglas}



% Inserir lista de símbolos

% Devam aparecer na mesma ordem de ocorrência no texto.

\begin{simbolos}
    \item[$ \Gamma $] Letra grega Gama
    \item[$ \Lambda $] Lambda
    \item[$ \zeta $] Letra grega minúscula zeta
    \item[$ \in $] Pertence
\end{simbolos}


% How to remove the self-reference of the ToC from the ToC?
% https://tex.stackexchange.com/questions/10943/how-to-remove-the-self-reference-of-the-toc-from-the-toc
\addtotextpreliminarycontent{\chooselang{Table of Contents}{Sumário}}
\begin{KeepFromToc}
    \pdfbookmark[0]{\contentsname}{toc}

    % What does “overfull hbox” mean?
    % https://tex.stackexchange.com/questions/35/what-does-overfull-hbox-mean
    %
    % How to avoid using \sloppy document-wide to fix overfull \hbox problems?
    % https://tex.stackexchange.com/questions/59122/how-to-avoid-using-sloppy-document-wide-to-fix-overfull-hbox-problems
    %
    % Adding color to table of contents and section headings
    % https://tex.stackexchange.com/questions/257007/adding-color-to-table-of-contents-and-section-headings
    {
        % underfull vbox (badness 10000) has occurred while \output is active with memoir
        % https://tex.stackexchange.com/questions/65711/underfull-vbox-badness-10000-with-memoir
        \raggedbottom

        % Overfull \hbox warning for TOC entries when using memoir documentclass
        % https://tex.stackexchange.com/questions/49887/overfull-hbox-warning-for-toc-entries-when-using-memoir-documentclass
        % \makeatletter
            % \renewcommand{\@pnumwidth}{2em}
            % \renewcommand{\@tocrmarg}{3em}
        % \makeatother

        % Memoir mysterious overfull hbox in TOC when mathptmx is used
        % https://tex.stackexchange.com/questions/57544/memoir-mysterious-overfull-hbox-in-toc-when-mathptmx-is-used
        % \setlength{\cftchapternumwidth}{2.25em}

        % Disable `colorlinks` locally (or just for the ToC)
        % https://tex.stackexchange.com/questions/179506/disable-colorlinks-locally-or-just-for-the-toc
        \hypersetup{hidelinks}

        \tableofcontents
    }

    \cleardoublepage
\end{KeepFromToc}



% How to fix destination with the same identifier (name{page.A}) has been already used, duplicate ignored?
% https://tex.stackexchange.com/questions/386446/how-to-fix-destination-with-the-same-identifier-namepage-a-has-been-already
\hypersetup{pageanchor=true}






    % ELEMENTOS TEXTUAIS
    %
    % Configura estilo das páginas.
    \textual

    \setlength\beforechapskip{50pt}
    \setlength\midchapskip{20pt}
    \setlength\afterchapskip{20pt}

    % To automatically put a [Go To Top/Back] ←← | ← on each section
    \addGoToSummary

    % Configura estilo das páginas com logos
    % \textualINEPUFSC

    % Introdução (exemplo de capítulo sem numeração, mas presente no Sumário)
    

% The \phantomsection command is needed to create a link to a place in the document that is not a
% figure, equation, table, section, subsection, chapter, etc.
%
% When do I need to invoke \phantomsection?
% https://tex.stackexchange.com/questions/44088/when-do-i-need-to-invoke-phantomsection
\cleardoublepage
\phantomsection


% Is it possible to keep my translation together with original text?
% https://tex.stackexchange.com/questions/5076/is-it-possible-to-keep-my-translation-together-with-original-text
\chapter{\chooselang{Introduction}{Introdução}}


\chooselang
{
    Questions like ``What are good programming practices?'' Or ``Why are these practices are good?''
    Are not easy to answer. But each programmer learns to write their codes in a certain way, with
    certain features like using 4 or 8 spaces to indent lines, always leave a blank line before each
    control structure as if or for statements, and alike rules. % TODO, put reference for this

    But entering the universe of good practices, there are many things for discoursing. So in this
    work implementation tool called ``Object Beautifier'' specifically dedicates on how to perform
    the best layout/display of programming code on the computer screen, so that maximize and
    facilitate the understanding of same. Therefore, allowing the programmer to disperse more tempe
    thinking about its coding algorithms problem, other than trying to decipher the information that
    is presented to it on the screen through infinit different code layouts. % TODO, put reference for this

    Within this work\textquotesingle s area, we need to also think long and hard about how to share
    the programming code of the programmers among you. Now, the problem of human diversity, like all
    big scientific questions -- how do you explain something like that -- It can be broken down into
    sub-questions. It happens many times, which is a good practice for a `Programmer A', is not the
    same to another `Programmer B'. For example, imagine some code where a programmer decided to put
    before each `if' statement, a blank line. It is therefore expected that whenever we see a blank
    line we can potentially find a matching `if', which can be considered a quite useful pattern
    matching as empty line may call better your attention. % TODO, put reference for this

    But again this is something heavily dependent of what each one learning through their life time.
    Imagine another programmer do not liked this rule, and when he was writing your code involving
    an `if', he did not put such blank line another programmer is expecting. So when the first
    programmer start reading its the code and look for `if', he will be expecting for blank lines
    before its if\textquotesingle s. But will lose some time searching until realize another
    programmer does not put them, or perhaps he forgot to insert them. % TODO, put reference for this

    These differences are due to the diversity of ways we learn programming, i.e., to the ways we
    are used to doing coding, as much as the abilities and objectives of every programmer developed.
    Hence, nowadays it becomes a big problem because we increasingly need more and more programmers
    working together developing several and diverse computing systems. Where the latter is due to
    the fact of the complexity of computer systems growing increasingly, therefore over requiring
    programmers working and sharing their codes and ideas. % TODO, put reference for this

    Moreover besides only worrying about how the code is displayed on their computer screen, we need
    to worry about on how it will be saved in the file system on its `plain-text' mode. Since for
    code sharing, it is vital for you to use a versioning control system which enable project
    manager\textquotesingle s and programmers themselves, take control of their code changes. It does allow to
    easily perform the tracking of code changes and allow you to better understand what each
    programmer is doing every time he formalizes a change in the code through a `commit', as in
    `git' systems for example. % TODO, put reference for this

    That is because while working with a versioning system like `git', we need to keep the code
    among a single style or which we may call a `good practice' set as standard for everybody, due
    the fact of letting each programmer to write as he pleases, there will be plenty of noise on the
    code review and we are figuring out what actually each programmer did. Hence, if every
    programmer re-writes the history making changes like inserting new lines before each if, we end
    up with too much noise and focus of a versioning system is to look at only those changes that
    are significant to the code, such as the creation of new functions and not the addition of new
    blank lines (ref find\_some). % TODO, put reference for this (ref coding\_horror)

    Talking about the last thing pointed out, we could also think about an approach to creating a
    new version control system which focuses only on significant changes to the code, while
    reviewing code changes. However, this approach could not be ideal, as for example, it would
    allow programmers to start tedious wars of unproductive code adjustments. For example, imagine
    how it would be for your every day and have to go through your code re-adding new lines before
    each one of your beloved if\textquotesingle s, just because some night shift programmer had just
    removed them? % TODO, put reference for this

    \section{Goals}

    Establish relationships between good programming practices and efficiency in programming, in
    addition to a new tool to support programmers in order to automate the long and diverse
    programming process in teams of developers with different programming `best practices'.

    \subsection{Specific Goals}

    \begin{enumerate}
        \item A study on universal programming tools, which from a single software, to work well
        behaved accross all programming languages. Moreover, explain the differences for other
        softwares and the beneficits of a unique tool, instead of several heavly different ones.

        \item Define, determine and classify which one are good programming practices and perform an
        in-depth study on the good practices on visual layout area, also known as code `Beautifying'.

        \item A study on the variety of existing tools for the support of good programming
        practices, beyond a comparative analysis between them, determining their weaknesses and
        strengths.

        \item The definition of a flow pattern of development allowing teams of developers with
        different programming best practices, to work without intervene with each other up to start
        wars of `best good practices'.

        \item Propose a unique tool that allowing several and distinct programming `best practices'
        being implemented in several programming languages, which can be configured and set
        accordindgly to their whishes.
    \end{enumerate}

    \section{Search Method}

    The work will be based on research in articles, books, theses, dissertations, trusted authors
    websites, and through new demonstrated evidences based on arguments in the monograph evolution
    road. Also, present results after building a new tool which proposes a solution for the problems
    presented and detailed.

    In this proposal last chapter which lies in the topic \autoref{sec:implementation}, there is a
    series of weblinks and references preselected and may be used in the release build of this work.
    Noticing the texts of the last section probably will end up gradually moved to the first section
    of the text where there is the theoretical research, while correlated research are incorporated
    in the main written work.

    Moreover, at the end of the first part of this work, the completion of the subject entitled of
    Course Conclusion Work 1, leaving only the information for the implementation of the proposed
    tool to be implemented in the second part of this named thesis on \nameref{sec:implementation}.
}
{ % Portuguese




    Perguntas como ``O que são boas práticas de programação?'' ou ainda ``O por quê estas práticas
    são boas?'', não são fáceis de responder. Mas cada programador aprende a escrever seus códigos
    em uma determinada maneira, com determinadas características como utilizar 4 ou 8 espaços para
    indentação de linhas, sempre deixar uma linha em branco antes de cada estrutura de controle como
    if\textquotesingle s, for\textquotesingle s, e afins.

    Mas entrando o universo de boas práticas, há muitos coisas sobre discorrer. Assim neste trabalho
    especificamente trabalhá-se sobre como realizar a melhor disposição/exibição do código de
    programação na tela do computador, de modo que maximize e facilite o entendimento do mesmo.
    Portanto permitindo que o programador dispersa mais tempe pensando sobre o problema, do que
    tentar decifrar a informação que é apresentado para ele na tela.

    Dentro desta área de trabalho, precisa-se também pensar muito bem sobre como compartilhar os
    códigos de programação dos programadores entre si. Isso por que entra agora o problema da
    diversidade de boas práticas de programação. Ela acontece por que muitas vezes, aquilo que é uma
    boa prática para um `programador A', não é para o outro `programador B'. Por exemplo, imagine um
    código onde um programador decidiu colocar antes de cada `if', uma linha em branco. Portanto é
    de se esperar que sempre que vemos uma linha em branco nos podemos potencialmente encontrar um
    `if'. Entretanto imagine que outro programador não gostou dessa regra e quando ele foi escrever
    seu código que envolvia um `if', ele não colocou a essa tal linha em branco que o outro
    programador vinha colocando. Então quando o primeiro programador for ler o código e procurar por
    `if'es, ele vai estar esperando por linhas em branco. Mas vai perder algum tempo procurando até
    perceber que o outro programador não as colocou.

    Essas diferenças dão-se devido a diversidade de meios de se aprender programação, tanto quanto
    aos gostos, aptidões e objetivos de cada programador. Assim hoje em dia isso torna-se um grande
    problema por que cada vez mais precisamos de mais e mais programadores trabalhem juntos entre
    si, desenvolvendo os mais diversos sistemas computações. Onde este último deve-se ao fato de que
    a complexidade dos sistemas computacionais cresce cada vez mais, portanto requer-se que mais e
    mais programadores trabalhem e compartilhem códigos.

    Então além de nos preocupar-mos somente como o código é exibido na tela do computador, nós
    precisamos nos preocupar sobre como ele será salvo no sistema de arquivos. Já que ao
    compartilhar o código, é vital o uso de um sistema de versionamento para permitir a gerências de
    projetos e os programadores em si, terem o controle de mudanças do código. O que permiti e
    facilmente possa realizar o rastreamento de mudanças e permitir que se possa entender melhor o
    que cada programador está fazendo a cada vez que ele formaliza um mudança no código através de
    uma `commit', como no sistemas `git` por exemplo.

    Isso por que quando trabalhos em um sistema de versionamento como `git' precisamos manter o
    código dentre um único estilo ou boa prática definida como padrão, devido ao fato de que se
    deixar-mos cada programador escrever como ele quiser, teremos muito ruído durante a revisão do
    código e estamos determinando o que o programador fez/escreveu, se cada programador re-escreve o
    histórico fazendo alterações como colocar linhas novas antes de cada if. Assim teremos ruído por
    que o foco de um sistema de versionamento é olhar somente as mudanças que são significativas
    para o código, como a criação de novas funções e não a adição de novas linhas em branco.

    Sobre o último ponto, podemos pensar também sobre uma abordagem da criação de um novo sistema de
    versão que foque somente nas mudanças significativas para o código, durante o momento da
    revisão. Entretanto essa abordagem não é ideal por que, por exemplo, ela dá margem para que
    programadores entrem em guerras tediantes e não produtivas de ajustes de código. Por exemplo,
    imagine o quão seria todo dia que você acorda e começa a trabalhar, você tem que passar pelo
    código colocando linhas novas antes de cada um dos if\textquotesingle s por que o programador do
    turno da noite tinha acabado de remover eles?


\section{Objetivos}

    Estabelecer relações entre boas práticas de programação e eficiência em programar, além de uma
    nova ferramenta ao apoio do programador com o intuito de automatizar o longo e diverso processo
    de programação em equipes de desenvolvedores com distintas boas práticas de programação.


\subsection{Objetivos específicos}

    \begin{enumerate}

        \item

        Um estudo sobre ferramentas universais de programação, que permitam que a partir de um único
        software, seja programado em todas as linguagens de programação. Assim explicar as
        diferenças para os outros softwares e os porquês de querer-se uma ferramenta única, ao invés
        de diversas.

        \item

        Definir, estudar, determinar e classificar o que são boas práticas de programação e realizar
        um estudo aprofundado sobre a as boas práticas da área de disposição visual, conhecidas
        também como `Beautifying'.

        \item

        Um estudo sobre as mais diversas ferramentas existentes para o apoio de boas práticas de
        programação, além de uma análise comparativa entre elas, determinando suas fraquezas e
        pontos fortes.

        \item

        A definição de um padrão de floxo de desenvolvimento que permita equipes de programadores
        com distintas boas práticas de programação, trabalhem em si sem intervir e iniciar guerras
        de boas práticas.

        \item

        Propor uma ferramenta única que permita diversas e distintas boas práticas de programação serem
        implementadas nas mais diversas linguagens de programação e que elas possam ser configuradas
        e definidas ao gosto dos programadores que a usa.

    \end{enumerate}


\section{Método de pesquisa}

    O trabalho será baseado em pesquisas em artigos, livros, teses, dissertações, sites de autores
    confiáveis, e por meio de novas provas demonstradas e baseadas através de argumentos no decorrer
    da evolução da monografia. Também sera apresentado os resultados decorridos da construção de uma
    nova ferramenta que proprõe a solução de um dos problemas apresentados e explicados.

    No último capítulo desta proposta encontra-se no tópico \autoref{sec:implementation} encontra-se
    uma série de links e referências que forma pré-selecionadas e poderão ser utilizadas na
    construção final deste trabalho. Notes que em si, as partes da última seção serão gradativamente
    movida para primeira parte do texto onde encontra-se pesquisa teórica, no decorrer que suas
    informações correlacionadas são incorporadas no trabalho escrito.

    Assim no final da primeira parte desta obra que dará-se no final da conclusão da disciplina
    intitulada de Trabalho de Conclusão de Curso 1, restarão somente as informações destinadas a
    implementação da ferramenta proposta, que serão implementadas na segunda parte da monografia
    denominada \nameref{sec:implementation}, que será desenvolvida no final da conclusão da
    disciplina de Trabalho de Conclusão de Curso 2.
}




    % PARTE
    % \part{Preparação da pesquisa}

    % Capitulo com exemplos de comandos inseridos de arquivo externo
    

% The \phantomsection command is needed to create a link to a place in the document that is not a
% figure, equation, table, section, subsection, chapter, etc.
\cleardoublepage
\phantomsection

% The \phantomsection command is needed to create a link to a place in the document that is not a
% figure, equation, table, section, subsection, chapter, etc.
\addcontentsline{toc}{chapter}{\HyperrefUppercase{Chapter 2}}

\chapter{Capítulo 2}

\newpage


\section{Implementação}
\label{sec:implementation}

    Este trabalho tem como objetivo criar um formatador (software único) de fácil configuração e
    expansão capaz de abranger as linguagens de programação que existem, baseado em um uso
    específico de expressões regulares.

    A metodologia abordada será de não ter a necessidade de ter-se conhecimento da sintaxe das
    linguagens de programação que se irão fazer o parsing. Isso porque trataremos elas como texto
    comum, e será o usuário final que fará a configuração das transformações que serão aplicados no
    texto, dando liberdade de facilmente se configurar várias linguagens de programação (senão
    todas), aproveitando o fato de que muitas deles compartilham estruturas semelhantes senão
    idênticas.

    Como resultado espera-se ter um Beautifier Universal capaz de abranger as linguagens que
    existem, senão que seja facilmente extensível para abrange-las. Os pontos positivos dessa
    abordagem são a reusabilidade de componentes entre as linguagens. Por exemplo, `if/for/while's
    em C++ e Java são da mesma estrutura. Assim temos que escrever somente uma vez a especificação
    para um componente da linguagem.

    A ideia de um software, que em certa extensão pode continuar um ramo do Trabalho de Conclusão de
    Curso do aluno `Lucas Boppre Niehues', orientado do Professor `Olinto José Varela Furtado'
    defendido em 2013/1, com o título: `Estudo e Criação de um Editor de Código Estruturado'. Donde
    durante a leitura de seu TCC, encontra-se o seguinte trecho que faz ligação com uma das
    propostas deste trabalho, no capítulo: `8.1.2 Separação de formato de exibição e de saída':

    \medskip
    \begin{myquote}
    ``As formas que o código é exibido ao usuário e que ele é salvo em disco são controladas
    por arquivos de configuração distintos. O arquivo ``theme.ini'' contém, entre outras
    configurações, informações de como serializar a árvore sintática.''
    \end{myquote}

    \vspace{-5mm}
    ...
    \begin{myquote}
    ``A configuração de formato de saída é dada da mesma forma, mas em um arquivo
    separado, chamado ``output\_format.ini''. A decisão desta separação foi em vista de equipes
    de programadores que queiram utilizar uma convenção única para os arquivos salvos,
    mas manter a exibição a escolha de cada um. Assim os integrantes desta equipe podem
    compartilhar os seus arquivos ``output\_format.ini'' enquanto personalizam o arquivo
    ``theme.ini'' a seu gosto.''
    \end{myquote}

    Com base nisso, pode-se pensar na escrita de plugins para editores de texto/IDEs comuns como
    Sublime Text. Assim ao carregar o arquivo do disco, este plugin chama o formatter e faz a
    formatação de acordo com as configurações de exibição para o usuário. Após isso, quando o
    usuário for salvar o arquivo, o arquivo com a formatação original é devolvido.

    Para auxilar nesse processo, um módulo de autoconfiguração é de grande ajuda. Ele detecta como o
    source code está formatado e cria arquivos de configuração para ele. Assim ao salvar o arquivo,
    automaticamente ele é salvo no formato que ele foi lido. Então temos o mesmo beneficio de
    editores estruturados, como proposto trabalho de `Lucas Boppre Niehues'. De inicio podemos
    pensar com os seguinte objetivo/ideia para um TCC:

    \medskip
    \begin{myquote}
    \begin{enumerate}[nolistsep]
        \item Criar um formatador de fácil configuração e expansão para as linguagens de
              programação que existem e que irão existir.
    \end{enumerate}
    \end{myquote}



\subsection{Problema}

    O problema proposto a se resolver é criar um Beautifier Universal. Os softwares atuais são
    limitados a um conjunto similar, ou mesmo à uma única linguagem, e além de muitos, serem
    limitados ao que eles podem fazer por você ao processar/formatar o código \cite{Terence}.

    To every developer in this world, the closest thing to their heart is the text editor of their
    choice. Over the last few years many new text editors has come into the market in both free and
    paid model, but unfortunately not all of them were able to make a real dent on the developer
    community. I remember in my college days we uses to use Notepad++ as our beloved text editor, as
    at that point of time it was one of the popular and free text editor with a lot of features for
    coding. But as time goes on, the entire development community started to lean towards sublime
    text since it’s launch.
    \url{https://www.isaumya.com/sublime-text-vs-atom-which-one-i-prefer-most-and-why/}

    As a developer, your code editor is one of the most important parts of your setup. It can save
    your wrists and fingers from repetitive strain injuries. It can save your eyes from going blind
    after a coding marathon.
    \url{https://hackernoon.com/virtualstudio-code-the-editor-i-didnt-think-i-needed-16970c8356d5}

    VS Code is an Editor while VS is an IDE.
    \url{https://stackoverflow.com/questions/30527522/what-are-the-differences-between-visual-studio-code-and-visual-studio}

    Implement tabstops with white space align.

    The solution - move tabstops to fit the text between them and align them with matching tabstops
    on adjacent lines.
    \url{http://nickgravgaard.com/elastic-tabstops/}
    \url{https://forum.sublimetext.com/t/elastic-tabs/128}

    What is the difference between VS Code and VS Community?
    Visual Studio Code is a streamlined code editor with support for development operations like
    debugging, task running and version control. It aims to provide just the tools a developer needs
    for a quick code-build-debug cycle and leaves more complex workflows to fuller featured IDEs.
    For more details about the goals of VS Code, see Why VS Code.
    \url{https://code.visualstudio.com/docs/supporting/faq#_licensing}

    Reg Replace is a plugin for Sublime Text 2 that allows the creating of commands consisting of
    sequences of find and replace instructions.
    \url{https://forum.sublimetext.com/t/regreplace-plugin/3810}

    The main reason I moved was that I find that it’s much slower, the simple things like opening a
    new window for a project should be instantaneous and sadly it’s far from it. As I've said before
    it's all about personal preference, I've gone back to Sublime but Adam for example is sticking
    with it...
    \url{http://engageinteractive.co.uk/blog/atom-review}

    Logo abaixo há algumas regras de formatação básica encontrados no serviço online
    \url{http://prettyprinter.de/} acessado em março/2017:

    \medskip
    \begin{myquote}
    \begin{enumerate}[nolistsep]
        \item Add new lines after ``\{'' and before ``\}''
        \item Add new lines before ``\{''
        \item Remove empty lines
        \item Add comment lines before function
        \item Add new lines after ``;''
        \item Add new lines after ``\}''
        \item Remove new lines
        \item Reduce whitespace
        \item Put the code again in the input box above after submit
    \end{enumerate}
    \end{myquote}

    A partir deste ponto, apresenta-se um esboço sobre o problema, soluções, informações como
    porquês de se querer fazer um software assim, ou ainda de querer-se o beautifying:

    \begin{enumerate}[leftmargin=*]

        \item

        Motivação: Existem muitas ferramentas distintas, por vezes pagas, e dificilmente completas
        \cite{Terence}.

        \item

        Muitas linguagens de programação existem, assim sempre ter fazer um software Beautifier para
        cada uma delas é muito trabalhoso \cite{Terence}. Mas a abordagem para um Beautifier
        Universal proposta nesse trabalho, permite que facilmente novas linguagens sejam
        adicionadas, sendo elas completamente diferentes das anteriores, ou similares. No caso de
        similaridades, basta reutilizar as estruturas de configuração das linguagens já existentes.

        \item

        Preocupa-se de fazer um Beautifier para cada uma delas por que programadores atualmente
        trabalham diariamente com varias dessas linguagens, e elas não são similares. Assim precisa-
        se configurar vários beautifiers para fazer a formatação. Isso é um problema por que,
        somente alguns beautifiers são mais completos, e toda vez que precisa-se fazer uma alteração
        no estilo de formatação, precisa-se propagar manualmente a mesma mudança ao longo de vários
        arquivos de configuração de programas distintos, o que é ruim pois toma ao usuário muito
        tempo de aprender a lidar com várias e muito diferentes tipos de configurações
        \cite{Schweitzer}.

        \item

        No caso do Beautifier que propõem-se, uma mudança no estilo é propagada para todas as
        linguagens. E caso queira-se deixar alguma linguagem fora da regra, basta remover ela da
        lista ao qual esse bloco da configuração se aplica, e `a)' deixar ela de fora assim nenhuma
        mudança é aplicada a ela. Ou `b)' criar um novo bloco que inclua somente ela com a
        configuração desejada.

        \item

        A seguir, temos algumas frases sobre o assunto:

        \begin{myquote}
        % \setlength{\itemindent}{5pt}
        ``One of absolute worst, worst methods of teamicide for software developers is to engage
        in these kinds of passive-aggressive formatting wars. I know because I've been there.
        They destroy peer relationships, and depending on the type of formatting, can also damage
        your ability to effectively compare revisions in source control, which is really scary.
        I can't even imagine how bad it would get if the lead was guilty of this behavior. That's
        leading by example, all right. Bad example.'', \cite{Atwood}.
        \end{myquote}
        \vspace{-5mm}
        ...
        \begin{myquote}
        ``So yes, absurd as it may sound, fighting over whitespace characters and other seemingly
        trivial issues of code layout is actually justified. Within reason of course -- when done
        openly, in a fair and concensus building way, and without stabbing your teammates in the
        face along the way.'', \cite{Atwood}.
        \end{myquote}

        \begin{myquote}``
        I'd say there are two main reasons to enforce a single code format in a project. First has
        to do with version control: with everybody formatting the code identically, all changes in
        the files are guaranteed to be meaningful. No more just adding or removing a space here or
        there, let alone reformatting an entire file as a `side effect' of actually changing just a
        line or two.'', \cite{Geukens}.
        \end{myquote}

    \end{enumerate}



\subsection{Objetivos}

    O objeto neste trabalho de TCC proposto aqui não é inicialmente suportar todas as regras de
    formatação de todas as linguagens de programação, mas a criação de uma estrutura básica inicial
    e robusta que sejam capaz de ser desenvolvida a ponto de ser facilmente expandida, tanto na
    adição de novos módulos de processamento no programa escrito, tanto pelo usuário final na
    escrita dos arquivos de programação.

    A teoria da técnica empregada é muito simples, mas diferente das atuais por que é atribuído ao
    usuário final a responsabilidade de dizer onde será realizado o beautifying do modulo que está
    se configurando. Esse é o preço a pagar para permitir a criação de um Beautifier Universal.
    Quando diz-se fácil configuração, refire-se a não necessidade de recorrer a programação ´C++',
    i.e., alterar o código fonte do programa para permitir/especificar onde devem ser realizadas as
    alterações de beautifying.


\subsubsection{Objetivos Gerais}

    \begin{enumerate}[leftmargin=*]

        \item

        Escrever o programa em C++ ou afins, para permitir também que a formação/beautifying seja
        (em trabalhos futuros/talvez nesse) dinâmico, isto é, na medida que você digita o texto, ele
        é formatado para você. Assim você pode focar mais em escrever o código, ao invés que se
        preocupar com o espaçamento, alinhamento, parenteses, linhas novas, e o que mais que seja.

        \item

        Utilizar o Framework `doctest` para escrita dos Testes de Unidade. Pois após procurar e
        testar alguns frameworks para testes de unidade em C++, entrou-se este como servindo muito
        bem as requisitos do projecto. Ele causa baixíssimo incremento no tempo de compilação e
        permite que os testes possam ser escritos no mesmo arquivo onde encontram-se o código do
        programa, sem que eles sejam compilados.

        \item

        Utilizar uma versão/algoritmo multi-core, então cada uma das regras pode ser processada em
        paralelo e sobre o mesmo source code original. Essa parte é bastante complexa de ser escrita
        por que as regras entre si podem gerar conflitos sobre o que elas estão fazendo. Para
        resolver esse problema, fazer com que cada regra processada gere um objeto de mudanças que
        essa regra está propondo. No final do processamento de todas as regras, será realizado um
        fusão das mudanças que cada uma decidiu realizer, e caso duas regras queriam mudar o mesmo
        pedaço/trecho de código, será lançada um exceção e uma nova classe de mudanças/regra deve
        estar disponível para resolver esse conflito. Caso não exista, ambas as mudanças são
        descartadas e somente as mudanças sem conflitos são refletidas no código.

    \end{enumerate}


\subsubsection{Objetivos Específicos}


    \begin{enumerate}[leftmargin=*]

        \item

        Um Produto de Software com uma ótima orientação a objetos e possibilidades de extensão das
        funcionalidades.

        \item

        Classificar todas classes e tipos de formatações (beautifying) de código aplicáveis com
        facilidade. Uma das partes a serem escritas e entregues na monografia. Um estudo sobre o que
        é beautifying, como fazer e por que fazer.

        \item

        Implementação de um núcleo funcional e de uma pesquisa decente sobre o estado da arte. Um
        dos pontos difíceis seria a marcação dos escopos, mas isso já é implementado pelo núcleo do
        editor Sublime Text, assim provado como possível de ser feito.

        \item

        Inicialmente devido a limitação de tempo em 1 ano e meio para um TCC, podemos pensar somente
        um núcleo básico, simples, reutilizável e que talvez possa ajudar no contexto da linguagem
        que vocês desenvolvem.

    \end{enumerate}


\subsubsection{Trabalhos Futuros}

    O número de recursos/funcionalidades e estratégias de otimizações para serem implementadas, e
    etc, são imensas. Mas esses trabalhos podem ser muito mais para frente depois da entrega do TCC.
    Hoje o controle de espaços em chamadas de funções, declarações de classes, comentários e etc,
    são mais tranquilos de se entender e pensar. Entretanto no requisito e ajuste de indentação,
    inserção/remoção de parenteses redundantes, etc ainda falta estudo sobre como deve ser
    implementado isso.

    Contudo essa especificação por parte do usuário é limitado a linguagens Livres de Contexto
    (máquinas de pilha). Assim caso as especificações de escope precisarem ser feitas em termos de
    linguagens Sensíveis ao Contexto ou ainda Recursivamente Enumeráveis, vai ser preciso tratar
    esses elementos diretamente em C++ (máquina de turing).

    Entretanto não consegue-se pensar facilmente em casos em que precise mais do que tratadores
    Livres de Contexto para realizar a especificação de quais partes do código deve ser necessário
    formatar. Sublime Text faz uso dessa técnica para o Highlight dos códigos das mais diversas
    linguagens e acredita-se que tenha um bom resultado.





\subsection{Método de pesquisa}

    A vantagem nesta abordagem é não ter a necessidade de ter-se conhecimento da sintaxe das
    linguagens de programação que se irão fazer o parsing. Isso porque trataremos elas como texto
    comum, e será o usuário final que fará a configuração das transformações que serão aplicados no
    texto, dando liberdade de facilmente se configurar várias linguagens de programação,
    aproveitando o fato de que muitas deles compartilham estruturas semelhantes senão idênticas.

    A literatura/programas atuais são dependentes de linguagem de programação. Minha proposta é
    fazer este processo independente de linguagem, mas de dialetos como este exemplo tirado do PDF
    em anexo a este e-mail `Initial check list tasks to do.pdf':

    \begin{lstlisting}
    // This is the name used to reference this scope around the settings files.
    Scope Name:
    %c++_like_block_comment

    // This set on which languages this block should be included. Setting it
    // to empty will allow it to be parsed for any languages.
    Language Inclusion:
    Java, C++, Pawn

    // Defines a expression which will map the beginning of a exclusion block.
    Scope Start:
    /\*\*

    // Defines a expression which will map the ending of a exclusion block.
    Scope End:
    \\\*
    \end{lstlisting}
    \vspace*{-4mm}

    A abordagem acima é uma abordagem ingênua, portanto somente brevemente ilustrativa. O real motor
    para o software é baseado em expressões regulares e um pilha de contextos. Esta ideia foi
    inicialmente desenvolvida pelo editor de texto `Sublime Text' \cite{Skinner}. Este editor
    utiliza essa estrutura de blocos para fazer a sintaxe highlighting do códigos das linguagens
    através de expressões regulares alocação de contextos/escopos. Essa mesma abordagem pode ser
    utilizada pelo usuário para definir em quais regiões uma Máquina de Turing (linguagens C++/Rust)
    devem fazer/propor as alterações no código.


\subsubsection{Pontos}

    Os pontos positivos dessa abordagem para um formatador de código são a reusabilidade de
    componentes entre as linguagens pelo usuário final da aplicação ao invés do programador, o que
    torna este software muito mais genérico e abre a possibilidades de maior sucesso para a criação
    definitiva de um formatador Universal de códigos das linguagens de programação, quaisquer sejam
    elas. Por exemplo, `if/for/while'\textquotesingle s em linguagens de programação como C++ e Java
    são da mesma estrutura. Assim temos que escrever somente uma vez a especificação para um
    componente da linguagem sem recorrer a programação de do código do programa. Isso tem a vantagem
    de por der ser configurado pelo usuário final ao invés do programador, assim fica mais simples
    de configurar e expandir o conjunto de linguagens disponíveis ao processamento/beautifying.

    Softwares existentes e similares:

    \medskip
    \begin{myquote}
    \begin{enumerate}[leftmargin=*]

        \item

        CodeBeautify is an online code beautifier which allows you to beautify your source code:
        \url{http://codebeautify.org/}.

        \item

        A universal code formatter, written in Dart: \url{https://pub.dartlang.org/packages/unifmt}.

        \item

        Google-java-format is a program that reformats Java source code to comply with Google Java
        Style: \url{https://github.com/google/google-java-format}.

        \item

        CodeFormatter is a Sublime Text 2/3 plugin that supports format (beautify) source code.
        \url{https://github.com/akalongman/sublimetext-codeformatter} and
        \url{https://github.com/aukaost/SublimePrettyYAML}

        \item

        UniversalIndentGUI offers a live preview for setting the parameters of nearly any indenter.
        You change the value of a parameter and directly see how your reformatted code will look
        like. Save your beauty looking code or create an anywhere usable batch/shell script to
        reformat whole directories or just one file even out of the editor of your choice that
        supports external tool calls: \url{http://universalindent.sourceforge.net/} and
        \url{https://github.com/danblakemore/universal-indent-gui}.

    \end{enumerate}
    \end{myquote}


\subsubsection{Listagens}

    Algumas bibliotecas existentes, e potencialmente utilizadas como `syntect` para o auxílio na
    construção do produto de software:

    \begin{myquote}
    \begin{enumerate}[leftmargin=*,parsep=0pt]

        \item \url{https://github.com/jbeder/yaml-cpp}
        \item \url{https://github.com/trishume/syntect}
        \item \url{https://github.com/onqtam/doctest}
        \item \url{https://github.com/c42f/tinyformat}
        \item \url{https://github.com/limetext/lime}
        \item \url{https://forum.sublimetext.com/t/disassembling-sublime-text/24824}

    \end{enumerate}
    \end{myquote}

    Segue-se uma lista básica de formatters/beautifiers acessado no endereço
    \lword{\url{http://www.softpanorama.org/Utilities/beautifiers.shtml}} em março/2017:

    \medskip
    \begin{sloppypar}
    \begin{myquote}\RaggedRight
    \begin{enumerate}[leftmargin=*,parsep=0pt]

        \item CB210.ZIP - C Beautifier 2.10 - polish C source code (19,406 bytes, 06/22/92)
        \item CL121.ZIP - Codelister 1.21 - print C code with stats (51,110 bytes, 01/10/94)

        \item CPC200.ZIP - CodePrint for C/C++ 2.00 is a full-featured command line driven source
        code reformatter and pretty printer for C++ and C; over 20 customization features including
        auto-indent, adjustable tab spacing, indent styles, flow lines, comment alignment, and line
        editing for consistent white space (140,605 bytes, 01/26/96)

        \item CSCOP120.ZIP - c-scope 1.20 analyzes C source code and produces various reports
        (48,505 bytes, 06/30/95)

        \item HTML : \url{http://www.digital-mines.com/htb/}
        \item HTML : \url{http://www.datacomm.ch/mwoog/software/perl/beautifier.html}
        \item HTML : \url{http://www.watson-net.com/free/perl/s_fhtml.asp}
        \item SQL : \url{http://www.netbula.com/products/sqlb}
        \item Oracle PLSQL : \url{http://www.revealnet.com}
        \item GPL \url{http://www.geocities.com/~starkville/vancbj.html}
        \item GPL \url{http://kevinkelley.mystarband.net/java/dent.html}
        \item Free \url{http://www.tiobe.com/jacobe.htm}
        \item Free \url{http://www.mmsindia.com/JPretty.html}
        \item Free \url{http://members.magnet.at/johann.langhofer/products/jxbeauty/overview.html} (has JBuilder support)
        \item Free \url{http://www.semdesigns.com/Products/Formatters/JavaFormatter.html}
        \item Commercial \$24.99 \url{http://smartbeautify.com}
        \item Commercial \$129 \url{http://www.jindent.com}
        \item Google \url{http://directory.google.com/Top/Computers/Programming/Languages/Java/Development_Tools/Code_Beautifiers/?tc=1}
        \item Java, SQL, HTML, C++ : \url{http://www.semdesigns.com/Products/DMS/DMSToolkit.html}
        \item Java JIndent \url{http://home.wtal.de/software-solutions/jindent}
        \item Java Pat \url{http://javaregex.com/cgi-bin/pat/jbeaut.asp}
        \item Java JStyle \url{http://www.redrival.com/greenrd/java/jstyle}
        \item Java JPrettyPrinter \url{http://www.epoch.com.tw/download/ms/java/java.htm}
        \item Java JxBeauty \url{http://members.nextra.at/johann.langhofer/download/jxbeauty} and the JxBeauty Home
        \item Java beautify percolator
        \item Java list \url{http://www.java.about.com/compute/java/library/weekly/aa102499.htm}
        \item Java html present VasJava2HTML
        \item Java code colorifier and beautifier \url{http://www.mycgiserver.com/~lisali/jccb}
        \item Perl : \url{http://www.consultix-inc.com/www.consultix-inc.com/talk.htm}
        \item Perl : \url{http://www.consultix-inc.com/www.consultix-inc.com/perl_beautifier.html}
        \item Fortran beautifier : \url{http://www.aeem.iastate.edu/Fortran/tools.html}

        \item C++ : BCPP site is at \url{http://dickey.his.com/bcpp/bcpp.html} or at \url{http://www.clark.net/pub/dickey}.
        BCPP ftp site is at \url{ftp://dickey.his.com/bcpp/bcpp.tar.gz}

        \item C++ : \url{http://www.consultix-inc.com/c++b.html}
        \item C : \url{http://www.chips.navy.mil/oasys/c/} and mirror at Oasys
        \item C++, C, Java, Oracle Pro-C Beautifier \url{http://www.geocities.com/~starkville/main.html}

        \item C++, C beautifier \url{http://users.erols.com/astronaut/vim/ccb-1.07.tar.gz} and site at
        \url{http://users.erols.com/astronaut/vim/#vimlinks_src}

        \item GC! GreatCode! is a powerful C/C++ source code beautifier Windows 95/98/NT/2000
        \url{http://perso.club-internet.fr/cbeaudet}

        \item C++ beautifier `SourceStyler' \url{https://web.archive.org/web/20061205061102/http://ochresoftware.com/}
        \item JavaScript : \url{http://jsbeautifier.org/}

    \end{enumerate}
    \end{myquote}
    \end{sloppypar}


\subsubsection{Trabalhos Correlatos}

    Após a busca do que há de publicações científicas sobre o assunto e entra-se alguns trabalhos na
    área específica e similar aos trabalhos feitos pelor formatadores de códigos (Beautifiers).
    Nessa modalidade de trabalho, pode-se confundir-se com artigos que tratam sobre o `Prettyprint`,
    que trata-se de colorir o texto e exibir-lo ao usuário. O que não é o que se busca nesse
    trabalho, mas sim fazer alterações no texto sobre a forma como ele é estruturado, apresentado ao
    usuário e salvo em disco. Seguem as seguintes publicações:

    % How to add `parsep` to `itemsep` and set `parsep` to 0pt, when declaring my list?
    % https://tex.stackexchange.com/questions/366904/how-to-add-parsep-to-itemsep-and-set-parsep-to-0pt-when-declaring-my-list
    \begin{sloppypar}
    \begin{myquote}\RaggedRight
    \begin{enumerate}[leftmargin=*,parsep=0pt]

    \item \url{https://www.researchgate.net/publication/228540036_An_industrial_application_of_context-sensitive_formatting}

    \item \url{http://www.suodenjoki.dk/us/archive/2010/cpp-checkstyle.htm}

    \item \url{http://www.basicinputoutput.com/2014/08/uncrustify-your-bios.html}

    \item \url{http://prettyprinter.de/}

    \item \url{https://github.com/ryanmaxwell/UncrustifyX}

    \item \url{http://www.softpanorama.org/Utilities/beautifiers.shtml}

    \item Understanding the Syntax Parsing
    \url{https://forum.sublimetext.com/t/understanding-the-syntax-parsing/28569}

    "So, part of what I've been working on is a code beautifier that, more or less, aligns and
    indents the code properly based on scanning through the source document."
    ...
    "It hasn't escaped my notice that this is to some degree exactly what the syntax file is doing."

    \item

    {\bfseries Towards a universal code formatter through machine learning:}
    In this paper, we solve the formatter construction problem using a novel approach, one that
    automatically derives formatters for any given language without intervention from a language
    expert. We introduce a code formatter called CODEBUFF that uses machine learning to abstract
    formatting rules from a representative corpus, using a carefully designed feature set. Our
    experiments on Java, SQL, and ANTLR grammars show that CODEBUFF is efficient, has excellent
    accuracy, and is grammar invariant for a given language. It also generalizes to a 4th language
    tested during manuscript preparation.
    \begin{enumerate}[nolistsep,topsep=0pt,label=$\star$]
        \item \url{http://dl.acm.org/citation.cfm?id=2997383}
        \item \url{http://homepages.cwi.nl/~jurgenv/papers/SLE16.pdf}
    \end{enumerate}

    \item \url{https://www.google.com/search?q=universal+source+code+formatter}
    \begin{enumerate}[nolistsep,topsep=0pt,label=$\star$]
        \item \url{https://www.google.com/search?q=universal+source+code+beautifier}
    \end{enumerate}

    \item \url{http://en.wikipedia.org/wiki/Indent_style}
    \begin{enumerate}[nolistsep,topsep=0pt,label=$\star$]
        \item \url{https://en.wikipedia.org/wiki/Programming_style}
        \item \url{https://en.wikipedia.org/wiki/Scope_(computer_science)}
    \end{enumerate}

    \item \url{http://wiki.c2.com/?CodingStyle}
    \begin{enumerate}[nolistsep,topsep=0pt,label=$\star$]
        \item \url{https://github.com/google/code-prettify}
        \item \url{https://github.com/uncrustify/uncrustify}
    \end{enumerate}

    \item \url{https://en.wikipedia.org/wiki/Prettyprint}
    \begin{enumerate}[nolistsep,topsep=0pt,label=$\star$]
        \item \url{https://www.researchgate.net/search.Search.html?query=formatting%20source%20code&type=publication}
        \item \url{https://www.researchgate.net/search.Search.html?query=pretty%20print%20source%20code&type=publication}
    \end{enumerate}

    \item \url{https://github.com/gchpaco/gopprint}
    \begin{enumerate}[nolistsep,topsep=0pt,label=$\star$]
        \item \url{http://dl.acm.org.sci-hub.io/citation.cfm?id=357115}
        \item \url{https://www.cs.indiana.edu/~sabry/papers/yield-pp.pdf}
    \end{enumerate}

    \item \url{http://www.worldcat.org/title/beautiful-code-a-customizable-code-beautifier-for-java/oclc/56564674}
    \begin{enumerate}[nolistsep,topsep=0pt,label=$\star$]
        \item \url{https://www.researchgate.net/publication/34736049_Beautiful_code_a_customizable_code_beautifier_for_Java}
        \item \url{https://vufind.carli.illinois.edu/vf-ncc/Record/ncc_118189/Holdings}
    \end{enumerate}

    \item \url{https://www.researchgate.net/publication/4283921_Smart_Formatter_Learning_Coding_Style_from_Existing_Source_Code}
    \begin{enumerate}[nolistsep,topsep=0pt,label=$\star$]
        \item \url{http://www.ing.unisannio.it/mdipenta/index.html}
        \item \url{https://github.com/iain/rspec-smart-formatter}
    \end{enumerate}

    \item \url{https://www.researchgate.net/publication/2543984_Source_Code_Files_as_Structured_Documents}
    \begin{enumerate}[nolistsep,topsep=0pt,label=$\star$]
        \item \url{https://en.wikipedia.org/wiki/SrcML}
    \end{enumerate}

    \item \url{https://www.researchgate.net/publication/228540036_An_industrial_application_of_context-sensitive_formatting}
    \begin{enumerate}[nolistsep,topsep=0pt,label=$\star$]
        \item \url{https://www.researchgate.net/publication/234809222_Program_indentation_and_comprehensibility}
    \end{enumerate}

    \end{enumerate}
    \end{myquote}
    \end{sloppypar}


\subsubsection{Obfuscators}

    Aqui encontra-se o lado oposto dessas ferramentas, Source Code Obfuscators, que servem para
    destruir o visual do código. Usualmente utilizado para dificultar a leitura por outras pessoas
    ou ainda reduzir o tamanho de códigos de linguagens scripting que devem ser carregadas/baixadas
    por navegadores de internet, assim diminuindo o tráfego de internet e salvando/economizando
    largura de banda para download:

    \begin{sloppypar}
    \begin{myquote}\RaggedRight
    \begin{enumerate}[leftmargin=*,parsep=0pt]

    \item \url{https://en.wikipedia.org/wiki/Obfuscation_(software)}

    \item \url{http://www.semdesigns.com/Products/Obfuscators/index.html}

    \end{enumerate}
    \end{myquote}
    \end{sloppypar}



    % PARTE
    % \part{Referenciais teóricos}

    % Capitulo de revisão de literatura
    % \include{chapters/chapter_3}

    % PARTE
    % \part{Resultados}

    % Primeiro capitulo de Resultados
    % \include{chapters/chapter_4}

    % Segundo capitulo de Resultados
    %\include{chapters/chapter_5}

    % Finaliza a parte no bookmark do PDF
    % para que se inicie o bookmark na raiz
    % e adiciona espaço de parte no Sumário
    \phantompart

    % Conclusão (outro exemplo de capítulo sem numeração e presente no sumário)
    

\chapter{\lang{Conclusion}{Conclusão}}
\label{chapter:conclusion}

\lang{%
    The difference from this proposal to remaining formatting tools,
    is the tradeoff between end\hyp{}users and developers responsibilities.
    Most tools rarely expose to end\hyp{}users their language syntax specification,
    in contrast,
    this proposal completely exposes the language to the end\hyp{}user as simple plain\hyp{}text,
    not requiring the tool to know any language syntax neither semantics.
    Moreover,
    with no syntax knowledge required,
    the tool be can used with any languages their user wishes to.

\begin{enumerate}[leftmargin=*]
    \item
        There are many different tools, sometimes paid, and difficult to
        complete. \cite{universalCodeFormatter};
    \item
        Many programming languages exist, so always having Beautifier
        software for each of them is very laborious
        \cite{universalCodeFormatter}. But the approach to a Universal
        Beautifier proposed in this work, would allow easily new languages to be
        added, being completely different from previous ones, or alike. And in
        case of similarities between them, it is enough to reuse their
        configuration structures already implemented;
    \item
        Looking for a Beautifier for each one of them because programmers
        currently work daily with several of these languages, and they are not
        similar. So you need to configure several beautifiers to do the
        formatting. This is a problem because only a few beautifiers are more
        complete, and every time you need to make a change in the formatting
        style, you must manually propagate the same change over several
        different program configuration files, which is bad because it takes the
        user a lot of time to learn how to handle many different types of
        settings \cite{universalIndentGUI};
    \item
        In the case of ideal Beautifier, a change in your styling is
        propagated to all languages. And if you want to leave some language out
        of it, you just need to remove it from the list on which the
        configuration block applies to, and `a)' leave it out so no change is
        applied to. Or `b)' create a new block including only the block within
        the desired settings.
\end{enumerate}
}{%
    Como esta ferramenta difere das demais já existentes?
    A diferença desta nova proposta de ferramenta de formatação de código~=fonte para as demais é a troca de responsabilidades entre usuários finais da ferramenta e
    os desenvolvedores desta ferramenta e
    das gramáticas para os usuários finais.
    Formatadores de código~=fonte usualmente expõem as mudanças que podem ocorrer ao formatar o código~=fonte sem quebrar a suas regras sintáticas ou
    semânticas.

    A maior parte das ferramentas raramente permite que usuários finais tenham o controle total das mudanças no código~=fonte.
    Enquanto utilizando a ferramenta desenvolvida neste trabalho,
    é possível escrever regras de formatação que quebrem a sintaxe e
    semântica da linguagem sendo formatada.
    Por exemplo,
    na linguagem ``Go'' \cite{programmingLanguageGolang},
    diferentemente de todas as demais linguagens,
    é um erro de sintaxe adicionar a chave ``\{'' de abertura de bloco em uma linha nova.

    Utilizando~=se as ferramentas usuais de formatação,
    fica impedido que configurações do usuário quebre o código~=fonte da linguagem sendo formatada,
    a não ser em casos de \textit{bugs} na ferramenta.
    As ferramentas em geral tentam reconstruir a árvore de sintaxe da linguagem a ser formatada.
    Neste trabalho,
    o usuário final pode escolher entre simplesmente especificar a gramática da linguagem com o mínimo necessário para atingir somente as suas necessidades.
    Mas ao mesmo tempo,
    ele também pode realizar a especificação completa de toda a sintaxe da sua linguagem.

    Mesmo com a especificação completa da sintaxe da linguagem,
    ainda não será o suficiente para impedir quebras nos códigos~=fonte sendo formatados,
    pois a sintaxe da linguagem não cobre os seus aspectos semânticos.
    Neste caso,
    o usuário final precisará conhecer quais são as regras semânticas da linguagem na qual ele está realizando a formatação,
    e configurar o formatador para que ele não quebre nenhuma das regras semânticas da linguagem.
}


\section{\lang{Future Works}{Trabalhos Futuros}}

O autor desse trabalho sugere alguns trabalhos futuros.
Antes de novos formatadores de código~=fonte sejam implementados,
estes pontos precisam ser revistos.
Ao realizar estas mudanças,
qualquer implementação já realizada para formatação será perdida devido ao número de mudanças realizadas,
como em um efeito borboleta \cite{standardButterflyEffect}.
Mais especificamente,
sobre a metalinguagem e
gerador de formatadores criados,
sugerem~=se as seguintes alterações:
\begin{enumerate}
\item Adicionar quaisquer regras semânticas da metalinguagem que não foram adicionadas no analisador semântico;
\item Reduzir o uso de memória e
otimizar a performance em tempo de execução,
uma vez que os algoritmos e
estratégias adotas não levaram estes pontos em consideração;
\item Corrigir erros de interpretação da metalinguagem ou
da especificação da metagramática quando alguns operadores como ``scope'' que tem um uso opcional,
são omitidos;
\item Implementar operadores como ``captures'' e
``set'' para tornar o uso da metalinguagem mais fácil ou
melhorar a sua performance em casos de uso específicos;
\item Adicionar suporte à especificação de múltiplos escopos a um mesmo trecho de código \cite{vsCodeSyntaxHighlighthing},
definindo alguma estrutura de dados adequada,
capaz de permitir consultas e
aritméticas de escopos \cite{textMateScopeExclusion} com performance constante $\Theta(1)$;
\item Melhorar a legibilidade e
facilitar a escrita das gramáticas,
removendo a necessidade de chaves de abertura ``\{'' e
fechamento ``\}'' de blocos,
fazendo a separação de blocos ser feita de acordo com a indentação como em linguagens como Python e
YAML.
\end{enumerate}%

A atual estrutura de composição dos formatadores de código~=fonte não suporta que diversos formatadores realizem a formatação simultaneamente.
Somente um formatador,
que estende da classe ``AbstractFormatter'' pode estar em funcionamento ao mesmo tempo.
Entretanto,
este formatador recebe como parâmetro de sua função ``format\_text'',
o trecho de código~=fonte a ser formatado e
o escopo dele.
Com essas informações ele poderia em tese,
chamar diferentes formatadores mais especializados de acordo com o seu parâmetro escopo.

Em relação aos formatadores gerados pelo gerador de formatadores,
eles precisaram ser completamente reescritos,
uma vez que as melhorias feitas no gerador de formatadores e
metalinguagem linguagem forem concluídas.
A implementação atual dos formatadores pode ser considerada nula,
uma vez que toda sua lógica de funcionamento é construída como um simples iteração descendente pela Árvore de Sintaxe Abstrata gerada pelo analisador semântico.

Em \citeonline{objectBeautifierFutureWorks},
pode ser encontrada a concepção inicial de implementação da ferramenta de formatação de código~=fonte deste trabalho.
Uma vez que este trabalho for publicado,
seu código~=fonte estará disponível.
Então,
seguindo as referências de \citeonline{objectBeautifierFutureWorks} será possível encontrar a especificação completa sobre última versão implementação de formatadores de código~=fonte com os requerimentos deste trabalho.





    % ELEMENTOS PÓS-TEXTUAIS
    %
    \postextual
    \setlength\beforechapskip{0pt}
    \setlength\midchapskip{15pt}
    \setlength\afterchapskip{15pt}

    % Referências bibliográficas
    \bibliography{refs}

    % Glossário, consulte o manual da classe abntex2 para orientações sobre o glossário.
    % \glossary

    % Apêndices, inicia os apêndices
    \begin{apendicesenv}

        % Imprime uma página indicando o início dos apêndices
        \partapendices

        \setlength\beforechapskip{50pt}
        \setlength\midchapskip{20pt}
        \setlength\afterchapskip{20pt}

        


%
% How to fix the Underfull \vbox badness has occurred while \output is active on my memoir chapter style?
% https://tex.stackexchange.com/questions/387881/how-to-fix-the-underfull-vbox-badness-has-occurred-while-output-is-active-on-m
%

% ---

\lang
{\chapter[Appendix A]{Since this page is not being completely filled, it is generating the bottom bottom of the page}}
{\chapter[Apêndice A]{Como esta página não está sendo completamente preenchida, ele está gerando a caixa inferior inferior da página}}
% ---


% Multiple-language document - babel - selectlanguage vs begin/end{otherlanguage}
% https://tex.stackexchange.com/questions/36526/multiple-language-document-babel-selectlanguage-vs-begin-endotherlanguage
\begin{otherlanguage*}{english}

\showfont

1. How to display the font size in use in the final output,
2. How to display the font size in use in the final output,
3. How to display the font size in use in the final output,
4. How to display the font size in use in the final output,
5. How to display the font size in use in the final output,
6. How to display the font size in use in the final output,
7. How to display the font size in use in the final output,
8. How to display the font size in use in the final output,
9. How to display the font size in use in the final output,


% As this page is not being completely filled, it is generating the page bottom bad box.
% Fix Underfull \vbox (badness 10000) has occurred while \output is active
%
% \flushbottom vs \raggedbottom
% https://tex.stackexchange.com/questions/65355/flushbottom-vs-raggedbottom
\newpage



\section[Some encoding tests]{\showfont}

1. How to display the font size in use in the final output,
2. How to display the font size in use in the final output,
3. How to display the font size in use in the final output,
4. How to display the font size in use in the final output,
5. How to display the font size in use in the final output,
6. How to display the font size in use in the final output,

7. How to display the font size in use in the final output,
8. How to display the font size in use in the final output,
9. How to display the font size in use in the final output,
10. How to display the font size in use in the final output,
11. How to display the font size in use in the final output,
12. How to display the font size in use in the final output,

\subsection{\showfont}

1. How to display the font size in use in the final output,
2. How to display the font size in use in the final output,
3. How to display the font size in use in the final output,
4. How to display the font size in use in the final output,
5. How to display the font size in use in the final output,
6. How to display the font size in use in the final output,

7. How to display the font size in use in the final output,
8. How to display the font size in use in the final output,
9. How to display the font size in use in the final output,
10. How to display the font size in use in the final output,
11. How to display the font size in use in the final output,
12. How to display the font size in use in the final output,

\subsubsection{\showfont}

1. How to display the font size in use in the final output,
2. How to display the font size in use in the final output,
3. How to display the font size in use in the final output,
4. How to display the font size in use in the final output,
5. How to display the font size in use in the final output,
6. How to display the font size in use in the final output,

7. How to display the font size in use in the final output,
8. How to display the font size in use in the final output,
9. How to display the font size in use in the final output,
10. How to display the font size in use in the final output,
11. How to display the font size in use in the final output,
12. How to display the font size in use in the final output,


Lipsum me [31-35]

\end{otherlanguage*}




    \end{apendicesenv}

    % Anexos, inicia os anexos
    \begin{anexosenv}

        % Imprime uma página indicando o início dos anexos
        \partanexos

        \setlength\beforechapskip{50pt}
        \setlength\midchapskip{20pt}
        \setlength\afterchapskip{20pt}

        


%
% How to fix the Underfull \vbox badness has occurred while \output is active on my memoir chapter style?
% https://tex.stackexchange.com/questions/387881/how-to-fix-the-underfull-vbox-badness-has-occurred-while-output-is-active-on-m
%

% ----------------------------------------------------------
\chooselang
{\chapter[Sample example]{How to display the font size in use in the final output}}
{\chapter[Anexo exemplo]{Como exibir o tamanho da fonte em uso na saída final}}
% ----------------------------------------------------------


% Multiple-language document - babel - selectlanguage vs begin/end{otherlanguage}
% https://tex.stackexchange.com/questions/36526/multiple-language-document-babel-selectlanguage-vs-begin-endotherlanguage
\begin{otherlanguage*}{english}

\showfont

1. How to display the font size in use in the final output,
2. How to display the font size in use in the final output,
3. How to display the font size in use in the final output,


\section[Some encoding tests]{\showfont}

1. How to display the font size in use in the final output,
2. How to display the font size in use in the final output,
3. How to display the font size in use in the final output,
4. How to display the font size in use in the final output,
5. How to display the font size in use in the final output,
6. How to display the font size in use in the final output,

7. How to display the font size in use in the final output,
8. How to display the font size in use in the final output,
9. How to display the font size in use in the final output,
10. How to display the font size in use in the final output,
11. How to display the font size in use in the final output,
12. How to display the font size in use in the final output,

\subsection{\showfont}

1. How to display the font size in use in the final output,
2. How to display the font size in use in the final output,
3. How to display the font size in use in the final output,
4. How to display the font size in use in the final output,
5. How to display the font size in use in the final output,
6. How to display the font size in use in the final output,

7. How to display the font size in use in the final output,
8. How to display the font size in use in the final output,
9. How to display the font size in use in the final output,
10. How to display the font size in use in the final output,
11. How to display the font size in use in the final output,
12. How to display the font size in use in the final output,

\subsubsection{\showfont}

1. How to display the font size in use in the final output,
2. How to display the font size in use in the final output,
3. How to display the font size in use in the final output,
4. How to display the font size in use in the final output,
5. How to display the font size in use in the final output,
6. How to display the font size in use in the final output,

7. How to display the font size in use in the final output,
8. How to display the font size in use in the final output,
9. How to display the font size in use in the final output,
10. How to display the font size in use in the final output,
11. How to display the font size in use in the final output,
12. How to display the font size in use in the final output,

\subsubsubsection{\showfont}

1. How to display the font size in use in the final output,
2. How to display the font size in use in the final output,
3. How to display the font size in use in the final output,
4. How to display the font size in use in the final output,
5. How to display the font size in use in the final output,
6. How to display the font size in use in the final output,
7. How to display the font size in use in the final output,

8. How to display the font size in use in the final output,
9. How to display the font size in use in the final output,
10. How to display the font size in use in the final output,
11. How to display the font size in use in the final output,
12. How to display the font size in use in the final output,


Lipsum me [55-65]

\end{otherlanguage*}




    \end{anexosenv}

    % INDICE REMISSIVO
    \phantompart
    \printindex

\end{document}
