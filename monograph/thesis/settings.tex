

% Thesis settings
\newcommand{\brazil}[1]{\foreignlanguage{brazil}{#1}}
\newcommand{\english}[1]{\foreignlanguage{english}{#1}}

\newcommand{\s}[0]{\textquotesingle{}s{ }}

% What is the difference between \def and \newcommand?
% https://tex.stackexchange.com/questions/655/what-is-the-difference-between-def-and-newcommand
\def\mytextpreliminarylistname{\lang{Brief Table of Contents}{Breve Sumário}}

% How to manually set where a word is split?
% https://tex.stackexchange.com/questions/182569/how-to-manually-set-where-a-word-is-split
\hyphenation{Ge-la-im}

% Informações de dados para CAPA e FOLHA DE ROSTO
\titulo
{%
    \lang
    {Good Programming Practices \& Style}
    {Boas Práticas de Programação \& Estilo}
}
\subtitulo
{%
    \lang
    {Universal Programming Tools}
    {Ferramentas Universais de Programação}
}

\data{\today}
\autor{\brazil{Evandro Coan}}
\local{\lang{\brazil{Florianópolis, Santa Catarina} -- Brazil}{Florianópolis, Santa Catarina -- Brasil}}

\biblioteca{\lang{University Library}{Biblioteca Universitária}}
\orientador{\lang{Prof. PhD. Ricardo Azambuja Silveira}{Prof. Dr. Ricardo Azambuja Silveira}}
\coorientador{\lang{M.S. Thiago Ângelo Gelaim}{M.S. Thiago Ângelo Gelaim}}

\instituicaosigla{UFSC}
\instituicao{\lang{Federal University of \brazil{Santa Catarina}}{Universidade Federal de Santa Catarina}}
\tipotrabalho{\lang{Bachelor's Thesis}{Tese de Graduação}}

\area{\lang{Formal Languages}{Linguagens Formais}}
\formacao{\lang{Bachelor of Science Degree in Computer Science}{Grau de Bacharel em Ciência da Computação}}
\programa{\lang{Undergraduate Program in Computer Science}{Trabalho de Conclusão de Curso}}
\centro{\lang{Department of Informatics and Statistics}{Departamento de Informática e Estatística}}

% O preambulo deve conter tipo do trabalho, objetivo, nome da instituição e a área de concentração.
\preambulo
{%
    \lang
    {Thesis submitted to the~\imprimirprograma~of the~\imprimirinstituicao~to obtain the~\imprimirformacao.}
    {Tese submetida ao \imprimirprograma da \imprimirinstituicao para a obtenção do Título de \imprimirformacao.}
}

% Keywords
\newcommand{\palavraschaveingles}
{%
    \item source. \item code. \item formatter. \item beautifier. \item prettyprint. \item universal.
    \item reuse. \item blocks. \item object. \item oriented. \item programming. \item structured.
    \item parsing. \item parse. \item regular. \item expression. \item regex. \item C. \item C++.
    \item grammar. \item Turing. \item machine. \item automata. \item lexer. \item syntax. \item
    Sublime. \item Java. \item Rust. \item Shell. \item script. \item obfuscators. \item learning.
    \item syntec. \item teamicide. \item concensus. \item indentação. \item settings.
}
\newcommand{\palavraschaveportugues}
{%
    \item fonte. \item código. \item formatador. \item embelezante. \item prettyprint. \item
    universal. \item reuso. \item blocos. \item objeto. \item orientado. \item programação. \item
    estruturada. \item análise. \item analisador. \item regular. \item expressão. \item regex. \item
    C. \item C++. \item gramática. \item Turing. \item máquina. \item autômatos. \item lexer. \item
    sintaxe. \item Sublime. \item Java. \item Rust. \item Shell. \item roteiro. \item ofuscadores.
    \item aprendizado. \item Syntec. \item teamicide. \item consenso. \item indentation. \item
    configurações.
}

% Remove the colon appended to theses variables, allowing us to use other separators
\addto\captionsbrazil
{
    \renewcommand{\orientadorname}{Orientador}
    \renewcommand{\coorientadorname}{Coorientador}
}

% Create caption English translations as the sections headers
% https://tex.stackexchange.com/questions/8564/what-is-the-right-way-to-redefine-macros-defined-by-babel
\addto\captionsenglish
{
    %% adjusts names from abnTeX2
    \renewcommand{\folhaderostoname}{Title page}
    \renewcommand{\epigraphname}{Epigraph}
    \renewcommand{\dedicatorianame}{Dedication}
    \renewcommand{\errataname}{Errata sheet}
    \renewcommand{\agradecimentosname}{Acknowledgements}
    \renewcommand{\anexoname}{ANNEX}
    \renewcommand{\anexosname}{Annex}
    \renewcommand{\apendicename}{APPENDIX}
    \renewcommand{\apendicesname}{Appendix}
    \renewcommand{\orientadorname}{Supervisor}
    \renewcommand{\coorientadorname}{Co\hyp{}supervisor}
    \renewcommand{\folhadeaprovacaoname}{Approval}
    \renewcommand{\resumoname}{Abstract}
    \renewcommand{\listadesiglasname}{List of abbreviations and acronyms}
    \renewcommand{\listadesimbolosname}{List of symbols}
    \renewcommand{\fontename}{Source}
    \renewcommand{\notaname}{Note}
    %% adjusts names used by \autoref
    \renewcommand{\pageautorefname}{page}
    \renewcommand{\sectionautorefname}{section}
    \renewcommand{\subsectionautorefname}{subsection}
    \renewcommand{\subsubsectionautorefname}{subsubsection}
    \renewcommand{\paragraphautorefname}{subsubsubsection}
}

% Source Code Settings in Document
\makeatletter
\@ifpackageloaded{listings}
{
\ifenglish
    % These default values are already in English
\else
    % Listing -> Codigo fonte
    \renewcommand{\lstlistingname}{Código--fonte}

    % List of Listings -> Lista de códigos-fonte
    \renewcommand{\lstlistlistingname}{Lista de códigos--fonte}

    % Calculate the size of the header
    \calculatelisteningsheader
\fi
}{}
\makeatother


% Backref package settings, pages with citations in bibliography
\makeatletter
\@ifpackageloaded{biblatex}
{
    \renewbibmacro*{pageref}
    {
        \iflistundef{pageref}
        {\printtext{\lang{\autocap{n}o citation in the text}{\autocap{n}enhuma citação no texto}.}}
        {%
            \printtext
            {%
                \ifnumgreater{\value{citecounter}}{1}
                    {\lang{\autocap{c}ited}{\autocap{c}itado} \arabic{citecounter} \lang{times}{vezes}}
                    {\lang{\autocap{c}ited}{\autocap{c}itado} \arabic{citecounter} \lang{time}{vez}}
            }%
            \setunit{\addspace}%
            \ifnumgreater{\value{pageref}}{1}
                {\bibstring{backrefpages}\ppspace}
                {\bibstring{backrefpage}\ppspace}%
            \printlist[pageref][-\value{listtotal}]{pageref}%
        }%
    }

    \DefineBibliographyStrings{brazil}
    {
        backrefpage  = {na página},
        backrefpages = {nas páginas},
    }

    \DefineBibliographyStrings{english}
    {
        backrefpage  = {on page},
        backrefpages = {on pages},
    }
}{}
\makeatother


% Espaçamentos entre linhas e parágrafos
%
% ifpackageloaded question
% https://tex.stackexchange.com/questions/70212/ifpackageloaded-question
\makeatletter
\@ifclassloaded{memoir}
{
    % Estilo de capítulos, ver classe para maiores detalhes.Veja outros estilos em:
    % http://mirrors.ibiblio.org/CTAN/macros/latex/contrib/memoir/memman.pdf
    \chapterstyle{VZ14}
    \setlength\beforechapskip{0pt}
    \setlength\midchapskip{15pt}
    \setlength\afterchapskip{15pt}

    % O tamanho do parágrafo é dado por:
    \setlength{\parindent}{1.3cm}

    % Controle do espaçamento entre um parágrafo e outro. Tente também
    % \onelineskip
    \setlength{\parskip}{0.2cm}

    % Memoir: Warnings “The material used in the headers is too large” w/ accented titles
    % https://tex.stackexchange.com/questions/387293/how-to-change-the-page-layout-with-memoir
    \setheadfoot{30.0pt}{\footskip}
    \checkandfixthelayout
}{}
\makeatother


% Color settings across the document
\makeatletter
\@ifpackageloaded{xcolor}
{
    % RGB colors in absolute values from 0 to 255 by using `RGB` tag
    \definecolor{darkblue}{RGB}{26,13,178}

    % Definição de cores, RGB colors in percentage notation by using `rgb` tag
    \definecolor{mygreen}{rgb}{0,0.6,0}
    \definecolor{mygray}{rgb}{0.5,0.5,0.5}
    \definecolor{mymauve}{rgb}{0.58,0,0.82}

    % Configurações de aparência do PDF final
    \definecolor{figcolor}{rgb}{1,0.4,0}  % orange
    \definecolor{tabcolor}{rgb}{1,0.4,0}  % orange
    \definecolor{eqncolor}{rgb}{1,0.4,0}  % orange
    \definecolor{linkcolor}{rgb}{1,0.4,0} % orange
    \definecolor{citecolor}{rgb}{1,0.4,0} % orange
    \definecolor{seccolor}{rgb}{0,0,1}    % blue
    \definecolor{abscolor}{rgb}{0,0,1}    % blue
    \definecolor{titlecolor}{rgb}{0,0,1}  % blue
    \definecolor{biocolor}{rgb}{0,0,1}    % blue

    % Alterando o aspecto da cor azul
    \definecolor{blue}{RGB}{41,5,195}

    % Informações do PDF
    \@ifpackageloaded{hyperref}
    {
        \hypersetup
        {
            pdftitle={\@title},
            colorlinks=true, % false: boxed links; true: colored links
            linkcolor=darkblue, % color of internal links
            citecolor=darkgreen, % color of links to bibliography
            filecolor=black, % color of file links
            urlcolor=linkcolor,
            bookmarksdepth=4
        }
        \ifenglish
            \hypersetup
            {
                pdfauthor={Author},
                pdfsubject={Thesis' Abstract},
                pdfcreator={LaTeX with abnTeX2 for UFSC},
                pdfkeywords={abnt}{latex}{UFSC}{abntex2}{thesis},
            }
        \else
            \hypersetup
            {
                pdfauthor={Autores},
                pdfsubject={Resumo da tese},
                pdfcreator={LaTeX com abnTeX2 para UFSC},
                pdfkeywords={abnt}{latex}{UFSC}{abntex2}{tese},
            }
        \fi
    }
}{}
\makeatother


% Fontes das entradas do sumario
\makeatletter
\renewcommand*{\l@chapter}[2]
{%
    \l@chapapp{\uppercase{#1}}{#2}{\cftchaptername}
}
\renewcommand*{\l@section}[2]
{%
    \l@chapapp{\ABNTEXsectionfont\uppercase{#1}}{#2}{\cftsectionname}
}
\makeatother

% Changing the font of the numbers in the ToC in the memoir class
% https://tex.stackexchange.com/questions/14314/changing-the-font-of-the-numbers-in-the-toc-in-the-memoir-class
\renewcommand{\cftpartfont}{\ABNTEXpartfont\color{darkblue}}
\renewcommand{\cftpartpagefont}{\ABNTEXpartfont\color{black}}

\renewcommand{\cftchapterfont}{\ABNTEXchapterfont\color{darkblue}}
\renewcommand{\cftchapterpagefont}{\ABNTEXchapterfont\color{black}}

\renewcommand{\cftsectionfont}{\ABNTEXsectionfont\color{darkblue}}
\renewcommand{\cftsectionpagefont}{\ABNTEXsectionfont\color{black}}

\renewcommand{\cftsubsectionfont}{\ABNTEXsubsectionfont\color{darkblue}}
\renewcommand{\cftsubsectionpagefont}{\ABNTEXsubsectionfont\color{black}}

\renewcommand{\cftsubsubsectionfont}{\ABNTEXsubsubsectionfont\color{darkblue}}
\renewcommand{\cftsubsubsectionpagefont}{\ABNTEXsubsubsectionfont\color{black}}

\renewcommand{\cftparagraphfont}{\ABNTEXsubsubsubsectionfont\color{darkblue}}
\renewcommand{\cftparagraphpagefont}{\ABNTEXsubsubsubsectionfont\color{black}}


