

% The \phantomsection command is needed to create a link to a place in the document that is not a
% figure, equation, table, section, subsection, chapter, etc.
%
% When do I need to invoke \phantomsection?
% https://tex.stackexchange.com/questions/44088/when-do-i-need-to-invoke-phantomsection
\cleardoublepage
\phantomsection


% Is it possible to keep my translation together with original text?
% https://tex.stackexchange.com/questions/5076/is-it-possible-to-keep-my-translation-together-with-original-text
\chapter{\lang{References' Abstracts}{Resumos das Referências}}


\begin{englishtext}
\begin{enumerate}

    \item In this paper we argue that there is a necessity for automating
    modifications to legacy assets. We propose a five layered process for the
    introduction and employment of tool support that enables automated
    modification to entire legacy systems. Furthermore, we elaborately discuss
    each layer on a conceptual level, and we make appropriate references to
    sources where technical contributions supporting that particular layer can
    be found. We sketch the perspective that more and more people working in the
    software engineering area will be contributing to working on existing
    systems and/or tools to support such work. \cite{legacyAssets}

    \item

\end{enumerate}
\end{englishtext}


