


\begin{abstract}

    Faz-se um estudo sobre o que é, para que servem os Beautifiers, assim como abordagens sobre o
    que são boas práticas de programação e por que devemos segui-las para um boa eficiência ao
    escrever códigos nas mais diversas linguagens de programação. Os softwares formatadores de
    código fonte atuais, também conhecidos como Beautifiers, são limitados a um conjunto similar, ou
    mesmo à uma única linguagem, e além de muitos, serem limitados ao que eles podem fazer por você
    ao processar/formatar o código \cite{Terence}.

    \medskip

    Portanto espera-se o final do trabalho, conhecer-se quais são as ferramentas que existem e quais
    delas são as melhores que podem ser utilizadas para o auxilio do programador durante a escrita
    de códigos das mais diversas linguagens de programação. Além de proposta de uma nova ferramenta
    com o intuído de centralizar em uma único programa o abordagem das mais diversas linguagens de
    programação.

    \medskip

    \textbf{Palavras-chave:}
    source, code, formatter, beautifier, prettyprint, universal, reuse, blocks, object, oriented,
    programming, structured, parsing, parse, regular, expression, regex, C, C++,  grammar, Turing,
    machine, automata, lexer, syntax, sublime, Java, Rust, shell, script, obfuscators, learning,
    syntec, teamicide, concensus, indent, settings.

\end{abstract}





